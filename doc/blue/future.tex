\documentclass{bluenote}
\usepackage[show]{ed}
\usepackage[hyperref=auto,style=alphabetic,isbn=false]{biblatex}
\usepackage{bibtweaks}
\usepackage{amstext}
\addbibresource{kwarcpubs.bib}
\addbibresource{extpubs.bib}
\addbibresource{kwarccrossrefs.bib}
\addbibresource{extcrossrefs.bib}
\usepackage{stex-logo}
\usepackage{xspace}
\usepackage{lstomdoc}
\usepackage{lststex}\lstset{language=[sTeX]TeX}

\title{Rethinking Modules and Semantic Macros in \protect\sTeX}
\author{Michael Kohlhase\\Computer Science, Jacobs University.de}

\def\smglom{\textsf{SMGloM}\xspace}
\def\omdoc{\textsf{OMDoc}\xspace}
\def\openmath{\textsf{OpenMath}\xspace}
\def\latexml{{\LaTeX}ML\xspace}
\def\lmh{\textsf{lmh}\xspace}
\blueProject{\sTeX}\blueURI{http://github.com/KWARC/sTeX}
\def\lstomdoc{\lstinline[language={[1.3]OMDoc},mathescape]}
\def\nlex#1{``\emph{#1}''}
\def\meta#1{\ensuremath{\langle\kern-.2em\langle\text{#1}\rangle\kern-.2em\rangle}}
\begin{document}
\maketitle
\begin{abstract}
  In this note, we document the state of rethinking the \sTeX infrastructure in terms of
  the \smglom.
\end{abstract}

\section{Introduction}
We have been using \sTeX as the encoding for the Semantic Multilingual Glossary of
Mathematics (\smglom; see~\cite{IanJucKoh:sps14}). The \smglom data model has been taxing
the representational capabilities of \sTeX with respect to multilingual support and
verbalization definitions; see~\cite{Kohlhase:dmesmgm14}, which we assume as background
reading for this note. Multilinguality support has been started with
in~\cite{KohGin:smss:svn} and will (no longer) be covered in this note.

\section{Mixed Presentation/Content Markup}
Currently, \sTeX produces content markup in the OpenMath encoding. But often \sTeX
formulae often contain bits of presentational {\LaTeX}, which \latexml has to convert into
OpenMath heuristically, which often leads to non-optimal results. Therefore we want to
rethink the representation of formulae, instead of insisting on homogeneous content markup
in OpenMath, we switch to MathML allow mixed presentation/content MathML, which conforms
much more closely to user input (preserving presentational bits) and postpones full
semantification to later stages of processing. Let us make an example: consider the
formula $(a+b)^n$ encoded as \lstinline|\exp{a+b}n|, where we have a semantic macro
\lstinline|\exp| defined by \lstinline|\symdef{exp}[2]{#1^{#2}}| in module
\lstinline|arith| Then we should create
\begin{lstlisting}[language=MathML]
<math>
  <apply>
     <csymbol cd="arith">exp</csymbol>
     <mrow><ci>a</ci><mo>+</mo><ci>b</ci></mrow>
     <ci>n</ci>
  </apply>
</math>
\end{lstlisting}
Note that MathML does indeed allow to freely mix content and presentation MathML, here we
have an application produced by the semantic macro \lstinline|\exp| applied to the
presentational $a+b$, where $a$ and $b$ are ``content identifiers''. 

A side effect of the switch to MathML is that complex variable names are much nicer in
MathML: $x_5$ is just 
\begin{lstlisting}[language=MathML]
<ci name="x5"><msub><mi>x</mi><mn>5</mn></msub></ci>
\end{lstlisting}

Finally, there is another effect of the switch to MathML: we finally have a good
representation of formulae with text in them, e.g. the set 
\[\{O\in\wp(X)\mid O \text{ is the union of open balls}\}\]%|
which we can encode as 
\begin{lstlisting}
\setst{O}{\inset{O}{\powerset{X}}}{\text{\ensuremath{O} is the union of open balls}}
\end{lstlisting}
given suitable semantic macros \lstinline|\setst|, \lstinline|\inset|, and
\lstinline|\powerset|. This should generate the mixed representation
\begin{lstlisting}[language=MathML]
<bind>
  <apply>
    <csymbol cd="sets">setst</csymbol>
    <apply><csymbol cd="sets">powerset</csymbol></apply>
  </apply>
  <bvar><ci>O</ci></bvar>
  <mtext><math><ci>O</ci></math> is the union of open balls</mtext>
</bind>
\end{lstlisting}

\section{Verbalization Definitions}

Currently, \sTeX only supports notation definitions for symbols, but we also need
verbalization definitions for flexiformal mathematics; see~\cite{Kohlhase:dmesmgm14} for a
description of the concept and background on their use and~\cite[section
5]{Kohlhase:dmsmglom14} for first ideas towards an \sTeX encoding. We will extend the
latter here.

\subsection{\omdoc markup for Notation and Verbalization Definitions}

In \omdoc, notation definitions are supplements to the \lstomdoc|symbol| declaration. We
have the following markup -- in a simple case. 

\begin{lstlisting}[language={[1.3]OMDoc},mathescape, 
  label=lst:nd-classical,caption=A classical Notation Definition]
<symbol name="foo"/>
<notation for="foo">
  <protoype>
    <om:OMS cd="$\meta{CD}$" name="foo"/>
  </prototype>
  <rendering>
    <msubsup><mi>f</mi><mi>o</mi><mi>o</mi></msubsup>
  </rendering>
<notation>
\end{lstlisting}
where \meta{CD} is the current theory. For functional/binding symbols, the prototype is an
\openmath application/binding expression, where the argument positions are meta-variables
\lstomdoc|<expr name="\meta{name}/>"| which are being picked up in the
\lstomdoc|rendering| element as corresponding recursive calls \lstomdoc|<render name="\meta{name}"/>|.

For verbalization definitions, we want to reuse notation definitions, so a mathematical
concept \nlex{big array raster} may be given a symbol name \lstomdoc|bar| and the notation
definition. 
\begin{lstlisting}[language={[1.3]OMDoc},mathescape,
  label=lst:vd-classical,caption=A classical Verbalization Definition]
<symbol name="bar"/>
<notation for="bar">
  <protoype>
    <om:OMS cd="$\meta{CD}$" name="bar"/>
  </prototype>
  <rendering>big array raster</rendering>
<notation>
\end{lstlisting}
Note that verbalizations are part of text, so the contents of the \lstomdoc|rendering|
element are as well. In cases, where a symbol has both notations and verbalizations,
e.g. addition, which has the notation $+$ and the verbalization \nlex{plus}, the notation
element has multiple \lstomdoc|rendering| children. 

But, in our new, multilingual infrastructure, symbol and notation both go into the module
signature, whereas the verbalizations go into the language bindings. Therefore we propose
to ungroup the notation definitions, use the \lstomdoc|prototype| and \lstomdoc|rendering|
elements directly, and cross-reference them. For plus this would give rise to the following
\omdoc markup 

\begin{lstlisting}[language={[1.3]OMDoc},mathescape, 
  label=lst:nd-new,caption=Proposed Notation Definition]
<symbol name="plus"/>
<protoype for="plus" name="plus.proto">
  <om:OMS cd="arithmetics" name="plus"/>
</prototype>
<rendering for="plus.proto">
  <msubsup><mi>f</mi><mi>o</mi><mi>o</mi></msubsup>
</rendering>
\end{lstlisting}
 
\noindent in the module signature and 

\begin{lstlisting}[language={[1.3]OMDoc},mathescape,
  label=lst:vd-classical,caption=Proposed Verbalization Definition]
<rendering for="plus.proto">plus</rendering>
\end{lstlisting}
\noindent in the (English) language binding. Note that with the ungrouping, we can also be
more flexible about where to put language-specific rendering (see Listing \ref{lst:gcd}
for an example).

\subsection{Direct \protect\sTeX Encoding of Verbalization Definitions}

In \sTeX use that the \lstinline|\symdef| macro for notation definitions,
e.g. \lstinline|\symdef{foo}{f^o_o}| creates a semantic macro \lstinline|\foo| that
expands to $f_o^o$. Note that such semantic macros are only intended to be used in math
mode (they usually lead to telltale errors in text mode). In the \latexml workflow does
two more things: it creates a \lstomdoc|symbol| element and a \lstomdoc|notation| element
as in Listing~\ref{lst:classical-nd}.

It seems natural to use the \lstinline|\symdef|/\lstinline|\symvariant| macros for
verbalization definitions as well. If there is already semantic macro for
\lstinline|\bar|, we can simply use
\begin{lstlisting}
\symvariant{bar}{en}{\text{big array raster}}
\end{lstlisting}
With this, \lstinline|\bar[de]| expands to \lstinline|big array raster|. Note that the
user has to keep track on which variants are math mode and which are text mode, and make
sure that he uses the right one in each situation. But we can hide this from the use by
making \lstinline|\bar| be mode-sensitive\footnote{Thanks to Deyan Ginev for this
  suggestion}: in math mode --this can be checked with \lstinline|\ifmmode| in \TeX, it
selects the internal variant defined by \lstinline|\symdef|/\lstinline|\symvariant|, and
in text mode it selects the internal macro defined by two new macros
\lstinline|\verbdef|/\lstinline|\verbvariant|. Note that as we are in a language binding
for verbalization definitions\footnote{We will not -- for the moment -- support
  verbalization definitions in the monolingual setting.}, we do no have to specify the
language; moreover, we can have variants for the math/text modes separately. This is
useful e.g. for greatest common divisors, which have language-sensitive notations and
verbalizations (see Listing~\ref{lst:gcd}).

\begin{lstlisting}[label=lst:gcd,caption=Notation and Verbalization Definitions for
  Greatest Common Divisor]
\begin{modsig}{gcd}
  \symdef[name=gcd]{gcdOp}{\text{gcd}}
  \symdef{gcd}[1]{\prefix\gcdOp{#1}}
\end{modsig}

\begin{modnl}{gcd}{en}
  \verbdef{gcdOp}{greatest common divisor}
  \verbdef{gcd}[2]{\gcdOp of #1 and #2}
  \verbvariant{gcdOp}{plural}{greatest common divisors}
  \verbvariant{gcd}[2]{plural}{\gcdOp[plural] of #1 and #2}
\end{modnl}

\begin{modnl}{gcd}{en}
  \verbdef{gcdOp}{\text{gcd}}
  \verbdef{gcd}[2]{\gcdOp von #1 und #2}
  \symvariant{gcdOp}{de}{\text{ggT}}
\end{modnl}
\end{lstlisting}
Note that the thus generated semantic macros are not only mode-sensitive, but also
language-sensitive: for every language there is a ``first'' \lstinline|\verbdef| followed
by \lstinline|\verbvariant|s. These generate semantic macros, that react to the value of
the switch \lstinline|\stex@lang| -- implicitly set by the \lstinline|modnl| environment
or explicitly by \lstinline|sTeXselectlanguage|\ednote{MK@MK: this still needs to be
  implemented; also implement variants of the other babel selection mechanisms like
  foreignlanguage, etc. best in smultiling.dtx.} and select the right variant in the right
language.

\subsection{Implicit Verbalization Definitions from Definienda}

But in most situations, an explicit \lstinline|\verbdef| is unnecessary, since we have the
definiendum markup. In the situation of Listing~\ref{lst:newmods}, we have a symbol
\lstinline|bar| generated by the \lstinline|\symdef| and a definiendum for the symbol
\lstinline|bar| marked up by the \lstinline|\defiii| macro -- see~\cite{Kohlhase:smms:svn}
for details on \lstinline|\def*|. Note that the optional argument of \lstinline|\defiii|
is used to specify the symbol name, here \lstinline|bar| here. We could let \latexml let
generate the equivalent of a \lstinline|verbdef| as above implicitly, freeing the user
from writing down specifications twice.

\begin{lstlisting}[label=lst:newmods,caption=Definiendum Markup in Language Bindings]
\begin{modnl}[creators=miko,primary]{foo}{en}
\begin{definition}
  A \defiii[bar]{big}{array}{raster} ($\bar$) is a\ldots, it is much bigger
  than a \defiii[sar]{small}{array}{raster}. 
\end{definition}
\end{modnl}

\begin{modnl}[creators=miko]{foo}{de}
\begin{definition}
  Ein \defiii[bar]{gro"ses}{Feld}{Raster} ($\bar$) ist ein\ldots, es
  ist viel gr"o"ser als ein \defiii[sar]{kleines}{Feld}{Raster}. 
\end{definition}
\end{modnl}
\end{lstlisting}

But let us also look at a more interesting symbol: the ``special linear group'' already
discussed in~\cite{Kohlhase:dmsmglom14}. Here the \sTeX verbalization definition would be 
\begin{lstlisting}
\verbdef[name=slgroup]{SLGroup}[2]{special linear group of order #1 over #2}
\end{lstlisting}
Here we have a problem with retrieving this from the definition without additional
markup. A normal definition would have the form
\begin{lstlisting}[label=lst:slg1,caption=A Discontiguous Definiendum]
\begin{definition}
  The \defiii[slgroup]{special}{linear}{group} \notatiendum{$\SLgroup{n}{F}$}
  of degree $n$ over a \trefi[field]{field} $F$ is ...
\end{definition}
\end{lstlisting}
In particular, the definiendum is discontiguous and usually only the ``head'' is
explicitly emphasized by boldface font. In this situation, we employ a a ``continuation
markup'', where we associate \nlex{of order} with the first argument and \nlex{over} with
the second. 
\begin{lstlisting}[label=lst:slg,caption=Continuation/Argument Markup for Discontiguous
  Definienda]
\begin{definition}
  The \defiii[slgroup]{special}{linear}{group}
  \notatiendum{$\SLgroup{n}{F}$}
  \defarg[for=slgroup,1,opt]{of degree \arg{$n$}} 
  \defarg[for=slgroup,2,opt]{over \arg{a \trefi[field]{field} $F$}} is ...
\end{definition}
\end{lstlisting}
With the \lstinline|\defarg| macro, we continue the definiendum started with the
\lstinline|\defiii| by two argument phrases. These in turn contain the actual argument
marked up by \lstinline|\arg|, which escapes the argument. This definiendum markup is
equivalent to an implicit call of \lstinline|\verbdef| and generates the notation
definition in Listing~\ref{lst:nd-slg}. 
\begin{lstlisting}[language={[1.3]OMDoc},mathescape,
  label=lst:nd-slg,caption=Verbalization Definition induced by Listing \ref{lst:slg}]
<protoype name="slgroup.proto">
  <om:OMA>
    <om:OMS cd="$\meta{CD}$" name="slgroup" cr="fun"/>
    <expr name="arg1"/>
    <expr name="arg2"/>
  </om:OMA>
</prototype>
<rendering for="slgroup.proto">
  <phrase cd="$\meta{CD}$" name="slgroup" cr="fun">special linear group</phrase>
  <phrase cr="arg1">of order <render name="arg1"/></phrase>
  <phrase cr="arg2">over <render name="arg2"/></phrase>
</rendering>
\end{lstlisting}
\ednote{MK: we need to have the phrase-level connections in this in a way that generates
  IDs in the presentation process}

\section{Conclusion}
We have described a set of new functionalities for \sTeX and specified some aspects of
them. Now, they need to be implemented and tested. 

\printbibliography
\end{document}

%%% Local Variables: 
%%% mode: latex
%%% TeX-master: t
%%% End: 

%  LocalWords:  maketitle smglom IanJucKoh sps14 dmesmgm14 latexml exp symdef arith mrow
%  LocalWords:  lstlisting csymbol csymbol ci ci ci ci mrow msub mn mn msub wp setst bvar
%  LocalWords:  powerset ensuremath bvar mtext mtext dmsmglom14 lstomdoc mathescape om es
%  LocalWords:  protoype msubsup msubsup expr omdoc nlex symvariant verbdef lst newmods
%  LocalWords:  defiii slgroup notatiendum trefi defarg defhead printbibliography nd-new
%  LocalWords:  nd-classical vd-classical plus.proto classical-nd ifmmode verbvariant von
%  LocalWords:  modsig modnl und ggT noindent stex@lang sTeXselectlanguage ednote miko gr
%  LocalWords:  foreignlanguage smultiling.dtx ldots sar Ein gro ses ist ein viel ser als
%  LocalWords:  kleines slg1 slg nl-slg nd-slg slgroup.proto cr
