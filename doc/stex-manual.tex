\makeatletter
\ifcsname if@infulldoc\endcsname\else
    \expandafter\newif\csname if@infulldoc\endcsname\@infulldocfalse
\fi
\makeatother

\csname if@infulldoc\endcsname\else

\def\bibfolder{../lib/bib}

\RequirePackage{paralist}
\ifcsname stexdocpath\endcsname\else\def\stexdocpath{.}\fi
\documentclass[full]{l3doc}
%\RequirePackage{document-structure}
\usepackage[hyperref=auto,style=alphabetic]{biblatex}
%\usepackage[mathhub=\stexdocpath/mh,usedeps]{stex}
\usepackage[lang={en,de}]{stex}

\usepackage{rustex}
\usepackage{stex-highlighting,stexthm}

\srefsetin[sTeX/Documentation]{documentation}{the \stex Documentation}

\makeatletter
\providecommand{\HTML}{\textsc{html}\xspace}%
\providecommand{\XML}{\textsc{xml}\xspace}%
\providecommand{\PDF}{\textsc{pdf}\xspace}%
\providecommand\openmath{\textsc{OpenMath}\xspace}
\providecommand\OMDoc{\textsc{OMDoc}\xspace}
\DeclareRobustCommand\LaTeXML{L\kern-.36em%
        {\sbox\z@ T%
         \vbox to\ht\z@{\hbox{\check@mathfonts
                              \fontsize\sf@size\z@
                              \math@fontsfalse\selectfont
                              A}%
                        \vss}%
        }%
        \kern-.15em%
%        T\kern-.1667em\lower.5ex\hbox{E}\kern-.125em\relax
%        {\tt XML}}
        T\kern-.1667em\lower.4ex\hbox{E}\kern-0.05em\relax
        {\scshape xml}\xspace}%
\def\mmt{\textsc{Mmt}\xspace}
\makeatother


\newif\ifhadtitle\hadtitlefalse

\def\stexversion{3.3.0}
\def\changedate{\today}
\def\stextoptitle#1#2{\title{#1\thanks{Version {\stexversion} (last revised {\changedate})} }\def\thispkg{#2}}

\author{Michael Kohlhase, Dennis Müller\\
 	FAU Erlangen-Nürnberg\\
 	\url{http://kwarc.info/}
}

\def\stexmaketitle{\ifhadtitle\else\hadtitletrue\maketitle\fi}

\ExplSyntaxOn

  \def\docmodule{
    \begin{document}
      \EnableManual
      \EnableDocumentation
      \EnableImplementation
      \stexmaketitle
      \tableofcontents
      \int_gincr:N \l_stex_docheader_sect
      \exp_args:Ne \__stex_mathhub_find_manifest:n {\stex_file_use:N \c_stex_mathhub_file / sTeX / Documentation}
      \str_if_empty:NF \l__stex_mathhub_manifest_str {
        \usemodule[sTeX/Documentation]{macros?AllMacros}
      }
      \DocInput{\jobname.dtx}
      \clearpage
      \PrintIndex
      \printbibliography
    \end{document}
  }

  \bool_new:N \g_stexdoc_typeset_manual_bool
  \NewDocumentCommand \EnableManual {}{
    \bool_gset_true:N \g_stexdoc_typeset_manual_bool
  }
  \NewDocumentCommand \DisableManual {}{
    \bool_gset_false:N \g_stexdoc_typeset_manual_bool
  }
  \NewDocumentEnvironment {stexmanual} {} {
    \bool_if:NTF \g_stexdoc_typeset_manual_bool
      {\bool_set_false:N \l__codedoc_in_implementation_bool}
      {\comment}
  }{
    \bool_if:NF \g_stexdoc_typeset_manual_bool {\endcomment}
  }
\ExplSyntaxOff

%\usepackage{makeidx}
%\makeindex

%\usepackage{document-structure}


\usepackage{lststex,mdframed}
\usepackage[most]{tcolorbox}

\lstset{literate=%
    {Ö}{{\"O}}1
    {Ä}{{\"A}}1
    {Ü}{{\"U}}1
    {ß}{{\ss}}1
    {ü}{{\"u}}1
    {ä}{{\"a}}1
    {ö}{{\"o}}1
    {~}{{\textasciitilde}}1
}

\newenvironment{framed}[1][]{
  \ifstexhtml\par\vbox\bgroup
    \csname exp_args:Nne\endcsname\begin{stex_annotate_env}{%
      style:border=solid 1px black,%
      style:width=var(--this-width),%
      style:min-width=var(--this-width),%
      style:--this-width=calc(var(--current-width) - 6px),%
      style:padding=3px,%
      style:margin-top=5px,%
      style:margin-bottom=5px%
    }
    \csname stex_annotate_invisible:n\endcsname{ }%
    \begin{stex_annotate_env}{%
      style:--current-width=var(--this-width);%
    }\csname stex_annotate_invisible:n\endcsname{ }
  \else\begin{mdframed}[#1]\fi
  %\begin{center}%
}{%
  %\end{center}%
  \ifstexhtml
    \end{stex_annotate_env}\end{stex_annotate_env}\egroup\par
  \else\end{mdframed}\fi
}
\newcommand{\scaled}[2][0.9\hsize]{\begin{center}\resizebox{#1}{!}{\begin{minipage}{\textwidth} #2 \end{minipage}}\end{center}}

\makeatletter
\ExplSyntaxOn

\def\doc_exbox:nnn#1#2#3{
  \begin{sexample}[#3]
    Input:
    \begin{framed}[linewidth=1pt,backgroundcolor=white]\small
      #1
    \end{framed}
    Output:
    \begin{framed}[linewidth=1pt,backgroundcolor=white]\small
      #2
    \end{framed}
  \end{sexample}
}


\NewDocumentCommand\stexexamplefile{O{} m O{} O{}}{
  \stex_resolve_path_pair:Nxx \l_@@_filepath_str {\tl_to_str:n{#1}} {\tl_to_str:n{#2}}
  \doc_exbox:nnn{
    \hfill File~\str_if_empty:nTF{#1}{
      \prop_if_exist:NT \l_stex_current_archive_prop {
        [\texttt{\prop_item:Nn \l_stex_current_archive_prop {id}}]
      }
    }{[#1]}\texttt{\tl_to_str:n{#2}}
    \_lststex_parse_args:n{#3}
    \exp_args:Nno \use:nn{\lstinputlisting[} \l_lststex_return_tl ]{\l_@@_filepath_str}
  }{
    \inputref[#1]{#2}
  }{#4}
}

\newwrite\testoutfile
\NewDocumentCommand\stexexample{O{} O{}}{
  \begingroup 
  \catcode`\\=12\relax
  \catcode`\#=12\relax
  \catcode`\&=12\relax
  \catcode`\$=12\relax
  \catcode`\^=12\relax
  \catcode`\_=12\relax
  \catcode`\ =12\relax
  \catcode`^^J=12\relax
  \endlinechar=`^^J
  \newlinechar=-1
%^^A    \everyeof{\noexpand}
  \example_a:nnn{#1}{#2}
}
\long\def\example_a:nnn #1 #2 #3 {
  \endgroup
  \immediate\openout\testoutfile=\jobname.exmpl
  \immediate\write\testoutfile{
    \c_backslash_str begin{stexcode}[#1]
    \detokenize{^^J}#3
    \c_backslash_str end{stexcode}
  }
  \immediate\closeout\testoutfile
  \doc_exbox:nnn{
    \catcode`\#=12\relax
    \csname @ @ input\endcsname{\jobname.exmpl}
  }{
    \immediate\openout\testoutfile=\jobname.exmpl
    \immediate\write\testoutfile{#3}
    \immediate\closeout\testoutfile
    \csname @ @ input\endcsname \jobname.exmpl\relax
  }{#2}
  \peek_charcode_remove:NT ^^J
}

\ExplSyntaxOff
\makeatother

\makeatletter
\newcount\example@counter\example@counter=0
\newtcolorbox{exampleborderbox}[1][]{
  empty,
  title={Example \the\example@counter #1},
  attach boxed title to top left,
     minipage boxed title,
  boxed title style={empty,size=minimal,toprule=0pt,top=1pt,left=3mm,overlay={}},
  coltitle=blue,fonttitle=\bfseries,
  parbox=false,boxsep=0pt,left=3mm,right=0mm,top=2pt,breakable,pad at break=0mm,
     before upper=\csname @totalleftmargin\endcsname0pt, 
  overlay unbroken={\draw[blue,line width=2pt] ([xshift=-0pt]title.north west) -- ([xshift=-0pt]frame.south west); },
  overlay first={\draw[blue,line width=2pt] ([xshift=-0pt]title.north west) -- ([xshift=-0pt]frame.south west); },
  overlay middle={\draw[blue,line width=2pt] ([xshift=-0pt]frame.north west) -- ([xshift=-0pt]frame.south west); },
  overlay last={\draw[blue,line width=2pt] ([xshift=-0pt]frame.north west) -- ([xshift=-0pt]frame.south west); },
  outer arc=4pt%
}

\ExplSyntaxOn
\stexstyleexample{
  \global\advance\example@counter by 1
  \tl_if_empty:NTF\thistitle{
    \begin{exampleborderbox}
  }{
    \begin{exampleborderbox}[ (\thistitle)]
  }
}{
    \end{exampleborderbox}
}

\ExplSyntaxOff\makeatother

\usetikzlibrary{calc}

\def\textwarning{\includegraphics[width=1.2em]{stex-dangerous-bend}\xspace}
\newtcolorbox{dangerbox}{
  breakable,
  enhanced,
  left=0pt,
  right=0pt,
  top=8pt,
  bottom=8pt,
  colback=white,
  colframe=red,
  width=\textwidth,
  enlarge left by=0mm,
  boxsep=5pt,
  fontupper=\small,
  arc=4pt,
  outer arc=4pt,
  leftupper=1.5cm,
  overlay={
    \node[anchor=west] at ([xshift=10pt]$(frame.north west)!0.5!(frame.south west)$)
       {\includegraphics[width=1cm,height=1cm]{stex-dangerous-bend}};}
}

\protected\def\TODO#1{\textcolor{red}{TODO}\footnote{\textcolor{red}{TODO: #1}}}

\definecolor{darkgreen}{rgb}{0.0, 0.5, 0.0}

\usepackage[solutions]{problem}
\usepackage{hwexam}
\newtcolorbox{problemborderbox}[1][]{
  empty,
  title={Exercise #1},
  attach boxed title to top left,
     minipage boxed title,
  boxed title style={empty,size=minimal,toprule=0pt,top=1pt,left=3mm,overlay={}},
  coltitle=darkgreen,fonttitle=\bfseries,
  parbox=false,boxsep=0pt,left=3mm,right=0mm,top=2pt,breakable,pad at break=0mm,
     before upper=\csname @totalleftmargin\endcsname0pt, 
  overlay unbroken={\draw[darkgreen,line width=2pt] ([xshift=-0pt]title.north west) -- ([xshift=-0pt]frame.south west); },
  overlay first={\draw[darkgreen,line width=2pt] ([xshift=-0pt]title.north west) -- ([xshift=-0pt]frame.south west); },
  overlay middle={\draw[darkgreen,line width=2pt] ([xshift=-0pt]frame.north west) -- ([xshift=-0pt]frame.south west); },
  overlay last={\draw[darkgreen,line width=2pt] ([xshift=-0pt]frame.north west) -- ([xshift=-0pt]frame.south west); },
  outer arc=4pt%
}

\ExplSyntaxOn
\stexstyleproblem{
  \tl_if_empty:NTF\thistitle{
    \begin{problemborderbox}
  }{
    \begin{problemborderbox}[ (\thistitle)]
  }
}{
    \end{problemborderbox}
}
\ExplSyntaxOff

\newtcolorbox{experimental}{
  breakable,
  enhanced,
  left=0pt,
  right=0pt,
  top=8pt,
  bottom=8pt,
  colback=white,
  colframe=gray,
  width=\textwidth,
  enlarge left by=0mm,
  boxsep=5pt,
  fontupper=\small,
  arc=4pt,
  outer arc=4pt,
  leftupper=1.5cm,
  overlay={
    \node[anchor=west] at ([xshift=10pt]$(frame.north west)!0.5!(frame.south west)$)
       {\includegraphics[height=1cm]{stex-experimental}};}
}


\usetikzlibrary{decorations.pathmorphing,shapes,arrows,calc}
% Taken from pgflibrarytikzmmt.code.tex
\newcommand{\mmtarrowtip}{angle 45}
\newcommand{\mmtarrowtipmonoright}{right hook}

\tikzstyle{include}=[\mmtarrowtipmonoright-\mmtarrowtip,thick]
\tikzstyle{morph}=[-\mmtarrowtip,thick]
\tikzstyle{preview}=[decorate, decoration={coil,aspect=0,amplitude=1pt,
                                                  segment length=6pt,
                                                  pre=lineto,pre length=3pt,
                                                  post=lineto,post length=5pt}, thick]
\tikzstyle{view}=[preview,-\mmtarrowtip]


% TIKZ RULES
\def\mmtlogo{
\begin{tikzpicture}

  % White Background (Margins are eyeballed)
  % This is necessary because we paste white over arrows later.
  % If somebody want's to do the full song and dance with
  % interrupted arrows to get transparent background, be my guest.

  \fill[white!] (-0.01,0.15) rectangle (1.11,-0.95);

  % Arrows
  \draw [blue, include] (0,0)     -- (1.1,0);
  \draw [green, morph] (0,-0.4)  -- (1.1,-0.4);
  \draw [red, view]   (-0,-0.8) -- (1.1,-0.8);

  % Cutout for letters
  \fill[white] (0.33,0.1) rectangle (0.66,-0.9);

  % Letters
  \node at (0.18,0)    (nodeM1) {\large M};
  \node at (0.18,-0.4) (nodeM2) {\large M};
  \node at (0.21,-0.8) (nodeT)  {\large T};

\end{tikzpicture}
}

\newtcolorbox{mmtbox}{
  breakable,
  enhanced,
  left=0pt,
  right=0pt,
  top=8pt,
  bottom=8pt,
  colback=white,
  colframe=green,
  width=\textwidth,
  enlarge left by=0mm,
  boxsep=5pt,
  fontupper=\small,
  arc=4pt,
  outer arc=4pt,
  leftupper=1.5cm,
  overlay={
    \node[anchor=west] at ([xshift=10pt]$(frame.north west)!0.5!(frame.south west)$)
       {\mmtlogo};}
}

\AtBeginDocument{\catcode`_=8}

\infulldoctrue

\begin{document}
  \csname if@infulldoc\endcsname\else
	\title{
		The {\stex{3}} Manual
		\thanks{Version {\fileversion} (last revised {\filedate})}
 	}
	\author{Michael Kohlhase, Dennis Müller\\
		FAU Erlangen-Nürnberg\\
		\url{http://kwarc.info/}
	}
	\pagenumbering{roman}
	\maketitle
	
	\begin{abstract}
  \sTeX is a collection of {\LaTeX} packages that allow to markup documents semantically
  without leaving the document format.

  Running `pdflatex` over sTeX-annotated documents formats them into normal-looking
  PDF. But sTeX also comes with a conversion pipeline into semantically annotated HTML5,
  which can host semantic added-value services that make the documents active
  (i.e. interactive and user-adaptive) and essentially turning {\LaTeX} into a document
  format for (mathematical) knowledge management (MKM).
  
  \sTeX augments {\LaTeX} with
  \begin{itemize}
  \item \emph{semantic macros} that denote and distinguish between mathematical concepts,
    operators, etc. independent of their notational presentation,
  \item a powerful \emph{module system} that allows for authoring and importing individual
    fragments containing document text and/or semantic macros, independent of -- and
    without hard coding -- directory paths relative to the current document, and
  \item a mechanism for exporting \sTeX documents to (modular) XHTML, preserving all the
    semantic information for semantically informed knowledge management services.
  \end{itemize}
\end{abstract}

%%% Local Variables:
%%% mode: latex
%%% TeX-master: "stex-manual"
%%% End:
\bigskip

  This is the user manual for the \sTeX package and 
  associated software. It is primarily directed at end-users 
  who want to use \sTeX to author semantically
  enriched documents. For the full documentation, see
  \href{\basedocurl/stex.pdf}{the \sTeX documentation}
	
	\makeatletter
		\renewcommand\part{%
    		\clearpage
  			\thispagestyle{plain}%
  			\@tempswafalse
  			\null\vfil
  			\secdef\@part\@spart%
  		}
		\newcounter{chapter}
		\numberwithin{section}{chapter}
		\renewcommand\thechapter{\@arabic\c@chapter}
		\renewcommand\thesection{\thechapter.\@arabic\c@section}
		\newcommand*\chaptermark[1]{}
		\setcounter{secnumdepth}{2}
		\newcommand\@chapapp{\chaptername}
		%\newcommand\chaptername{Chapter}
  		\def\ps@headings{%
    		\let\@oddfoot\@empty
    		\def\@oddhead{{\slshape\rightmark}\hfil\thepage}%
    		\let\@mkboth\markboth
    		\def\chaptermark##1{%
      			\markright{\MakeUppercase{%
        			\ifnum \c@secnumdepth >\m@ne
            			\@chapapp\ \thechapter. \ %
        			\fi
        		##1}}%
        	}%
        }
		\newcommand\chapter{\clearpage
			\thispagestyle{plain}%
			\global\@topnum\z@
			\@afterindentfalse
			\secdef\@chapter\@schapter%
		}
		\def\@chapter[#1]#2{\refstepcounter{chapter}%
			\typeout{\@chapapp\space\thechapter.}%
			\addcontentsline{toc}{chapter}%
				{\protect\numberline{\thechapter}#1}%
			\chaptermark{#1}%
			\addtocontents{lof}{\protect\addvspace{10\p@}}%
			\addtocontents{lot}{\protect\addvspace{10\p@}}%
			\@makechapterhead{#2}%
			\@afterheading%
		}
		\def\@makechapterhead#1{%
			\vspace*{50\p@}%
			{\parindent \z@ \raggedright \normalfont
				\huge\bfseries \@chapapp\space \thechapter
				\par\nobreak
				\vskip 20\p@
				\interlinepenalty\@M
				\Huge \bfseries #1\par\nobreak
				\vskip 40\p@
			}%
		}
\newcommand*\l@chapter[2]{%
  \ifnum \c@tocdepth >\m@ne
    \addpenalty{-\@highpenalty}%
    \vskip 1.0em \@plus\p@
    \setlength\@tempdima{1.5em}%
    \begingroup
      \parindent \z@ \rightskip \@pnumwidth
      \parfillskip -\@pnumwidth
      \leavevmode \bfseries
      \advance\leftskip\@tempdima
      \hskip -\leftskip
      #1\nobreak\hfil
      \nobreak\hb@xt@\@pnumwidth{\hss #2%
                                 \kern-\p@\kern\p@}\par
      \penalty\@highpenalty
    \endgroup
  \fi}
\renewcommand*\l@section{\@dottedtocline{1}{1.5em}{2.8em}}
\renewcommand*\l@subsection{\@dottedtocline{2}{3.8em}{3.2em}}
\renewcommand*\l@subsubsection{\@dottedtocline{3}{7.0em}{4.1em}}
\def\partname{Part}
\def\toclevel@part{-1}
\def\maketitle{\chapter{\@title}}
\let\thanks\@gobble
\let\DelayPrintIndex\PrintIndex
\let\PrintIndex\@empty
\providecommand*{\hexnum}[1]{\text{\texttt{\char`\"}#1}}
\makeatother

\ExplSyntaxOn
\int_set:Nn \l_document_structure_section_level_int {1}
\ExplSyntaxOff

\clearpage

{%
  \def\\{:}% fix "newlines" in the ToC
  \tableofcontents
}

\clearpage
\pagenumbering{arabic}
	
\fi

\long\def\ignore#1{}


\begin{dangerbox}
  Boxes like this one contain implementation details that are
  mostly relevant for more advanced use cases, might be useful 
  to know when debugging, or might be good to know to better understand
  how something works. They can easiyl be skipped on a first read.
\end{dangerbox}

\begin{mmtbox}
  Boxes like this one explain how some \sTeX concept
  relates to the \mmt/\omdoc system, philosophy or language.
\end{mmtbox}


\begin{sfragment}{What is \sTeX?}
  
Formal systems for mathematics (such as interactive theorem provers)
have the potential to significantly increase both the accessibility
of published knowledge, as well as the confidence in its veracity,
by rendering the precise semantics of statements machine actionable.
This allows for a plurality of added-value services, from semantic
search up to verification and automated theorem proving.
Unfortunately, their usefulness is hidden behind severe barriers
to accessibility; primarily related to their surface languages
reminiscent of programming languages and very unlike informal
standards of presentation.

\sTeX minimizes this gap between informal and formal 
mathematics by integrating formal methods into established
and widespread authoring workflows, primarily \LaTeX, via 
non-intrusive semantic
annotations of arbitrary informal document fragments. That way
formal knowledge management services become available for informal
documents, accessible via an IDE for authors and via generated
\emph{active} documents for readers, while remaining fully compatible
with existing authoring workflows and publishing systems.

Additionally, an extensible library of reusable
document fragments is being developed, that serve as reference targets
for global disambiguation, intermediaries for content exchange
between systems and other services.

Every component of the system is designed modularly and extensibly,
and thus lay the groundwork for a potential full integration of
interactive theorem proving systems into established informal document
authoring workflows.

\paragraph{} The general \sTeX workflow combines functionalities
provided by several pieces of software:
\begin{itemize}
  \item The \sTeX package to use semantic annotations in
    {\LaTeX} documents,
  \item \RusTeX to convert |tex| sources to (semantically enriched)
    |xhtml|,
  \item The \mmt software, that extracts semantic information
    from the thus generated |xhtml| and provides semantically informed
    added value services.
\end{itemize}


% ----------------------------

\ignore{The objectives of this project will be achieved by developing a 
language and system 
that uses non-intrusive annotations
to augment informal documents with semantic information
(ranging from \textbf{fully formal} to \textbf{purely informal})
 without
impacting linguistic presentation or document layout. 
That way, the system
remains compatible with established publishing
pipelines and practices, while additionally providing flexiformal 
information that
enables formal knowledge management services, and hence produces 
\emph{rich active documents}, satisfying \textbf{R3}, \textbf{R4} and 
\textbf{R5}.
In particular, it will avoid commitment to a fixed logical foundation.
Instead, it will be designed as a modular pipeline of consecutive
and compositional
annotations, semantics extraction and translation steps, extensible
via new structuring mechanisms (\textbf{R1}), library content 
(\textbf{R2}),
NLP techniques, foundations, translation methods and 
end-user services.

Naturally, the benefits of formal knowledge management services scale 
with the amount of mathematics involved. Consequently I will primarily 
focus on those 
STEM fields in which mathematical methods are most prominently
used (e.g. mathematics, physics, computer science). Since in those fields
\LaTeX~is the most commonly used scientific writing tool, I will also
primarily focus on \LaTeX~as a development and evaluation target, but 
the system will be designed such that all components apart from
the surface language will be integrable with other writing tools 
(e.g. WYSIWYG word processors).

\paragraph{} The basic architecture of the proposed system is sketched in
\autoref{fig:architecture}.
\begin{figure}\centering
  \resizebox{0.95\textwidth}{!}{\tikzinput[]{diagram}}
  {\small (Note, that the syntax used
    in the box on the top right is prototypical and subject to change during the project.
    Details and open questions regarding the syntax are discussed here:
    \url{https://github.com/KWARC/FoMID/issues/1})}
  \caption{Basic Architecture of the Proposed System}\label{fig:architecture}
\end{figure}
A user can write their content using standard \LaTeX\ in an IDE;
ideally using semantic annotations provided by \sTeX
%and the library developed in \OBJref{smglom}
(as in the upper right of 
\autoref{fig:architecture}), but not necessarily so.

The document is converted to xhtml with \omdoc annotations
using \LaTeX ML in the background,
thus becoming actionable by the \mmt system. Both the source document
as well as the generated xhtml/\omdoc are accessible to a natural language
processing pripeline that can supply additional inferred semantic 
information or suggest annotations to the user, in the latter case 
augmenting the source document directly. This pipeline can use both 
classical NLP techniques using the GLIF system, as well as machine 
learning models such as \cite{own:fifom}.

A semiformal fragment is converted 
into an appropriate syntax tree (possibly containing opaque
informal nodes), 
thus becoming amenable
to flexiformal knowledge management services. In a consecutive step
-- if sufficiently annotated --, these are
additionally translated
to a fully formal foundation, e.g. using the techniques from 
\cite{DMueller:phd:19,own:translations}, allowing
more powerful services and conversion to established formal
systems. All three representations
are thus available from within the \mmt system for various
knowledge management services, interfaces for which can be
implemented in the IDE.

Importantly, every non-trivial arrow in the figure is 
composable and extensible -- 
translations to a foundation can be provided
by supplying an appropriate formalization and alignment-based
translations (or entirely new methods),
services can be implemented generically using the \mmt API,
NLP techniques can be implemented both inside and alongside of
GLIF, and the concrete syntax within \sTeX can be extended
by convenience macros in \LaTeX\ (enabling new
structuring mechanisms as in \textbf{R1} via 
\mmt extensions, see
\cite{MueRabRot:rslffml20}) as well as via additions to
the library, which will be extensible both from within the IDE
as well as on MathHub,
remaining backwards compatible with existing content in a surface 
language. Additionally, sufficiently disambiguated
statements can be translated to the syntax of 
external systems (such as interactive theorem prover systems
or computer algebra systems),
which can thus be integrated as additional services into the system.
}

\end{sfragment}

\begin{sfragment}{Quickstart}

	\begin{sfragment}{Setup}
		\begin{sfragment}{The \sTeX IDE}
      TODO: VSCode Plugin
    \end{sfragment}
    \begin{sfragment}{Manual Setup}
      Foregoing on the \sTeX IDE, we will need several
      pieces of software; namely:
      \begin{itemize}
        \item \textbf{The \sTeX-Package} available 
          \href{https://github.com/slatex/sTeX/blob/latex3/doc/stex.pdf}{here}.

          \sTeX is also available on CTAN and in \TeX Live.

        \item To make sure that \sTeX too knows where to find its
          archives, we need to set a global system variable |MATHHUB|,
          that points to your local |MathHub|-directory 
          (see \sref{sec.stexarchives}).
          %If you are only interested in using semantic macros in (ultimately)
          %|pdf|s generated by |pdflatex|, this is all you need.

        \item \textbf{The \mmt System} available
          \href{https://github.com/uniformal/MMT/tree/sTeX}{here}%
          \ednote{For now, we require the \texttt{sTeX}-branch, requiring manually
          compiling the MMT sources}. We recommend following
          the setup routine documented 
          \href{https://uniformal.github.io//doc/setup/}{here}.

          Following the setup routine (Step 3) will entail designating
          a |MathHub|-directory on your local file system, where
          the \mmt system will look for \sTeX/\mmt content archives.

        \item \textbf{\sTeX Archives} If we only care about {\LaTeX} and generating |pdf|s, we do not
          technically need \mmt at all; however, we still need the |MATHHUB|
          system variable to be set. Furthermore, \mmt can make downloading
          content archives we might want to use significantly easier, since
          it makes sure that all dependencies of (often highly interrelated)
          \sTeX archives are cloned as well.

          Once set up, we can run |mmt| in a shell and download an archive along with
          all of its dependencies like this: |lmh install <name-of-repository>|,
          or a whole \emph{group} of archives; for example,
          |lmh install smglom| will download all smglom archives.
        \item \textbf{\RusTeX} The \mmt system will also set up \RusTeX for you,
          which is used to generate (semantically annotated)
          |xhtml| from tex sources. In lieu of using \mmt, you
          can also download and use \RusTeX directly
          \href{https://github.com/slatex/RusTeX}{here}.

      \end{itemize}
    \end{sfragment}
	\end{sfragment}

    \begin{sfragment}{A First \sTeX Document}
    Having set everything up, we can write a first
    \sTeX document. As an example, we will use the
    |smglom/calculus| and |smglom/arithmetics| archives, 
    which should be present in the designated |MathHub|-folder,
    and write a small fragment defining the \emph{geometric series}:

    \textcolor{red}{TODO: use some sTeX-archive instead of smglom,
    use a convergence-notion that includes the limit,
    mark-up the theorem properly}

    \begin{framed}\begin{latexcode}[gobble=8]
        \documentclass{article}
        \usepackage{stex,xcolor,stexthm}

        \begin{document}
        \begin{smodule}{GeometricSeries}
            \importmodule[smglom/calculus]{series}
            \importmodule[smglom/arithmetics]{realarith}

            \symdef{geometricSeries}[name=geometric-series]{\comp{S}}

            \begin{sdefinition}[for=geometricSeries]
                The \definame{geometricSeries} is the \symname{?series}
                \[\defeq{\geometricSeries}{\definiens{
                    \infinitesum{\svar{n}}{1}{
                        \realdivide[frac]{1}{
                            \realpower{2}{\svar{n}}
                    }}
                }}.\]
            \end{sdefinition}

            \begin{sassertion}[name=geometricSeriesConverges,type=theorem]
            The \symname{geometricSeries} \symname{converges} towards $1$.
            \end{sassertion}
        \end{smodule}
        \end{document}
    \end{latexcode}\end{framed}

    Compiling this document with |pdflatex| should yield
    the output

    \begin{mdframed}
        \noindent\textbf{Definition 0.1. }\ The 
        \pdftooltip{\textcolor{blue}{\textbf{geometric series}}}{URI: file://your/file/name/here?GeometricSeries?geometric-series}
        is the 
        \pdftooltip{\textcolor{blue}{series}}{URI: http://mathhub.info/smglom/calculus?series?series}
        \[
        \pdftooltip{\textcolor{blue}S}{URI: file://your/file/name/here?GeometricSeries?geometric-series}
        \pdftooltip{\textcolor{blue}{:=}}{URI: http://mathhub.info/smglom/mv?defeq?definitional-equation}
        \mathop{\pdftooltip{\textcolor{blue}{\sum}}{URI: http://mathhub.info/smglom/calculus?series?infinitesum}
        }_{
            \pdftooltip{\textcolor{gray}{n}}{Variable var://n}=1
        }^{
          \pdftooltip{\textcolor{blue}\infty}{URI: http://mathhub.info/smglom/calculus?series?infinitesum}
        } \frac{1}{2^{\pdftooltip{\textcolor{gray}{n}}{Variable var://n}}}
        .\]
        \noindent\textbf{Theorem 0.2. }\ The 
        \pdftooltip{\textcolor{blue}{geometric series}}{URI: file://your/file/name/here?GeometricSeries?geometric-series}
        \pdftooltip{\textcolor{blue}{converges}}{URI: http://mathhub.info/smglom/calculus?sequenceConvergence?converges} towards $1$.
    \end{mdframed}

    Feel free to move your cursor over the various highlighted parts
    of the document -- depending on your pdf viewer, this should
    yield some interesting (but possibly for now cryptic) information.

    \begin{sparagraph}[type=remark]
      Note that all of the highlighting, tooltips, coloring and the environment headers
      come from \pkg{stexthm} -- by default, the amount of additional packages loaded
      is kept to a minimum and all the presentations can be customized.
    \end{sparagraph}

    Let's investigate this document in detail now:\bigskip

    \begin{latexcode}[numbers=none,aboveskip=0pt,belowskip=0pt,gobble=8]
        \begin{smodule}{GeometricSeries}
        ...
        \end{smodule}
    \end{latexcode}
    \begin{environment}{smodule}
      First, we open a new \emph{module} called |GeometricSeries|.
      This module is assigned a \emph{globally unique} identifier (URI),
      which (depending on your pdf viewer) should pop up in a tooltip
      if you hover over the word 
      \pdftooltip{\textcolor{blue}{\textbf{geometric series}}}{URI: file://your/file/name/here?GeometricSeries?geometric-series}.
    \end{environment}\bigskip

    \begin{latexcode}[numbers=none,aboveskip=0pt,belowskip=0pt,gobble=8]
        \importmodule[smglom/calculus]{series}
        \importmodule[smglom/arithmetics]{realarith}
    \end{latexcode}
    \begin{function}{\importmodule}
      Next, we \emph{import} two modules -- 
      |series| in the |smglom/calculus|-archive, and |realarith| in
      the |smglom/arithmetics|-archive. If we investigate these archives,
      we find the files |series.en.tex| and |realarith.en.tex| (respectively) 
      in their respective |source|-folders, which contain 
      the statements \stexcode"\begin{smodule}{series}" and \stexcode"\begin{smodule}{realarith}"
      (respectively).
      \iffalse\end{smodule}\end{smodule}\fi

      The \stexcode"\importmodule"-statements make all \stex symbols and associated
      semantic macros (e.g. \stexcode"\infinitesum", \stexcode"\realdivide", 
      \stexcode"\realpower")
      in the desired module available. Additionally, they ``export''
      these symbols to all further modules which include the \emph{current}
      module -- i.e. if in some future module we would put
      \stexcode"\importmodule{GeometricSeries}", we would also have \stexcode"\infinitesum"
      etc. at our disposal.
    \end{function}

    \begin{function}{\usemodule}
      If we only want to \emph{use} the content of some module |Foo|,
      e.g. in remarks or examples, but none
      of the symbols in our current module actually \emph{depend} on
      the content of |Foo|, we can use \stexcode"\usemodule" instead -- like
      \stexcode"\importmodule", this will make the module content available,
      but will \emph{not} export it to other modules.
    \end{function}\bigskip

    \begin{latexcode}[numbers=none,aboveskip=0pt,belowskip=0pt,gobble=6]
      \symdef{GeometricSeries}[name=geometric-series]{\comp{S}}
    \end{latexcode}
    \begin{function}{\symdef}
      Next, we introduce a new \emph{symbol} with name
      |geometric-series| and assign it the semantic macro
      \stexcode"\geometricSeries".
      \stexcode"\symdef" also immediately assigns this symbol a \emph{notation},
      namely $S$.
    \end{function}

    \begin{function}{\comp}
      The macro \stexcode"\comp" marks the $S$ in the notation as a
      \emph{notational component}, as opposed to e.g. arguments
      to \stexcode"\geometricSeries".
      It is the notational components that get highlighted
      and associated with the corresponding symbol (i.e. in this
      case |geometricSeries|). Since \stexcode"\geometricSeries" takes
      no arguments, we can wrap the whole notation in a \stexcode"\comp".
    \end{function}\bigskip

    \begin{latexcode}[numbers=none,aboveskip=0pt,belowskip=0pt,gobble=8]
        \begin{sdefinition}[for=geometricSeries]
        ...
        \end{sdefinition}
        \begin{sassertion}[name=geometricSeriesConverges,type=theorem]
        ...
        \end{sassertion}
    \end{latexcode}
    What follows are two \sTeX-\emph{statements} (e.g. definitions,
    theorems, examples, proofs, ...). These are semantically marked-up
    variants of the usual environments, which take additional optional
    arguments (e.g. |for=|, |type=|, |name=|). Since many \LaTeX\xspace templates
    predefine environments like |definition| or |theorem| with
    different syntax, we use \stexcode"sdefinition", 
    \stexcode"sassertion", \stexcode"sexample"
    etc. instead. You can customize these environments to e.g.
    simply wrap around some predefined |theorem|-environment.
    That way, we can still use \stexcode"sassertion" to provide semantic
    information, while being fully compatible with (and using
    the document presentation of) predefined environments.

    In our case, the \pkg{stexthm}-package patches
    e.g. \stexcode"\begin{sassertion}[type=theorem]" to use
    a |theorem|-environment defined (as usual) using \pkg{amsthm}.
    \bigskip \iffalse \end{sassertion}\fi

    \begin{latexcode}[numbers=none,aboveskip=0pt,belowskip=0pt,gobble=6]
      The \definame{geometricSeries} is the \symname{?series}
    \end{latexcode}
    \begin{function}{\symname}
      The \stexcode"\symname"-command prints the name of a symbol,
      highlights it (based on customizable settings)
      and associates the text printed with the corresponding
      symbol. If you hover over the word
      \pdftooltip{\textcolor{blue}{series}}{URI: http://mathhub.info/smglom/calculus?series?series}
      in the pdf output, you should see a tooltip showing the full URI
      of the symbol used.
    \end{function}
    \begin{function}{\symref}
      The \stexcode"\symname"-command is a special case of the more general
      \stexcode"\symref"-command, which allows customizing the precise
      text associated with a symbol.
    \end{function}
    \begin{function}{\definame,\definiendum}
      The \stexcode"sdefinition"-environment provides two additional
      macros, \stexcode"\definame" and \stexcode"\definiendum" which behave
      similar to \stexcode"\symname" and \stexcode"\symref", but explicitly mark
      the symbols as \emph{being defined} in this environment,
      to allow for special highlighting.
    \end{function}\bigskip

    \begin{latexcode}[numbers=none,aboveskip=0pt,belowskip=0pt,gobble=8]
        \[\defeq{\geometricSeries}{\definiens{
            \infinitesum{\svar{n}}{1}{
                \realdivide[frac]{1}{
                    \realpower{2}{\svar{n}}
            }}
        }}.\]
    \end{latexcode}
    The next snippet -- set in a math environment -- uses
    several semantic macros imported from (or recursively via) 
    |series| and |realarithmetics|, such as \stexcode"\defeq", 
    \stexcode"\infinitesum",
    etc. In math mode, using a semantic macro inserts its (default)
    definition. A semantic macro can have several notations -- in
    that case, we can explicitly choose a specific notation by
    providing its identifier as an optional argument; e.g.
    \stexcode"\realdivide[frac]{a}{b}" will use the explicit notation named |frac|
    of the semantic macro \stexcode"\realdivide", which yields $\frac ab$
    instead of $a/b$.
    \begin{function}{\svar}
      The \stexcode"\svar{n}" command marks up the |n| as a variable
      with name |n| and notation |n|.
    \end{function}
    \begin{function}{\definiens}
      The \stexcode"sdefinition"-environment additionally provides the
      \stexcode"\definiens"-command, which allows for explicitly
      marking up its argument as the \emph{definiens} of the
      symbol currently being defined.
    \end{function}

    \begin{sfragment}{\omdoc/xhtml Conversion}
      So, if we run |pdflatex| on our document, then \sTeX yields 
      pretty colors and
      tooltips\footnote{...and hyperlinks for symbols, and indices,
      and allows reusing document fragments modularly, and...}.
      But \sTeX becomes a lot more powerful if we additionally
      convert our document to |xhtml|.

      \textcolor{red}{TODO VSCode Plugin}

      Using \rustex, we can convert the document to |xhtml|
      using the command |rustex -i /path/to/file.tex -o /path/to/outfile.xhtml|.
      Investigating the resulting file, we notice additionaly semantic
      information resulting from our usage of semantic macros,
      \stexcode"\symref" etc. Below is the (abbreviated) snippet inside
      our \stexcode"\definiens" block:

      \begin{lstlisting}[escapechar=!]
<mrow resource="" property="stex:definiens">
 <mrow resource="...?series?infinitesum##" property="stex:OMBIND">
  <munderover displaystyle="true">
   <mo resource="...?series?infinitesum" property="stex:comp">!$\Sigma$!</mo>
   <mrow>
    <mi resource="1" property="stex:arg">n</mi>
    <mo class="rel">=</mo>
    <mi resource="2" property="stex:arg">1</mi>
   </mrow>
   <mi resource="...?series?infinitesum" property="stex:comp">!$\infty$!</mi>
  </munderover>
   <mrow resource="3" property="stex:arg">
    <mfrac resource="...?realarith?division#frac#" property="stex:OMA">
     <mi resource="1" property="stex:arg">1</mi>
     <mrow resource="2" property="stex:arg">
     <mo class="opening">(</mo>
     <msup resource="...realarith?exponentiation##" property="stex:OMA">
      <mi resource="1" property="stex:arg">2</mi>
      <mi resource="2" property="stex:arg">n</mi>
     </msup>
     <mo class="closing">)</mo>
    </mrow>
   </mfrac>
  </mrow>
 </mrow>
</mrow>
      \end{lstlisting}
      ...containing all the semantic information. The \mmt system
      can extract from this the following \openmath snippet:

      \begin{lstlisting}[escapechar=!]
<OMBIND>
  <OMID name="...?series?infinitesum"/>
  <OMV name="n"/>
  <OMLIT name="1"/>
  <OMA>
    <OMS name="...?realarith?division"/>
    <OMLIT name="1"/>
    <OMA>
      <OMS name="...realarith?exponentiation"/>
      <OMLIT name="2"/>
      <OMV name="n"/>
    </OMA>
  </OMA>
</OMBIND>
      \end{lstlisting}
      ...giving us the full semantics of the snippet, allowing for
      a plurality of knowledge management services -- in particular
      when serving the |xhtml|.

      \begin{remark}
          Note that the |html| when opened in a browser will
          look slightly different than the |pdf| when it comes
          to highlighting semantic content -- that is because
          naturally |html| allows for much more powerful
          features than |pdf| does. Consequently, the |html|
          is intended to be served by a system like \mmt,
          which can pick up on the semantic information and
          offer much more powerful highlighting, linking
          and similar features, and being customizable by
          \emph{readers} rather than being prescribed by an author.

          Additionally, not all browsers (most notably Chrome)
          support \mathml natively, and might require
          additional external JavaScript libraries such as
          MathJax to render mathematical formulas properly.
      \end{remark}
    \end{sfragment}
\end{sfragment}

\end{sfragment}

\begin{sfragment}{Creating \sTeX Content}

  % \iffalse meta-comment
% An Infrastructure for Semantic Macros and Module Scoping
% Copyright (c) 2019 Michael Kohlhase, all rights reserved
%                this file is released under the
%                LaTeX Project Public License (LPPL)
% 
% The original of this file is in the public repository at 
% http://github.com/sLaTeX/sTeX/
%
% TODO update copyright  
%
%<*driver>
\def\libfolder#1{../../lib/#1}
\RequirePackage{paralist}
\ifcsname stexdocpath\endcsname\else\def\stexdocpath{.}\fi
\documentclass[full]{l3doc}
%\RequirePackage{document-structure}
\usepackage[hyperref=auto,style=alphabetic]{biblatex}
%\usepackage[mathhub=\stexdocpath/mh,usedeps]{stex}
\usepackage[lang={en,de}]{stex}

\usepackage{rustex}
\usepackage{stex-highlighting,stexthm}

\srefsetin[sTeX/Documentation]{documentation}{the \stex Documentation}

\makeatletter
\providecommand{\HTML}{\textsc{html}\xspace}%
\providecommand{\XML}{\textsc{xml}\xspace}%
\providecommand{\PDF}{\textsc{pdf}\xspace}%
\providecommand\openmath{\textsc{OpenMath}\xspace}
\providecommand\OMDoc{\textsc{OMDoc}\xspace}
\DeclareRobustCommand\LaTeXML{L\kern-.36em%
        {\sbox\z@ T%
         \vbox to\ht\z@{\hbox{\check@mathfonts
                              \fontsize\sf@size\z@
                              \math@fontsfalse\selectfont
                              A}%
                        \vss}%
        }%
        \kern-.15em%
%        T\kern-.1667em\lower.5ex\hbox{E}\kern-.125em\relax
%        {\tt XML}}
        T\kern-.1667em\lower.4ex\hbox{E}\kern-0.05em\relax
        {\scshape xml}\xspace}%
\def\mmt{\textsc{Mmt}\xspace}
\makeatother


\newif\ifhadtitle\hadtitlefalse

\def\stexversion{3.3.0}
\def\changedate{\today}
\def\stextoptitle#1#2{\title{#1\thanks{Version {\stexversion} (last revised {\changedate})} }\def\thispkg{#2}}

\author{Michael Kohlhase, Dennis Müller\\
 	FAU Erlangen-Nürnberg\\
 	\url{http://kwarc.info/}
}

\def\stexmaketitle{\ifhadtitle\else\hadtitletrue\maketitle\fi}

\ExplSyntaxOn

  \def\docmodule{
    \begin{document}
      \EnableManual
      \EnableDocumentation
      \EnableImplementation
      \stexmaketitle
      \tableofcontents
      \int_gincr:N \l_stex_docheader_sect
      \exp_args:Ne \__stex_mathhub_find_manifest:n {\stex_file_use:N \c_stex_mathhub_file / sTeX / Documentation}
      \str_if_empty:NF \l__stex_mathhub_manifest_str {
        \usemodule[sTeX/Documentation]{macros?AllMacros}
      }
      \DocInput{\jobname.dtx}
      \clearpage
      \PrintIndex
      \printbibliography
    \end{document}
  }

  \bool_new:N \g_stexdoc_typeset_manual_bool
  \NewDocumentCommand \EnableManual {}{
    \bool_gset_true:N \g_stexdoc_typeset_manual_bool
  }
  \NewDocumentCommand \DisableManual {}{
    \bool_gset_false:N \g_stexdoc_typeset_manual_bool
  }
  \NewDocumentEnvironment {stexmanual} {} {
    \bool_if:NTF \g_stexdoc_typeset_manual_bool
      {\bool_set_false:N \l__codedoc_in_implementation_bool}
      {\comment}
  }{
    \bool_if:NF \g_stexdoc_typeset_manual_bool {\endcomment}
  }
\ExplSyntaxOff

%\usepackage{makeidx}
%\makeindex

%\usepackage{document-structure}


\usepackage{lststex,mdframed}
\usepackage[most]{tcolorbox}

\lstset{literate=%
    {Ö}{{\"O}}1
    {Ä}{{\"A}}1
    {Ü}{{\"U}}1
    {ß}{{\ss}}1
    {ü}{{\"u}}1
    {ä}{{\"a}}1
    {ö}{{\"o}}1
    {~}{{\textasciitilde}}1
}

\newenvironment{framed}[1][]{
  \ifstexhtml\par\vbox\bgroup
    \csname exp_args:Nne\endcsname\begin{stex_annotate_env}{%
      style:border=solid 1px black,%
      style:width=var(--this-width),%
      style:min-width=var(--this-width),%
      style:--this-width=calc(var(--current-width) - 6px),%
      style:padding=3px,%
      style:margin-top=5px,%
      style:margin-bottom=5px%
    }
    \csname stex_annotate_invisible:n\endcsname{ }%
    \begin{stex_annotate_env}{%
      style:--current-width=var(--this-width);%
    }\csname stex_annotate_invisible:n\endcsname{ }
  \else\begin{mdframed}[#1]\fi
  %\begin{center}%
}{%
  %\end{center}%
  \ifstexhtml
    \end{stex_annotate_env}\end{stex_annotate_env}\egroup\par
  \else\end{mdframed}\fi
}
\newcommand{\scaled}[2][0.9\hsize]{\begin{center}\resizebox{#1}{!}{\begin{minipage}{\textwidth} #2 \end{minipage}}\end{center}}

\makeatletter
\ExplSyntaxOn

\def\doc_exbox:nnn#1#2#3{
  \begin{sexample}[#3]
    Input:
    \begin{framed}[linewidth=1pt,backgroundcolor=white]\small
      #1
    \end{framed}
    Output:
    \begin{framed}[linewidth=1pt,backgroundcolor=white]\small
      #2
    \end{framed}
  \end{sexample}
}


\NewDocumentCommand\stexexamplefile{O{} m O{} O{}}{
  \stex_resolve_path_pair:Nxx \l_@@_filepath_str {\tl_to_str:n{#1}} {\tl_to_str:n{#2}}
  \doc_exbox:nnn{
    \hfill File~\str_if_empty:nTF{#1}{
      \prop_if_exist:NT \l_stex_current_archive_prop {
        [\texttt{\prop_item:Nn \l_stex_current_archive_prop {id}}]
      }
    }{[#1]}\texttt{\tl_to_str:n{#2}}
    \_lststex_parse_args:n{#3}
    \exp_args:Nno \use:nn{\lstinputlisting[} \l_lststex_return_tl ]{\l_@@_filepath_str}
  }{
    \inputref[#1]{#2}
  }{#4}
}

\newwrite\testoutfile
\NewDocumentCommand\stexexample{O{} O{}}{
  \begingroup 
  \catcode`\\=12\relax
  \catcode`\#=12\relax
  \catcode`\&=12\relax
  \catcode`\$=12\relax
  \catcode`\^=12\relax
  \catcode`\_=12\relax
  \catcode`\ =12\relax
  \catcode`^^J=12\relax
  \endlinechar=`^^J
  \newlinechar=-1
%^^A    \everyeof{\noexpand}
  \example_a:nnn{#1}{#2}
}
\long\def\example_a:nnn #1 #2 #3 {
  \endgroup
  \immediate\openout\testoutfile=\jobname.exmpl
  \immediate\write\testoutfile{
    \c_backslash_str begin{stexcode}[#1]
    \detokenize{^^J}#3
    \c_backslash_str end{stexcode}
  }
  \immediate\closeout\testoutfile
  \doc_exbox:nnn{
    \catcode`\#=12\relax
    \csname @ @ input\endcsname{\jobname.exmpl}
  }{
    \immediate\openout\testoutfile=\jobname.exmpl
    \immediate\write\testoutfile{#3}
    \immediate\closeout\testoutfile
    \csname @ @ input\endcsname \jobname.exmpl\relax
  }{#2}
  \peek_charcode_remove:NT ^^J
}

\ExplSyntaxOff
\makeatother

\makeatletter
\newcount\example@counter\example@counter=0
\newtcolorbox{exampleborderbox}[1][]{
  empty,
  title={Example \the\example@counter #1},
  attach boxed title to top left,
     minipage boxed title,
  boxed title style={empty,size=minimal,toprule=0pt,top=1pt,left=3mm,overlay={}},
  coltitle=blue,fonttitle=\bfseries,
  parbox=false,boxsep=0pt,left=3mm,right=0mm,top=2pt,breakable,pad at break=0mm,
     before upper=\csname @totalleftmargin\endcsname0pt, 
  overlay unbroken={\draw[blue,line width=2pt] ([xshift=-0pt]title.north west) -- ([xshift=-0pt]frame.south west); },
  overlay first={\draw[blue,line width=2pt] ([xshift=-0pt]title.north west) -- ([xshift=-0pt]frame.south west); },
  overlay middle={\draw[blue,line width=2pt] ([xshift=-0pt]frame.north west) -- ([xshift=-0pt]frame.south west); },
  overlay last={\draw[blue,line width=2pt] ([xshift=-0pt]frame.north west) -- ([xshift=-0pt]frame.south west); },
  outer arc=4pt%
}

\ExplSyntaxOn
\stexstyleexample{
  \global\advance\example@counter by 1
  \tl_if_empty:NTF\thistitle{
    \begin{exampleborderbox}
  }{
    \begin{exampleborderbox}[ (\thistitle)]
  }
}{
    \end{exampleborderbox}
}

\ExplSyntaxOff\makeatother

\usetikzlibrary{calc}

\def\textwarning{\includegraphics[width=1.2em]{stex-dangerous-bend}\xspace}
\newtcolorbox{dangerbox}{
  breakable,
  enhanced,
  left=0pt,
  right=0pt,
  top=8pt,
  bottom=8pt,
  colback=white,
  colframe=red,
  width=\textwidth,
  enlarge left by=0mm,
  boxsep=5pt,
  fontupper=\small,
  arc=4pt,
  outer arc=4pt,
  leftupper=1.5cm,
  overlay={
    \node[anchor=west] at ([xshift=10pt]$(frame.north west)!0.5!(frame.south west)$)
       {\includegraphics[width=1cm,height=1cm]{stex-dangerous-bend}};}
}

\protected\def\TODO#1{\textcolor{red}{TODO}\footnote{\textcolor{red}{TODO: #1}}}

\definecolor{darkgreen}{rgb}{0.0, 0.5, 0.0}

\usepackage[solutions]{problem}
\usepackage{hwexam}
\newtcolorbox{problemborderbox}[1][]{
  empty,
  title={Exercise #1},
  attach boxed title to top left,
     minipage boxed title,
  boxed title style={empty,size=minimal,toprule=0pt,top=1pt,left=3mm,overlay={}},
  coltitle=darkgreen,fonttitle=\bfseries,
  parbox=false,boxsep=0pt,left=3mm,right=0mm,top=2pt,breakable,pad at break=0mm,
     before upper=\csname @totalleftmargin\endcsname0pt, 
  overlay unbroken={\draw[darkgreen,line width=2pt] ([xshift=-0pt]title.north west) -- ([xshift=-0pt]frame.south west); },
  overlay first={\draw[darkgreen,line width=2pt] ([xshift=-0pt]title.north west) -- ([xshift=-0pt]frame.south west); },
  overlay middle={\draw[darkgreen,line width=2pt] ([xshift=-0pt]frame.north west) -- ([xshift=-0pt]frame.south west); },
  overlay last={\draw[darkgreen,line width=2pt] ([xshift=-0pt]frame.north west) -- ([xshift=-0pt]frame.south west); },
  outer arc=4pt%
}

\ExplSyntaxOn
\stexstyleproblem{
  \tl_if_empty:NTF\thistitle{
    \begin{problemborderbox}
  }{
    \begin{problemborderbox}[ (\thistitle)]
  }
}{
    \end{problemborderbox}
}
\ExplSyntaxOff

\newtcolorbox{experimental}{
  breakable,
  enhanced,
  left=0pt,
  right=0pt,
  top=8pt,
  bottom=8pt,
  colback=white,
  colframe=gray,
  width=\textwidth,
  enlarge left by=0mm,
  boxsep=5pt,
  fontupper=\small,
  arc=4pt,
  outer arc=4pt,
  leftupper=1.5cm,
  overlay={
    \node[anchor=west] at ([xshift=10pt]$(frame.north west)!0.5!(frame.south west)$)
       {\includegraphics[height=1cm]{stex-experimental}};}
}


\usetikzlibrary{decorations.pathmorphing,shapes,arrows,calc}
% Taken from pgflibrarytikzmmt.code.tex
\newcommand{\mmtarrowtip}{angle 45}
\newcommand{\mmtarrowtipmonoright}{right hook}

\tikzstyle{include}=[\mmtarrowtipmonoright-\mmtarrowtip,thick]
\tikzstyle{morph}=[-\mmtarrowtip,thick]
\tikzstyle{preview}=[decorate, decoration={coil,aspect=0,amplitude=1pt,
                                                  segment length=6pt,
                                                  pre=lineto,pre length=3pt,
                                                  post=lineto,post length=5pt}, thick]
\tikzstyle{view}=[preview,-\mmtarrowtip]


% TIKZ RULES
\def\mmtlogo{
\begin{tikzpicture}

  % White Background (Margins are eyeballed)
  % This is necessary because we paste white over arrows later.
  % If somebody want's to do the full song and dance with
  % interrupted arrows to get transparent background, be my guest.

  \fill[white!] (-0.01,0.15) rectangle (1.11,-0.95);

  % Arrows
  \draw [blue, include] (0,0)     -- (1.1,0);
  \draw [green, morph] (0,-0.4)  -- (1.1,-0.4);
  \draw [red, view]   (-0,-0.8) -- (1.1,-0.8);

  % Cutout for letters
  \fill[white] (0.33,0.1) rectangle (0.66,-0.9);

  % Letters
  \node at (0.18,0)    (nodeM1) {\large M};
  \node at (0.18,-0.4) (nodeM2) {\large M};
  \node at (0.21,-0.8) (nodeT)  {\large T};

\end{tikzpicture}
}

\newtcolorbox{mmtbox}{
  breakable,
  enhanced,
  left=0pt,
  right=0pt,
  top=8pt,
  bottom=8pt,
  colback=white,
  colframe=green,
  width=\textwidth,
  enlarge left by=0mm,
  boxsep=5pt,
  fontupper=\small,
  arc=4pt,
  outer arc=4pt,
  leftupper=1.5cm,
  overlay={
    \node[anchor=west] at ([xshift=10pt]$(frame.north west)!0.5!(frame.south west)$)
       {\mmtlogo};}
}

\AtBeginDocument{\catcode`_=8}

\begin{document}
  \DocInput{\jobname.dtx}
\end{document}
%</driver>
% \fi
%
% \title{ \sTeX-Basics
% 	\thanks{Version {\fileversion} (last revised {\filedate})} 
% }
%
% \author{Michael Kohlhase, Dennis Müller\\
% 	FAU Erlangen-Nürnberg\\
% 	\url{http://kwarc.info/}
% }
%
% \maketitle
%
%\ifinfulldoc\else
% This is the documentation for the \pkg{stex-basics} package.
% For a more high-level introduction, 
%  see \href{\basedocurl/manual.pdf}{the \sTeX Manual} or the
% \href{\basedocurl/stex.pdf}{full \sTeX documentation}.
% \fi
%
% \begin{documentation}\label{pkg:basics:doc}
%
% This sub package provides general set up code, auxiliary methods
% and abstractions for |xhtml| annotations.
%
%\ifinfulldoc\else
% We can use \sTeX by simply including the package with |\usepackage{stex}|,
or -- primarily for individual fragments to be included in other
documents -- by using the \sTeX document class with |\documentclass{stex}|
which combines the \pkg{standalone} document class with the \pkg{stex}
package.

Both the \pkg{stex} package and document class offer the following
options:

\begin{description}
   \item[\texttt{lang}] (\meta{language}$\ast$) Languages
     to load with the \pkg{babel} package.
   \item[\texttt{mathhub}] (\meta{directory}) MathHub folder
     to search for repositories -- this is not necessary if the
     |MATHHUB| system variable is set.
   \item[\texttt{sms}] (\meta{boolean}) use \emph{persisted}
     mode (not yet implemented).
   \item[\texttt{image}] (\meta{boolean}) passed on to
     \pkg{tikzinput}.
   \item[\texttt{debug}] (\meta{log-prefix}$\ast$) Logs debugging
     information with the given prefixes to the terminal,
     or all if |all| is given. Largely irrelevant for the
     majority of users.
\end{description}
% \fi
%
%
% \section{Macros and Environments}\label{pkg:basics:doc:macros}
%
% \begin{function}{\sTeX , \stex}
%   Both print this \stex logo.
% \end{function}
%
% \begin{function}{\stex_debug:nn}
%   \begin{syntax}
%     \cs{stex_debug:nn} \Arg{log-prefix} \Arg{message} ^^A \meta{comma list}
%   \end{syntax}
% Logs \meta{message}, if the package option |debug| contains \meta{log-prefix}.
% \end{function}
%
% \subsection{HTML Annotations}
%
% \begin{function}{\if@latexml}
%   \LaTeX2e conditional for \latexml
% \end{function}
%
% \begin{function}[pTF]{\latexml_if:}
%   \LaTeX3 conditionals for \latexml.
% \end{function}
%
% \begin{function}[pTF]{\stex_if_do_html:}
%   Whether to currently produce any HTML annotations (can be false
%   in some advanced structuring environments, for example)
% \end{function}
%
% \begin{function}{\stex_suppress_html:n}
%   Temporarily disables HTML annotations in its argument code
% \end{function}
% 
%
% We have four macros for annotating generated HTML (via \latexml
% or \rustex) with attributes:
%
% \begin{function}{\stex_annotate:nnn, \stex_annotate_invisible:nnn,
%   \stex_annotate_invisible:n}
%   \begin{syntax} \cs{stex_annotate:nnn} \Arg{property} \Arg{resource} \Arg{content} \end{syntax}
% Annotates the HTML generated by \meta{content} with\\
% \begin{center}
%  |property="stex:|\meta{property}|", resource="|\meta{resource}|"|.
% \end{center}
%
% \cs{stex_annotate_invisible:n} adds the attributes\\
% \begin{center}
% |stex:visible="false", style="display:none"|.
% \end{center}
%
% \cs{stex_annotate_invisible:nnn} combines the functionality of both.
% \end{function}
%
% \begin{environment}{stex_annotate_env}
%   \begin{syntax} \cs{begin}|{stex_annotate_env}|\Arg{property}\Arg{resource}
%       \meta{content}
%     \cs{end}|{stex_annotate_env}|
%   \end{syntax}
% behaves like \cs{stex_annotate:nnn} \Arg{property} \Arg{resource}
%     \Arg{content}.
% \end{environment}
%
% \subsection{Babel Languages}
%
% \begin{variable}{\c_stex_languages_prop,\c_stex_language_abbrevs_prop}
%   Map language abbreviations to their full babel names and vice versa.
%   e.g. \cs{c_stex_languages_prop}|{en}| yields |english|, and
%   \cs{c_stex_language_abbrevs_prop}|{english}| yields |en|.
% \end{variable}
%
% \subsection{Auxiliary Methods}
%
% \begin{function}{\stex_deactivate_macro:Nn , \stex_reactivate_macro:N}
%   \begin{syntax}\cs{stex_deactivate_macro:Nn}\meta{cs}\Arg{environments}\end{syntax}
%   Makes the macro \meta{cs} throw an error, indicating that it
%   is only allowed in the context of \meta{environments}.
%
%   \cs{stex_reactivate_macro:N}\meta{cs} reactivates it again, i.e.
%   this happens ideally in the \meta{begin}-code of the associated
%   environments.
% \end{function}
%
% \begin{function}{\ignorespacesandpars}
%   ignores white space characters and |\par| control sequences.
%   Expands tokens in the process.
% \end{function}
%
% \end{documentation}
%
% \begin{implementation}
%
% \section{\sTeX-Basics Implementation}\label{pkg:basics:impl}
%
%   \subsection{The \sTeX Document Class}
%
% The \cls{stex} document class is pretty straight-forward: It largely extends the \cls{standalone} package
% and loads the \pkg{stex} package, passing all provided options on to the package.
%
%    \begin{macrocode}
%<*cls>

%%%%%%%%%%%%%   basics.dtx   %%%%%%%%%%%%%

\RequirePackage{expl3,l3keys2e}
\ProvidesExplClass{stex}{2022/08/08}{3.2.0}{sTeX document class}

\DeclareOption*{\PassOptionsToPackage{\CurrentOption}{stex}}
\ProcessOptions

\bool_set_true:N \c_stex_document_class_bool

\RequirePackage{stex}

\stex_html_backend:TF {
  \LoadClass{article}
}{
  \LoadClass[border=1px,varwidth,crop=false]{standalone}
  \setlength\textwidth{15cm}
}
\RequirePackage{standalone}


\clist_if_empty:NT \c_stex_languages_clist {
  \seq_get_right:NN \g_stex_currentfile_seq \l_tmpa_str
  \seq_set_split:NnV \l_tmpa_seq . \l_tmpa_str
  \seq_pop_right:NN \l_tmpa_seq \l_tmpa_str % .tex
  \exp_args:No \str_if_eq:nnF \l_tmpa_str {tex} {
    \exp_args:No \str_if_eq:nnF \l_tmpa_str {dtx} {
      \exp_args:NNo \seq_put_right:Nn \l_tmpa_seq \l_tmpa_str
    }
  }
  \seq_pop_left:NN \l_tmpa_seq \l_tmpa_str % <filename>
  \seq_if_empty:NF \l_tmpa_seq { %remaining element should be [<something>.]language
    \seq_pop_right:NN \l_tmpa_seq \l_tmpa_str
    \prop_if_in:NoT \c_stex_languages_prop \l_tmpa_str {
      \stex_debug:nn{language} {Language~\l_tmpa_str~
        inferred~from~file~name}
      \exp_args:NNo \stex_set_language:Nn \l_tmpa_str \l_tmpa_str
    }
  }
}
%</cls>
%    \end{macrocode}
%
% \subsection{Preliminaries}
%
%    \begin{macrocode}
%<*package>

%%%%%%%%%%%%%   basics.dtx   %%%%%%%%%%%%%

\RequirePackage{expl3,l3keys2e,ltxcmds}
\ProvidesExplPackage{stex}{2022/08/08}{3.2.0}{sTeX package}

\bool_if_exist:NF \c_stex_document_class_bool {
  \bool_set_false:N \c_stex_document_class_bool
  \RequirePackage{standalone}
}

\message{^^J*~This~is~sTeX~version~3.2.0~*^^J}

%\RequirePackage{morewrites}
%\RequirePackage{amsmath}

%    \end{macrocode}
%
% Package options:
%
%    \begin{macrocode}
\keys_define:nn { stex } {
  debug     .clist_set:N  = \c_stex_debug_clist ,
  lang      .clist_set:N  = \c_stex_languages_clist ,
  mathhub   .tl_set_x:N   = \mathhub ,
  usesms    .bool_set:N   = \c_stex_persist_mode_bool ,
  writesms  .bool_set:N   = \c_stex_persist_write_mode_bool ,
  image     .bool_set:N   = \c_tikzinput_image_bool,
  unknown   .code:n       = {}
}
\ProcessKeysOptions { stex }
%    \end{macrocode}
%
% \begin{macro}{\stex,\sTeX}
%   The \sTeX logo:
%
%    \begin{macrocode}
\RequirePackage{stex-logo} % externalized for backwards-compatibility reasons
%    \end{macrocode}
% \end{macro}
%
%
% \subsection{Messages and logging}
%
%    \begin{macrocode}
%<@@=stex_log>
%    \end{macrocode}
%
% Warnings and error messages
%
%    \begin{macrocode}
\msg_new:nnn{stex}{error/unknownlanguage}{
  Unknown~language:~#1
}
\msg_new:nnn{stex}{warning/nomathhub}{
  MATHHUB~system~variable~not~found~and~no~
  \detokenize{\mathhub}-value~set!
}
\msg_new:nnn{stex}{error/deactivated-macro}{
  The~\detokenize{#1}~command~is~only~allowed~in~#2!
}
%    \end{macrocode}
% 
% \begin{macro}{\stex_debug:nn}
%
%  A simple macro issuing package messages with subpath.
%
%    \begin{macrocode}
\cs_new_protected:Nn \stex_debug:nn {
  \clist_if_in:NnTF \c_stex_debug_clist { all } {
    \msg_set:nnn{stex}{debug / #1}{
      \\Debug~#1:~#2\\
    }
    \msg_none:nn{stex}{debug / #1}
  }{
    \clist_if_in:NnT \c_stex_debug_clist { #1 } {
      \msg_set:nnn{stex}{debug / #1}{
        \\Debug~#1:~#2\\
      }
      \msg_none:nn{stex}{debug / #1}
    }  
  }
}
%    \end{macrocode}
% \end{macro}
%
% Redirecting messages:
%
%    \begin{macrocode}
\clist_if_in:NnTF \c_stex_debug_clist {all} {
    \msg_redirect_module:nnn{ stex }{ none }{ term }
}{
  \clist_map_inline:Nn \c_stex_debug_clist {
    \msg_redirect_name:nnn{ stex }{ debug / #1 }{ term }
  }
}

\stex_debug:nn{log}{debug~mode~on}
%    \end{macrocode}
%
%
% \subsection{HTML Annotations}
%    \begin{macrocode}
%<@@=stex_annotate>
%    \end{macrocode}
%
%    \begin{macrocode}
%    \end{macrocode}
%
% \begin{variable}{\l_stex_html_arg_tl, \c_stex_html_emptyarg_tl}
%
% Used by annotation macros to ensure that the HTML output to annotate
% is not empty.
%
%    \begin{macrocode}
\tl_new:N \l_stex_html_arg_tl
%    \end{macrocode}
% \end{variable}
%
% \begin{macro}{\_stex_html_checkempty:n}
%    \begin{macrocode}
\cs_new_protected:Nn \_stex_html_checkempty:n {
  \tl_set:Nn \l_stex_html_arg_tl { #1 }
  \tl_if_empty:NT \l_stex_html_arg_tl {
    \tl_set_eq:NN \l_stex_html_arg_tl \c_stex_html_emptyarg_tl
  }
}
%    \end{macrocode}
% \end{macro}
%
% \begin{macro}[pTF]{\stex_if_do_html:}
%  Whether to (locally) produce HTML output
%    \begin{macrocode}
\bool_new:N \_stex_html_do_output_bool
\bool_set_true:N \_stex_html_do_output_bool

\prg_new_conditional:Nnn \stex_if_do_html: {p,T,F,TF} {
  \bool_if:nTF \_stex_html_do_output_bool
    \prg_return_true: \prg_return_false:
}
%    \end{macrocode}
% \end{macro}
%
% \begin{macro}{\stex_suppress_html:n}
%  Whether to (locally) produce HTML output
%    \begin{macrocode}
\cs_new_protected:Nn \stex_suppress_html:n {
  \exp_args:Nne \use:nn {
    \bool_set_false:N \_stex_html_do_output_bool
    #1
  }{
    \stex_if_do_html:T {
      \bool_set_true:N \_stex_html_do_output_bool
    }
  }
}
%    \end{macrocode}
% \end{macro}
%
%
% \begin{environment}{stex_annotate_env}
% \begin{macro}{\stex_annotate:nnn, \stex_annotate_invisible:n,
%    \stex_annotate_invisible:nnn}
%
% We define four macros for introducing attributes in the HTML
% output. The definitions depend on the ``backend'' used
% (\latexml, \rustex, \texttt{pdflatex}). 
%
% The \texttt{pdflatex}-macros largely do nothing; the
% \rustex-implementations are pretty clear in what they do,
%  the \latexml-implementations resort to perl bindings.
%
%    \begin{macrocode}
\ifcsname if@rustex\endcsname\else
  \expandafter\newif\csname if@rustex\endcsname
  \@rustexfalse
\fi
\ifcsname if@latexml\endcsname\else
  \expandafter\newif\csname if@latexml\endcsname
  \@latexmlfalse
\fi
\tl_if_exist:NF\stex@backend{
  \if@rustex
    \def\stex@backend{rustex}
  \else
    \if@latexml
      \def\stex@backend{latexml}
    \else
      \cs_if_exist:NTF\HCode{
        \def\stex@backend{tex4ht}
      }{
        \def\stex@backend{pdflatex}
      }
    \fi
  \fi
}
\input{stex-backend-\stex@backend.cfg}

\newif\ifstexhtml
\stex_html_backend:TF\stexhtmltrue\stexhtmlfalse

%    \end{macrocode}
% \end{macro}
% \end{environment}
%
% \subsection{Babel Languages}
%    \begin{macrocode}
%<@@=stex_language>
%    \end{macrocode}
%
% \begin{variable}{\c_stex_languages_prop,\c_stex_language_abbrevs_prop}
%
% We store language abbreviations in two (mutually inverse) 
% property lists:
%    \begin{macrocode}
\exp_args:NNx \prop_const_from_keyval:Nn \c_stex_languages_prop { \tl_to_str:n {
  en = english ,
  de = ngerman ,
  ar = arabic ,
  bg = bulgarian ,
  ru = russian ,
  fi = finnish ,
  ro = romanian ,
  tr = turkish ,
  fr = french
}}

\exp_args:NNx \prop_const_from_keyval:Nn \c_stex_language_abbrevs_prop { \tl_to_str:n {
  english   = en ,
  ngerman   = de ,
  arabic    = ar ,
  bulgarian = bg ,
  russian   = ru ,
  finnish   = fi ,
  romanian  = ro ,
  turkish   = tr ,
  french    = fr
}}
% todo: chinese simplified (zhs)
%       chinese traditional (zht)
%    \end{macrocode}
% \end{variable}
%
% we use the |lang|-package option to load the corresponding
% babel languages:
%
%    \begin{macrocode}
\cs_new_protected:Nn \stex_set_language:Nn {
  \str_set:Nx \l_tmpa_str {#2}
  \prop_get:NoNT \c_stex_languages_prop \l_tmpa_str #1 {
    \ifx\@onlypreamble\@notprerr 
      \ltx@ifpackageloaded{babel}{
        \exp_args:No \selectlanguage #1
      }{}
    \else
      \exp_args:No \str_if_eq:nnTF #1 {turkish} {
        \RequirePackage[#1,shorthands=:!]{babel}
      }{
        \RequirePackage[#1]{babel}
      }
    \fi
  }
}

\clist_if_empty:NF \c_stex_languages_clist {
  \bool_set_false:N \l_tmpa_bool
  \clist_clear:N \l_tmpa_clist
  \clist_map_inline:Nn \c_stex_languages_clist {
    \str_set:Nx \l_tmpa_str {#1}
    \str_if_eq:nnT {#1}{tr}{
      \bool_set_true:N \l_tmpa_bool
    }
    \prop_get:NoNTF \c_stex_languages_prop \l_tmpa_str \l_tmpa_str {
      \clist_put_right:No \l_tmpa_clist \l_tmpa_str
    } {
      \msg_error:nnx{stex}{error/unknownlanguage}{\l_tmpa_str}
    }
  }
  \stex_debug:nn{lang} {Languages:~\clist_use:Nn \l_tmpa_clist {,~} }
  \bool_if:NTF \l_tmpa_bool {
    \RequirePackage[\clist_use:Nn \l_tmpa_clist,,shorthands=:!]{babel}
  }{
    \RequirePackage[\clist_use:Nn \l_tmpa_clist,]{babel}
  }
}

\AtBeginDocument{
  \stex_html_backend:T {
    \seq_get_right:NN \g_stex_currentfile_seq \l_tmpa_str
    \seq_set_split:NnV \l_tmpa_seq . \l_tmpa_str
    \seq_pop_right:NN \l_tmpa_seq \l_tmpa_str % .tex
    \seq_pop_left:NN \l_tmpa_seq \l_tmpa_str % <filename>
    \seq_if_empty:NF \l_tmpa_seq { %remaining element should be language
      \seq_pop_right:NN \l_tmpa_seq \l_tmpa_str
      \stex_debug:nn{basics} {Language~\l_tmpa_str~
        inferred~from~file~name}
      \stex_annotate_invisible:nnn{language}{ \l_tmpa_str }{}
    }
  }
}

%    \end{macrocode}
%
% \subsection{Persistence}
%
%    \begin{macrocode}
%<@@=stex_persist>
\bool_if:NTF \c_stex_persist_mode_bool {
  \def \stex_persist:n #1 {}
  \def \stex_persist:x #1 {}
}{
  \bool_if:NTF \c_stex_persist_write_mode_bool {
  \iow_new:N \c_@@_iow
  \iow_open:Nn \c_@@_iow{\jobname.sms}
  \AtEndDocument{
    \iow_close:N \c_@@_iow
  }
  \cs_new_protected:Nn \stex_persist:n {
    \tl_set:Nn \l_tmpa_tl { #1 }
    \regex_replace_all:nnN { \cP\# } { \cO\# } \l_tmpa_tl
    \regex_replace_all:nnN { \  } { \~ } \l_tmpa_tl
    \exp_args:NNo \iow_now:Nn \c_@@_iow \l_tmpa_tl
  }
  \cs_generate_variant:Nn \stex_persist:n {x}
  }{
    \def \stex_persist:n #1 {}
    \def \stex_persist:x #1 {}
  }
}
%    \end{macrocode}
%
% \subsection{Auxiliary Methods}
%
% \begin{macro}{\stex_deactivate_macro:Nn}
%    \begin{macrocode}
\cs_new_protected:Nn \stex_deactivate_macro:Nn {
  \exp_after:wN\let\csname \detokenize{#1} - orig\endcsname#1
  \def#1{
    \msg_error:nnnn{stex}{error/deactivated-macro}{\detokenize{#1}}{#2}
  }
}
%    \end{macrocode}
% \end{macro}
%
% \begin{macro}{\stex_reactivate_macro:N}
%    \begin{macrocode}
\cs_new_protected:Nn \stex_reactivate_macro:N {
  \exp_after:wN\let\exp_after:wN#1\csname \detokenize{#1} - orig\endcsname
}
%    \end{macrocode}
% \end{macro}
%
% \begin{macro}{\ignorespacesandpars}
%    \begin{macrocode}
\protected\def\ignorespacesandpars{
  \begingroup\catcode13=10\relax
  \@ifnextchar\par{
    \endgroup\expandafter\ignorespacesandpars\@gobble
  }{
    \endgroup
  }
}

\cs_new_protected:Nn \stex_copy_control_sequence:NNN {
  \tl_set:Nx \_tmp_args_tl {\cs_argument_spec:N #2}
  \exp_args:NNo \tl_remove_all:Nn \_tmp_args_tl \c_hash_str
  \int_set:Nn \l_tmpa_int {\tl_count:N \_tmp_args_tl}

  \tl_clear:N \_tmp_args_tl
  \int_step_inline:nn \l_tmpa_int {
    \tl_put_right:Nx \_tmp_args_tl {{\exp_not:n{####}\exp_not:n{##1}}}
  }

  \tl_set:Nn #3 {\cs_generate_from_arg_count:NNnn #1 \cs_set:Npn}
  \tl_put_right:Nx #3 { {\int_use:N \l_tmpa_int}{
      \exp_after:wN\exp_after:wN\exp_after:wN \exp_not:n 
      \exp_after:wN\exp_after:wN\exp_after:wN {
        \exp_after:wN #2 \_tmp_args_tl
      }
  }}
}
\cs_generate_variant:Nn \stex_copy_control_sequence:NNN {cNN}
\cs_generate_variant:Nn \stex_copy_control_sequence:NNN {NcN}
\cs_generate_variant:Nn \stex_copy_control_sequence:NNN {ccN}

\cs_new_protected:Nn \stex_copy_control_sequence_ii:NNN {
  \tl_set:Nx \_tmp_args_tl {\cs_argument_spec:N #2}
  \exp_args:NNo \tl_remove_all:Nn \_tmp_args_tl \c_hash_str
  \int_set:Nn \l_tmpa_int {\tl_count:N \_tmp_args_tl}

  \tl_clear:N \_tmp_args_tl
  \int_step_inline:nn \l_tmpa_int {
    \tl_put_right:Nx \_tmp_args_tl {{\exp_not:n{########}\exp_not:n{##1}}}
  }

  \edef \_tmp_args_tl {
    \exp_after:wN\exp_after:wN\exp_after:wN \exp_not:n 
    \exp_after:wN\exp_after:wN\exp_after:wN {
      \exp_after:wN #2 \_tmp_args_tl
    }
  }
  
  \exp_after:wN \def \exp_after:wN \_tmp_args_tl
  \exp_after:wN ##\exp_after:wN 1 \exp_after:wN ##\exp_after:wN 2
  \exp_after:wN  { \_tmp_args_tl }

  \edef \_tmp_args_tl {
    \exp_after:wN \exp_not:n \exp_after:wN {
      \_tmp_args_tl {####1}{####2}
    }
  }

  \tl_set:Nn #3 {\cs_generate_from_arg_count:NNnn #1 \cs_set:Npn}
  \tl_put_right:Nx #3 { {\int_use:N \l_tmpa_int}{
    \exp_after:wN\exp_not:n\exp_after:wN{\_tmp_args_tl}
  }}
}

\cs_generate_variant:Nn \stex_copy_control_sequence_ii:NNN {cNN}
\cs_generate_variant:Nn \stex_copy_control_sequence_ii:NNN {NcN}
\cs_generate_variant:Nn \stex_copy_control_sequence_ii:NNN {ccN}
%    \end{macrocode}
% \end{macro}
%
% \begin{macro}{\MMTrule}
%    \begin{macrocode}
\NewDocumentCommand \MMTrule {m m}{
  \seq_set_split:Nnn \l_tmpa_seq , {#2}
  \int_zero:N \l_tmpa_int
  \stex_annotate_invisible:nnn{mmtrule}{scala://#1}{
    \seq_if_empty:NF \l_tmpa_seq {
      $\seq_map_inline:Nn \l_tmpa_seq {
        \int_incr:N \l_tmpa_int
        \stex_annotate:nnn{arg}{i\int_use:N \l_tmpa_int}{##1}
      }$
    }
  }
}

\NewDocumentCommand \MMTinclude {m}{
  \stex_annotate_invisible:nnn{import}{#1}{}
}

\tl_new:N \g_stex_document_title
\cs_new_protected:Npn \STEXtitle #1 {
  \tl_if_empty:NT \g_stex_document_title {
    \tl_gset:Nn \g_stex_document_title { #1 }
  }
}
\cs_new_protected:Nn \stex_document_title:n {
  \tl_if_empty:NT \g_stex_document_title {
    \tl_gset:Nn \g_stex_document_title { #1 }
    \stex_annotate_invisible:n{\noindent
      \stex_annotate:nnn{doctitle}{}{ #1 }
    \par}
  }
}
\AtBeginDocument {
  \let \STEXtitle \stex_document_title:n
  \tl_if_empty:NF \g_stex_document_title {
    \stex_annotate_invisible:n{\noindent
      \stex_annotate:nnn{doctitle}{}{ \g_stex_document_title }
    \par}
  }
  \let\_stex_maketitle:\maketitle
  \def\maketitle{
    \tl_if_empty:NF \@title {
      \exp_args:No \stex_document_title:n \@title
    }
    \_stex_maketitle:
  }
}

\cs_new_protected:Nn \stex_par: {
  \mode_if_vertical:F{
    \if@minipage\else\if@nobreak\else\par\fi\fi
  }
}

%</package>
%    \end{macrocode}
% \end{macro}
%
% \end{implementation}
% \ifinfulldoc\else\printbibliography\fi
%
% \PrintIndex


  \begin{sfragment}{How Knowledge is Organized in \sTeX}

    \sTeX content is organized on multiple levels:
    \begin{itemize}
      \item \sTeX \textbf{archives} (see \sref{sec.stexarchives})
        contain individual |.tex|-files.
      \item These may contain \sTeX \textbf{modules}, introduced via 
      \stexcode"\begin{smodule}{ModuleName}".\iffalse\end{smodule}\fi
      \item Modules contain \sTeX \textbf{symbol declarations}, introduced via
        \stexcode"\symdecl{symbolname}", \stexcode"\symdef{symbolname}" and some other
        constructions. Most symbols have a \emph{notation} that can
        be used via a \emph{semantic macro} \stexcode"\symbolname" generated
        by symbol declarations.
      \item \sTeX \textbf{expressions} finally are built up from
        usages of semantic macros.
    \end{itemize}

    \begin{mmtbox}
      \begin{itemize}
        \item \sTeX archives are simultaneously \mmt archives, and the same
          directory structure is consequently used.
        \item \sTeX modules correspond to \omdoc/\mmt \emph{theories}.
          \stexcode"\importmodule"s (and similar constructions) induce 
          \mmt |include|s and other \emph{theory morphisms},
          thus giving rise to a \emph{theory graph} in the \omdoc sense.
        \item Symbol declarations induce \omdoc/\mmt \emph{constants},
          with optional (formal) \emph{type} and \emph{definiens} components.
        \item Finally, \sTeX expressions are converted to \omdoc/\mmt terms,
          which use the syntax of \openmath.
      \end{itemize}
    \end{mmtbox}

	\end{sfragment}

  \begin{sfragment}[id=sec.stexarchives]{\sTeX Archives}
    % \iffalse meta-comment
% An Infrastructure for Semantic Macros and Module Scoping
% Copyright (c) 2019 Michael Kohlhase, all rights reserved
%                this file is released under the
%                LaTeX Project Public License (LPPL)
% 
% The original of this file is in the public repository at 
% http://github.com/sLaTeX/sTeX/
%
% TODO update copyright  
%
%<*driver>
\def\bibfolder#1{../../lib/bib/#1}
\RequirePackage{paralist}
\ifcsname stexdocpath\endcsname\else\def\stexdocpath{.}\fi
\documentclass[full]{l3doc}
%\RequirePackage{document-structure}
\usepackage[hyperref=auto,style=alphabetic]{biblatex}
%\usepackage[mathhub=\stexdocpath/mh,usedeps]{stex}
\usepackage[lang={en,de}]{stex}

\usepackage{rustex}
\usepackage{stex-highlighting,stexthm}

\srefsetin[sTeX/Documentation]{documentation}{the \stex Documentation}

\makeatletter
\providecommand{\HTML}{\textsc{html}\xspace}%
\providecommand{\XML}{\textsc{xml}\xspace}%
\providecommand{\PDF}{\textsc{pdf}\xspace}%
\providecommand\openmath{\textsc{OpenMath}\xspace}
\providecommand\OMDoc{\textsc{OMDoc}\xspace}
\DeclareRobustCommand\LaTeXML{L\kern-.36em%
        {\sbox\z@ T%
         \vbox to\ht\z@{\hbox{\check@mathfonts
                              \fontsize\sf@size\z@
                              \math@fontsfalse\selectfont
                              A}%
                        \vss}%
        }%
        \kern-.15em%
%        T\kern-.1667em\lower.5ex\hbox{E}\kern-.125em\relax
%        {\tt XML}}
        T\kern-.1667em\lower.4ex\hbox{E}\kern-0.05em\relax
        {\scshape xml}\xspace}%
\def\mmt{\textsc{Mmt}\xspace}
\makeatother


\newif\ifhadtitle\hadtitlefalse

\def\stexversion{3.3.0}
\def\changedate{\today}
\def\stextoptitle#1#2{\title{#1\thanks{Version {\stexversion} (last revised {\changedate})} }\def\thispkg{#2}}

\author{Michael Kohlhase, Dennis Müller\\
 	FAU Erlangen-Nürnberg\\
 	\url{http://kwarc.info/}
}

\def\stexmaketitle{\ifhadtitle\else\hadtitletrue\maketitle\fi}

\ExplSyntaxOn

  \def\docmodule{
    \begin{document}
      \EnableManual
      \EnableDocumentation
      \EnableImplementation
      \stexmaketitle
      \tableofcontents
      \int_gincr:N \l_stex_docheader_sect
      \exp_args:Ne \__stex_mathhub_find_manifest:n {\stex_file_use:N \c_stex_mathhub_file / sTeX / Documentation}
      \str_if_empty:NF \l__stex_mathhub_manifest_str {
        \usemodule[sTeX/Documentation]{macros?AllMacros}
      }
      \DocInput{\jobname.dtx}
      \clearpage
      \PrintIndex
      \printbibliography
    \end{document}
  }

  \bool_new:N \g_stexdoc_typeset_manual_bool
  \NewDocumentCommand \EnableManual {}{
    \bool_gset_true:N \g_stexdoc_typeset_manual_bool
  }
  \NewDocumentCommand \DisableManual {}{
    \bool_gset_false:N \g_stexdoc_typeset_manual_bool
  }
  \NewDocumentEnvironment {stexmanual} {} {
    \bool_if:NTF \g_stexdoc_typeset_manual_bool
      {\bool_set_false:N \l__codedoc_in_implementation_bool}
      {\comment}
  }{
    \bool_if:NF \g_stexdoc_typeset_manual_bool {\endcomment}
  }
\ExplSyntaxOff

%\usepackage{makeidx}
%\makeindex

%\usepackage{document-structure}


\usepackage{lststex,mdframed}
\usepackage[most]{tcolorbox}

\lstset{literate=%
    {Ö}{{\"O}}1
    {Ä}{{\"A}}1
    {Ü}{{\"U}}1
    {ß}{{\ss}}1
    {ü}{{\"u}}1
    {ä}{{\"a}}1
    {ö}{{\"o}}1
    {~}{{\textasciitilde}}1
}

\newenvironment{framed}[1][]{
  \ifstexhtml\par\vbox\bgroup
    \csname exp_args:Nne\endcsname\begin{stex_annotate_env}{%
      style:border=solid 1px black,%
      style:width=var(--this-width),%
      style:min-width=var(--this-width),%
      style:--this-width=calc(var(--current-width) - 6px),%
      style:padding=3px,%
      style:margin-top=5px,%
      style:margin-bottom=5px%
    }
    \csname stex_annotate_invisible:n\endcsname{ }%
    \begin{stex_annotate_env}{%
      style:--current-width=var(--this-width);%
    }\csname stex_annotate_invisible:n\endcsname{ }
  \else\begin{mdframed}[#1]\fi
  %\begin{center}%
}{%
  %\end{center}%
  \ifstexhtml
    \end{stex_annotate_env}\end{stex_annotate_env}\egroup\par
  \else\end{mdframed}\fi
}
\newcommand{\scaled}[2][0.9\hsize]{\begin{center}\resizebox{#1}{!}{\begin{minipage}{\textwidth} #2 \end{minipage}}\end{center}}

\makeatletter
\ExplSyntaxOn

\def\doc_exbox:nnn#1#2#3{
  \begin{sexample}[#3]
    Input:
    \begin{framed}[linewidth=1pt,backgroundcolor=white]\small
      #1
    \end{framed}
    Output:
    \begin{framed}[linewidth=1pt,backgroundcolor=white]\small
      #2
    \end{framed}
  \end{sexample}
}


\NewDocumentCommand\stexexamplefile{O{} m O{} O{}}{
  \stex_resolve_path_pair:Nxx \l_@@_filepath_str {\tl_to_str:n{#1}} {\tl_to_str:n{#2}}
  \doc_exbox:nnn{
    \hfill File~\str_if_empty:nTF{#1}{
      \prop_if_exist:NT \l_stex_current_archive_prop {
        [\texttt{\prop_item:Nn \l_stex_current_archive_prop {id}}]
      }
    }{[#1]}\texttt{\tl_to_str:n{#2}}
    \_lststex_parse_args:n{#3}
    \exp_args:Nno \use:nn{\lstinputlisting[} \l_lststex_return_tl ]{\l_@@_filepath_str}
  }{
    \inputref[#1]{#2}
  }{#4}
}

\newwrite\testoutfile
\NewDocumentCommand\stexexample{O{} O{}}{
  \begingroup 
  \catcode`\\=12\relax
  \catcode`\#=12\relax
  \catcode`\&=12\relax
  \catcode`\$=12\relax
  \catcode`\^=12\relax
  \catcode`\_=12\relax
  \catcode`\ =12\relax
  \catcode`^^J=12\relax
  \endlinechar=`^^J
  \newlinechar=-1
%^^A    \everyeof{\noexpand}
  \example_a:nnn{#1}{#2}
}
\long\def\example_a:nnn #1 #2 #3 {
  \endgroup
  \immediate\openout\testoutfile=\jobname.exmpl
  \immediate\write\testoutfile{
    \c_backslash_str begin{stexcode}[#1]
    \detokenize{^^J}#3
    \c_backslash_str end{stexcode}
  }
  \immediate\closeout\testoutfile
  \doc_exbox:nnn{
    \catcode`\#=12\relax
    \csname @ @ input\endcsname{\jobname.exmpl}
  }{
    \immediate\openout\testoutfile=\jobname.exmpl
    \immediate\write\testoutfile{#3}
    \immediate\closeout\testoutfile
    \csname @ @ input\endcsname \jobname.exmpl\relax
  }{#2}
  \peek_charcode_remove:NT ^^J
}

\ExplSyntaxOff
\makeatother

\makeatletter
\newcount\example@counter\example@counter=0
\newtcolorbox{exampleborderbox}[1][]{
  empty,
  title={Example \the\example@counter #1},
  attach boxed title to top left,
     minipage boxed title,
  boxed title style={empty,size=minimal,toprule=0pt,top=1pt,left=3mm,overlay={}},
  coltitle=blue,fonttitle=\bfseries,
  parbox=false,boxsep=0pt,left=3mm,right=0mm,top=2pt,breakable,pad at break=0mm,
     before upper=\csname @totalleftmargin\endcsname0pt, 
  overlay unbroken={\draw[blue,line width=2pt] ([xshift=-0pt]title.north west) -- ([xshift=-0pt]frame.south west); },
  overlay first={\draw[blue,line width=2pt] ([xshift=-0pt]title.north west) -- ([xshift=-0pt]frame.south west); },
  overlay middle={\draw[blue,line width=2pt] ([xshift=-0pt]frame.north west) -- ([xshift=-0pt]frame.south west); },
  overlay last={\draw[blue,line width=2pt] ([xshift=-0pt]frame.north west) -- ([xshift=-0pt]frame.south west); },
  outer arc=4pt%
}

\ExplSyntaxOn
\stexstyleexample{
  \global\advance\example@counter by 1
  \tl_if_empty:NTF\thistitle{
    \begin{exampleborderbox}
  }{
    \begin{exampleborderbox}[ (\thistitle)]
  }
}{
    \end{exampleborderbox}
}

\ExplSyntaxOff\makeatother

\usetikzlibrary{calc}

\def\textwarning{\includegraphics[width=1.2em]{stex-dangerous-bend}\xspace}
\newtcolorbox{dangerbox}{
  breakable,
  enhanced,
  left=0pt,
  right=0pt,
  top=8pt,
  bottom=8pt,
  colback=white,
  colframe=red,
  width=\textwidth,
  enlarge left by=0mm,
  boxsep=5pt,
  fontupper=\small,
  arc=4pt,
  outer arc=4pt,
  leftupper=1.5cm,
  overlay={
    \node[anchor=west] at ([xshift=10pt]$(frame.north west)!0.5!(frame.south west)$)
       {\includegraphics[width=1cm,height=1cm]{stex-dangerous-bend}};}
}

\protected\def\TODO#1{\textcolor{red}{TODO}\footnote{\textcolor{red}{TODO: #1}}}

\definecolor{darkgreen}{rgb}{0.0, 0.5, 0.0}

\usepackage[solutions]{problem}
\usepackage{hwexam}
\newtcolorbox{problemborderbox}[1][]{
  empty,
  title={Exercise #1},
  attach boxed title to top left,
     minipage boxed title,
  boxed title style={empty,size=minimal,toprule=0pt,top=1pt,left=3mm,overlay={}},
  coltitle=darkgreen,fonttitle=\bfseries,
  parbox=false,boxsep=0pt,left=3mm,right=0mm,top=2pt,breakable,pad at break=0mm,
     before upper=\csname @totalleftmargin\endcsname0pt, 
  overlay unbroken={\draw[darkgreen,line width=2pt] ([xshift=-0pt]title.north west) -- ([xshift=-0pt]frame.south west); },
  overlay first={\draw[darkgreen,line width=2pt] ([xshift=-0pt]title.north west) -- ([xshift=-0pt]frame.south west); },
  overlay middle={\draw[darkgreen,line width=2pt] ([xshift=-0pt]frame.north west) -- ([xshift=-0pt]frame.south west); },
  overlay last={\draw[darkgreen,line width=2pt] ([xshift=-0pt]frame.north west) -- ([xshift=-0pt]frame.south west); },
  outer arc=4pt%
}

\ExplSyntaxOn
\stexstyleproblem{
  \tl_if_empty:NTF\thistitle{
    \begin{problemborderbox}
  }{
    \begin{problemborderbox}[ (\thistitle)]
  }
}{
    \end{problemborderbox}
}
\ExplSyntaxOff

\newtcolorbox{experimental}{
  breakable,
  enhanced,
  left=0pt,
  right=0pt,
  top=8pt,
  bottom=8pt,
  colback=white,
  colframe=gray,
  width=\textwidth,
  enlarge left by=0mm,
  boxsep=5pt,
  fontupper=\small,
  arc=4pt,
  outer arc=4pt,
  leftupper=1.5cm,
  overlay={
    \node[anchor=west] at ([xshift=10pt]$(frame.north west)!0.5!(frame.south west)$)
       {\includegraphics[height=1cm]{stex-experimental}};}
}


\usetikzlibrary{decorations.pathmorphing,shapes,arrows,calc}
% Taken from pgflibrarytikzmmt.code.tex
\newcommand{\mmtarrowtip}{angle 45}
\newcommand{\mmtarrowtipmonoright}{right hook}

\tikzstyle{include}=[\mmtarrowtipmonoright-\mmtarrowtip,thick]
\tikzstyle{morph}=[-\mmtarrowtip,thick]
\tikzstyle{preview}=[decorate, decoration={coil,aspect=0,amplitude=1pt,
                                                  segment length=6pt,
                                                  pre=lineto,pre length=3pt,
                                                  post=lineto,post length=5pt}, thick]
\tikzstyle{view}=[preview,-\mmtarrowtip]


% TIKZ RULES
\def\mmtlogo{
\begin{tikzpicture}

  % White Background (Margins are eyeballed)
  % This is necessary because we paste white over arrows later.
  % If somebody want's to do the full song and dance with
  % interrupted arrows to get transparent background, be my guest.

  \fill[white!] (-0.01,0.15) rectangle (1.11,-0.95);

  % Arrows
  \draw [blue, include] (0,0)     -- (1.1,0);
  \draw [green, morph] (0,-0.4)  -- (1.1,-0.4);
  \draw [red, view]   (-0,-0.8) -- (1.1,-0.8);

  % Cutout for letters
  \fill[white] (0.33,0.1) rectangle (0.66,-0.9);

  % Letters
  \node at (0.18,0)    (nodeM1) {\large M};
  \node at (0.18,-0.4) (nodeM2) {\large M};
  \node at (0.21,-0.8) (nodeT)  {\large T};

\end{tikzpicture}
}

\newtcolorbox{mmtbox}{
  breakable,
  enhanced,
  left=0pt,
  right=0pt,
  top=8pt,
  bottom=8pt,
  colback=white,
  colframe=green,
  width=\textwidth,
  enlarge left by=0mm,
  boxsep=5pt,
  fontupper=\small,
  arc=4pt,
  outer arc=4pt,
  leftupper=1.5cm,
  overlay={
    \node[anchor=west] at ([xshift=10pt]$(frame.north west)!0.5!(frame.south west)$)
       {\mmtlogo};}
}

\AtBeginDocument{\catcode`_=8}

\begin{document}
  \DocInput{\jobname.dtx}
\end{document}
%</driver>
% \fi
%
% \title{ \sTeX-MathHub
% 	\thanks{Version {\fileversion} (last revised {\filedate})} 
% }
%
% \author{Michael Kohlhase, Dennis Müller\\
% 	FAU Erlangen-Nürnberg\\
% 	\url{http://kwarc.info/}
% }
%
% \maketitle
%
%\ifinfulldoc\else
% This is the documentation for the \pkg{stex-mathhub} package.
% For a more high-level introduction, 
%  see \href{\basedocurl/manual.pdf}{the \sTeX Manual} or the
% \href{\basedocurl/stex.pdf}{full \sTeX documentation}.
% \fi
%
%
% \begin{documentation}\label{pkg:mathhub:doc}
% \changes{3.1.0}{2022/03/09}{Fixed a bug with \textbackslash inputref outside of archives}
%
% This sub package provides code for handling \sTeX archives,
% files, file paths and related methods.
%
% \ifinfulldoc\else
% \begin{sfragment}{Manual}\begin{sfragment}{The Local MathHub-Directory}
    \stexcode"\usemodule", \stexcode"\importmodule", 
    \stexcode"\inputref" etc. allow for 
    including content modularly without having to specify absolute
    paths, which would differ between users and machines. Instead,
    \sTeX uses \emph{archives} that determine the global
    namespaces for symbols and statements and make it possible
    for \sTeX to find content referenced via such URIs.

    All \sTeX archives need to exist in the local |MathHub|-directory.
    \sTeX knows where this folder is via one of four means:

    \begin{enumerate}
    \item If the \sTeX package is loaded with the option |mathhub=/path/to/mathhub|, then
      \sTeX will consider |/path/to/mathhub| as the local |MathHub|-directory.
    \item If the |mathhub| package option is \emph{not} set, but the macro |\mathhub|
      exists when the \sTeX-package is loaded, then this macro is assumed to point to the
      local |MathHub|-directory; i.e.
      \stexcode"\def\mathhub{/path/to/mathhub}\usepackage{stex}" will set the
      |MathHub|-directory as |path/to/mathhub|.
    \item Otherwise, \sTeX will attempt to retrieve the system variable |MATHHUB|,
      assuming it will point to the local |MathHub|-directory. Since this variant needs
      setting up only \emph{once} and is machine-specific (rather than defined in tex
      code), it is compatible with collaborating and sharing tex content, and hence
      recommended.
    \item Finally, if all else fails, \sTeX will look for a file
      |~/.stex/mathhub.path|. If this file exists, \sTeX will assume that it contains the
      path to the local |MathHub|-directory. This method is recommended on systems where
      it is difficult to set environment variables.
    \end{enumerate}
\end{sfragment}

\begin{sfragment}{The Structure of \sTeX Archives}
    An \sTeX archive |group/name| is stored in the
    directory |/path/to/mathhub/group/name|; e.g. assuming your
    local |MathHub|-directory is set as |/user/foo/MathHub|, then
    in order for the |smglom/calculus|-archive to be found by the
    \sTeX system, it needs to be in |/user/foo/MathHub/smglom/calculus|.

    Each such archive needs two subdirectories:
    \begin{itemize}
        \item |/source| -- this is where all your tex files go.
        \item |/META-INF| -- a directory containing a single file
            |MANIFEST.MF|, the content of which we will consider shortly
    \end{itemize}
    An additional |lib|-directory is optional, and is where \sTeX will
    look for files included via \stexcode"\libinput".

    Additionally a \emph{group} of archives |group/name| may have
    an additional archive |group/meta-inf|. If this |meta-inf|-archive
    has a |/lib|-subdirectory, it too will be searched by \stexcode"\libinput"
    from all tex files in any archive in the |group/*|-group.

    \paragraph{} We recommend the following additional directory structure in the
    |source|-folder of an \sTeX archive:
    \begin{itemize}
        \item |/source/mod/| -- individual \sTeX modules, containing
            symbol declarations, notations, and 
            \stexcode"\begin{sparagraph}[type=symdoc,for=...]"
            environments for ``encyclopaedic'' symbol documentations
            \iffalse\end{sparagraph}\fi
        \item |/source/def/| -- definitions
        \item |/source/ex/| -- examples
        \item |/source/thm/| -- theorems, lemmata and proofs; preferably
            proofs in separate files to allow for multiple proofs for the
            same statement
        \item |/source/snip/| -- individual text snippets such as remarks,
            explanations etc.
        \item |/source/frag/| -- individual document fragments,
            ideally only \stexcode"\inputref"ing snippets, definitions,
            examples etc. in some desirable order
        \item |/source/tikz/| -- tikz images, as individual |.tex|-files
        \item |/source/pic/| -- image files.\ednote{MK: bisher habe ich immer PIC subdirs,
          soll ich das ändern?}
    \end{itemize}

\end{sfragment}

\begin{sfragment}{MANIFEST.MF-Files}
  The |MANIFEST.MF| in the |META-INF|-directory consists of key-value-pairs, informing
  \sTeX (and associated software) of various properties of an archive. For example, the
  |MANIFEST.MF| of the |smglom/calculus|-archive looks like this:

    \begin{framed}
        \begin{verbatim}
    id: smglom/calculus
    source-base: http://mathhub.info/smglom/calculus
    narration-base: http://mathhub.info/smglom/calculus
    dependencies: smglom/arithmetics,smglom/sets,smglom/topology,
                smglom/mv,smglom/linear-algebra,smglom/algebra
    responsible: Michael.Kohlhase@FAU.de
    title: Elementary Calculus
    teaser: Terminology for the mathematical study of change. 
    description: desc.html
        \end{verbatim}
    \end{framed}

    Many of these are in fact ignored by \sTeX, but some are important:
    \begin{itemize}
        \item[|id|:] The name of the archive, including its group (e.g. |smglom/calculus|),
        \item[|source-base|] or
        \item[|ns|:] The namespace from which all symbol and module URIs
            in this repository are formed, see (\textcolor{red}{TODO}),
        \item[|narration-base:|] The namespace from which all document
            URIs in this repository are formed, see (\textcolor{red}{TODO}),
        \item[|url-base|:] The URL that is formed as a basis for \emph{external references},
            see (\textcolor{red}{TODO}),
        \item[|dependencies|:] All archives that this archive depends on. \sTeX ignores
            this field, but \mmt can pick up on them to resolve dependencies,
            e.g. for |lmh install|.  
    \end{itemize}

\end{sfragment}

\begin{sfragment}{Using Files in \sTeX Archives Directly}
    Several macros provided by \sTeX allow for directly including
    files in repositories. These are:
    \begin{function}{\mhinput}
        \stexcode"\mhinput[Some/Archive]{some/file}" directly
        inputs the file |some/file| in the |source|-folder of
        |Some/Archive|.
    \end{function}
    \begin{function}{\inputref}
      \stexcode"\inputref[Some/Archive]{some/file}" behaves like \stexcode"\mhinput", but
      wraps the input in a |\begingroup ... \endgroup|. When converting to |xhtml|, the
      file is not input at all, and instead an |html|-annotation is inserted that
      references the file, e.g. for lazy loading. 

      In the majority of practical cases \stexcode"\inputref" is likely to be preferred
      over \stexcode"\mhinput" because it leads to less duplication in the generated
      |xhtml|.
    \end{function}
    \begin{function}{\ifinput}
        Both \stexcode"\mhinput" and \stexcode"\inputref"
        set \stexcode"\ifinput" to ``true'' during input. This allows
        for selectively including e.g. bibliographies only if the
        current file is not being currently included in a larger document.
    \end{function}
    \begin{function}{\addmhbibresource}
      \stexcode"\addmhbibresource[Some/Archive]{some/file}" searches for a file like
      \stexcode"\mhinput" does, but calls |\addbibresource| to the result and looks for
      the file in the archive root directory directly, rather than the |source|
      directory. Typical invocations are
      \begin{itemize}
      \item |\addmhbibresource{lib/refs.bib}|, which specifies a bibliography in the |lib|
        folder in the local archive or
      \item |\addmhbibresource[HW/meta-inf]{lib/refs.bib}| in another.
      \end{itemize}
    \end{function}
    \begin{function}{\libinput}
        \stexcode"\libinput{some/file}" 
        searches for a file |some/file| in
        \begin{itemize}
            \item the |lib|-directory of the current archive, and
            \item the |lib|-directory of a |meta-inf|-archive in
                (any of) the archive groups containing the current archive
        \end{itemize}
        and include all found files in reverse order; 
        e.g. \stexcode"\libinput{preamble}" in a |.tex|-file in
        |smglom/calculus| will \emph{first} input |.../smglom/meta-inf/lib/preamble.tex|
        and then |../smglom/calculus/lib/preamble.tex|. 

        \stexcode|\libinput| will throw an error if \emph{no} candidate for |some/file|
        is found.
    \end{function}
    \begin{function}{\libusepackage}
      \stexcode"\libusepackage[package-options]{some/file}" searches for a file
      |some/file.sty| in the same way that \stexcode"\libinput" does, but will
      call\\
      |\usepackage[package-options]{path/to/some/file}| instead of |\input|.

      \stexcode|\libusepackage| throws an error if not \emph{exactly one} candidate for
      |some/file| is found.
    \end{function}

    \begin{remark}
        A good practice is to have individual \sTeX fragments
        follow basically this document frame:
        \begin{latexcode}[gobble=12]
            \documentclass{stex}
            \libinput{preamble}
            \begin{document}
                ... 
                \ifinputref \else \libinput{postamble} \fi
            \end{document}
        \end{latexcode}
        Then the |preamble.tex| files can take care of loading the generally required
        packages, setting presentation customizations etc. (per archive or archive group
        or both), and |postamble.tex| can e.g. print the bibliography, index etc.

        \stexcode|\libusepackage| is particularly useful in |preamble.tex| when we want to
        use custom packages that are not part of {\TeX}Live. In this case we commit the
        respective packages in one of the |lib| folders and use \stexcode|\libusepackage|
        to load them.
    \end{remark}
\end{sfragment}

%%% Local Variables:
%%% mode: latex
%%% TeX-master: "../stex-manual"
%%% End:

%%% LocalWords:  mathhub symdoc,for lmh subdirs arithmetics,smglom sets,smglom mv,smglom
%%% LocalWords:  linear-algebra,smglom
\end{sfragment}
% \fi
%
% \section{Macros and Environments}\label{pkg:mathhub:doc:macros}
%
% \begin{function}{\stex_kpsewhich:n}
% |\stex_kpsewhich:n| executes kpsewhich and stores the return
% in\\ |\l_stex_kpsewhich_return_str|. This does not require
% shell escaping.
% \end{function}
%
% \subsection{Files, Paths, URIs}
%
% \begin{function}{\stex_path_from_string:Nn}
%
%   \begin{syntax} \cs{stex_path_from_string:Nn} \meta{path-variable} \Arg{string} \end{syntax}
%   turns the \meta{string} into a path by splitting it at |/|-characters
%   and stores the result in \meta{path-variable}. Also applies
%   \cs{stex_path_canonicalize:N}.
% \end{function}
%
% \begin{function}{\stex_path_to_string:NN, \stex_path_to_string:N}
%   The inverse; turns a path into a string and stores it in the second
% argument variable, or leaves it in the input stream.
% \end{function}
%
% \begin{function}{\stex_path_canonicalize:N}
%   Canonicalizes the path provided; in particular, resolves |.| and |..|
%   path segments.
% \end{function}
%
% \begin{function}[pTF]{\stex_path_if_absolute:N}
%   Checks whether the path provided is \emph{absolute}, i.e. starts
%   with an empty segment
% \end{function}
%
% \begin{variable}{\c_stex_pwd_seq, \c_stex_pwd_str, \c_stex_mainfile_seq, \c_stex_mainfile_str}
%   Store the current working directory as path-sequence and string,
%   respectively, and the (heuristically guessed) full path to the
%   main file, based on the PWD and |\jobname|.
% \end{variable}
%
% \begin{variable}{\g_stex_currentfile_seq}
%   The file being currently processed (respecting |\input| etc.)
% \end{variable}
%
% \begin{function}{\stex_filestack_push:n,\stex_filestack_pop:}
% Push and pop (repsectively) a file path to the file stack,
% to keep track of the current file. Are called in hooks |file/before|
% and |file/after|, respectively.
% \end{function}
%
%  \subsection{MathHub Archives}
%
% \begin{variable}{\mathhub, \c_stex_mathhub_seq, \c_stex_mathhub_str}
% We determine the path to the local MathHub folder via one of
% four means, in order of precedence:
% \begin{enumerate}
%   \item The |mathhub| package option, or
%   \item the |\mathhub|-macro, if it has been defined before
%     the |\usepackage{stex}|-statement, or
%   \item the |MATHHUB| system variable, or
%   \item a path specified in |~/.stex/mathhub.path|.
% \end{enumerate}
% In all four cases, \cs{c_stex_mathhub_seq} and
% \cs{c_stex_mathhub_str} are set accordingly.
% \end{variable}
%
% \begin{variable}{\l_stex_current_repository_prop}
%   Always points to the \emph{current} MathHub repository (if
%   we currently are in one). Has the following fields corresponding
%   to the entries in the |MANIFEST.MF|-file:
%   \begin{itemize}
%     \item[|id|:] The name of the archive, including its group (e.g. |smglom/calculus|),
%     \item[|ns|:] The content namespace (for modules and symbols),
%     \item[|narr|:] the narration namespace (for document references),
%     \item[|docurl|:] The URL that is used as a basis for \emph{external references},
%     \item[|deps|:] All archives that this archive depends on (currently not in use).
%   \end{itemize}
% \end{variable}
%
% \begin{function}{\stex_set_current_repository:n}
%   Sets the current repository to the one with the provided ID.
%   calls \cs{__stex_mathhub_do_manifest:n}, so works whether this
%   repository's |MANIFEST.MF|-file has already been read or not.
% \end{function}
%
% \begin{function}{\stex_require_repository:n}
%   Calls \cs{__stex_mathhub_do_manifest:n} iff the corresponding
%   archive property list does not already exist, and
%   adds a corresponding definition to the |.sms|-file.
% \end{function}
%
% \begin{function}{\stex_in_repository:nn}
%   \begin{syntax}\cs{stex_in_repository:nn}\Arg{repository-name}\Arg{code}\end{syntax}
% Change the current repository to \Arg{repository-name} (or not, if \Arg{repository-name} is
% empty), and passes its ID on to \Arg{code} as |#1|. Switches back
% to the previous repository after executing \Arg{code}.
% \end{function}
%
%  \subsection{Using Content in Archives}
%
% \begin{function}[EXP]{\mhpath}
%   \begin{syntax}\cs{mhpath}\Arg{archive-ID}\Arg{filename}\end{syntax}
% Expands to the full path of file \meta{filename} in repository \meta{archive-ID}.
% Does not check whether the file or the repository exist. 
% \end{function}
%
% \begin{function}{\inputref,\mhinput}
%   \begin{syntax}\cs{inputref}|[|\meta{archive-ID}|]|\Arg{filename}\end{syntax}
% Both \cs{input} the file \meta{filename} in archive \meta{archive-ID} (relative
% to the |source|-subdirectory). \cs{mhinput} does so directly.
% \cs{inputref} does so within an |\begingroup|...|\endgroup|-block,
% and skips it in |html|-mode, inserting a \emph{reference} to the
% file instead.
%
%   Both also set |\ifinputref| to true.
% \end{function}
%
% \begin{function}{\addmhbibresource}
%   \begin{syntax}\cs{inputref}|[|\meta{archive-ID}|]|\Arg{filename}\end{syntax}
% Adds a |.bib|-file \meta{filename} in archive \meta{archive-ID} (relative
% to the top-directory of the archive!).
% \end{function}
%
% \begin{function}{\libinput}
%   \begin{syntax} \cs{libinput}\Arg{filename} \end{syntax}
%   Inputs \meta{filename}|.tex| from the |lib| folders in the
%   current archive and the |meta-inf|-archive of the current archive group(s)
%   (if existent) in descending order. Throws an error if no file by that name exists in
%   any of the relevant |lib|-folders.
% \end{function}
%
% \begin{function}{\libusepackage}
%   \begin{syntax} \cs{libusepackage}[\meta{args}]\Arg{filename} \end{syntax}
%   Like \cs{libinput}, but looks for |.sty|-files and calls
%   |\usepackage[\meta{args}]\Arg{filename}| instead of \cs{input}.
%
%   Throws an error, if none or more than one suitable package file is found.
% \end{function}
%
% \begin{function}{\mhgraphics,\cmhgraphics}
%   \emph{If} the \pkg{graphicx} package is loaded, these
%   macros are defined at |\begin{document}|.
%
%   \cs{mhgraphics} takes the same arguments as \cs{includegraphics},
%   with the additional optional key |mhrepos|. It then resolves
%   the file path in |\mhgraphics[mhrepos=Foo/Bar]{foo/bar.png}|
%   relative to the |source|-folder of the |Foo/Bar|-archive.
%
%   \cs{cmhgraphics} additional wraps the image in a |center|-environment.
% \end{function}
%
% \begin{function}{\lstinputmhlisting,\clstinputmhlisting}
%   Like \cs{mhgraphics}, but only defined if the \pkg{listings}-package
%   is loaded, and with \cs{lstinputlisting} instead of \cs{includegraphics}.
% \end{function}
%
% \end{documentation}
%
% \begin{implementation}
%
% \section{\sTeX-MathHub Implementation}\label{pkg:mathhub:doc:impl}
%
%    \begin{macrocode}
%<*package>

%%%%%%%%%%%%%   mathhub.dtx   %%%%%%%%%%%%%

%<@@=stex_path>
%    \end{macrocode}
%
% Warnings and error messages
%
%    \begin{macrocode}
\msg_new:nnn{stex}{error/norepository}{
  No~archive~#1~found~in~#2
}
\msg_new:nnn{stex}{error/notinarchive}{
  Not~currently~in~an~archive,~but~\detokenize{#1}~
  needs~one!
}
\msg_new:nnn{stex}{error/nofile}{
  \detokenize{#1}~could~not~find~file~#2
}
\msg_new:nnn{stex}{error/twofiles}{
  \detokenize{#1}~found~two~candidates~for~#2
}
%    \end{macrocode}
%
% \subsubsection{Generic Path Handling}
%
% We treat paths as \LaTeX3-sequences (of the individual
% path segments, i.e. separated by a /-character) unix-style;
% i.e. a path is absolute if the sequence starts with an empty 
% entry.
%
% \begin{macro}{\stex_path_from_string:Nn}
%    \begin{macrocode}
\cs_new_protected:Nn \stex_path_from_string:Nn {
  \str_set:Nx \l_tmpa_str { #2 }
  \str_if_empty:NTF \l_tmpa_str {
    \seq_clear:N #1
  }{
    \exp_args:NNNo \seq_set_split:Nnn #1 / { \l_tmpa_str }    
    \sys_if_platform_windows:T{
      \seq_clear:N \l_tmpa_tl
      \seq_map_inline:Nn #1 {
        \seq_set_split:Nnn \l_tmpb_tl \c_backslash_str { ##1 }
        \seq_concat:NNN \l_tmpa_tl \l_tmpa_tl \l_tmpb_tl
      }
      \seq_set_eq:NN #1 \l_tmpa_tl
    }
    \stex_path_canonicalize:N #1
  }
}

%    \end{macrocode}
% \end{macro}
%
% \begin{macro}{\stex_path_to_string:NN,\stex_path_to_string:N}
%    \begin{macrocode}
\cs_new_protected:Nn \stex_path_to_string:NN {
  \exp_args:NNe \str_set:Nn #2 { \seq_use:Nn #1 / }
}

\cs_new:Nn \stex_path_to_string:N {
  \seq_use:Nn #1 /
}
%    \end{macrocode}
% \end{macro}
%
% \begin{variable}{\c_@@_dot_str,\c_@@_up_str}
%
% |.| and |..|, respectively.
%
%    \begin{macrocode}
\str_const:Nn \c_@@_dot_str {.}
\str_const:Nn \c_@@_up_str {..}
%    \end{macrocode}
% \end{variable}
%
% \begin{macro}{\stex_path_canonicalize:N}
%
%  Canonicalizes the path provided; in particular, resolves |.| and |..|
%  path segments.
%
%    \begin{macrocode}
\cs_new_protected:Nn \stex_path_canonicalize:N {
  \seq_if_empty:NF #1 {
    \seq_clear:N \l_tmpa_seq
    \seq_get_left:NN #1 \l_tmpa_tl
    \str_if_empty:NT \l_tmpa_tl {
      \seq_put_right:Nn \l_tmpa_seq {}
    }
    \seq_map_inline:Nn #1 {
      \str_set:Nn \l_tmpa_tl { ##1 }
      \str_if_eq:NNF \l_tmpa_tl \c_@@_dot_str {
        \str_if_eq:NNTF \l_tmpa_tl \c_@@_up_str {
          \seq_if_empty:NTF \l_tmpa_seq {
            \exp_args:NNo \seq_put_right:Nn \l_tmpa_seq {
              \c_@@_up_str
            }
          }{
            \seq_get_right:NN \l_tmpa_seq \l_tmpa_tl
            \str_if_eq:NNTF \l_tmpa_tl \c_@@_up_str {
              \exp_args:NNo \seq_put_right:Nn \l_tmpa_seq {
                \c_@@_up_str
              }
            }{
              \seq_pop_right:NN \l_tmpa_seq \l_tmpb_tl
            }
          }
        }{
          \str_if_empty:NF \l_tmpa_tl {
            \exp_args:NNo \seq_put_right:Nn \l_tmpa_seq { \l_tmpa_tl }
          }
        }
      }
    }
    \seq_gset_eq:NN #1 \l_tmpa_seq
  }
}
%    \end{macrocode}
% \end{macro}
%
% \begin{macro}[pTF]{\stex_path_if_absolute:N}
%    \begin{macrocode}
\prg_new_conditional:Nnn \stex_path_if_absolute:N {p, T, F, TF} {
  \seq_if_empty:NTF #1 {
    \prg_return_false:
  }{
    \seq_get_left:NN #1 \l_tmpa_tl
    \sys_if_platform_windows:TF{
      \str_if_in:NnTF \l_tmpa_tl {:}{
        \prg_return_true:
      }{
        \prg_return_false:
      }
    }{
      \str_if_empty:NTF \l_tmpa_tl {
        \prg_return_true:
      }{
        \prg_return_false:
      }
    }
  }
}
%    \end{macrocode}
% \end{macro}
%
% \subsubsection{PWD and kpsewhich}
%
% \begin{macro}{\stex_kpsewhich:n}
%    \begin{macrocode}
\str_new:N\l_stex_kpsewhich_return_str
\cs_new_protected:Nn \stex_kpsewhich:n {\begingroup
  \catcode`\ =12
  \sys_get_shell:nnN { kpsewhich ~ #1 } { } \l_tmpa_tl
  \tl_gset_eq:NN \l_tmpa_tl \l_tmpa_tl
  \endgroup
  \exp_args:NNo\str_set:Nn\l_stex_kpsewhich_return_str{\l_tmpa_tl}
  \tl_trim_spaces:N \l_stex_kpsewhich_return_str
}
%    \end{macrocode}
% \end{macro}
%
% We determine the PWD
%
% \begin{variable}{\c_stex_pwd_seq,\c_stex_pwd_str}
%    \begin{macrocode}
\sys_if_platform_windows:TF{
  \begingroup\escapechar=-1\catcode`\\=12
  \exp_args:Nx\stex_kpsewhich:n{-expand-var~\c_percent_str CD\c_percent_str}
  \exp_args:NNx\str_replace_all:Nnn\l_stex_kpsewhich_return_str{\c_backslash_str}/
  \exp_args:Nnx\use:nn{\endgroup}{\str_set:Nn\exp_not:N\l_stex_kpsewhich_return_str{\l_stex_kpsewhich_return_str}}
}{
  \stex_kpsewhich:n{-var-value~PWD}
}

\stex_path_from_string:Nn\c_stex_pwd_seq\l_stex_kpsewhich_return_str
\stex_path_to_string:NN\c_stex_pwd_seq\c_stex_pwd_str
\stex_debug:nn {mathhub} {PWD:~\str_use:N\c_stex_pwd_str}
%    \end{macrocode}
% \end{variable}
%
% \subsubsection{File Hooks and Tracking}
%    \begin{macrocode}
%<@@=stex_files>
%    \end{macrocode}
%
% We introduce hooks for file inputs that keep track of the
% absolute paths of files used. This will be useful to keep track
% of modules, their archives, namespaces etc.
%
% Note that the absolute paths are only accurate in |\input|-statements
% for paths relative to the PWD, so they shouldn't be relied upon
% in any other setting than for \sTeX-purposes.
%
% \begin{variable}{\g_@@_stack}
%
% keeps track of file changes
%
%    \begin{macrocode}
\seq_gclear_new:N\g_@@_stack
%    \end{macrocode}
% \end{variable}
%
% \begin{variable}{\c_stex_mainfile_seq, \c_stex_mainfile_str}
%    \begin{macrocode}
\str_set:Nx \c_stex_mainfile_str {\c_stex_pwd_str/\jobname.tex}
\stex_path_from_string:Nn \c_stex_mainfile_seq 
  \c_stex_mainfile_str
%    \end{macrocode}
% \end{variable}
%
% \begin{variable}{\g_stex_currentfile_seq}
%    \begin{macrocode}
\seq_gclear_new:N\g_stex_currentfile_seq
%    \end{macrocode}
% \end{variable}
%
% \begin{macro}{\stex_filestack_push:n}
%    \begin{macrocode}
\cs_new_protected:Nn \stex_filestack_push:n {
  \stex_path_from_string:Nn\g_stex_currentfile_seq{#1}
  \stex_path_if_absolute:NF\g_stex_currentfile_seq{
    \stex_path_from_string:Nn\g_stex_currentfile_seq{
      \c_stex_pwd_str/#1
    }
  }
  \seq_gset_eq:NN\g_stex_currentfile_seq\g_stex_currentfile_seq
  \exp_args:NNo\seq_gpush:Nn\g_@@_stack\g_stex_currentfile_seq
}
%    \end{macrocode}
% \end{macro}
%
% \begin{macro}{\stex_filestack_pop:}
%    \begin{macrocode}
\cs_new_protected:Nn \stex_filestack_pop: {
  \seq_if_empty:NF\g_@@_stack{
    \seq_gpop:NN\g_@@_stack\l_tmpa_seq
  }
  \seq_if_empty:NTF\g_@@_stack{
    \seq_gset_eq:NN\g_stex_currentfile_seq\c_stex_mainfile_seq
  }{
    \seq_get:NN\g_@@_stack\l_tmpa_seq
    \seq_gset_eq:NN\g_stex_currentfile_seq\l_tmpa_seq
  }
}
%    \end{macrocode}
% \end{macro}
%
% Hooks for the current file:
%
%    \begin{macrocode}
\AddToHook{file/before}{
  \stex_filestack_push:n{\CurrentFilePath/\CurrentFile}
}
\AddToHook{file/after}{
  \stex_filestack_pop:
}
%    \end{macrocode}
%
% \subsection{MathHub Repositories}
%    \begin{macrocode}
%<@@=stex_mathhub>
%    \end{macrocode}
%
% \begin{variable}{\mathhub, \c_stex_mathhub_seq, \c_stex_mathhub_str}
% The path to the mathhub directory. If the \cs{mathhub}-macro is not set,
% we query |kpsewhich| for the |MATHHUB| system variable.
%    \begin{macrocode}
\str_if_empty:NTF\mathhub{
  \sys_if_platform_windows:TF{
    \begingroup\escapechar=-1\catcode`\\=12
    \exp_args:Nx\stex_kpsewhich:n{-expand-var~\c_percent_str MATHHUB\c_percent_str}
    \exp_args:NNx\str_replace_all:Nnn\l_stex_kpsewhich_return_str{\c_backslash_str}/
    \exp_args:Nnx\use:nn{\endgroup}{\str_set:Nn\exp_not:N\l_stex_kpsewhich_return_str{\l_stex_kpsewhich_return_str}}
  }{
    \stex_kpsewhich:n{-var-value~MATHHUB}
  }
  \str_set_eq:NN\c_stex_mathhub_str\l_stex_kpsewhich_return_str

  \str_if_empty:NT \c_stex_mathhub_str {
    \sys_if_platform_windows:TF{
      \begingroup\escapechar=-1\catcode`\\=12
      \exp_args:Nx\stex_kpsewhich:n{-var-value~HOME}
      \exp_args:NNx\str_replace_all:Nnn\l_stex_kpsewhich_return_str{\c_backslash_str}/
      \exp_args:Nnx\use:nn{\endgroup}{\str_set:Nn\exp_not:N\l_stex_kpsewhich_return_str{\l_stex_kpsewhich_return_str}}
    }{
      \stex_kpsewhich:n{-var-value~HOME}
    }
    \ior_open:NnT \l_tmpa_ior{\l_stex_kpsewhich_return_str / .stex / mathhub.path}{
      \begingroup\escapechar=-1\catcode`\\=12
      \ior_str_get:NN \l_tmpa_ior \l_tmpa_str
      \sys_if_platform_windows:T{
        \exp_args:NNx\str_replace_all:Nnn\l_tmpa_str{\c_backslash_str}/
      }
      \str_gset_eq:NN \c_stex_mathhub_str\l_tmpa_str
      \endgroup
      \ior_close:N \l_tmpa_ior
    }
  }
  \str_if_empty:NTF\c_stex_mathhub_str{
    \msg_warning:nn{stex}{warning/nomathhub}
  }{
    \stex_debug:nn{mathhub}{MathHub:~\str_use:N\c_stex_mathhub_str}
    \exp_args:NNo \stex_path_from_string:Nn\c_stex_mathhub_seq\c_stex_mathhub_str
  }
}{
  \stex_path_from_string:Nn \c_stex_mathhub_seq \mathhub
  \stex_path_if_absolute:NF \c_stex_mathhub_seq {
    \exp_args:NNx \stex_path_from_string:Nn \c_stex_mathhub_seq {
      \c_stex_pwd_str/\mathhub
    }
  }
  \stex_path_to_string:NN\c_stex_mathhub_seq\c_stex_mathhub_str
  \stex_debug:nn{mathhub} {MathHub:~\str_use:N\c_stex_mathhub_str}
}
%    \end{macrocode}
% \end{variable}
%
% \begin{macro}{\_@@_do_manifest:n}
% Checks whether the manifest for archive |#1| already exists, and
% if not, finds and parses the corresponding manifest file
%    \begin{macrocode}
\cs_new_protected:Nn \_@@_do_manifest:n {
  \prop_if_exist:cF {c_stex_mathhub_#1_manifest_prop} {
    \str_set:Nx \l_tmpa_str { #1 }
    \prop_new:c { c_stex_mathhub_#1_manifest_prop }
    \seq_set_split:NnV \l_tmpa_seq / \l_tmpa_str
    \seq_concat:NNN \l_tmpa_seq \c_stex_mathhub_seq \l_tmpa_seq
    \_@@_find_manifest:N \l_tmpa_seq
    \seq_if_empty:NTF \l_@@_manifest_file_seq {
      \msg_error:nnxx{stex}{error/norepository}{#1}{
        \stex_path_to_string:N \c_stex_mathhub_str
      }
      \input{Fatal~Error!}
    } {
      \exp_args:No \_@@_parse_manifest:n { \l_tmpa_str }
    }
  }
}
%    \end{macrocode}
% \end{macro}
%
% \begin{variable}{\l_@@_manifest_file_seq}
%    \begin{macrocode}
\seq_new:N\l_@@_manifest_file_seq
%    \end{macrocode}
% \end{variable}
%
% \begin{macro}{\_@@_find_manifest:N}
%
% Attempts to find the |MANIFEST.MF| in some file path and
% stores its path in \cs{l_@@_manifest_file_seq}:
%
%    \begin{macrocode}
\cs_new_protected:Nn \_@@_find_manifest:N {
  \seq_set_eq:NN\l_tmpa_seq #1
  \bool_set_true:N\l_tmpa_bool
  \bool_while_do:Nn \l_tmpa_bool {
    \seq_if_empty:NTF \l_tmpa_seq {
      \bool_set_false:N\l_tmpa_bool
    }{
      \file_if_exist:nTF{
        \stex_path_to_string:N\l_tmpa_seq/MANIFEST.MF
      }{
        \seq_put_right:Nn\l_tmpa_seq{MANIFEST.MF}
        \bool_set_false:N\l_tmpa_bool
      }{
        \file_if_exist:nTF{
          \stex_path_to_string:N\l_tmpa_seq/META-INF/MANIFEST.MF
        }{
          \seq_put_right:Nn\l_tmpa_seq{META-INF}
          \seq_put_right:Nn\l_tmpa_seq{MANIFEST.MF}
          \bool_set_false:N\l_tmpa_bool
        }{
          \file_if_exist:nTF{
            \stex_path_to_string:N\l_tmpa_seq/meta-inf/MANIFEST.MF
          }{
            \seq_put_right:Nn\l_tmpa_seq{meta-inf}
            \seq_put_right:Nn\l_tmpa_seq{MANIFEST.MF}
            \bool_set_false:N\l_tmpa_bool
          }{
            \seq_pop_right:NN\l_tmpa_seq\l_tmpa_tl
          }
        }
      }
    }
  }
  \seq_set_eq:NN\l_@@_manifest_file_seq\l_tmpa_seq
}
%    \end{macrocode}
% \end{macro}
%
% \begin{variable}{\c_@@_manifest_ior}
%
%   File variable used for |MANIFEST|-files
%
%    \begin{macrocode}
\ior_new:N \c_@@_manifest_ior
%    \end{macrocode}
% \end{variable}
%
% \begin{macro}{\_@@_parse_manifest:n}
%
% Stores the entries in manifest file in the
% corresponding property list:
%
%    \begin{macrocode}
\cs_new_protected:Nn \_@@_parse_manifest:n {
  \seq_set_eq:NN \l_tmpa_seq \l_@@_manifest_file_seq
  \ior_open:Nn \c_@@_manifest_ior {\stex_path_to_string:N \l_tmpa_seq}
  \ior_map_inline:Nn \c_@@_manifest_ior {
    \str_set:Nn \l_tmpa_str {##1}
    \exp_args:NNoo \seq_set_split:Nnn 
        \l_tmpb_seq \c_colon_str \l_tmpa_str
    \seq_pop_left:NNTF \l_tmpb_seq \l_tmpa_tl {
      \exp_args:NNe \str_set:Nn \l_tmpb_tl { 
        \exp_args:NNo \seq_use:Nn \l_tmpb_seq \c_colon_str 
      }
      \exp_args:No \str_case:nnTF \l_tmpa_tl {
        {id} {
          \prop_gput:cno { c_stex_mathhub_#1_manifest_prop } 
            { id } \l_tmpb_tl
        }
        {narration-base} {
          \prop_gput:cno { c_stex_mathhub_#1_manifest_prop } 
            { narr } \l_tmpb_tl
        }
        {url-base} {
          \prop_gput:cno { c_stex_mathhub_#1_manifest_prop } 
            { docurl } \l_tmpb_tl
        }
        {source-base} {
          \prop_gput:cno { c_stex_mathhub_#1_manifest_prop } 
            { ns } \l_tmpb_tl
        }
        {ns} {
          \prop_gput:cno { c_stex_mathhub_#1_manifest_prop } 
            { ns } \l_tmpb_tl
        }
        {dependencies} {
          \prop_gput:cno { c_stex_mathhub_#1_manifest_prop } 
            { deps } \l_tmpb_tl
        }
      }{}{}
    }{}
  }
  \ior_close:N \c_@@_manifest_ior
  \stex_persist:x {
    \prop_set_from_keyval:cn{ c_stex_mathhub_#1_manifest_prop }{
      \exp_after:wN \prop_to_keyval:N \csname c_stex_mathhub_#1_manifest_prop\endcsname
    }
  }
}
%    \end{macrocode}
% \end{macro}
% 
%
% \begin{macro}{\stex_set_current_repository:n}
%    \begin{macrocode}
\cs_new_protected:Nn \stex_set_current_repository:n {
  \stex_require_repository:n { #1 }
  \prop_set_eq:Nc \l_stex_current_repository_prop { 
    c_stex_mathhub_#1_manifest_prop 
  }
}
%    \end{macrocode}
% \end{macro}
%
% \begin{macro}{\stex_require_repository:n}
%    \begin{macrocode}
\cs_new_protected:Nn \stex_require_repository:n {
  \prop_if_exist:cF { c_stex_mathhub_#1_manifest_prop } {
    \stex_debug:nn{mathhub}{Opening~archive:~#1}
    \_@@_do_manifest:n { #1 }
  }
}
%    \end{macrocode}
% \end{macro}
%
%\begin{variable}{\l_stex_current_repository_prop}
%
% Current MathHub repository
%
%    \begin{macrocode}
%\prop_new:N \l_stex_current_repository_prop
\bool_if:NF \c_stex_persist_mode_bool {
  \_@@_find_manifest:N \c_stex_pwd_seq
  \seq_if_empty:NTF \l_@@_manifest_file_seq {
    \stex_debug:nn{mathhub}{Not~currently~in~a~MathHub~repository}
  } {
    \_@@_parse_manifest:n { main }
    \prop_get:NnN \c_stex_mathhub_main_manifest_prop {id} 
      \l_tmpa_str
    \prop_set_eq:cN { c_stex_mathhub_\l_tmpa_str _manifest_prop }
      \c_stex_mathhub_main_manifest_prop
    \exp_args:Nx \stex_set_current_repository:n { \l_tmpa_str }
    \stex_debug:nn{mathhub}{Current~repository:~
      \prop_item:Nn \l_stex_current_repository_prop {id}
    }
  }
}
%    \end{macrocode}
% \end{variable}
%
% \begin{macro}{\stex_in_repository:nn}
% Executes the code in the second argument in the context
% of the repository whose ID is provided as the first argument.
%    \begin{macrocode}
\cs_new_protected:Nn \stex_in_repository:nn {
  \str_set:Nx \l_tmpa_str { #1 }
  \cs_set:Npn \l_tmpa_cs ##1 { #2 }
  \str_if_empty:NTF \l_tmpa_str {
    \prop_if_exist:NTF \l_stex_current_repository_prop {
      \stex_debug:nn{mathhub}{do~in~current~repository:~\prop_item:Nn \l_stex_current_repository_prop { id }}
      \exp_args:Ne \l_tmpa_cs{
        \prop_item:Nn \l_stex_current_repository_prop { id }
      }
    }{
      \l_tmpa_cs{}
    }
  }{
    \stex_debug:nn{mathhub}{in~repository:~\l_tmpa_str}
    \stex_require_repository:n \l_tmpa_str
    \str_set:Nx \l_tmpa_str { #1 }
    \exp_args:Nne \use:nn {
      \stex_set_current_repository:n \l_tmpa_str
      \exp_args:Nx \l_tmpa_cs{\l_tmpa_str}
    }{
      \stex_debug:nn{mathhub}{switching~back~to:~
        \prop_if_exist:NTF \l_stex_current_repository_prop {
          \prop_item:Nn \l_stex_current_repository_prop { id }:~
          \meaning\l_stex_current_repository_prop
        }{
          no~repository
        }
      }
      \prop_if_exist:NTF \l_stex_current_repository_prop {
       \stex_set_current_repository:n {
        \prop_item:Nn \l_stex_current_repository_prop { id }
       }
      }{
        \let\exp_not:N\l_stex_current_repository_prop\exp_not:N\undefined
      }
    }
  }
}
%    \end{macrocode}
% \end{macro}
%
%  \subsection{Using Content in Archives}
%
% \begin{macro}{\mhpath}
%    \begin{macrocode}
\def \mhpath #1 #2 {
  \exp_args:Ne \tl_if_empty:nTF{#1}{
    \c_stex_mathhub_str / 
      \prop_item:Nn \l_stex_current_repository_prop { id }
      / source / #2
  }{
    \c_stex_mathhub_str / #1 / source / #2
  }
}
%    \end{macrocode}
% \end{macro}
%
% \begin{macro}{\inputref,\mhinput}
%    \begin{macrocode}
\newif \ifinputref \inputreffalse

\cs_new_protected:Nn \_@@_mhinput:nn {
  \stex_in_repository:nn {#1} {
    \ifinputref
      \input{ \c_stex_mathhub_str / ##1 / source / #2 }
    \else
      \inputreftrue
      \input{ \c_stex_mathhub_str / ##1 / source / #2 }
      \inputreffalse
    \fi
  }
}
\NewDocumentCommand \mhinput { O{} m}{
  \_@@_mhinput:nn{ #1 }{ #2 }
}

\cs_new_protected:Nn \_@@_inputref:nn {
  \stex_in_repository:nn {#1} {
    \stex_html_backend:TF {
      \str_clear:N \l_tmpa_str
      \prop_get:NnNF \l_stex_current_repository_prop { narr } \l_tmpa_str {
        \prop_get:NnNF \l_stex_current_repository_prop { ns } \l_tmpa_str {}
      }

      \tl_if_empty:nTF{ ##1 }{
        \IfFileExists{#2}{
          \stex_annotate_invisible:nnn{inputref}{
            \l_tmpa_str / #2
          }{}
        }{
          \input{#2}
        }
      }{
        \IfFileExists{ \c_stex_mathhub_str / ##1 / source / #2 }{
          \stex_annotate_invisible:nnn{inputref}{
            \l_tmpa_str / #2
          }{}
        }{
          \input{ \c_stex_mathhub_str / ##1 / source / #2 }
        }
      }

    }{
      \begingroup
        \inputreftrue
        \tl_if_empty:nTF{ ##1 }{
          \input{#2}
        }{
          \input{ \c_stex_mathhub_str / ##1 / source / #2 }
        }
      \endgroup
    }
  }
}
\NewDocumentCommand \inputref { O{} m}{
  \_@@_inputref:nn{ #1 }{ #2 }
}
%    \end{macrocode}
% \end{macro}
%
% \begin{macro}{\addmhbibresource}
%    \begin{macrocode}
\cs_new_protected:Nn \_@@_mhbibresource:nn {
  \stex_in_repository:nn {#1} {
    \addbibresource{ \c_stex_mathhub_str / ##1 / #2 }
  }
}
\newcommand\addmhbibresource[2][]{
  \_@@_mhbibresource:nn{ #1 }{ #2 }
}
%    \end{macrocode}
% \end{macro}
%
% \begin{macro}{\libinput}
%    \begin{macrocode}
\cs_new_protected:Npn \libinput #1 {
  \prop_if_exist:NF \l_stex_current_repository_prop {
    \msg_error:nnn{stex}{error/notinarchive}\libinput
  } 
  \prop_get:NnNF \l_stex_current_repository_prop {id} \l_tmpa_str {
    \msg_error:nnn{stex}{error/notinarchive}\libinput
  }
  \seq_clear:N \l_@@_libinput_files_seq
  \seq_set_eq:NN \l_tmpa_seq \c_stex_mathhub_seq
  \seq_set_split:NnV \l_tmpb_seq / \l_tmpa_str

  \bool_while_do:nn { ! \seq_if_empty_p:N \l_tmpb_seq }{
    \str_set:Nx \l_tmpa_str {\stex_path_to_string:N \l_tmpa_seq / meta-inf / lib / #1.tex}
    \IfFileExists{ \l_tmpa_str }{
      \seq_put_right:No \l_@@_libinput_files_seq \l_tmpa_str
    }{}
    \seq_pop_left:NN \l_tmpb_seq \l_tmpa_str
    \seq_put_right:No \l_tmpa_seq \l_tmpa_str
  }

  \str_set:Nx \l_tmpa_str {\stex_path_to_string:N \l_tmpa_seq / lib / #1.tex}
  \IfFileExists{ \l_tmpa_str }{
    \seq_put_right:No \l_@@_libinput_files_seq \l_tmpa_str
  }{}

  \seq_if_empty:NTF \l_@@_libinput_files_seq {
    \msg_error:nnxx{stex}{error/nofile}{\exp_not:N\libinput}{#1.tex}
  }{
    \seq_map_inline:Nn \l_@@_libinput_files_seq {
      \input{ ##1 }
    }
  }
}
%    \end{macrocode}
% \end{macro}
%
% \begin{macro}{\libusepackage}
%    \begin{macrocode}
\NewDocumentCommand \libusepackage {O{} m} {
  \prop_if_exist:NF \l_stex_current_repository_prop {
    \msg_error:nnn{stex}{error/notinarchive}\libusepackage
  } 
  \prop_get:NnNF \l_stex_current_repository_prop {id} \l_tmpa_str {
    \msg_error:nnn{stex}{error/notinarchive}\libusepackage
  }
  \seq_clear:N \l_@@_libinput_files_seq
  \seq_set_eq:NN \l_tmpa_seq \c_stex_mathhub_seq
  \seq_set_split:NnV \l_tmpb_seq / \l_tmpa_str

  \bool_while_do:nn { ! \seq_if_empty_p:N \l_tmpb_seq }{
    \str_set:Nx \l_tmpa_str {\stex_path_to_string:N \l_tmpa_seq / meta-inf / lib / #2}
    \IfFileExists{ \l_tmpa_str.sty }{
      \seq_put_right:No \l_@@_libinput_files_seq \l_tmpa_str
    }{}
    \seq_pop_left:NN \l_tmpb_seq \l_tmpa_str
    \seq_put_right:No \l_tmpa_seq \l_tmpa_str
  }

  \str_set:Nx \l_tmpa_str {\stex_path_to_string:N \l_tmpa_seq / lib / #2}
  \IfFileExists{ \l_tmpa_str.sty }{
    \seq_put_right:No \l_@@_libinput_files_seq \l_tmpa_str
  }{}

  \seq_if_empty:NTF \l_@@_libinput_files_seq {
    \msg_error:nnxx{stex}{error/nofile}{\exp_not:N\libusepackage}{#2.sty}
  }{
    \int_compare:nNnTF {\seq_count:N \l_@@_libinput_files_seq} = 1 {
      \seq_map_inline:Nn \l_@@_libinput_files_seq {
        \usepackage[#1]{ ##1 }
      }
    }{
      \msg_error:nnxx{stex}{error/twofiles}{\exp_not:N\libusepackage}{#2.sty}
    }
  }
}
%    \end{macrocode}
% \end{macro}
%
% \begin{macro}{\mhgraphics,\cmhgraphics}
%    \begin{macrocode}

\AddToHook{begindocument}{
	\ltx@ifpackageloaded{graphicx}{
    \define@key{Gin}{mhrepos}{\def\Gin@mhrepos{#1}}
    \newcommand\mhgraphics[2][]{%
      \def\Gin@mhrepos{}\setkeys{Gin}{#1}%
      \includegraphics[#1]{\mhpath\Gin@mhrepos{#2}}}
    \newcommand\cmhgraphics[2][]{\begin{center}\mhgraphics[#1]{#2}\end{center}}    
  }{}
%    \end{macrocode}
% \end{macro}
%
% \begin{macro}{\lstinputmhlisting,\clstinputmhlisting}
%    \begin{macrocode}
	\ltx@ifpackageloaded{listings}{
    \define@key{lst}{mhrepos}{\def\lst@mhrepos{#1}}
    \newcommand\lstinputmhlisting[2][]{%
      \def\lst@mhrepos{}\setkeys{lst}{#1}%
      \lstinputlisting[#1]{\mhpath\lst@mhrepos{#2}}}
    \newcommand\clstinputmhlisting[2][]{\begin{center}\lstinputmhlisting[#1]{#2}\end{center}}
  }{}
}

%</package>
%    \end{macrocode}
% \end{macro}
%
% \end{implementation}
% \ifinfulldoc\else\printbibliography\fi
%
% \PrintIndex

  \end{sfragment}

  \begin{sfragment}[id=sec.decls]{Module, Symbol and Notation Declarations}
    \begin{sfragment}{The \texttt{smodule}-Environment}
    \begin{environment}{smodule}
        A new module is declared using the basic syntax
        \begin{center}
        \stexcode"\begin{smodule}[options]{ModuleName}...\end{smodule}".
        \end{center}
        A module is required to declare any new formal content such as
        symbols or notations (but not variables, which may be introduced
        anywhere).

        The |smodule|-environment takes several optional arguments,
        all of which are optional:

        \begin{itemize}
            \item[|title|] (\meta{token list}) to display in customizations.
            \item[|type|] (\meta{string}$\ast$)  for use in customizations.
            \item[|deprecate|] (\meta{module}) if set, will throw a warning
            when loaded, urging to use \meta{module} instead.
            \item[|id|] (\meta{string}) for cross-referencing.
            \item[|ns|] (\meta{URI}) the namespace to use. \emph{Should not be used,
            unless you know precisely what you're doing}. If not explicitly set, is
            computed using \cs{stex_modules_current_namespace:}.
            \item[|lang|] (\meta{language}) if not set, computed from the current file name (e.g. |foo.en.tex|).
            \item[|sig|] (\meta{language}) if the current file is a translation of a file with the same base name
            but a different language suffix, setting |sig=<lang>| will preload the module
            from that language file. This helps ensuring that the (formal) content of both modules
            is (almost) identical across languages and avoids duplication.
            \item[|creators|] (\meta{string}$\ast$) names of the creators.
            \item[|contributors|] (\meta{string}$\ast$) names of contributors.
            \item[|srccite|] (\meta{string}) a source citation for the content of this module.
        \end{itemize}
    \end{environment}

    \begin{mmtbox}
        An \sTeX module corresponds to an \mmt/\omdoc \emph{theory}.
    \end{mmtbox}

    By default, opening a module will produce no output whatsoever,
    e.g.:
    \stexexample{
\begin{smodule}[title={This is Some Module}]{SomeModule}
    Hello World
\end{smodule}
    }

    \begin{function}{\stexpatchmodule}
        We can customize this behavior either for all modules or
        only for modules with a specific |type| using the command
        \stexcode"\stexpatchmodule[optional-type]{begin-code}{end-code}".
        Some optional parameters are then available in |\smodule*|-macros,
        specifically |\smoduletitle|, |\smoduletype| and |\smoduleid|.
        For example:

        \stexexample{
\stexpatchmodule[display]
  {\textbf{Module (\smoduletitle)}\par}
  {\par\noindent\textbf{End of Module (\smoduletitle)}}

\begin{smodule}[type=display,title={Some New Module}]{SomeModule2}
    Hello World
\end{smodule}
        }

    \end{function}
\end{sfragment}

    % \iffalse meta-comment
% An Infrastructure for Semantic Macros and Module Scoping
% Copyright (c) 2019 Michael Kohlhase, all rights reserved
%                this file is released under the
%                LaTeX Project Public License (LPPL)
% 
% The original of this file is in the public repository at 
% http://github.com/sLaTeX/sTeX/
%
% TODO update copyright  
%
%<*driver>
\providecommand\bibfolder{../../lib/bib}
\RequirePackage{paralist}
\ifcsname stexdocpath\endcsname\else\def\stexdocpath{.}\fi
\documentclass[full]{l3doc}
%\RequirePackage{document-structure}
\usepackage[hyperref=auto,style=alphabetic]{biblatex}
%\usepackage[mathhub=\stexdocpath/mh,usedeps]{stex}
\usepackage[lang={en,de}]{stex}

\usepackage{rustex}
\usepackage{stex-highlighting,stexthm}

\srefsetin[sTeX/Documentation]{documentation}{the \stex Documentation}

\makeatletter
\providecommand{\HTML}{\textsc{html}\xspace}%
\providecommand{\XML}{\textsc{xml}\xspace}%
\providecommand{\PDF}{\textsc{pdf}\xspace}%
\providecommand\openmath{\textsc{OpenMath}\xspace}
\providecommand\OMDoc{\textsc{OMDoc}\xspace}
\DeclareRobustCommand\LaTeXML{L\kern-.36em%
        {\sbox\z@ T%
         \vbox to\ht\z@{\hbox{\check@mathfonts
                              \fontsize\sf@size\z@
                              \math@fontsfalse\selectfont
                              A}%
                        \vss}%
        }%
        \kern-.15em%
%        T\kern-.1667em\lower.5ex\hbox{E}\kern-.125em\relax
%        {\tt XML}}
        T\kern-.1667em\lower.4ex\hbox{E}\kern-0.05em\relax
        {\scshape xml}\xspace}%
\def\mmt{\textsc{Mmt}\xspace}
\makeatother


\newif\ifhadtitle\hadtitlefalse

\def\stexversion{3.3.0}
\def\changedate{\today}
\def\stextoptitle#1#2{\title{#1\thanks{Version {\stexversion} (last revised {\changedate})} }\def\thispkg{#2}}

\author{Michael Kohlhase, Dennis Müller\\
 	FAU Erlangen-Nürnberg\\
 	\url{http://kwarc.info/}
}

\def\stexmaketitle{\ifhadtitle\else\hadtitletrue\maketitle\fi}

\ExplSyntaxOn

  \def\docmodule{
    \begin{document}
      \EnableManual
      \EnableDocumentation
      \EnableImplementation
      \stexmaketitle
      \tableofcontents
      \int_gincr:N \l_stex_docheader_sect
      \exp_args:Ne \__stex_mathhub_find_manifest:n {\stex_file_use:N \c_stex_mathhub_file / sTeX / Documentation}
      \str_if_empty:NF \l__stex_mathhub_manifest_str {
        \usemodule[sTeX/Documentation]{macros?AllMacros}
      }
      \DocInput{\jobname.dtx}
      \clearpage
      \PrintIndex
      \printbibliography
    \end{document}
  }

  \bool_new:N \g_stexdoc_typeset_manual_bool
  \NewDocumentCommand \EnableManual {}{
    \bool_gset_true:N \g_stexdoc_typeset_manual_bool
  }
  \NewDocumentCommand \DisableManual {}{
    \bool_gset_false:N \g_stexdoc_typeset_manual_bool
  }
  \NewDocumentEnvironment {stexmanual} {} {
    \bool_if:NTF \g_stexdoc_typeset_manual_bool
      {\bool_set_false:N \l__codedoc_in_implementation_bool}
      {\comment}
  }{
    \bool_if:NF \g_stexdoc_typeset_manual_bool {\endcomment}
  }
\ExplSyntaxOff

%\usepackage{makeidx}
%\makeindex

%\usepackage{document-structure}


\usepackage{lststex,mdframed}
\usepackage[most]{tcolorbox}

\lstset{literate=%
    {Ö}{{\"O}}1
    {Ä}{{\"A}}1
    {Ü}{{\"U}}1
    {ß}{{\ss}}1
    {ü}{{\"u}}1
    {ä}{{\"a}}1
    {ö}{{\"o}}1
    {~}{{\textasciitilde}}1
}

\newenvironment{framed}[1][]{
  \ifstexhtml\par\vbox\bgroup
    \csname exp_args:Nne\endcsname\begin{stex_annotate_env}{%
      style:border=solid 1px black,%
      style:width=var(--this-width),%
      style:min-width=var(--this-width),%
      style:--this-width=calc(var(--current-width) - 6px),%
      style:padding=3px,%
      style:margin-top=5px,%
      style:margin-bottom=5px%
    }
    \csname stex_annotate_invisible:n\endcsname{ }%
    \begin{stex_annotate_env}{%
      style:--current-width=var(--this-width);%
    }\csname stex_annotate_invisible:n\endcsname{ }
  \else\begin{mdframed}[#1]\fi
  %\begin{center}%
}{%
  %\end{center}%
  \ifstexhtml
    \end{stex_annotate_env}\end{stex_annotate_env}\egroup\par
  \else\end{mdframed}\fi
}
\newcommand{\scaled}[2][0.9\hsize]{\begin{center}\resizebox{#1}{!}{\begin{minipage}{\textwidth} #2 \end{minipage}}\end{center}}

\makeatletter
\ExplSyntaxOn

\def\doc_exbox:nnn#1#2#3{
  \begin{sexample}[#3]
    Input:
    \begin{framed}[linewidth=1pt,backgroundcolor=white]\small
      #1
    \end{framed}
    Output:
    \begin{framed}[linewidth=1pt,backgroundcolor=white]\small
      #2
    \end{framed}
  \end{sexample}
}


\NewDocumentCommand\stexexamplefile{O{} m O{} O{}}{
  \stex_resolve_path_pair:Nxx \l_@@_filepath_str {\tl_to_str:n{#1}} {\tl_to_str:n{#2}}
  \doc_exbox:nnn{
    \hfill File~\str_if_empty:nTF{#1}{
      \prop_if_exist:NT \l_stex_current_archive_prop {
        [\texttt{\prop_item:Nn \l_stex_current_archive_prop {id}}]
      }
    }{[#1]}\texttt{\tl_to_str:n{#2}}
    \_lststex_parse_args:n{#3}
    \exp_args:Nno \use:nn{\lstinputlisting[} \l_lststex_return_tl ]{\l_@@_filepath_str}
  }{
    \inputref[#1]{#2}
  }{#4}
}

\newwrite\testoutfile
\NewDocumentCommand\stexexample{O{} O{}}{
  \begingroup 
  \catcode`\\=12\relax
  \catcode`\#=12\relax
  \catcode`\&=12\relax
  \catcode`\$=12\relax
  \catcode`\^=12\relax
  \catcode`\_=12\relax
  \catcode`\ =12\relax
  \catcode`^^J=12\relax
  \endlinechar=`^^J
  \newlinechar=-1
%^^A    \everyeof{\noexpand}
  \example_a:nnn{#1}{#2}
}
\long\def\example_a:nnn #1 #2 #3 {
  \endgroup
  \immediate\openout\testoutfile=\jobname.exmpl
  \immediate\write\testoutfile{
    \c_backslash_str begin{stexcode}[#1]
    \detokenize{^^J}#3
    \c_backslash_str end{stexcode}
  }
  \immediate\closeout\testoutfile
  \doc_exbox:nnn{
    \catcode`\#=12\relax
    \csname @ @ input\endcsname{\jobname.exmpl}
  }{
    \immediate\openout\testoutfile=\jobname.exmpl
    \immediate\write\testoutfile{#3}
    \immediate\closeout\testoutfile
    \csname @ @ input\endcsname \jobname.exmpl\relax
  }{#2}
  \peek_charcode_remove:NT ^^J
}

\ExplSyntaxOff
\makeatother

\makeatletter
\newcount\example@counter\example@counter=0
\newtcolorbox{exampleborderbox}[1][]{
  empty,
  title={Example \the\example@counter #1},
  attach boxed title to top left,
     minipage boxed title,
  boxed title style={empty,size=minimal,toprule=0pt,top=1pt,left=3mm,overlay={}},
  coltitle=blue,fonttitle=\bfseries,
  parbox=false,boxsep=0pt,left=3mm,right=0mm,top=2pt,breakable,pad at break=0mm,
     before upper=\csname @totalleftmargin\endcsname0pt, 
  overlay unbroken={\draw[blue,line width=2pt] ([xshift=-0pt]title.north west) -- ([xshift=-0pt]frame.south west); },
  overlay first={\draw[blue,line width=2pt] ([xshift=-0pt]title.north west) -- ([xshift=-0pt]frame.south west); },
  overlay middle={\draw[blue,line width=2pt] ([xshift=-0pt]frame.north west) -- ([xshift=-0pt]frame.south west); },
  overlay last={\draw[blue,line width=2pt] ([xshift=-0pt]frame.north west) -- ([xshift=-0pt]frame.south west); },
  outer arc=4pt%
}

\ExplSyntaxOn
\stexstyleexample{
  \global\advance\example@counter by 1
  \tl_if_empty:NTF\thistitle{
    \begin{exampleborderbox}
  }{
    \begin{exampleborderbox}[ (\thistitle)]
  }
}{
    \end{exampleborderbox}
}

\ExplSyntaxOff\makeatother

\usetikzlibrary{calc}

\def\textwarning{\includegraphics[width=1.2em]{stex-dangerous-bend}\xspace}
\newtcolorbox{dangerbox}{
  breakable,
  enhanced,
  left=0pt,
  right=0pt,
  top=8pt,
  bottom=8pt,
  colback=white,
  colframe=red,
  width=\textwidth,
  enlarge left by=0mm,
  boxsep=5pt,
  fontupper=\small,
  arc=4pt,
  outer arc=4pt,
  leftupper=1.5cm,
  overlay={
    \node[anchor=west] at ([xshift=10pt]$(frame.north west)!0.5!(frame.south west)$)
       {\includegraphics[width=1cm,height=1cm]{stex-dangerous-bend}};}
}

\protected\def\TODO#1{\textcolor{red}{TODO}\footnote{\textcolor{red}{TODO: #1}}}

\definecolor{darkgreen}{rgb}{0.0, 0.5, 0.0}

\usepackage[solutions]{problem}
\usepackage{hwexam}
\newtcolorbox{problemborderbox}[1][]{
  empty,
  title={Exercise #1},
  attach boxed title to top left,
     minipage boxed title,
  boxed title style={empty,size=minimal,toprule=0pt,top=1pt,left=3mm,overlay={}},
  coltitle=darkgreen,fonttitle=\bfseries,
  parbox=false,boxsep=0pt,left=3mm,right=0mm,top=2pt,breakable,pad at break=0mm,
     before upper=\csname @totalleftmargin\endcsname0pt, 
  overlay unbroken={\draw[darkgreen,line width=2pt] ([xshift=-0pt]title.north west) -- ([xshift=-0pt]frame.south west); },
  overlay first={\draw[darkgreen,line width=2pt] ([xshift=-0pt]title.north west) -- ([xshift=-0pt]frame.south west); },
  overlay middle={\draw[darkgreen,line width=2pt] ([xshift=-0pt]frame.north west) -- ([xshift=-0pt]frame.south west); },
  overlay last={\draw[darkgreen,line width=2pt] ([xshift=-0pt]frame.north west) -- ([xshift=-0pt]frame.south west); },
  outer arc=4pt%
}

\ExplSyntaxOn
\stexstyleproblem{
  \tl_if_empty:NTF\thistitle{
    \begin{problemborderbox}
  }{
    \begin{problemborderbox}[ (\thistitle)]
  }
}{
    \end{problemborderbox}
}
\ExplSyntaxOff

\newtcolorbox{experimental}{
  breakable,
  enhanced,
  left=0pt,
  right=0pt,
  top=8pt,
  bottom=8pt,
  colback=white,
  colframe=gray,
  width=\textwidth,
  enlarge left by=0mm,
  boxsep=5pt,
  fontupper=\small,
  arc=4pt,
  outer arc=4pt,
  leftupper=1.5cm,
  overlay={
    \node[anchor=west] at ([xshift=10pt]$(frame.north west)!0.5!(frame.south west)$)
       {\includegraphics[height=1cm]{stex-experimental}};}
}


\usetikzlibrary{decorations.pathmorphing,shapes,arrows,calc}
% Taken from pgflibrarytikzmmt.code.tex
\newcommand{\mmtarrowtip}{angle 45}
\newcommand{\mmtarrowtipmonoright}{right hook}

\tikzstyle{include}=[\mmtarrowtipmonoright-\mmtarrowtip,thick]
\tikzstyle{morph}=[-\mmtarrowtip,thick]
\tikzstyle{preview}=[decorate, decoration={coil,aspect=0,amplitude=1pt,
                                                  segment length=6pt,
                                                  pre=lineto,pre length=3pt,
                                                  post=lineto,post length=5pt}, thick]
\tikzstyle{view}=[preview,-\mmtarrowtip]


% TIKZ RULES
\def\mmtlogo{
\begin{tikzpicture}

  % White Background (Margins are eyeballed)
  % This is necessary because we paste white over arrows later.
  % If somebody want's to do the full song and dance with
  % interrupted arrows to get transparent background, be my guest.

  \fill[white!] (-0.01,0.15) rectangle (1.11,-0.95);

  % Arrows
  \draw [blue, include] (0,0)     -- (1.1,0);
  \draw [green, morph] (0,-0.4)  -- (1.1,-0.4);
  \draw [red, view]   (-0,-0.8) -- (1.1,-0.8);

  % Cutout for letters
  \fill[white] (0.33,0.1) rectangle (0.66,-0.9);

  % Letters
  \node at (0.18,0)    (nodeM1) {\large M};
  \node at (0.18,-0.4) (nodeM2) {\large M};
  \node at (0.21,-0.8) (nodeT)  {\large T};

\end{tikzpicture}
}

\newtcolorbox{mmtbox}{
  breakable,
  enhanced,
  left=0pt,
  right=0pt,
  top=8pt,
  bottom=8pt,
  colback=white,
  colframe=green,
  width=\textwidth,
  enlarge left by=0mm,
  boxsep=5pt,
  fontupper=\small,
  arc=4pt,
  outer arc=4pt,
  leftupper=1.5cm,
  overlay={
    \node[anchor=west] at ([xshift=10pt]$(frame.north west)!0.5!(frame.south west)$)
       {\mmtlogo};}
}

\AtBeginDocument{\catcode`_=8}

\begin{document}
  \DocInput{\jobname.dtx}
\end{document}
%</driver>
% \fi
%
% \title{ \sTeX-Symbols
% 	\thanks{Version {\fileversion} (last revised {\filedate})} 
% }
%
% \author{Michael Kohlhase, Dennis Müller\\
% 	FAU Erlangen-Nürnberg\\
% 	\url{http://kwarc.info/}
% }
%
% \maketitle
%
%\ifinfulldoc\else
% This is the documentation for the \pkg{stex-symbols} package.
% For a more high-level introduction, 
%  see \href{\basedocurl/manual.pdf}{the \sTeX Manual} or the
% \href{\basedocurl/stex.pdf}{full \sTeX documentation}.
%
% \begin{smodule}[ns=https://github.com/slatex/sTeX/doc]{SymbolsAndNotations}
\begin{sfragment}{Declaring New Symbols and Notations}
  Inside an \stexcode"smodule" environment, we can declare new \sTeX symbols.

\begin{function}{\symdecl}
  The most basic command for doing so is using \stexcode"\symdecl{symbolname}". This
  introduces a new symbol with name |symbolname|, arity $0$ and semantic macro
  \stexcode"\symbolname".

  The starred variant \stexcode"\symdecl*{symbolname}" will declare a symbol, but not
  introduce a semantic macro.  If we don't want to supply a notation (for example to
  introduce concepts like ``abelian'', which is not something that has a notation), the
  starred variant is likely to be what we want.
\end{function}
\begin{mmtbox}
  \stexcode"\symdecl" introduces a new \omdoc/\mmt constant in the current module
  (=\omdoc/\mmt theory).  Correspondingly, they get assigned the URI
  |<module-URI>?<constant-name>|.
\end{mmtbox}

Without a semantic macro or a notation, the only meaningful way to reference a symbol is
via \stexcode"\symref",\stexcode"\symname" etc.

\stexexample{%
\symdecl*{foo}
Given a \symname{foo}, we can...
}

Obviously, most semantic macros should take actual \emph{arguments}, implying that the
symbol we introduce is an \emph{operator} or \emph{function}. We can let
\stexcode"\symdecl" know the \emph{arity} (i.e. number of arguments) of a symbol like
this:

\stexexample{%
\symdecl{binarysymbol}[args=2]
\symref{binarysymbol}{this} is a symbol taking two arguments.
}

So far we have gained exactly \ldots nothing by adding the arity information: we cannot do
anything with the arguments in the text.

We will now see what we can gain with more machinery.
    
\begin{function}{\notation}
  We probably want to supply a notation as well, in which case we can finally actually use
  the semantic macro in math mode.  We can do so using the \stexcode"\notation" command,
  like this:

\stexexample{%
\notation{binarysymbol}{\text{First: }#1\text{; Second: }#2}
$\binarysymbol{a}{b}$ }
\end{function}

\begin{mmtbox}
  Applications of semantic macros, such as \stexcode"\binarysymbol{a}{b}" are translated
  to \mmt/\omdoc as |OMA|-terms with head |<OMS name="...?binarysymbol"/>|.

  Semantic macros with no arguments correspond to |OMS| directly.
\end{mmtbox}

\begin{function}{\comp}
  For many semantic services e.g. semantic highlighting or \defemph{wikification} (linking
  user-visible notation components to the definition of the respective symbol they come
  from), we need to specify the notation components. Unfortunately, there is currently no
  way the \sTeX engine can infer this by itself, so we have to specify it manually in the
  notation specification.  We can do so with the \stexcode"\comp" command.
\end{function}

We can introduce a new notation |highlight| for \stexcode"\binarysymbol" that fixes this
flaw, which we can subsequently use with \stexcode"\binarysymbol[highlight]":

\stexexample{%
\notation{binarysymbol}[highlight]
    {\comp{\text{First: }}#1\comp{\text{; Second: }}#2}
$\binarysymbol[highlight]{a}{b}$
}

\begin{dangerbox}
  Ideally, \stexcode"\comp" would not be necessary: Everything in a notation that is
  \emph{not} an argument should be a notation component. Unfortunately, it is
  computationally expensive to determine where an argument begins and ends, and the
  argument markers |#n| may themselves be nested in other macro applications or
  \TeX\xspace groups, making it ultimately almost impossible to determine them
  automatically while also remaining compatible with arbitrary highlighting customizations
  (such as tooltips, hyperlinks, colors) that users might employ, and that are ultimately
  invoked by \stexcode"\comp".
\end{dangerbox}

\begin{dangerbox}
  Note that it is required that
  \begin{enumerate}
  \item the argument markers |#n| never occur inside a \stexcode"\comp", and
  \item no semantic arguments may ever occur inside a notation.
  \end{enumerate}
  Both criteria are not just required for technical reasons, but conceptionally
  meaningful:
        
  The underlying principle is that the arguments to a semantic macro represent
  \emph{arguments to the mathematical operation} represented by a symbol. For example, a
  semantic macro \stexcode"\addition{a}{b}" taking two arguments would represent \emph{the
    actual addition of (mathematical objects) $a$ and $b$}.  It should therefore be
  impossible for $a$ or $b$ to be part of a notation component of \stexcode"\addition".

  Similarly, a semantic macro can not conceptually be part of the notation of
  \stexcode"\addition", since a semantic macro represents a \emph{distinct mathematical
    concept} with \emph{its own semantics}, whereas notations are syntactic
  representations of the very symbol to which the notation belongs.

  If you want an argument to a semantic macro to be a purely syntactic parameter, then you
  are likely somewhat confused with respect to the distinction between the precise
  \emph{syntax} and \emph{semantics} of the symbol you are trying to declare (which
  happens quite often even to experienced \sTeX users), and might want to give those
  another thought - quite likely, the macro you aim to implement does not actually
  represent a semantically meaningful mathematical concept, and you will want to use
  \stexcode"\def" and similar native \LaTeX\xspace macro definitions rather than semantic
  macros.
\end{dangerbox}

\begin{function}{\symdef}
  In the vast majority of cases where a symbol declaration should come with a semantic
  macro, we will want to supply a notation immediately. For that reason, the
  \stexcode"\symdef" command combines the functionality of both \stexcode"\symdecl" and
  \stexcode"\notation" with the optional arguments of both:
\end{function}

\stexexample{%
\symdef{newbinarysymbol}[hl,args=2]
    {\comp{\text{1.: }}#1\comp{\text{; 2.: }}#2}
$\newbinarysymbol{a}{b}$
}

We just declared a new symbol |newbinarysymbol| with |args=2| and immediately provided it
with a notation with identifier |hl|. Since |hl| is the \emph{first} (and so far, only)
notation supplied for |newbinarysymbol|, using \stexcode"\newbinarysymbol" without
optional argument defaults to this notation.\bigskip

But one man's meat is another man's poison: it is very subjective what the ``default
notation'' of an operator should be. Different communities have different practices. For
instance, the complex unit is written as $i$ in Mathematics and as $j$ in electrical
engineering. So to allow modular specification and facilitate re-use of document fragments
\sTeX allows to re-set notation defaults.

\begin{function}{\setnotation}
  The first notation provided will stay the default notation unless explicitly changed --
  this is enabled by the \stexcode"\setnotation" command:
  \stexcode"\setnotation{symbolname}{notation-id}" sets the default notation of
  \stexcode"\symbolname" to |notation-id|, i.e. henceforth, \stexcode"\symbolname" behaves
  like \stexcode"\symbolname[notation-id]" from now on.
\end{function}

Often, a default notation is set right after the corresponding notation is introduced --
the starred version \stexcode"\notation*" for that reason introduces a new notation and
immediately sets it to be the new default notation. So expressed differently, the
\emph{first} \stexcode"\notation" for a symbol behaves exactly like \stexcode"\notation*",
and \stexcode"\notation*{foo}[bar]{...}" behaves exactly like
\stexcode"\notation{foo}[bar]{...}\setnotation{foo}{bar}".

\begin{function}{\textsymdecl}
  In the less mathematical settings where we want a symbol and
  semantic macro for some concept with a notation \emph{beyond}
  its mere name, but which should also be available in \TeX's text
  mode, the command \stexcode"\textsymdecl" is useful.
  For example, we can declare a symbol \stexcode"openmath"
  with the notation \stexcode"\textsc{OpenMath}" using
  \textsymdecl{openmath}[name=OpenMath]{\textsc{OpenMath}}
  \stexcode"\textsymdecl{openmath}[name=OpenMath]{\textsc{OpenMath}}".
  The \stexcode"\openmath" yields \openmath both in text and math
  mode.
\end{function}
    
\begin{sfragment}{Operator Notations}
  Once we have a semantic macro with arguments, such as \stexcode"\newbinarysymbol", the
  semantic macro represents the \emph{application} of the symbol to a list of
  arguments. What if we want to refer to the operator \emph{itself}, though?

  We can do so by supplying the \stexcode"\notation" (or \stexcode"\symdef") with an
  \emph{operator notation}, indicated with the optional argument |op=|.  We can then
  invoke the operator notation using \stexcode"\symbolname![notation-identifier]".  Since
  operator notations never take arguments, we do not need to use \stexcode"\comp" in it,
  the whole notation is wrapped in a \stexcode"\comp" automatically:

  \stexexample{%
    \notation{newbinarysymbol}[ab, op={\text{a:}\cdot\text{; b:}\cdot}]
    {\comp{\text{a:}}#1\comp{\text{; b:}}#2} \symname{newbinarysymbol} is also
    occasionally written $\newbinarysymbol![ab]$
  }

  \begin{mmtbox}
    \stexcode"\symbolname!" is translated to \omdoc/\mmt as |<OMS name="...?symbolname"/>|
    directly.
  \end{mmtbox}

\end{sfragment}
\end{sfragment}

\begin{sfragment}{Argument Modes}
  The notations so far used \emph{simple} arguments which we call \emph{mode}-|i|
  arguments. Declaring a new symbol with \stexcode"\symdecl{foo}[args=3]" is equivalent to
  writing \stexcode"\symdecl{foo}[args=iii]", indicating that the semantic macro takes
  three mode-|i| arguments. However, there are three more argument modes which we will
  investigate now, namely mode-|b|, mode-|a| and mode-|B| arguments.

\begin{sfragment}{Mode-\texttt b Arguments}

A mode-|b| argument represents a \emph{variable} that is \emph{bound} by the symbol in
its application, making the symbol a \emph{binding operator}. Typical examples of
binding operators are e.g. sums $\sum$, products $\prod$, integrals $\int$, quantifiers
like $\forall$ and $\exists$, that $\lambda$-operator, etc.

\begin{mmtbox}
  Mode-|b| arguments behave exactly like mode-|i| arguments within \TeX, but applications
  of binding operators, i.e. symbols with mode-|b| arguments, are translated to
  |OMBIND|-terms in \omdoc/\mmt, rather than |OMA|.
\end{mmtbox}

For example, we can implement a summation operator binding an index variable and taking
lower and upper index bounds and the expression to sum over like this:

\stexexample{%
\symdef{summation}[args=biii]
  {\mathop{\comp{\sum}}_{#1\comp{=}#2}^{#3}#4}
  $\summation{\svar{x}}{1}{\svar{n}}{\svar{x}}^2$
}

where the variable $\svar{x}$ is now \emph{bound} by the \stexcode"\summation"-symbol in
the expression.
\end{sfragment}
    
\begin{sfragment}{Mode-\texttt a Arguments}
  Mode-|a| arguments represent a \emph{flexary argument sequence}, i.e. a sequence of
  arguments of arbitrary length.  Formally, operators that take arbitrarily many arguments
  don't ``exist'', but in informal mathematics, they are ubiquitous.  Mode-|a| arguments
  allow us to write e.g.  \stexcode"\addition{a,b,c,d,e}" rather than having to write
  something like \stexcode"\addition{a}{\addition{b}{\addition{c}{\addition{d}{e}}}}"!

  \stexcode"\notation" (and consequently \stexcode"\symdef", too) take one additional
  argument for each mode-|a| argument that indicates how to ``accumulate'' a
  comma-separated sequence of arguments. This is best demonstrated on an example.

  Let's say we want an operator representing quantification over an ascending chain of
  elements in some set, i.e.  \stexcode"\ascendingchain{S}{a,b,c,d,e}{t}" should yield
  $\forall a{<_S}b{<_S}c{<_S}d{<_S}e.\,t$. The ``base''-notation for this operator is
  simply\\ \stexcode"{\comp{\forall} #2\comp{.\,}#3}", where |#2| represents the full
  notation fragment \emph{accumulated} from |{a,b,c,d,e}|.
        
  The \emph{additional} argument to \stexcode"\notation" (or \stexcode"\symdef") takes the
  same arguments as the base notation and two \emph{additional} arguments |##1| and |##2|
  representing successive pairs in the mode-|a| argument, and accumulates them into |#2|,
  i.e. to produce $a<_Sb<_Sc<_Sd<_Se$, we do \stexcode"{##1 \comp{<}_{#1} ##2}":

  \stexexample{%
\symdef{ascendingchain}[args=iai]
  {\comp{\forall} #2\comp{.\,}#3}
  {##1 \comp{<}_{#1} ##2}

Tadaa: $\ascendingchain{S}{a,b,c,d,e}{t}$
}

If this seems overkill, keep in mind that you will rarely need the single-hash arguments
|#1|,|#2| etc.  in the |a|-notation-argument.  For a much more representative and simpler
example, we can introduce flexary addition via:
\stexexample{%
  \symdef{addition}[args=a]{#1}{##1 \comp{+} ##2}
  
Tadaa: $\addition{a,b,c,d,e}$
}

\begin{sfragment}{The \texttt{assoc}-key}
  We mentioned earlier that ``formally'', flexary arguments don't really
  ``exist''. Indeed, formally, addition is usually defined as a binary operation,
  quantifiers bind a single variable etc.

  Consequently, we can tell \sTeX (or, rather, \mmt/\omdoc) how to ``resolve'' flexary
  arguments by providing \stexcode"\symdecl" or \stexcode"\symdef" with an optional
  |assoc|-argument, as in \stexcode"\symdecl{addition}[args=a,assoc=bin]".  The possible
  values for the |assoc|-key are:
  \begin{itemize}
  \item[|bin|:] A binary, associative argument, e.g.  as in \stexcode"\addition"
  \item[|binl|:] A binary, left-associative argument, e.g.
    $a^{\scriptstyle b^{\scriptstyle c^d}}$, which stands for $((a^b)^c)^d$
  \item[|binr|:] A binary, right-associative argument, e.g. as in $A\to B\to C\to D$,
    which stands for $A \to (B \to (C \to D))$
  \item[|pre|:] Successively prefixed, e.g. as in $\forall x,y,z.\,P$, which stands for
    $\forall x.\, \forall y.\, \forall z.\,P$
  \item[|conj|:] Conjunctive, e.g. as in $a=b=c=d$ or $a,b,c,d\in A$, which stand for
    $a=d\wedge b=d\wedge c=d$ and $a\in A\wedge b\in A \wedge c\in A\wedge d\in A$,
    respectively
  \item[|pwconj|:] Pairwise conjunctive, e.g. as in $a\neq b\neq c\neq d$, which stands
    for $a\neq b\wedge a\neq c\wedge a\neq d\wedge b\neq c\wedge b\neq d\wedge c\neq d$
  \end{itemize}
  As before, at the PDF level, this annotation is invisible (and without effect), but at
  the level of the generated OMDoc/MMT this leads to more semantical expressions.
\end{sfragment}
\end{sfragment}
    
\begin{sfragment}{Mode-\texttt B Arguments}
  Finally, mode-|B| arguments simply combine the functionality of both |a| and |b| -
  i.e. they represent an arbitrarily long sequence of variables to be bound, e.g. for
  implementing quantifiers:

  \stexexample{%
\symdef{quantforall}[args=Bi]
  {\comp{\forall}#1\comp{.}#2}
  {##1\comp,##2}

$\quantforall{\svar{x},\svar{y},\svar{z}}{P}$
}
\end{sfragment}
\end{sfragment}

\begin{sfragment}{Type and Definiens Components}
  \stexcode"\symdecl" and \stexcode"\symdef" take two more optional arguments. \TeX\xspace
  largely ignores them (except for special situations we will talk about later), but \mmt
  can pick up on them for additional services. These are the |type| and |def| keys, which
  expect expressions in math-mode (ideally using semantic macros, of course!)

  \begin{mmtbox}
    The |type| and |def| keys correspond to the |type| and |definiens| components of
    \omdoc/\mmt constants.

    Correspondingly, the name ``type'' should be taken with a grain of salt, since
    \omdoc/\mmt -- being foundation-independent -- does not a priori implement a fixed
    typing system.
  \end{mmtbox}

  \symdef{funtype}[args=ai]{#1 \comp\to #2}{##1 \comp\times ##2}
  \symdef{fun}[args=bi]{#1 \comp\mapsto #2}
  \symdef{set}{\comp{\texttt{Set}}}
  
  The |type|-key allows us to provide additional information
  (given the necessary \sTeX symbols), e.g. for
  addition on natural numbers:
  
  \stexexample{%
\symdef{Nat}[type=\set]{\comp{\mathbb N}}
\symdef{addition}[
    type=\funtype{\Nat,\Nat}{\Nat},
    op=+,
    args=a
]{#1}{##1 \comp+ ##2}

\symname{addition} is an operation $\funtype{\Nat,\Nat}{\Nat}$
}

The |def|-key allows for declaring symbols as abbreviations:
\stexexample{%
\symdef{successor}[
    type=\funtype{\Nat}{\Nat},
    def=\fun{\svar{x}}{\addition{\svar{x},1}},
    op=\mathtt{succ},
    args=1
]{\comp{\mathtt{succ(}#1\comp{)}}}

The \symname{successor} operation $\funtype{\Nat}{\Nat}$
is defined as $\fun{\svar{x}}{\addition{\svar{x},1}}$
}
\end{sfragment}

\begin{sfragment}{Precedences and Automated Bracketing}
  Having done \stexcode"\addition", the obvious next thing to implement is
  \stexcode"\multiplication".  This is straight-forward in theory:

  \stexexample{%
\symdef{multiplication}[
    type=\funtype{\Nat,\Nat}{\Nat},
    op=\cdot,
    args=a
]{#1}{##1 \comp\cdot ##2}

\symname{multiplication} is an operation $\funtype{\Nat,\Nat}{\Nat}$
}

However, if we \emph{combine} \stexcode"\addition" and \stexcode"\multiplication", we
notice a problem:

\stexexample{%
$\addition{a,\multiplication{b,\addition{c,\multiplication{d,e}}}}$
}

We all know that $\multiplication!$ binds stronger than $\addition!$, so the output
$\addition{a,\multiplication{b,\addition{c,\multiplication{d,e}}}}$ does not actually
reflect the term we wrote. We can of course insert parentheses manually

\stexexample{%
$\addition{a,\multiplication{b,(\addition{c,\multiplication{d,e}})}}$
}
but we can also do better by supplying \emph{precedences} and
have \sTeX insert parentheses automatically.

For that purpose, \stexcode"\notation" (and hence \stexcode"\symdef") take an optional
argument |prec=<opprec>;<argprec1>x...x<argprec n>|.

We will investigate the precise meaning of |<opprec>| and the |<argprec>|s shortly -- in
the vast majority of cases, it is perfectly sufficient to think of |prec=| taking a single
number and having that be \emph{the} precedence of the notation, where lower precedences
(somewhat counterintuitively) bind stronger than higher precedences.  So fixing our
notations for \stexcode"\addition" and \stexcode"\multiplication", we get:

\stexexample{%
\notation{multiplication}[
    op=\cdot,
    prec=50
]{#1}{##1 \comp\cdot ##2}
\notation{addition}[
    op=+,
    prec=100
]{#1}{##1 \comp+ ##2}

$\addition{a,\multiplication{b,\addition{c,\multiplication{d,e}}}}$
}

Note that the precise numbers used for precedences are pretty arbitrary - what matters is
which precedences are higher than which other precedences when used in conjunction.
\begin{variable}{\infprec,\neginfprec}
  It is occasionally useful to have ``infinitely'' high or low precedences to enforce or
  forbid automated bracketing entirely, e.g. for bracket-like notations such as intervals
  -- for those purposes, \stexcode"\infprec" and \stexcode"\neginfprec" exist (which are
  implemented as the maximal and minimal integer values accordingly).g
\end{variable}

\begin{dangerbox}
  More precisely, each notation takes
  \begin{enumerate}
  \item One \emph{operator precedence} and
  \item one \emph{argument precedence} for each argument.
  \end{enumerate}
  By default, all precedences are $0$, unless the symbol takes no argument, in which case
  the operator precedence is \stexcode"\neginfprec" (negative infinity). If we only
  provide a single number, this is taken as both the operator precedence and all argument
  precedences.

  \sTeX decides whether to insert parentheses by comparing operator precedences to a
  \emph{downward precedence} $p_d$ with initial value \stexcode"\infprec".  When
  encountering a semantic macro, \sTeX takes the operator precedence $p_{op}$ of the
  notation used and checks whether $p_{op}>p_d$. If so, \sTeX insert parentheses.

  When \sTeX steps into an argument of a semantic macro, it sets $p_d$ to the respective
  argument precedence of the notation used.

  In the example above:
  \begin{enumerate}
  \item \sTeX starts out with $p_d=$\stexcode"\infprec".
  \item \sTeX encounters \stexcode"\addition" with $p_{op}=100$. Since
    $100\not>$\stexcode"\infprec", it inserts no parentheses.
  \item Next, \sTeX encounters the two arguments for \stexcode"\addition".  Both have no
    specifically provided argument precedence, so \sTeX uses $p_d=p_{op}=100$ for both and
    recurses.
  \item Next, \sTeX encounters \stexcode"\multiplication{b,...}", whose notation has
    $p_{op}=50$.
  \item We compare to the current downward precedence $p_d$ set by \stexcode"\addition",
    arriving at $p_{op}=50\not>100=p_d$, so \sTeX again inserts no parentheses.
  \item Since the notation of \stexcode"\multiplication" has no explicitly set argument
    precedences, \sTeX uses the operator precedence for all arguments of
    \stexcode"\multiplication", hence sets $p_d=p_{op}=50$ and recurses.
  \item Next, \sTeX encounters the inner \stexcode"\addition{c,...}" whose notation has
    $p_{op}=100$.
  \item We compare to the current downward precedence $p_d$ set by
    \stexcode"\multiplication", arriving at $p_{op}=100>50=p_d$ -- which finally prompts
    \sTeX to insert parentheses, and we proceed as before.
  \end{enumerate}
\end{dangerbox}
\end{sfragment}

\begin{sfragment}{Variables}
  All symbol and notation declarations require a module with which they are associated,
  hence the commands \stexcode"\symdecl", \stexcode"\notation", \stexcode"\symdef"
  etc. are disabled outside of |smodule|-environments.

  Variables are different -- variables are allowed everywhere, are not exported when the
  current module (if one exists) is imported (via \stexcode"\importmodule" or
  \stexcode"\usemodule") and (also unlike symbol declarations) ``disappear'' at the end of
  the current \TeX\xspace group.

  \begin{function}{\svar}
    So far, we have always used variables using \stexcode"\svar{n}", which marks-up $n$ as
    a variable with name |n|. More generally, \stexcode"\svar[foo]{<texcode>}" marks-up
    the arbitrary |<texcode>| as representing a variable with name |foo|.
  \end{function}

  Of course, this makes it difficult to reuse variables, or introduce ``functional''
  variables with arities $>0$, or provide them with a type or definiens.

  \begin{function}{\vardef}
    For that, we can use the \stexcode"\vardef" command. Its syntax is largely the same as
    that of \stexcode"\symdef", but unlike symbols, variables have only one notation
    (\textcolor{red}{TODO: so far?}), hence there is only \stexcode"\vardef" and no
    \stexcode"\vardecl".
  \end{function}

\stexexample{%
\vardef{varf}[
    name=f,
    type=\funtype{\Nat}{\Nat},
    op=f,
    args=1,
    prec=0;\neginfprec
]{\comp{f}#1}
\vardef{varn}[name=n,type=\Nat]{\comp{n}}
\vardef{varx}[name=x,type=\Nat]{\comp{x}}

Given a function $\varf!:\funtype{\Nat}{\Nat}$, 
by $\addition{\varf!,\varn}$ we mean the function\rustexBREAK
$\fun{\varx}{\varf{\addition{\varx,\varn}}}$
}

(of course, ``lifting'' addition in the way described in the previous example is an
operation that deserves its own symbol rather than abusing \stexcode"\addition",
but... well.)

\textcolor{red}{TODO: bind=forall/exists}
\end{sfragment}

\begin{sfragment}{Variable Sequences}
  Variable \emph{sequences} occur quite frequently in informal mathematics, hence they
  deserve special support. Variable sequences behave like variables in that they disappear
  at the end of the current \TeX\xspace group and are not exported from modules, but their
  declaration is quite different.

  \begin{function}{\varseq}
    A variable sequence is introduced via the command \stexcode"\varseq", which takes the
    usual optional arguments |name| and |type|. It then takes a starting index, an end
    index and a \emph{notation} for the individual elements of the sequence parametric in
    an index. Note that both the starting as well as the ending index may be variables.
  \end{function}

  This is best shown by example:
  \stexexample{%
\vardef{varn}[name=n,type=\Nat]{\comp{n}}
\varseq{seqa}[name=a,type=\Nat]{1}{\varn}{\comp{a}_{#1}}

The $i$th index of $\seqa!$ is $\seqa{i}$.
}

Note that the syntax |\seqa!| now automatically generates a presentation based on the
starting and ending index.
        
\textcolor{red}{TODO: more notations for invoking sequences}.

\vardef{varn}[name=n,type=\Nat]{\comp{n}}
\varseq{seqa}[name=a]{1}{\varn}{\comp{a}_{#1}}

Notably, variable sequences are nicely compatible with |a|-type arguments, so we can do
the following:

\stexexample{%
$\addition{\seqa}$
}

Sequences can be \emph{multidimensional} using the |args|-key, in which case the
notation's arity increases and starting and ending indices have to be provided as a
comma-separated list:

\stexexample{%
\vardef{varm}[name=m,type=\Nat]{\comp{m}}
\varseq{seqa}[
    name=a,
    args=2,
    type=\Nat,
]{1,1}{\varn,\varm}{\comp{a}_{#1}^{#2}}

$\seqa!$ and $\addition{\seqa}$
}
\vardef{varm}[name=m,type=\Nat]{\comp{m}}

We can also explicitly provide a ``middle'' segment to be used, like such:

\stexexample{%
\varseq{seqa}[
    name=a,
    type=\Nat,
    args=2,
    mid={\comp{a}_{\varn}^1,\comp{a}_1^2,\ellipses,\comp{a}_{1}^{\varm}}
]{1,1}{\varn,\varm}{\comp{a}_{#1}^{#2}}

$\seqa!$ and $\addition{\seqa}$
}
\end{sfragment}
\end{smodule}

%%% Local Variables:
%%% mode: latex
%%% TeX-master: "../stex-manual"
%%% End:

%  LocalWords:  binarysymbol newbinarysymbol hl,args a,b,c,d,e ascendingchain assoc binl
%  LocalWords:  a,assoc binr x,y,z conj a,b,c,d pwconj funtype succ prec opprec argprec1
%  LocalWords:  argprec texcode varf varn n,type varx x,type varseq seqa a,type th m,type

% \fi
%
% \begin{documentation}\label{pkg:symbols:doc}
% \changes{3.1.0}{2022/03/01}{Fixed bug with precedences in variables}
% \changes{3.1.0}{2022/03/01}{Introduced \detokenize{\varemph, \varemph@uri}}
%
% Code related to symbol declarations and notations
%
% \section{Macros and Environments}\label{pkg:symbols:doc:macros}
%
% \begin{function}{\symdecl}
%   \begin{syntax} \cs{symdecl}\Arg{macroname}|[|\meta{args}|]| \end{syntax}
%   Declares a new symbol with semantic macro \cs{macroname}. Optional
%   arguments are:
%   \begin{itemize}
%     \item |name|: An (\omdoc) name. By default equal to \meta{macroname}.
%     \item |type|: An (ideally semantic) term. Not used by \sTeX, but
%         passed on to \mmt for semantic services.
%     \item |local|: A boolean (by default false). If set, this declaration
%         will not be added to the module content, i.e. importing
%         the current module will not make this declaration available.
%     \item |args|: Specifies the ``signature'' of the semantic macro.
%       Can be either an integer $0 \leq n \leq 9$, or a (more precise)
%       sequence of the following characters:
%         \begin{itemize}
%           \item[|i|] a ``normal'' argument, e.g.
%             |\symdecl{plus}[args=ii]| allows for |\plus{2}{2}|.
%           \item[|a|] an \emph{associative} argument; i.e. a sequence of
%             arbitrarily many arguments provided as a comma-separated list,
%             e.g.
%             |\symdecl{plus}[args=a]| allows for |\plus{2,2,2}|.
%           \item[|b|] a \emph{variable} argument. Is treated by \sTeX
%             like an |i|-argument, but an application is turned into
%             an |OMBind| in \omdoc, binding the provided variable
%             in the subsequent arguments of the operator; e.g.
%             |\symdecl{forall}[args=bi]| allows for |\forall{x\in\Nat}{x\geq0}|.
%         \end{itemize}
%   \end{itemize}
% \end{function}
%
% \begin{function}{\stex_symdecl_do:n}
%   Implements the core functionality of \cs{symdecl}, and is
%   called by \cs{symdecl} and \cs{symdef}.
%
%   Ultimately stores the symbol \meta{URI} in the property
%   list |\l_stex_symdecl_|\meta{URI}|_prop| with fields:
%   \begin{itemize}
%     \item |name| (string),
%     \item |module| (string),
%     \item |notations| (sequence of strings; initially empty),
%     \item |local| (boolean),
%     \item |type| (token list),
%     \item |args| (string of |i|s, |a|s and |b|s),
%     \item |arity| (integer string),
%     \item |assocs| (integer string; number of associative arguments),
%   \end{itemize}
% \end{function}
%
% \begin{function}{\stex_all_symbols:n}
%   Iterates over all currently available symbols.
%   Requires two |\seq_map_break:| to break fully.
% \end{function}
%
% \begin{function}{\stex_get_symbol:n}
%   Computes the full URI of a symbol from a macro argument, e.g.
%   the macro name, the macro itself, the full URI...
% \end{function}
%
% \begin{function}{\notation}
%   \begin{syntax} \cs{notation}|[|\meta{args}|]|\Arg{symbol}\Arg{notations$^+$} \end{syntax}
%   Introduces a new notation for \meta{symbol}, see \cs{stex_notation_do:nn}
% \end{function}
%
% \begin{function}{\stex_notation_do:nn}
%   \begin{syntax} \cs{stex_notation_do:nn}\Arg{URI}\Arg{notations$^+$}\end{syntax}
%
%   Implements the core functionality of \cs{notation}, and is
%   called by \cs{notation} and \cs{symdef}.
%
%   Ultimately stores the notation in the property
%   list\\ |\g_stex_notation_|\meta{URI}|#|\meta{variant}|#|^^A
%   \meta{lang}|_prop| with fields:
%   \begin{itemize}
%     \item |symbol| (URI string),
%     \item |language| (string),
%     \item |variant| (string),
%     \item |opprec| (integer string),
%     \item |argprecs| (sequence of integer strings)
%   \end{itemize}
% \end{function}
%
% \begin{function}{\symdef}
%   \begin{syntax} \cs{symdef}|[|\meta{args}|]|\Arg{symbol}\Arg{notations$^+$} \end{syntax}
%   Combines \cs{symdecl} and \cs{notation} by introducing a new
%   symbol and assigning a new notation for it.
% \end{function}
%
% \end{documentation}
%
% \begin{implementation}\label{pkg:symbols:impl}
%
% \section{\sTeX-Symbols Implementation}
%
%    \begin{macrocode}
%<*package>

%%%%%%%%%%%%%   symbols.dtx   %%%%%%%%%%%%%

%    \end{macrocode}
%
% Warnings and error messages
%
%    \begin{macrocode}
\msg_new:nnn{stex}{error/wrongargs}{
  args~value~in~symbol~declaration~for~#1~
  needs~to~be~i,~a,~b~or~B,~but~#2~given
}
\msg_new:nnn{stex}{error/unknownsymbol}{
  No~symbol~#1~found!
}
%    \end{macrocode}
%
% \subsection{Symbol Declarations}
%    \begin{macrocode}
%<@@=stex_symdecl>
%    \end{macrocode}
%
% \begin{macro}{\stex_all_symbols:n}
%   Map over all available symbols
%    \begin{macrocode}
\cs_new_protected:Nn \stex_all_symbols:n {
  \def \_@@_all_symbols_cs ##1 {#1}
  \seq_map_inline:Nn \l_stex_all_modules_seq {
    \seq_map_inline:cn{c_stex_module_##1_constants}{
      \_@@_all_symbols_cs{##1?####1}
    }
  } 
}
%    \end{macrocode}
% \end{macro}
%
% \begin{macro}{\STEXsymbol}
%    \begin{macrocode}
\NewDocumentCommand \STEXsymbol { m } {
  \stex_get_symbol:n { #1 }
  \exp_args:No
  \stex_invoke_symbol:n { \l_stex_get_symbol_uri_str }
}
%    \end{macrocode}
% \end{macro}
%
% |symdecl| arguments:
%
%    \begin{macrocode}
\keys_define:nn { stex / symdecl } {
  name        .str_set_x:N  = \l_stex_symdecl_name_str ,
  local       .bool_set:N   = \l_stex_symdecl_local_bool ,
  args        .str_set_x:N  = \l_stex_symdecl_args_str ,
  type        .tl_set:N     = \l_stex_symdecl_type_tl ,
  deprecate   .str_set_x:N  = \l_stex_symdecl_deprecate_str ,
  align       .str_set:N    = \l_stex_symdecl_align_str , % TODO(?)
  gfc         .str_set:N    = \l_stex_symdecl_gfc_str , % TODO(?)
  specializes .str_set:N    = \l_stex_symdecl_specializes_str , % TODO(?)
  def         .tl_set:N     = \l_stex_symdecl_definiens_tl ,
  assoc       .choices:nn   = 
      {bin,binl,binr,pre,conj,pwconj}
      {\str_set:Nx \l_stex_symdecl_assoctype_str {\l_keys_choice_tl}}
}

\bool_new:N \l_stex_symdecl_make_macro_bool

\cs_new_protected:Nn \_@@_args:n {
  \str_clear:N \l_stex_symdecl_name_str
  \str_clear:N \l_stex_symdecl_args_str
  \str_clear:N \l_stex_symdecl_deprecate_str
  \str_clear:N \l_stex_symdecl_assoctype_str
  \bool_set_false:N \l_stex_symdecl_local_bool
  \tl_clear:N \l_stex_symdecl_type_tl
  \tl_clear:N \l_stex_symdecl_definiens_tl
  
  \keys_set:nn { stex / symdecl } { #1 }
}
%    \end{macrocode}
%
% \begin{macro}{\symdecl}
%
% Parses the optional arguments and passes them on to
% \cs{stex_symdecl_do:} (so that \cs{symdef}
% can do the same)
%
%    \begin{macrocode}

\NewDocumentCommand \symdecl { s m O{}} {
  \_@@_args:n { #3 }
  \IfBooleanTF #1 {
    \bool_set_false:N \l_stex_symdecl_make_macro_bool
  } {
    \bool_set_true:N \l_stex_symdecl_make_macro_bool
  }
  \stex_symdecl_do:n { #2 }
  \stex_smsmode_do:
}

\cs_new_protected:Nn \stex_symdecl_do:nn {
  \_@@_args:n{#1}
  \bool_set_false:N \l_stex_symdecl_make_macro_bool
  \stex_symdecl_do:n{#2}
}

\stex_deactivate_macro:Nn \symdecl {module~environments}
%    \end{macrocode}
% \end{macro}
%
%
% \begin{macro}{\stex_symdecl_do:n}
%    \begin{macrocode}
\cs_new_protected:Nn \stex_symdecl_do:n {
  \stex_if_in_module:F {
    % TODO throw error? some default namespace?
  }
  
  \str_if_empty:NT \l_stex_symdecl_name_str {
    \str_set:Nx \l_stex_symdecl_name_str { #1 }
  }

  \prop_if_exist:cT { l_stex_symdecl_ 
      \l_stex_current_module_str ?
      \l_stex_symdecl_name_str
    _prop
  }{
    % TODO throw error (beware of circular dependencies)
  }

  \prop_clear:N \l_tmpa_prop
  \prop_put:Nnx \l_tmpa_prop { module } { \l_stex_current_module_str }
  \seq_clear:N \l_tmpa_seq
  \prop_put:Nno \l_tmpa_prop { name } \l_stex_symdecl_name_str
  \prop_put:Nno \l_tmpa_prop { type } \l_stex_symdecl_type_tl

  \str_if_empty:NT \l_stex_symdecl_deprecate_str {
    \str_if_empty:NF \l_stex_module_deprecate_str {
      \str_set_eq:NN \l_stex_symdecl_deprecate_str \l_stex_module_deprecate_str
    }
  }
  \prop_put:Nno \l_tmpa_prop { deprecate } \l_stex_symdecl_deprecate_str

  \exp_args:No \stex_add_constant_to_current_module:n {
    \l_stex_symdecl_name_str
  }

  % arity/args
  \int_zero:N \l_tmpb_int

  \bool_set_true:N \l_tmpa_bool
  \str_map_inline:Nn \l_stex_symdecl_args_str {
    \token_case_meaning:NnF ##1 {
      0 {} 1 {} 2 {} 3 {} 4 {} 5 {} 6 {} 7 {} 8 {} 9 {}
      {\tl_to_str:n i} { \bool_set_false:N \l_tmpa_bool }
      {\tl_to_str:n b} { \bool_set_false:N \l_tmpa_bool }
      {\tl_to_str:n a} { 
        \bool_set_false:N \l_tmpa_bool
        \int_incr:N \l_tmpb_int
      }
      {\tl_to_str:n B} { 
        \bool_set_false:N \l_tmpa_bool
        \int_incr:N \l_tmpb_int
      }
    }{
      \msg_error:nnxx{stex}{error/wrongargs}{
        \l_stex_current_module_str ?
        \l_stex_symdecl_name_str
      }{##1}
    }
  }
  \bool_if:NTF \l_tmpa_bool {
    % possibly numeric
    \str_if_empty:NTF \l_stex_symdecl_args_str {
      \prop_put:Nnn \l_tmpa_prop { args } {}
      \prop_put:Nnn \l_tmpa_prop { arity } { 0 }
    }{
      \int_set:Nn \l_tmpa_int { \l_stex_symdecl_args_str }
      \prop_put:Nnx \l_tmpa_prop { arity } { \int_use:N \l_tmpa_int }
      \str_clear:N \l_tmpa_str
      \int_step_inline:nn \l_tmpa_int {
        \str_put_right:Nn \l_tmpa_str i
      }
      \prop_put:Nnx \l_tmpa_prop { args } { \l_tmpa_str }
    }
  } {
    \prop_put:Nnx \l_tmpa_prop { args } { \l_stex_symdecl_args_str }
    \prop_put:Nnx \l_tmpa_prop { arity }
      { \str_count:N \l_stex_symdecl_args_str }
  }
  \prop_put:Nnx \l_tmpa_prop { assocs } { \int_use:N \l_tmpb_int }
  

  % semantic macro

  \bool_if:NT \l_stex_symdecl_make_macro_bool {
    \exp_args:Nx \stex_do_up_to_module:n {
      \tl_set:cn { #1 } { \stex_invoke_symbol:n {
        \l_stex_current_module_str ? \l_stex_symdecl_name_str
      }}
    }

    \bool_if:NF \l_stex_symdecl_local_bool {
      \exp_args:Nx \stex_add_to_current_module:n {
        \tl_set:cn { #1 } { \stex_invoke_symbol:n {
          \l_stex_current_module_str ? \l_stex_symdecl_name_str
        } }
      }
    }
  }

  \stex_debug:nn{symbols}{New~symbol:~
    \l_stex_current_module_str ? \l_stex_symdecl_name_str^^J
    Type:~\exp_not:o { \l_stex_symdecl_type_tl }^^J
    Args:~\prop_item:Nn \l_tmpa_prop { args }
  }

  % circular dependencies require this:

  \prop_if_exist:cF {
    l_stex_symdecl_ 
    \l_stex_current_module_str ? \l_stex_symdecl_name_str
    _prop 
  } {
    \exp_args:Nx \stex_do_up_to_module:n {
      \prop_set_from_keyval:cn {
        l_stex_symdecl_ 
        \l_stex_current_module_str ? \l_stex_symdecl_name_str
        _prop 
      } {\prop_to_keyval:N \l_tmpa_prop}
    }
  }

  \seq_clear:c {
    l_stex_symdecl_ 
    \l_stex_current_module_str ? \l_stex_symdecl_name_str
    _notations
  }

  \bool_if:NF \l_stex_symdecl_local_bool {
    \exp_args:Nx
    \stex_add_to_current_module:n {
      \seq_clear:c {
        l_stex_symdecl_ 
        \l_stex_current_module_str ? \l_stex_symdecl_name_str
        _notations
      }
      \prop_set_from_keyval:cn {
        l_stex_symdecl_ 
        \l_stex_current_module_str ? \l_stex_symdecl_name_str
        _prop 
      } {
        name      = \prop_item:Nn \l_tmpa_prop { name }       ,
        module    = \prop_item:Nn \l_tmpa_prop { module }     ,
        type      = \prop_item:Nn \l_tmpa_prop { type }       ,
        args      = \prop_item:Nn \l_tmpa_prop { args }       ,
        arity     = \prop_item:Nn \l_tmpa_prop { arity }      ,
        assocs    = \prop_item:Nn \l_tmpa_prop { assocs }
      }
    }
  }

  \stex_if_smsmode:F {
%    \exp_args:Nx \stex_do_up_to_module:n {
%        \seq_put_right:Nn \exp_not:N \l_stex_all_symbols_seq {
%        \l_stex_current_module_str ? \l_stex_symdecl_name_str
%      }
%    }
    \stex_if_do_html:T {
      \stex_annotate_invisible:nnn {symdecl} {
        \l_stex_current_module_str ? \l_stex_symdecl_name_str
      } {
        \tl_if_empty:NF \l_stex_symdecl_type_tl {\stex_annotate_invisible:nnn{type}{}{$\l_stex_symdecl_type_tl$}}
        \stex_annotate_invisible:nnn{args}{}{
          \prop_item:Nn \l_tmpa_prop { args }
        }
        \stex_annotate_invisible:nnn{macroname}{#1}{}
        \tl_if_empty:NF \l_stex_symdecl_definiens_tl {
          \stex_annotate_invisible:nnn{definiens}{}
            {$\l_stex_symdecl_definiens_tl$}
        }
        \str_if_empty:NF \l_stex_symdecl_assoctype_str {
          \stex_annotate_invisible:nnn{assoctype}{\l_stex_symdecl_assoctype_str}{}
        }
      }
    }
  }
}
%    \end{macrocode}
% \end{macro}
%
% \begin{macro}{\stex_get_symbol:n}
%
%    \begin{macrocode}
\str_new:N \l_stex_get_symbol_uri_str

\cs_new_protected:Nn \stex_get_symbol:n {
  \tl_if_head_eq_catcode:nNTF { #1 } \relax {
    \tl_set:Nn \l_tmpa_tl { #1 }
    \_@@_get_symbol_from_cs:
  }{
    % argument is a string
    % is it a command name?
    \cs_if_exist:cTF { #1 }{
      \cs_set_eq:Nc \l_tmpa_tl { #1 }
      \str_set:Nx \l_tmpa_str { \cs_argument_spec:N \l_tmpa_tl }
      \str_if_empty:NTF \l_tmpa_str {
        \exp_args:Nx \cs_if_eq:NNTF {
          \tl_head:N \l_tmpa_tl
        } \stex_invoke_symbol:n {
          \_@@_get_symbol_from_cs:
        }{
          \_@@_get_symbol_from_string:n { #1 }
        }
      } {
        \_@@_get_symbol_from_string:n { #1 }
      }
    }{
      % argument is not a command name
      \_@@_get_symbol_from_string:n { #1 }
      % \l_stex_all_symbols_seq
    }
  }
  \str_if_eq:eeF {
    \prop_item:cn {
      l_stex_symdecl_\l_stex_get_symbol_uri_str _prop
    }{ deprecate }
  }{}{
    \msg_warning:nnxx{stex}{warning/deprecated}{
      Symbol~\l_stex_get_symbol_uri_str
    }{
      \prop_item:cn {l_stex_symdecl_\l_stex_get_symbol_uri_str _prop}{ deprecate }
    }
  }
}

\cs_new_protected:Nn \_@@_get_symbol_from_string:n {
  \tl_set:Nn \l_tmpa_tl {
    \msg_error:nnn{stex}{error/unknownsymbol}{#1}
  }
  \str_set:Nn \l_tmpa_str { #1 }
  \int_set:Nn \l_tmpa_int { \str_count:N \l_tmpa_str }

  \stex_all_symbols:n {
    \str_if_eq:eeT { \l_tmpa_str }{ \str_range:nnn {##1}{-\l_tmpa_int}{-1}}{
      \seq_map_break:n{\seq_map_break:n{
        \tl_set:Nn \l_tmpa_tl {
          \str_set:Nn \l_stex_get_symbol_uri_str { ##1 }
        }
      }}
    }
  }

  \l_tmpa_tl
}

\cs_new_protected:Nn \_@@_get_symbol_from_cs: {
  \exp_args:NNx \tl_set:Nn \l_tmpa_tl 
    { \tl_tail:N \l_tmpa_tl }
  \tl_if_single:NTF \l_tmpa_tl {
    \exp_args:No \tl_if_head_is_group:nTF \l_tmpa_tl {
      \exp_after:wN \str_set:Nn \exp_after:wN
        \l_stex_get_symbol_uri_str \l_tmpa_tl
    }{
      % TODO
      % tail is not a single group
    }
  }{
    % TODO
    % tail is not a single group
  }
}
%    \end{macrocode}
% \end{macro}
%
% \subsection{Notations}
%    \begin{macrocode}
%<@@=stex_notation>
%    \end{macrocode}
%
% |notation| arguments:
%
%    \begin{macrocode}
\keys_define:nn { stex / notation } {
  lang    .tl_set_x:N  = \l_@@_lang_str ,
  variant .tl_set_x:N  = \l_@@_variant_str ,
  prec    .str_set_x:N = \l_@@_prec_str ,
  op      .tl_set:N    = \l_@@_op_tl ,
  primary .bool_set:N  = \l_@@_primary_bool ,
  primary .default:n   = {true} ,
  unknown .code:n      = \str_set:Nx 
      \l_@@_variant_str \l_keys_key_str
}

\cs_new_protected:Nn \_stex_notation_args:n {
  \str_clear:N \l_@@_lang_str
  \str_clear:N \l_@@_variant_str
  \str_clear:N \l_@@_prec_str
  \tl_clear:N \l_@@_op_tl
  \bool_set_false:N \l_@@_primary_bool
  
  \keys_set:nn { stex / notation } { #1 }
}
%    \end{macrocode}
%
%
%
% \begin{macro}{\notation}
%    \begin{macrocode}
\NewDocumentCommand \notation { s m O{}} {
  \_stex_notation_args:n { #3 }
  \tl_clear:N \l_stex_symdecl_definiens_tl
  \stex_get_symbol:n { #2 }
  \tl_set:Nn \l_stex_notation_after_do_tl {
    \_@@_final:
    \IfBooleanTF#1{
      \stex_setnotation:n {\l_stex_get_symbol_uri_str}
    }{}
    \stex_smsmode_do:
  }
  \stex_notation_do:nnnnn
    { \prop_item:cn {l_stex_symdecl_\l_stex_get_symbol_uri_str _prop } { args } }
    { \prop_item:cn { l_stex_symdecl_\l_stex_get_symbol_uri_str _prop } { arity } }
    { \l_@@_variant_str \c_hash_str \l_@@_lang_str }
    { \l_@@_prec_str}
}
\stex_deactivate_macro:Nn \notation {module~environments}
%    \end{macrocode}
% \end{macro}
%
% \begin{macro}{\stex_notation_do:nnnnn}
%    \begin{macrocode}
\seq_new:N \l_@@_precedences_seq
\tl_new:N \l_@@_opprec_tl
\int_new:N \l_@@_currarg_int
\tl_new:N \stex_symbol_after_invokation_tl

\cs_new_protected:Nn \stex_notation_do:nnnnn {
  \let\l_stex_current_symbol_str\relax
  \seq_clear:N \l_@@_precedences_seq
  \tl_clear:N \l_@@_opprec_tl
  \str_set:Nx \l_@@_args_str { #1 }
  \str_set:Nx \l_@@_arity_str { #2 }
  \str_set:Nx \l_@@_suffix_str { #3 }
  \str_set:Nx \l_@@_prec_str { #4 }

  % precedences
  \str_if_empty:NTF \l_@@_prec_str {
    \int_compare:nNnTF \l_@@_arity_str = 0 {
      \tl_set:No \l_@@_opprec_tl { \neginfprec }
    }{
      \tl_set:Nn \l_@@_opprec_tl { 0 }
    }
  } {
    \str_if_eq:onTF \l_@@_prec_str {nobrackets}{
      \tl_set:No \l_@@_opprec_tl { \neginfprec }
      \int_step_inline:nn { \l_@@_arity_str } {
        \exp_args:NNo
        \seq_put_right:Nn \l_@@_precedences_seq { \infprec }
      }
    }{
      \seq_set_split:NnV \l_tmpa_seq ; \l_@@_prec_str
      \seq_pop_left:NNTF \l_tmpa_seq \l_tmpa_str {
        \tl_set:No \l_@@_opprec_tl { \l_tmpa_str }
        \seq_pop_left:NNT \l_tmpa_seq \l_tmpa_str {
          \exp_args:NNNo \exp_args:NNno \seq_set_split:Nnn 
            \l_tmpa_seq {\tl_to_str:n{x} } { \l_tmpa_str }
          \seq_map_inline:Nn \l_tmpa_seq {
            \seq_put_right:Nn \l_tmpb_seq { ##1 }
          }
        }
      }{
        \int_compare:nNnTF \l_@@_arity_str = 0 {
          \tl_set:No \l_@@_opprec_tl { \infprec }
        }{
          \tl_set:No \l_@@_opprec_tl { 0 }
        }
      }
    }
  }

  \seq_set_eq:NN \l_tmpa_seq \l_@@_precedences_seq
  \int_step_inline:nn { \l_@@_arity_str } {
    \seq_pop_left:NNF \l_tmpa_seq \l_tmpb_str {
      \exp_args:NNo
      \seq_put_right:No \l_@@_precedences_seq { 
        \l_@@_opprec_tl
      }
    }
  }
  \tl_clear:N \l_stex_notation_dummyargs_tl

  \int_compare:nNnTF \l_@@_arity_str = 0 {
    \exp_args:NNe
    \cs_set:Npn \l_stex_notation_macrocode_cs {
      \_stex_term_math_oms:nnnn { \l_stex_current_symbol_str } 
        { \l_@@_suffix_str }
        { \l_@@_opprec_tl } 
        { \exp_not:n { #5 } }
    }
    \l_stex_notation_after_do_tl
  }{
    \str_if_in:NnTF \l_@@_args_str b {
      \exp_args:Nne \use:nn
      {
      \cs_generate_from_arg_count:NNnn \l_stex_notation_macrocode_cs
      \cs_set:Npn \l_@@_arity_str } { {
        \_stex_term_math_omb:nnnn { \l_stex_current_symbol_str } 
          { \l_@@_suffix_str }
          { \l_@@_opprec_tl } 
          { \exp_not:n { #5 } }
      }}
    }{
      \str_if_in:NnTF \l_@@_args_str B {
        \exp_args:Nne \use:nn
        {
        \cs_generate_from_arg_count:NNnn \l_stex_notation_macrocode_cs
        \cs_set:Npn \l_@@_arity_str } { {
          \_stex_term_math_omb:nnnn { \l_stex_current_symbol_str } 
            { \l_@@_suffix_str }
            { \l_@@_opprec_tl } 
            { \exp_not:n { #5 } }
        } }
      }{
        \exp_args:Nne \use:nn
        {
        \cs_generate_from_arg_count:NNnn \l_stex_notation_macrocode_cs
        \cs_set:Npn \l_@@_arity_str } { {
          \_stex_term_math_oma:nnnn { \l_stex_current_symbol_str } 
            { \l_@@_suffix_str }
            { \l_@@_opprec_tl } 
            { \exp_not:n { #5 } }
        } }
      }
    }

    \str_set_eq:NN \l_@@_remaining_args_str \l_@@_args_str
    \int_zero:N \l_@@_currarg_int
    \seq_set_eq:NN \l_@@_remaining_precs_seq \l_@@_precedences_seq
    \_@@_arguments:
  }
}
%    \end{macrocode}
% \end{macro}
%
% \begin{macro}{\_@@_arguments:}
%
% Takes care of annotating the arguments in a
% notation macro
%
%    \begin{macrocode}
\cs_new_protected:Nn \_@@_arguments: {
  \int_incr:N \l_@@_currarg_int
  \str_if_empty:NTF \l_@@_remaining_args_str {
    \l_stex_notation_after_do_tl
  }{
    \str_set:Nx \l_tmpa_str { \str_head:N \l_@@_remaining_args_str }
    \str_set:Nx \l_@@_remaining_args_str { \str_tail:N \l_@@_remaining_args_str }
    \str_if_eq:VnTF \l_tmpa_str a {
      \_@@_argument_assoc:n
    }{
      \str_if_eq:VnTF \l_tmpa_str B {
        \_@@_argument_assoc:n
      }{
        \seq_pop_left:NN \l_@@_remaining_precs_seq \l_tmpa_str
        \tl_put_right:Nx \l_stex_notation_dummyargs_tl {
          { \_stex_term_math_arg:nnn
            { \int_use:N \l_@@_currarg_int }
            { \l_tmpa_str }
            { ####\int_use:N \l_@@_currarg_int }
          }
        }
        \_@@_arguments:
      }
    }
  }
}
%    \end{macrocode}
% \end{macro}
%
% \begin{macro}{\_@@_argument_assoc:n}
%    \begin{macrocode}
\cs_new_protected:Nn \_@@_argument_assoc:n {

  \cs_generate_from_arg_count:NNnn \l_tmpa_cs \cs_set:Npn 
    {\l_@@_arity_str}{
    #1
  }
  \int_zero:N \l_tmpa_int
  \tl_clear:N \l_tmpa_tl
  \str_map_inline:Nn \l_@@_args_str {
    \int_incr:N \l_tmpa_int
    \tl_put_right:Nx \l_tmpa_tl {
      \str_if_eq:nnTF {##1}{a}{ {} }{
        \str_if_eq:nnTF {##1}{B}{ {} }{
          {\_stex_term_arg:nn{\int_use:N \l_tmpa_int}{################ \int_use:N \l_tmpa_int}}
        }
      }
    }
  }
  \exp_after:wN\exp_after:wN\exp_after:wN \def 
  \exp_after:wN\exp_after:wN\exp_after:wN \l_tmpa_cs 
  \exp_after:wN\exp_after:wN\exp_after:wN ## 
  \exp_after:wN\exp_after:wN\exp_after:wN 1 
  \exp_after:wN\exp_after:wN\exp_after:wN ## 
  \exp_after:wN\exp_after:wN\exp_after:wN 2 
  \exp_after:wN\exp_after:wN\exp_after:wN {
    \exp_after:wN \exp_after:wN \exp_after:wN 
    \exp_not:n \exp_after:wN \exp_after:wN \exp_after:wN {
      \exp_after:wN \l_tmpa_cs \l_tmpa_tl
    }
  }

  \seq_pop_left:NN \l_@@_remaining_precs_seq \l_tmpa_str
  \tl_put_right:Nx \l_stex_notation_dummyargs_tl { {
    \_stex_term_math_assoc_arg:nnnn
      { \int_use:N \l_@@_currarg_int }
      { \l_tmpa_str }
      { ####\int_use:N \l_@@_currarg_int }
      { \l_tmpa_cs {####1} {####2} }
  } }
  \_@@_arguments:
}
%    \end{macrocode}
% \end{macro}
%
% \begin{macro}{\_@@_final:}
%
% Called after processing all notation arguments
%
%    \begin{macrocode}
\cs_new_protected:Nn \_@@_final: {
  \exp_args:Nne \use:nn
  {
  \cs_generate_from_arg_count:cNnn {
      stex_notation_ \l_stex_get_symbol_uri_str \c_hash_str 
      \l_@@_suffix_str
      _cs
    }
    \cs_set:Npn \l_@@_arity_str } { {
      \exp_after:wN \exp_after:wN \exp_after:wN
      \exp_not:n \exp_after:wN \exp_after:wN \exp_after:wN 
      { \exp_after:wN \l_stex_notation_macrocode_cs \l_stex_notation_dummyargs_tl \stex_symbol_after_invokation_tl}
  } }

  \tl_if_empty:NF \l_@@_op_tl {
    \cs_set:cpx {
      stex_op_notation_ \l_stex_get_symbol_uri_str \c_hash_str
      \l_@@_suffix_str
      _cs
    } { \exp_not:N \comp{ \exp_args:No \exp_not:n { \l_@@_op_tl } } }
  }

  \exp_args:Ne
  \stex_add_to_current_module:n {
    \cs_generate_from_arg_count:cNnn {
      stex_notation_ \l_stex_get_symbol_uri_str \c_hash_str 
      \l_@@_suffix_str
      _cs
    } \cs_set:Npn {\l_@@_arity_str} {
        \exp_after:wN \exp_after:wN \exp_after:wN
        \exp_not:n \exp_after:wN \exp_after:wN \exp_after:wN 
        { \exp_after:wN \l_stex_notation_macrocode_cs \l_stex_notation_dummyargs_tl \stex_symbol_after_invokation_tl}
    }
    \tl_if_empty:NF \l_@@_op_tl {
      \cs_set:cpn {
        stex_op_notation_\l_stex_get_symbol_uri_str \c_hash_str
        \l_@@_suffix_str
        _cs
      } { \exp_not:N \comp{ \exp_args:No \exp_not:n { \l_@@_op_tl } } }
    }
  }
  %\exp_args:Nx
 % \stex_do_up_to_module:n {
    \seq_put_right:cx {
      l_stex_symdecl_ \l_stex_get_symbol_uri_str
      _notations
    } {
      \l_@@_suffix_str
    }
 % }

  \stex_debug:nn{symbols}{
    Notation~\l_@@_suffix_str
    ~for~\l_stex_get_symbol_uri_str^^J
    Operator~precedence:~\l_@@_opprec_tl^^J
    Argument~precedences:~
      \seq_use:Nn \l_@@_precedences_seq {,~}^^J
    Notation: \cs_meaning:c {
      stex_notation_ \l_stex_get_symbol_uri_str \c_hash_str 
      \l_@@_suffix_str
      _cs
    }
  }
  
  \exp_args:Ne
  \stex_add_to_current_module:n {
    \seq_put_right:cn {
      l_stex_symdecl_\l_stex_get_symbol_uri_str
      _notations
    } { \l_@@_suffix_str }
  }

  \stex_if_smsmode:F {

    % HTML annotations
    \stex_if_do_html:T {
      \stex_annotate_invisible:nnn { notation }
      { \l_stex_get_symbol_uri_str } {
        \stex_annotate_invisible:nnn { notationfragment }
          { \l_@@_suffix_str }{}
        \stex_annotate_invisible:nnn { precedence }
          { \l_@@_prec_str }{}

        \int_zero:N \l_tmpa_int
        \str_set_eq:NN \l_@@_remaining_args_str \l_@@_args_str
        \tl_clear:N \l_tmpa_tl
        \int_step_inline:nn { \l_@@_arity_str }{
          \int_incr:N \l_tmpa_int
          \str_set:Nx \l_tmpb_str { \str_head:N \l_@@_remaining_args_str }
          \str_set:Nx \l_@@_remaining_args_str { \str_tail:N \l_@@_remaining_args_str }
          \str_if_eq:VnTF \l_tmpb_str a {
            \tl_set:Nx \l_tmpa_tl { \l_tmpa_tl { 
              \c_hash_str \c_hash_str \int_use:N \l_tmpa_int a ,
              \c_hash_str \c_hash_str \int_use:N \l_tmpa_int b
            } }
          }{
            \str_if_eq:VnTF \l_tmpb_str B {
              \tl_set:Nx \l_tmpa_tl { \l_tmpa_tl { 
                \c_hash_str \c_hash_str \int_use:N \l_tmpa_int a ,
                \c_hash_str \c_hash_str \int_use:N \l_tmpa_int b
              } }
            }{
              \tl_set:Nx \l_tmpa_tl { \l_tmpa_tl { 
                \c_hash_str \c_hash_str \int_use:N \l_tmpa_int
              } }
            }
          }
        }
        \stex_annotate_invisible:nnn { notationcomp }{}{
          \str_set:Nx \l_stex_current_symbol_str {\l_stex_get_symbol_uri_str }
          $ \exp_args:Nno \use:nn { \use:c {
            stex_notation_ \l_stex_current_symbol_str
            \c_hash_str \l_@@_suffix_str _cs
          } } { \l_tmpa_tl } $
        }
      }
    }
  }
}
%    \end{macrocode}
% \end{macro}
%
% \begin{macro}{\setnotation}
%    \begin{macrocode}
\keys_define:nn { stex / setnotation } {
  lang    .tl_set_x:N  = \l_@@_lang_str ,
  variant .tl_set_x:N  = \l_@@_variant_str ,
  unknown .code:n      = \str_set:Nx 
      \l_@@_variant_str \l_keys_key_str
}

\cs_new_protected:Nn \_stex_setnotation_args:n {
  \str_clear:N \l_@@_lang_str
  \str_clear:N \l_@@_variant_str
  \keys_set:nn { stex / setnotation } { #1 }
}

\cs_new_protected:Nn \stex_setnotation:n {
  \exp_args:Nnx \seq_if_in:cnTF { l_stex_symdecl_#1 _notations }
    { \l_@@_variant_str \c_hash_str \l_@@_lang_str }{
      \exp_args:Nnx \seq_remove_all:cn { l_stex_symdecl_#1 _notations }
        { \l_@@_variant_str \c_hash_str \l_@@_lang_str }
      \exp_args:Nnx \seq_remove_all:cn { l_stex_symdecl_#1 _notations }
        { \c_hash_str }
      \exp_args:Nnx \seq_put_left:cn { l_stex_symdecl_#1 _notations }
        { \l_@@_variant_str \c_hash_str \l_@@_lang_str }
      \exp_args:Nx \stex_add_to_current_module:n {
        \exp_args:Nnx \seq_remove_all:cn { l_stex_symdecl_#1 _notations }
          { \l_@@_variant_str \c_hash_str \l_@@_lang_str }
        \exp_args:Nnx \seq_put_left:cn { l_stex_symdecl_#1 _notations }
          { \l_@@_variant_str \c_hash_str \l_@@_lang_str }
        \exp_args:Nnx \seq_remove_all:cn { l_stex_symdecl_#1 _notations }
          { \c_hash_str }
      }
      \stex_debug:nn {notations}{
        Setting~default~notation~
        {\l_@@_variant_str \c_hash_str \l_@@_lang_str}~for~
        #1 \\
        \expandafter\meaning\csname
        l_stex_symdecl_#1 _notations\endcsname
      }
    }{
      % todo throw error
    }
}

\NewDocumentCommand \setnotation {m m} {
  \stex_get_symbol:n { #1 }
  \_stex_setnotation_args:n { #2 }
  \stex_setnotation:n{\l_stex_get_symbol_uri_str}
  \stex_smsmode_do:
}

\cs_new_protected:Nn \stex_copy_notations:nn {
  \stex_debug:nn {notations}{
    Copying~notations~from~#2~to~#1\\
    \seq_use:cn{l_stex_symdecl_#2_notations}{,~}
  }
  \tl_clear:N \l_tmpa_tl
  \int_step_inline:nn { \prop_item:cn {l_stex_symdecl_#2_prop}{ arity } } {
    \tl_put_right:Nn \l_tmpa_tl { {## ##1} }
  }
  \seq_map_inline:cn {l_stex_symdecl_#2_notations}{
    \cs_set_eq:Nc \l_tmpa_cs { stex_notation_ #2 \c_hash_str ##1 _cs }
    \edef \l_tmpa_tl {
      \exp_after:wN\exp_after:wN\exp_after:wN \exp_not:n 
      \exp_after:wN\exp_after:wN\exp_after:wN {
        \exp_after:wN \l_tmpa_cs \l_tmpa_tl
      }
    }
    \exp_args:Nx
    \stex_do_up_to_module:n {
      \seq_put_right:cn{l_stex_symdecl_#1_notations}{##1}
      \cs_generate_from_arg_count:cNnn {
        stex_notation_ #1 \c_hash_str ##1 _cs
      } \cs_set:Npn { \prop_item:cn {l_stex_symdecl_#2_prop}{ arity } }{
        \exp_after:wN\exp_not:n\exp_after:wN{\l_tmpa_tl}
      }
    }
  }
}

\NewDocumentCommand \copynotation {m m} {
  \stex_get_symbol:n { #1 }
  \str_set_eq:NN \l_tmpa_str \l_stex_get_symbol_uri_str
  \stex_get_symbol:n { #2 }
  \exp_args:Noo
  \stex_copy_notations:nn \l_tmpa_str \l_stex_get_symbol_uri_str
  \exp_args:Nx \stex_add_import_to_current_module:n{
    \stex_copy_notations:nn {\l_tmpa_str} {\l_stex_get_symbol_uri_str}
  }
  \stex_smsmode_do:
}

%    \end{macrocode}
% \end{macro}
%
% \begin{macro}{\symdef}
%    \begin{macrocode}
\keys_define:nn { stex / symdef } {
  name    .str_set_x:N = \l_stex_symdecl_name_str ,
  local   .bool_set:N  = \l_stex_symdecl_local_bool ,
  args    .str_set_x:N = \l_stex_symdecl_args_str ,
  type    .tl_set:N    = \l_stex_symdecl_type_tl ,
  def     .tl_set:N    = \l_stex_symdecl_definiens_tl ,
  op      .tl_set:N    = \l_@@_op_tl ,
  lang    .str_set_x:N = \l_@@_lang_str ,
  variant .str_set_x:N = \l_@@_variant_str ,
  prec    .str_set_x:N = \l_@@_prec_str ,
  assoc   .choices:nn  = 
      {bin,binl,binr,pre,conj,pwconj}
      {\str_set:Nx \l_stex_symdecl_assoctype_str {\l_keys_choice_tl}},
  unknown .code:n      = \str_set:Nx 
      \l_@@_variant_str \l_keys_key_str
}

\cs_new_protected:Nn \_@@_symdef_args:n {
  \str_clear:N \l_stex_symdecl_name_str
  \str_clear:N \l_stex_symdecl_args_str
  \str_clear:N \l_stex_symdecl_assoctype_str
  \bool_set_false:N \l_stex_symdecl_local_bool
  \tl_clear:N \l_stex_symdecl_type_tl
  \tl_clear:N \l_stex_symdecl_definiens_tl
  \str_clear:N \l_@@_lang_str
  \str_clear:N \l_@@_variant_str
  \str_clear:N \l_@@_prec_str
  \tl_clear:N \l_@@_op_tl
  
  \keys_set:nn { stex / symdef } { #1 }
}

\NewDocumentCommand \symdef { m O{} } {
  \_@@_symdef_args:n { #2 }
  \bool_set_true:N \l_stex_symdecl_make_macro_bool
  \stex_symdecl_do:n { #1 }
  \tl_set:Nn \l_stex_notation_after_do_tl {
    \_@@_final:
    \stex_smsmode_do:
  }
  \str_set:Nx \l_stex_get_symbol_uri_str {
    \l_stex_current_module_str ? \l_stex_symdecl_name_str
  }
  \exp_args:Nx \stex_notation_do:nnnnn
    { \prop_item:cn {l_stex_symdecl_\l_stex_get_symbol_uri_str _prop } { args } }
    { \prop_item:cn { l_stex_symdecl_\l_stex_get_symbol_uri_str _prop } { arity } }
    { \l_@@_variant_str \c_hash_str \l_@@_lang_str }
    { \l_@@_prec_str}
}
\stex_deactivate_macro:Nn \symdef {module~environments}
%    \end{macrocode}
% \end{macro}
%
% \subsection{Variables}
%
%    \begin{macrocode}
%<@@=stex_variables>

\keys_define:nn { stex / vardef } {
  name    .str_set_x:N  = \l_@@_name_str ,
  args    .str_set_x:N  = \l_@@_args_str ,
  type    .tl_set:N     = \l_@@_type_tl ,
  def     .tl_set:N     = \l_@@_def_tl ,
  op      .tl_set:N     = \l_@@_op_tl ,
  prec    .str_set_x:N  = \l_@@_prec_str ,
  assoc   .choices:nn   = 
      {bin,binl,binr,pre,conj,pwconj}
      {\str_set:Nx \l_@@_assoctype_str {\l_keys_choice_tl}},
  bind    .choices:nn   =
      {forall,exists}
      {\str_set:Nx \l_@@_bind_str {\l_keys_choice_tl}}
}

\cs_new_protected:Nn \_@@_args:n {
  \str_clear:N \l_@@_name_str
  \str_clear:N \l_@@_args_str
  \str_clear:N \l_@@_prec_str
  \str_clear:N \l_@@_assoctype_str
  \str_clear:N \l_@@_bind_str
  \tl_clear:N \l_@@_type_tl
  \tl_clear:N \l_@@_def_tl
  \tl_clear:N \l_@@_op_tl

  \keys_set:nn { stex / vardef } { #1 }
}

\NewDocumentCommand \_@@_do_simple:nnn { m O{}} {
  \_@@_args:n {#2}
  \str_if_empty:NT \l_@@_name_str {
    \str_set:Nx \l_@@_name_str { #1 }
  }
  \prop_clear:N \l_tmpa_prop
  \prop_put:Nno \l_tmpa_prop { name } \l_@@_name_str
  
  \int_zero:N \l_tmpb_int
  \bool_set_true:N \l_tmpa_bool
  \str_map_inline:Nn \l_@@_args_str {
    \token_case_meaning:NnF ##1 {
      0 {} 1 {} 2 {} 3 {} 4 {} 5 {} 6 {} 7 {} 8 {} 9 {}
      {\tl_to_str:n i} { \bool_set_false:N \l_tmpa_bool }
      {\tl_to_str:n b} { \bool_set_false:N \l_tmpa_bool }
      {\tl_to_str:n a} { 
        \bool_set_false:N \l_tmpa_bool
        \int_incr:N \l_tmpb_int
      }
      {\tl_to_str:n B} { 
        \bool_set_false:N \l_tmpa_bool
        \int_incr:N \l_tmpb_int
      }
    }{
      \msg_error:nnxx{stex}{error/wrongargs}{
        variable~\l_@@_name_str
      }{##1}
    }
  }
  \bool_if:NTF \l_tmpa_bool {
    % possibly numeric
    \str_if_empty:NTF \l_@@_args_str {
      \prop_put:Nnn \l_tmpa_prop { args } {}
      \prop_put:Nnn \l_tmpa_prop { arity } { 0 }
    }{
      \int_set:Nn \l_tmpa_int { \l_@@_args_str }
      \prop_put:Nnx \l_tmpa_prop { arity } { \int_use:N \l_tmpa_int }
      \str_clear:N \l_tmpa_str
      \int_step_inline:nn \l_tmpa_int {
        \str_put_right:Nn \l_tmpa_str i
      }
      \str_set_eq:NN \l_@@_args_str \l_tmpa_str
      \prop_put:Nnx \l_tmpa_prop { args } { \l_@@_args_str }
    }
  } {
    \prop_put:Nnx \l_tmpa_prop { args } { \l_@@_args_str }
    \prop_put:Nnx \l_tmpa_prop { arity }
      { \str_count:N \l_@@_args_str }
  }
  \prop_put:Nnx \l_tmpa_prop { assocs } { \int_use:N \l_tmpb_int }
  \tl_set:cx { #1 }{ \stex_invoke_variable:n { \l_@@_name_str } }

  \prop_set_eq:cN { l_stex_variable_\l_@@_name_str _prop} \l_tmpa_prop

  \tl_if_empty:NF \l_@@_op_tl {
    \cs_set:cpx {
      stex_var_op_notation_ \l_@@_name_str _cs
    } { \exp_not:N\comp{ \exp_args:No \exp_not:n { \l_@@_op_tl } } }
  }

  \tl_set:Nn \l_stex_notation_after_do_tl {
    \exp_args:Nne \use:nn {
      \cs_generate_from_arg_count:cNnn { stex_var_notation_\l_@@_name_str _cs }
        \cs_set:Npn { \prop_item:Nn \l_tmpa_prop { arity } }
    } {{
      \exp_after:wN \exp_after:wN \exp_after:wN
      \exp_not:n \exp_after:wN \exp_after:wN \exp_after:wN 
      { \exp_after:wN \l_stex_notation_macrocode_cs \l_stex_notation_dummyargs_tl \stex_symbol_after_invokation_tl}
    }}
    \stex_if_do_html:T {
      \stex_annotate_invisible:nnn {vardecl}{\l_@@_name_str}{
        \stex_annotate_invisible:nnn { precedence }
          { \l_@@_prec_str }{}
        \tl_if_empty:NF \l_@@_type_tl {\stex_annotate_invisible:nnn{type}{}{$\l_@@_type_tl$}}
        \stex_annotate_invisible:nnn{args}{}{ \l_@@_args_str }
        \stex_annotate_invisible:nnn{macroname}{#1}{}
        \tl_if_empty:NF \l_@@_def_tl {
          \stex_annotate_invisible:nnn{definiens}{}
            {$\l_@@_def_tl$}
        }
        \str_if_empty:NF \l_@@_assoctype_str {
          \stex_annotate_invisible:nnn{assoctype}{\l_@@_assoctype_str}{}
        }
        \int_zero:N \l_tmpa_int
        \str_set_eq:NN \l_@@_remaining_args_str \l_@@_args_str
        \tl_clear:N \l_tmpa_tl
        \int_step_inline:nn { \prop_item:Nn \l_tmpa_prop { arity } }{
          \int_incr:N \l_tmpa_int
          \str_set:Nx \l_tmpb_str { \str_head:N \l_@@_remaining_args_str }
          \str_set:Nx \l_@@_remaining_args_str { \str_tail:N \l_@@_remaining_args_str }
          \str_if_eq:VnTF \l_tmpb_str a {
            \tl_set:Nx \l_tmpa_tl { \l_tmpa_tl { 
              \c_hash_str \c_hash_str \int_use:N \l_tmpa_int a ,
              \c_hash_str \c_hash_str \int_use:N \l_tmpa_int b
            } }
          }{
            \str_if_eq:VnTF \l_tmpb_str B {
              \tl_set:Nx \l_tmpa_tl { \l_tmpa_tl { 
                \c_hash_str \c_hash_str \int_use:N \l_tmpa_int a ,
                \c_hash_str \c_hash_str \int_use:N \l_tmpa_int b
              } }
            }{
              \tl_set:Nx \l_tmpa_tl { \l_tmpa_tl { 
                \c_hash_str \c_hash_str \int_use:N \l_tmpa_int
              } }
            }
          }
        }
        \stex_annotate_invisible:nnn { notationcomp }{}{
          \str_set:Nx \l_stex_current_symbol_str {var://\l_@@_name_str }
          $ \exp_args:Nno \use:nn { \use:c {
            stex_var_notation_\l_@@_name_str _cs
          } } { \l_tmpa_tl } $
        }
      }
    }
  }

  \stex_notation_do:nnnnn { \l_@@_args_str } { \prop_item:Nn \l_tmpa_prop { arity } } {}{ \l_@@_prec_str}
}

\cs_new:Nn \_@@_reset:N {
  \tl_if_exist:NTF #1 {
    \def \exp_not:N #1 { \exp_args:No \exp_not:n #1 }
  }{
    \let \exp_not:N #1 \exp_not:N \undefined
  }
}

\NewDocumentCommand \_@@_do_complex:nn { m m }{
  \clist_set:Nx \l_@@_names { \tl_to_str:n {#1} }
  \exp_args:Nnx \use:nn {
    % TODO
    \stex_annotate_invisible:nnn {vardecls}{\clist_use:Nn\l_@@_names,}{
      #2
    }
  }{
    \_@@_reset:N \varnot
    \_@@_reset:N \vartype
    \_@@_reset:N \vardefi
  }
}

\NewDocumentCommand \vardef { s } {
  \IfBooleanTF#1 {
    \_@@_do_complex:nn
  }{
    \_@@_do_simple:nnn
  }
}

\NewDocumentCommand \svar { O{} m }{
  \tl_if_empty:nTF {#1}{
    \str_set:Nn \l_tmpa_str { #2 }
  }{
    \str_set:Nn \l_tmpa_str { #1 }
  }
  \_stex_term_omv:nn {
        var://\l_tmpa_str
    }{ \comp{ #2 } }
}

%    \end{macrocode}
%
%    \begin{macrocode}
%</package>
%    \end{macrocode}
%
% \end{implementation}
%
% \PrintIndex

% \endinput
% Local Variables:
% mode: doctex
% TeX-master: t
% End:

  \end{sfragment}

  \begin{sfragment}{Module Inheritance and Structures}

    \begin{sfragment}{Multilinguality and Translations}

      If we load the \sTeX document class or package with the option
      |lang=<lang>|, \sTeX will load the appropriate \pkg{babel}
      language for you -- e.g. |lang=de| will load the babel
      language |ngerman|. Additionally, it makes \sTeX aware
      of the current document being set in (in this example)
      \emph{german}. This matters for reasons other than mere
      \pkg{babel}-purposes, though:

      Every \emph{module} is assigned a language. If no \sTeX
      package option is set that allows for inferring a language,
      \sTeX will check whether the current file name ends in
      e.g. |.en.tex| (or |.de.tex| or |.fr.tex|, or...) and
      set the language accordingly. Alternatively, a language
      can be explicitly assigned via 
      \stexcode"\begin{smodule}[lang=<language>]{Foo}".
      \iffalse\end{smodule}\fi

      \begin{mmtbox}
        Technically, each |smodule|-environment induces \emph{two}
        \omdoc/\mmt theories:
        \stexcode"\begin{smodule}[lang=<lang>]{Foo}"
        \iffalse\end{smodule}\fi
        generates a theory |some/namespace?Foo| that only contains
        the ``formal'' part of the module -- i.e. exactly the
        content that is exported when using \stexcode"\importmodule".

        Additionally, \mmt generates a \emph{language theory} 
        |some/namespace/Foo?<lang>| that includes |some/namespace?Foo|
        and contains all the other document content -- variable
        declarations, includes for each \stexcode"\usemodule", etc.
      \end{mmtbox}

      Notably, the language suffix in a filename is ignored
      for \stexcode"\usemodule", \stexcode"\importmodule"
      and in generating/computing URIs for modules. This however
      allows for providing \emph{translations} for modules
      between languages without needing to duplicate content:

      If a module |Foo| exists in e.g. english in a file |Foo.en.tex|,
      we can provide a file |Foo.de.tex| right next to it, and write
      \stexcode"\begin{smodule}[sig=en]{Foo}".
      \iffalse\end{smodule}\fi
      The |sig|-key then signifies, that the ``signature'' of the
      module is contained in the \emph{english} version of the module,
      which is immediately imported from there, just like
      \stexcode"\importmodule" would.

      Additionally to translating the informal content of a module
      file to different languages, it also allows for customizing
      notations between languages. For example,
      the \emph{least common multiple} of two numbers is often
      denoted as $\mathtt{lcm}(a,b)$ in english, but is
      called \emph{kleinstes gemeinsames Vielfaches} in german
      and consequently denoted as $\mathtt{kgV}(a,b)$ there.

      We can therefore imagine a german version of an lcm-module
      looking something like this:

      \begin{latexcode}[gobble=8]
        \begin{smodule}[sig=en]{lcm}
          \notation*{lcm}[de]{\comp{\mathtt{kgV}}(#1,#2)}

          Das \symref{lcm}{kleinste gemeinsame Vielfache}
          $\lcm{a,b}$ von zwei Zahlen $a,b$ ist... 
        \end{smodule}
      \end{latexcode}

      If we now do \stexcode"\importmodule{lcm}"
      (or \stexcode"\usemodule{lcm}") within a \emph{german} document,
      it will also load the content of the german translation,
      including the |de|-notation for \stexcode"\lcm".

    \end{sfragment}

    \textcolor{red}{TODO: inheritance documentation}
    \begin{sfragment}{The \texttt{mathstructure} Environment}
\begin{smodule}[ns=https://github.com/slatex/sTeX/doc]{MathStructures}
    A common occurence in mathematics is bundling several
    interrelated ``declarations'' together into \emph{structures}.
    For example:
    \begin{itemize}
        \item A \emph{monoid} is a structure $\mathstruct{M,\circ,e}$
            with $\circ:M\times M\to M$ and $e\in M$ such that...
        \item A \emph{topological space} is a structure
            $\mathstruct{X,\mathcal T}$ where $X$ is a set and
            $\mathcal T$ is a topology on $X$
        \item A \emph{partial order} is a structure $\mathstruct{S,\leq}$
            where $\leq$ is a binary relation on $S$ such that...
    \end{itemize}

    This phenomenon is important and common enough to warrant special
    support, in particular because it requires being able
    to \emph{instantiate} such structures (or, ratherer,
    structure \emph{signatures}) in order to talk about (concrete
    or variable) \emph{particular} monoids, topological spaces,
    partial orders etc.

    \begin{environment}{mathstructure}
        The \stexcode"mathstructure" environment allows us to do
        exactly that. It behaves exactly like the
        \stexcode"smodule" environment, but is itself only allowed
        inside an \stexcode"smodule" environment, and allows
        for instantiation later on.
    \end{environment}

    How this works is again best demonstrated by example:
        \symdef{funtype}[args=ai]{#1 \comp\to #2}{##1 \comp\times ##2}
        \symdef{fun}[args=bi]{#1 \comp\mapsto #2}
        \symdef{set}{\comp{\texttt{Set}}}

        \stexexample{
\begin{mathstructure}{monoid}
    \symdef{universe}[type=\set]{\comp{U}}
    \symdef{op}[
        args=2,
        type=\funtype{\universe,\universe}{\universe},
        op=\circ
    ]{#1 \comp{\circ} #2}
    \symdef{unit}[type=\universe]{\comp{e}}
\end{mathstructure}

A \symname{monoid} is...
        }
        Note that the \stexcode"\symname{monoid}" is appropriately
        highlighted and (depending on your pdf viewer)
        shows a URI on hovering -- implying that the \stexcode"mathstructure"
        environment has generated a \emph{symbol} |monoid| for us.
        It has not generated a semantic macro though, since
        we can not use the |monoid|-symbol \emph{directly}. Instead,
        we can instantiate it, for example for integers:

        \stexexample{
\symdef{Int}[type=\set]{\comp{\mathbb Z}}
\symdef{addition}[
    type=\funtype{\Int,\Int}{\Int},
    args=2,
    op=+
]{##1 \comp{+} ##2}
\symdef{zero}[type=\Int]{\comp{0}}

$\mathstruct{\Int,\addition!,\zero}$ is a \symname{monoid}.
        }

        So far, we have not actually instantiated |monoid|, but now
        that we have all the symbols to do so, we can:

        \stexexample{
\instantiate{intmonoid}{
    universe = Int ,
    op = addition ,
    unit = zero
}{monoid}{\mathbb{Z}_{+,0}}

    $\intmonoid{universe}$, $\intmonoid{unit}$ and $\intmonoid{op}{a}{b}$.

    Also: $\intmonoid!$
        }
        \begin{function}{\instantiate}
            So summarizing:
            \stexcode"\instantiate" takes four arguments: The 
            (macro-)name of the instance, a key-value pair assigning
            declarations in the corresponding \stexcode"mathstructure"
            to symbols currently in scope, the name of the \stexcode"mathstructure"
            to instantiate, and lastly a notation for the instance itself.

            It then generates a semantic macro that takes as argument
            the name of a declaration in the instantiated \stexcode"mathstructure"
            and resolves it to the corresponding instance of that particular declaration.
        \end{function}

        \begin{mmtbox}
            \stexcode"\instantiate" and \stexcode"mathstructure"
            make use of the \emph{Theories-as-Types} paradigm:

            \stexcode"mathstructure{<name>}" does in fact simply create
            a nested theory with name |<name>-structure|. The \emph{constant}
            |<name>| is defined as |Mod(<name>-structure)| -- 
            a \emph{dependent record type with manifest fields}, the fields
            of which are generated from (and correspond to) the constants
            in |<name>-structure|.

            \stexcode"\instantiate" appropriately generates a constant
            whose definiens is a record term of type |Mod(<name>-structure)|,
            with the fields assigned appropriately based on the key-value-list.
        \end{mmtbox}

        Notably, \stexcode"\instantiate" throws an error if not \emph{every}
        declaration in the instantiated \stexcode"mathstructure" is being assigned.
        
        You might consequently ask what the usefulness of \stexcode"mathstructure"
        even is.

        \begin{function}{\varinstantiate}
            The answer is that we can also instantiate a 
            \stexcode"mathstructure" with a \emph{variable}.
            The syntax of \stexcode"\varianstantiate" is equivalent
            to that of \stexcode"\instantiate", but all of the key-value-pairs
            are optional, and if not explicitly assigned (to a symbol \emph{or}
            a variable declared with \stexcode"\vardef") inherit their notation
            from the one in the \stexcode"mathstructure" environment.
        \end{function}

        This allows us to do things like:

        \stexexample{
\varinstantiate{varM}{}{monoid}{M}

A \symname{monoid} is a structure 
$\varM!:=\mathstruct{\varM{universe},\varM{op}!,\varM{unit}}$
such that 
$\varM{op}!:\funtype{\varM{universe},\varM{universe}}{\varM{universe}}$
 and...

 \varinstantiate{varMb}{universe = Int}{monoid}{M_2}

 \noindent Let $\varMb!:=\mathstruct{\varMb{universe},\varMb{op}!,\varMb{unit}}$
 a \symname{monoid} on $\Int$...
        }

        We will return to this example later, when we also know
        how to handle the \emph{axioms} of a monoid.
\end{smodule}
\end{sfragment}

\begin{sfragment}{The \texttt{copymodule} Environment}

    \textcolor{red}{TODO: explain}

Given modules:

\stexexample{
\begin{smodule}{magma}
    \symdef{universe}{\comp{\mathcal U}}
    \symdef{operation}[args=2,op=\circ]{#1 \comp\circ #2}
\end{smodule}
\begin{smodule}{monoid}
    \importmodule{magma}
    \symdef{unit}{\comp e}
\end{smodule}
\begin{smodule}{group}
    \importmodule{monoid}
    \symdef{inverse}[args=1]{{#1}^{\comp{-1}}}
\end{smodule}
}

We can form a module for \emph{rings} by ``cloning''
an instance of |group| (for addition) and |monoid| (for multiplication),
respectively, and ``glueing them together'' to ensure they share the
same universe:

\stexexample{
\begin{smodule}{ring}
    \begin{copymodule}{group}{addition}
        \renamedecl[name=universe]{universe}{runiverse}
        \renamedecl[name=plus]{operation}{rplus}
        \renamedecl[name=zero]{unit}{rzero}
        \renamedecl[name=uminus]{inverse}{ruminus}
    \end{copymodule}
    \notation*{rplus}[plus,op=+,prec=60]{#1 \comp+ #2}
    %\setnotation{rplus}{plus}
    \notation*{rzero}[zero]{\comp0}
    %\setnotation{rzero}{zero}
    \notation*{ruminus}[uminus,op=-]{\comp- #1}
    %\setnotation{ruminus}{uminus}
    \begin{copymodule}{monoid}{multiplication}
        \assign{universe}{\runiverse}
        \renamedecl[name=times]{operation}{rtimes}
        \renamedecl[name=one]{unit}{rone}
    \end{copymodule}
    \notation*{rtimes}[cdot,op=\cdot,prec=50]{#1 \comp\cdot #2}
    %\setnotation{rtimes}{cdot}
    \notation*{rone}[one]{\comp1}
    %\setnotation{rone}{one}
    Test: $\rtimes a{\rplus c{\rtimes de}}$
\end{smodule}
}

\textcolor{red}{TODO: explain donotclone}
    
\end{sfragment}

\begin{sfragment}{The \texttt{interpretmodule} Environment}

    \textcolor{red}{TODO: explain}

\stexexample{
\begin{smodule}{int}
    \symdef{Integers}{\comp{\mathbb Z}}
    \symdef{plus}[args=2,op=+]{#1 \comp+ #2}
    \symdef{zero}{\comp0}
    \symdef{uminus}[args=1,op=-]{\comp-#1}

    \begin{interpretmodule}{group}{intisgroup}
        \assign{universe}{\Integers}
        \assign{operation}{\plus!}
        \assign{unit}{\zero}
        \assign{inverse}{\uminus!}
    \end{interpretmodule}
\end{smodule}
}
    
\end{sfragment}
  \end{sfragment}

  \begin{sfragment}{Primitive Symbols (The \sTeX Metatheory)}
    % \iffalse meta-comment
% An Infrastructure for Semantic Macros and Module Scoping
% Copyright (c) 2019 Michael Kohlhase, all rights reserved
%                this file is released under the
%                LaTeX Project Public License (LPPL)
% 
% The original of this file is in the public repository at 
% http://github.com/sLaTeX/sTeX/
%
% TODO update copyright  
%
%<*driver>
\def\bibfolder#1{../../lib/bib/#1}
\RequirePackage{paralist}
\ifcsname stexdocpath\endcsname\else\def\stexdocpath{.}\fi
\documentclass[full]{l3doc}
%\RequirePackage{document-structure}
\usepackage[hyperref=auto,style=alphabetic]{biblatex}
%\usepackage[mathhub=\stexdocpath/mh,usedeps]{stex}
\usepackage[lang={en,de}]{stex}

\usepackage{rustex}
\usepackage{stex-highlighting,stexthm}

\srefsetin[sTeX/Documentation]{documentation}{the \stex Documentation}

\makeatletter
\providecommand{\HTML}{\textsc{html}\xspace}%
\providecommand{\XML}{\textsc{xml}\xspace}%
\providecommand{\PDF}{\textsc{pdf}\xspace}%
\providecommand\openmath{\textsc{OpenMath}\xspace}
\providecommand\OMDoc{\textsc{OMDoc}\xspace}
\DeclareRobustCommand\LaTeXML{L\kern-.36em%
        {\sbox\z@ T%
         \vbox to\ht\z@{\hbox{\check@mathfonts
                              \fontsize\sf@size\z@
                              \math@fontsfalse\selectfont
                              A}%
                        \vss}%
        }%
        \kern-.15em%
%        T\kern-.1667em\lower.5ex\hbox{E}\kern-.125em\relax
%        {\tt XML}}
        T\kern-.1667em\lower.4ex\hbox{E}\kern-0.05em\relax
        {\scshape xml}\xspace}%
\def\mmt{\textsc{Mmt}\xspace}
\makeatother


\newif\ifhadtitle\hadtitlefalse

\def\stexversion{3.3.0}
\def\changedate{\today}
\def\stextoptitle#1#2{\title{#1\thanks{Version {\stexversion} (last revised {\changedate})} }\def\thispkg{#2}}

\author{Michael Kohlhase, Dennis Müller\\
 	FAU Erlangen-Nürnberg\\
 	\url{http://kwarc.info/}
}

\def\stexmaketitle{\ifhadtitle\else\hadtitletrue\maketitle\fi}

\ExplSyntaxOn

  \def\docmodule{
    \begin{document}
      \EnableManual
      \EnableDocumentation
      \EnableImplementation
      \stexmaketitle
      \tableofcontents
      \int_gincr:N \l_stex_docheader_sect
      \exp_args:Ne \__stex_mathhub_find_manifest:n {\stex_file_use:N \c_stex_mathhub_file / sTeX / Documentation}
      \str_if_empty:NF \l__stex_mathhub_manifest_str {
        \usemodule[sTeX/Documentation]{macros?AllMacros}
      }
      \DocInput{\jobname.dtx}
      \clearpage
      \PrintIndex
      \printbibliography
    \end{document}
  }

  \bool_new:N \g_stexdoc_typeset_manual_bool
  \NewDocumentCommand \EnableManual {}{
    \bool_gset_true:N \g_stexdoc_typeset_manual_bool
  }
  \NewDocumentCommand \DisableManual {}{
    \bool_gset_false:N \g_stexdoc_typeset_manual_bool
  }
  \NewDocumentEnvironment {stexmanual} {} {
    \bool_if:NTF \g_stexdoc_typeset_manual_bool
      {\bool_set_false:N \l__codedoc_in_implementation_bool}
      {\comment}
  }{
    \bool_if:NF \g_stexdoc_typeset_manual_bool {\endcomment}
  }
\ExplSyntaxOff

%\usepackage{makeidx}
%\makeindex

%\usepackage{document-structure}


\usepackage{lststex,mdframed}
\usepackage[most]{tcolorbox}

\lstset{literate=%
    {Ö}{{\"O}}1
    {Ä}{{\"A}}1
    {Ü}{{\"U}}1
    {ß}{{\ss}}1
    {ü}{{\"u}}1
    {ä}{{\"a}}1
    {ö}{{\"o}}1
    {~}{{\textasciitilde}}1
}

\newenvironment{framed}[1][]{
  \ifstexhtml\par\vbox\bgroup
    \csname exp_args:Nne\endcsname\begin{stex_annotate_env}{%
      style:border=solid 1px black,%
      style:width=var(--this-width),%
      style:min-width=var(--this-width),%
      style:--this-width=calc(var(--current-width) - 6px),%
      style:padding=3px,%
      style:margin-top=5px,%
      style:margin-bottom=5px%
    }
    \csname stex_annotate_invisible:n\endcsname{ }%
    \begin{stex_annotate_env}{%
      style:--current-width=var(--this-width);%
    }\csname stex_annotate_invisible:n\endcsname{ }
  \else\begin{mdframed}[#1]\fi
  %\begin{center}%
}{%
  %\end{center}%
  \ifstexhtml
    \end{stex_annotate_env}\end{stex_annotate_env}\egroup\par
  \else\end{mdframed}\fi
}
\newcommand{\scaled}[2][0.9\hsize]{\begin{center}\resizebox{#1}{!}{\begin{minipage}{\textwidth} #2 \end{minipage}}\end{center}}

\makeatletter
\ExplSyntaxOn

\def\doc_exbox:nnn#1#2#3{
  \begin{sexample}[#3]
    Input:
    \begin{framed}[linewidth=1pt,backgroundcolor=white]\small
      #1
    \end{framed}
    Output:
    \begin{framed}[linewidth=1pt,backgroundcolor=white]\small
      #2
    \end{framed}
  \end{sexample}
}


\NewDocumentCommand\stexexamplefile{O{} m O{} O{}}{
  \stex_resolve_path_pair:Nxx \l_@@_filepath_str {\tl_to_str:n{#1}} {\tl_to_str:n{#2}}
  \doc_exbox:nnn{
    \hfill File~\str_if_empty:nTF{#1}{
      \prop_if_exist:NT \l_stex_current_archive_prop {
        [\texttt{\prop_item:Nn \l_stex_current_archive_prop {id}}]
      }
    }{[#1]}\texttt{\tl_to_str:n{#2}}
    \_lststex_parse_args:n{#3}
    \exp_args:Nno \use:nn{\lstinputlisting[} \l_lststex_return_tl ]{\l_@@_filepath_str}
  }{
    \inputref[#1]{#2}
  }{#4}
}

\newwrite\testoutfile
\NewDocumentCommand\stexexample{O{} O{}}{
  \begingroup 
  \catcode`\\=12\relax
  \catcode`\#=12\relax
  \catcode`\&=12\relax
  \catcode`\$=12\relax
  \catcode`\^=12\relax
  \catcode`\_=12\relax
  \catcode`\ =12\relax
  \catcode`^^J=12\relax
  \endlinechar=`^^J
  \newlinechar=-1
%^^A    \everyeof{\noexpand}
  \example_a:nnn{#1}{#2}
}
\long\def\example_a:nnn #1 #2 #3 {
  \endgroup
  \immediate\openout\testoutfile=\jobname.exmpl
  \immediate\write\testoutfile{
    \c_backslash_str begin{stexcode}[#1]
    \detokenize{^^J}#3
    \c_backslash_str end{stexcode}
  }
  \immediate\closeout\testoutfile
  \doc_exbox:nnn{
    \catcode`\#=12\relax
    \csname @ @ input\endcsname{\jobname.exmpl}
  }{
    \immediate\openout\testoutfile=\jobname.exmpl
    \immediate\write\testoutfile{#3}
    \immediate\closeout\testoutfile
    \csname @ @ input\endcsname \jobname.exmpl\relax
  }{#2}
  \peek_charcode_remove:NT ^^J
}

\ExplSyntaxOff
\makeatother

\makeatletter
\newcount\example@counter\example@counter=0
\newtcolorbox{exampleborderbox}[1][]{
  empty,
  title={Example \the\example@counter #1},
  attach boxed title to top left,
     minipage boxed title,
  boxed title style={empty,size=minimal,toprule=0pt,top=1pt,left=3mm,overlay={}},
  coltitle=blue,fonttitle=\bfseries,
  parbox=false,boxsep=0pt,left=3mm,right=0mm,top=2pt,breakable,pad at break=0mm,
     before upper=\csname @totalleftmargin\endcsname0pt, 
  overlay unbroken={\draw[blue,line width=2pt] ([xshift=-0pt]title.north west) -- ([xshift=-0pt]frame.south west); },
  overlay first={\draw[blue,line width=2pt] ([xshift=-0pt]title.north west) -- ([xshift=-0pt]frame.south west); },
  overlay middle={\draw[blue,line width=2pt] ([xshift=-0pt]frame.north west) -- ([xshift=-0pt]frame.south west); },
  overlay last={\draw[blue,line width=2pt] ([xshift=-0pt]frame.north west) -- ([xshift=-0pt]frame.south west); },
  outer arc=4pt%
}

\ExplSyntaxOn
\stexstyleexample{
  \global\advance\example@counter by 1
  \tl_if_empty:NTF\thistitle{
    \begin{exampleborderbox}
  }{
    \begin{exampleborderbox}[ (\thistitle)]
  }
}{
    \end{exampleborderbox}
}

\ExplSyntaxOff\makeatother

\usetikzlibrary{calc}

\def\textwarning{\includegraphics[width=1.2em]{stex-dangerous-bend}\xspace}
\newtcolorbox{dangerbox}{
  breakable,
  enhanced,
  left=0pt,
  right=0pt,
  top=8pt,
  bottom=8pt,
  colback=white,
  colframe=red,
  width=\textwidth,
  enlarge left by=0mm,
  boxsep=5pt,
  fontupper=\small,
  arc=4pt,
  outer arc=4pt,
  leftupper=1.5cm,
  overlay={
    \node[anchor=west] at ([xshift=10pt]$(frame.north west)!0.5!(frame.south west)$)
       {\includegraphics[width=1cm,height=1cm]{stex-dangerous-bend}};}
}

\protected\def\TODO#1{\textcolor{red}{TODO}\footnote{\textcolor{red}{TODO: #1}}}

\definecolor{darkgreen}{rgb}{0.0, 0.5, 0.0}

\usepackage[solutions]{problem}
\usepackage{hwexam}
\newtcolorbox{problemborderbox}[1][]{
  empty,
  title={Exercise #1},
  attach boxed title to top left,
     minipage boxed title,
  boxed title style={empty,size=minimal,toprule=0pt,top=1pt,left=3mm,overlay={}},
  coltitle=darkgreen,fonttitle=\bfseries,
  parbox=false,boxsep=0pt,left=3mm,right=0mm,top=2pt,breakable,pad at break=0mm,
     before upper=\csname @totalleftmargin\endcsname0pt, 
  overlay unbroken={\draw[darkgreen,line width=2pt] ([xshift=-0pt]title.north west) -- ([xshift=-0pt]frame.south west); },
  overlay first={\draw[darkgreen,line width=2pt] ([xshift=-0pt]title.north west) -- ([xshift=-0pt]frame.south west); },
  overlay middle={\draw[darkgreen,line width=2pt] ([xshift=-0pt]frame.north west) -- ([xshift=-0pt]frame.south west); },
  overlay last={\draw[darkgreen,line width=2pt] ([xshift=-0pt]frame.north west) -- ([xshift=-0pt]frame.south west); },
  outer arc=4pt%
}

\ExplSyntaxOn
\stexstyleproblem{
  \tl_if_empty:NTF\thistitle{
    \begin{problemborderbox}
  }{
    \begin{problemborderbox}[ (\thistitle)]
  }
}{
    \end{problemborderbox}
}
\ExplSyntaxOff

\newtcolorbox{experimental}{
  breakable,
  enhanced,
  left=0pt,
  right=0pt,
  top=8pt,
  bottom=8pt,
  colback=white,
  colframe=gray,
  width=\textwidth,
  enlarge left by=0mm,
  boxsep=5pt,
  fontupper=\small,
  arc=4pt,
  outer arc=4pt,
  leftupper=1.5cm,
  overlay={
    \node[anchor=west] at ([xshift=10pt]$(frame.north west)!0.5!(frame.south west)$)
       {\includegraphics[height=1cm]{stex-experimental}};}
}


\usetikzlibrary{decorations.pathmorphing,shapes,arrows,calc}
% Taken from pgflibrarytikzmmt.code.tex
\newcommand{\mmtarrowtip}{angle 45}
\newcommand{\mmtarrowtipmonoright}{right hook}

\tikzstyle{include}=[\mmtarrowtipmonoright-\mmtarrowtip,thick]
\tikzstyle{morph}=[-\mmtarrowtip,thick]
\tikzstyle{preview}=[decorate, decoration={coil,aspect=0,amplitude=1pt,
                                                  segment length=6pt,
                                                  pre=lineto,pre length=3pt,
                                                  post=lineto,post length=5pt}, thick]
\tikzstyle{view}=[preview,-\mmtarrowtip]


% TIKZ RULES
\def\mmtlogo{
\begin{tikzpicture}

  % White Background (Margins are eyeballed)
  % This is necessary because we paste white over arrows later.
  % If somebody want's to do the full song and dance with
  % interrupted arrows to get transparent background, be my guest.

  \fill[white!] (-0.01,0.15) rectangle (1.11,-0.95);

  % Arrows
  \draw [blue, include] (0,0)     -- (1.1,0);
  \draw [green, morph] (0,-0.4)  -- (1.1,-0.4);
  \draw [red, view]   (-0,-0.8) -- (1.1,-0.8);

  % Cutout for letters
  \fill[white] (0.33,0.1) rectangle (0.66,-0.9);

  % Letters
  \node at (0.18,0)    (nodeM1) {\large M};
  \node at (0.18,-0.4) (nodeM2) {\large M};
  \node at (0.21,-0.8) (nodeT)  {\large T};

\end{tikzpicture}
}

\newtcolorbox{mmtbox}{
  breakable,
  enhanced,
  left=0pt,
  right=0pt,
  top=8pt,
  bottom=8pt,
  colback=white,
  colframe=green,
  width=\textwidth,
  enlarge left by=0mm,
  boxsep=5pt,
  fontupper=\small,
  arc=4pt,
  outer arc=4pt,
  leftupper=1.5cm,
  overlay={
    \node[anchor=west] at ([xshift=10pt]$(frame.north west)!0.5!(frame.south west)$)
       {\mmtlogo};}
}

\AtBeginDocument{\catcode`_=8}

\begin{document}
  \DocInput{\jobname.dtx}
\end{document}
%</driver>
% \fi
%
% \title{ \sTeX-Metatheory
% 	\thanks{Version {\fileversion} (last revised {\filedate})} 
% }
%
% \author{Michael Kohlhase, Dennis Müller\\
% 	FAU Erlangen-Nürnberg\\
% 	\url{http://kwarc.info/}
% }
%
% \maketitle
%
%\ifinfulldoc\else
% This is the documentation for the \pkg{stex-metatheory} package.
% For a more high-level introduction, 
%  see \href{\basedocurl/manual.pdf}{the \sTeX Manual} or the
% \href{\basedocurl/stex.pdf}{full \sTeX documentation}.
%
% \textcolor{red}{TODO: metatheory documentation}
% \fi
%
% \begin{documentation}\label{pkg:metatheory:doc}
%
% \section{Symbols}\label{pkg:metatheory:symbols}
%
% \end{documentation}
%
% \begin{implementation}\label{pkg:metatheory:impl}
%
% \section{\sTeX-Metatheory Implementation}
%
%    \begin{macrocode}
%<*package>
%<@@=stex_modules>

%%%%%%%%%%%%%   metatheory.dtx   %%%%%%%%%%%%%

\str_const:Nn \c_stex_metatheory_ns_str {http://mathhub.info/sTeX/meta}
\begingroup
\stex_module_setup:nn{
  ns=\c_stex_metatheory_ns_str,
  meta=NONE
}{Metatheory}
\stex_reactivate_macro:N \symdecl
\stex_reactivate_macro:N \notation
\stex_reactivate_macro:N \symdef
\ExplSyntaxOff
\csname stex_suppress_html:n\endcsname{
  % is-a (a:A, a \in A, a is an A, etc.)
  \symdecl{isa}[args=ai]
  \notation{isa}[typed,op=:]{#1 \comp{:} #2}{##1 \comp, ##2}
  \notation{isa}[in]{#1 \comp\in #2}{##1 \comp, ##2}
  \notation{isa}[pred]{#2\comp(#1 \comp)}{##1 \comp, ##2}

  % bind (\forall, \Pi, \lambda etc.)
  \symdecl{bind}[args=Bi]
  \notation{bind}[forall]{\comp\forall #1.\;#2}{##1 \comp, ##2}
  \notation{bind}[Pi]{\comp\prod_{#1}#2}{##1 \comp, ##2}
  \notation{bind}[depfun]{\comp( #1 \comp{)\;\to\;} #2}{##1 \comp, ##2}

  % implicit bind
  \symdef{implicitbind}[args=Bi]{\comp\prod_{#1}#2}{##1\comp,##2}

  % dummy variable
  \symdecl{dummyvar}
  \notation{dummyvar}[underscore]{\comp\_}
  \notation{dummyvar}[dot]{\comp\cdot}
  \notation{dummyvar}[dash]{\comp{{\rm --}}}

  %fromto (function space, Hom-set, implication etc.)
  \symdecl{fromto}[args=ai]
  \notation{fromto}[xarrow]{#1 \comp\to #2}{##1 \comp\times ##2}
  \notation{fromto}[arrow]{#1 \comp\to #2}{##1 \comp\to ##2}

  % mapto (lambda etc.)
  %\symdecl{mapto}[args=Bi]
  %\notation{mapto}[mapsto]{#1 \comp\mapsto #2}{#1 \comp, #2}
  %\notation{mapto}[lambda]{\comp\lambda #1 \comp.\; #2}{#1 \comp, #2}
  %\notation{mapto}[lambdau]{\comp\lambda_{#1} \comp.\; #2}{#1 \comp, #2}

  % function/operator application
  \symdecl{apply}[args=ia]
  \notation{apply}[prec=0;0x\infprec,parens]{#1 \comp( #2 \comp)}{##1 \comp, ##2}
  \notation{apply}[prec=0;0x\infprec,lambda]{#1 \; #2 }{##1 \; ##2}

  % collection of propositions/booleans/truth values
  \symdecl{prop}[name=proposition]
  \notation{prop}[prop]{\comp{{\rm prop}}}
  \notation{prop}[BOOL]{\comp{{\rm BOOL}}}

  \symdecl{judgmentholds}[args=1]
  \notation{judgmentholds}[vdash,op=\vdash]{\comp\vdash\; #1}

  % sequences
  \symdecl{seqtype}[args=1]
  \notation{seqtype}[kleene]{#1^{\comp\ast}}

  \symdecl{seqexpr}[args=a]
  \notation{seqexpr}[angle,prec=nobrackets]{\comp\langle #1\comp\rangle}{##1\comp,##2}

  \symdef{sequence-index}[args=2,li,prec=nobrackets]{{#1}_{#2}}
  \notation{sequence-index}[ui,prec=nobrackets]{{#1}^{#2}}

  \symdef{aseqdots}[args=a,prec=nobrackets]{#1\comp{,\ellipses}}{##1\comp,##2}
  \symdef{aseqfromto}[args=ai,prec=nobrackets]{#1\comp{,\ellipses,}#2}{##1\comp,##2}
  \symdef{aseqfromtovia}[args=aii,prec=nobrackets]{#1\comp{,\ellipses,}#2\comp{,\ellipses,}#3}{##1\comp,##2}

  % letin (``let'', local definitions, variable substitution)
  \symdecl{letin}[args=bii]
  \notation{letin}[let]{\comp{{\rm let}}\;#1\comp{=}#2\;\comp{{\rm in}}\;#3}
  \notation{letin}[subst]{#3 \comp[ #1 \comp/ #2 \comp]}
  \notation{letin}[frac]{#3 \comp[ \frac{#2}{#1} \comp]}

  % structures
  \symdecl*{module-type}[args=1]
  \notation{module-type}{\comp{\mathtt{MOD}} #1}
  \symdecl{mathstruct}[name=mathematical-structure,args=a] % TODO
  \notation{mathstruct}[angle,prec=nobrackets]{\comp\langle #1 \comp\rangle}{##1 \comp, ##2}

  % objects
  \symdecl{object}
  \notation{object}{\comp{\mathtt{OBJECT}}}

}
  \ExplSyntaxOn
  \stex_add_to_current_module:n{
    \let\nappa\apply
    \def\nappli#1#2#3#4{\apply{#1}{\naseqli{#2}{#3}{#4}}}
    \def\nappui#1#2#3#4{\apply{#1}{\nasequi{#2}{#3}{#4}}}
    \def\livar{\csname sequence-index\endcsname[li]}
    \def\uivar{\csname sequence-index\endcsname[ui]}
    \def\naseqli#1#2#3{\aseqfromto{\livar{#1}{#2}}{\livar{#1}{#3}}}
    \def\nasequi#1#2#3{\aseqfromto{\uivar{#1}{#2}}{\uivar{#1}{#3}}}
    \def\nappe#1#2#3{\apply{#1}{\aseqfromto{#2}{#3}}}
  }
\_@@_end_module:
\endgroup
%    \end{macrocode}
%
%
%    \begin{macrocode}
%</package>
%    \end{macrocode}
%
% \end{implementation}
%
% \PrintIndex
% \ifinfulldoc\else\printbibliography\fi
% \endinput
% Local Variables:
% mode: doctex
% TeX-master: t
% End:

  \end{sfragment}
  
\end{sfragment}

\begin{sfragment}[id=sec.textsymbols]{Using \sTeX Symbols}
  \begin{sfragment}{Using \sTeX Symbols in Text Mode}

\end{sfragment}

\begin{sfragment}[id=sec.customhighlight]{Customizing Highlighting}

\end{sfragment}

  \begin{sfragment}[id=sec.references]{Referencing Symbols and Statements}
\textcolor{red}{TODO: references documentation}
\end{sfragment}

%%% Local Variables:
%%% mode: latex
%%% TeX-master: "../stex-manual"
%%% End:

\end{sfragment}

\begin{sfragment}{\sTeX Statements (Definitions, Theorems, Examples, ...)}
  \begin{sfragment}{Definitions, Theorems, Examples, Paragraphs}
\begin{smodule}{Statements}
    As mentioned earlier, we can semantically mark-up
    \emph{statements} such as definitions, theorems, lemmata, examples, etc.

    The corresponding environments for that are:
    \begin{itemize}
        \item \stexcode"sdefinition" for definitions,
        \item \stexcode"sassertion" for assertions, i.e.
            propositions that are declared to be \emph{true},
            such as theorems, lemmata, axioms,
        \item \stexcode"sexample" for examples, and
        \item \stexcode"sparagraph" for other semantic paragraphs,
            such as comments, remarks, conjectures, etc.
    \end{itemize}

    The \emph{presentation} of these environments can be customized
    to use e.g. predefined |theorem|-environments, see \sref{sec.customhighlight}
    for details.

    All of these environments take optional arguments
    in the form of |key=value|-pairs. Common to all of them are
    the keys |id=| (for cross-referencing, see \sref{sec.references}),
    |type=| for customization (see \sref{sec.customhighlight})
    and additional information (e.g. definition principles,
    ``difficulty'' etc), |title=|, and |for=|.

    The |for=| key expects a comma-separated list of existing
    symbols, allowing for e.g. things like
    \symdef{addition}[args=a,prec=100]{#1}{##1 \comp+ ##2}
    \symdef{multiplication}[args=a,prec=50]{#1}{##1 \comp\cdot ##2}
    \stexexample{
\begin{sexample}[
    id=additionandmultiplication.ex,
    for={addition,multiplication},
    type={trivial,boring},
    title={An Example}
]
    $\addition{2,3}$ is $5$, $\multiplication{2,3}$ is $6$.
\end{sexample}
    }


\end{smodule}
\end{sfragment}

  The \pkg{stex-proof} package supplies macros and environment that allow to annotate the
structure of mathematical proofs in \sTeX document. This structure can be used by MKM
systems for added-value services, either directly from the \sTeX sources, or after
translation.

We will go over the general intuition by way of a running example: 

\begin{latexcode}
\begin{sproof}[id=simple-proof]
   {We prove that $\sum_{i=1}^n{2i-1}=n^{2}$ by induction over $n$}
  \begin{spfcases}{For the induction we have to consider three cases:}
   \begin{spfcase}{$n=1$}
    \begin{spfstep}[type=inline] then we compute $1=1^2$\end{spfstep}
   \end{spfcase}
   \begin{spfcase}{$n=2$}
      \begin{spfcomment}[type=inline]
        This case is not really necessary, but we do it for the
        fun of it (and to get more intuition).
      \end{spfcomment}
      \begin{spfstep}[type=inline] We compute $1+3=2^{2}=4$.\end{spfstep}
   \end{spfcase}
   \begin{spfcase}{$n>1$}
      \begin{spfstep}[type=assumption,id=ind-hyp]
        Now, we assume that the assertion is true for a certain $k\geq 1$,
        i.e. $\sum_{i=1}^k{(2i-1)}=k^{2}$.
      \end{spfstep}
      \begin{spfcomment}
        We have to show that we can derive the assertion for $n=k+1$ from
        this assumption, i.e. $\sum_{i=1}^{k+1}{(2i-1)}=(k+1)^{2}$.
      \end{spfcomment}
      \begin{spfstep}
        We obtain $\sum_{i=1}^{k+1}{2i-1}=\sum_{i=1}^k{2i-1}+2(k+1)-1$
        \spfjust[method=arith:split-sum]{by splitting the sum}.
      \end{spfstep}
      \begin{spfstep}
        Thus we have $\sum_{i=1}^{k+1}{(2i-1)}=k^2+2k+1$
        \spfjust[method=fertilize]{by inductive hypothesis}.
      \end{spfstep}
      \begin{spfstep}[type=conclusion]
        We can \spfjust[method=simplify]{simplify} the right-hand side to
        ${k+1}^2$, which proves the assertion.
      \end{spfstep}
   \end{spfcase}
    \begin{spfstep}[type=conclusion]
      We have considered all the cases, so we have proven the assertion.
    \end{spfstep}
  \end{spfcases}
\end{sproof}
\end{latexcode}

This yields the following result: 

\begin{mdframed}
  \begin{sproof}[id=simple-proof]
  {We prove that $\sum_{i=1}^n{2i-1}=n^{2}$ by induction over $n$}
  \begin{spfcases}{For the induction we have to consider the following cases:}
    \begin{spfcase}{$n=1$}
      \begin{spfstep}[type=inline] then we compute $1=1^2$\end{spfstep}
    \end{spfcase}
    \begin{spfcase}{$n=2$}
      \begin{spfcomment}[type=inline]
         This case is not really necessary, but we do it for the fun
         of it (and to get more intuition).
      \end{spfcomment}
      \begin{spfstep}[type=inline]
         We compute $1+3=2^{2}=4$
      \end{spfstep}
    \end{spfcase}
    \begin{spfcase}{$n>1$}
      \begin{spfstep}[type=hypothesis,id=ind-hyp]
        Now, we assume that the assertion is true for a certain $k\geq 1$, i.e.
        $\sum_{i=1}^k{(2i-1)}=k^{2}$.
      \end{spfstep}
      \begin{spfcomment}
        We have to show that we can derive the assertion for $n=k+1$ from this
        assumption, i.e.  $\sum_{i=1}^{k+1}{(2i-1)}=(k+1)^{2}$.
      \end{spfcomment}
      \begin{spfstep}[id=splitit]
        We obtain $\sum_{i=1}^{k+1}{(2i-1)}=\sum_{i=1}^k{(2i-1)}+2(k+1)-1$
       \spfjust[method=arith:split-sum]{by splitting the sum}.
     \end{spfstep}
     \begin{spfstep}[id=byindhyp]
       Thus we have $\sum_{i=1}^{k+1}{(2i-1)}=k^2+2k+1$
       \spfjust[method=fertilize]{by \premise[ind-hyp]{inductive hypothesis}}.
     \end{spfstep}
     \begin{spfstep}[type=conclusion]
       We can \spfjust[method=simplify-eq]{simplify the \justarg[rhs]{right-hand side}} to
       $(k+1)^2$, which proves the assertion.
     \end{spfstep}
   \end{spfcase}
   \begin{spfstep}[type=conclusion]
     We have considered all the cases, so we have proven the assertion.
   \end{spfstep}
  \end{spfcases}
\end{sproof}
\end{mdframed}

\begin{environment}{sproof}
  The |sproof| environment is the main container for proofs. It takes an optional |KeyVal|
  argument that allows to specify the |id| (identifier) and |for| (for which assertion is
  this a proof) keys. The regular argument of the |proof| environment contains an
  introductory comment, that may be used to announce the proof style. The |proof|
  environment contains a sequence of |spfstep|, |spfcomment|, and |spfcases| environments
  that are used to markup the proof steps.
\end{environment}
  
\begin{function}{\spfidea}
  The |\spfidea| macro allows to give a one-paragraph description of the proof idea.
\end{function}

\begin{function}{\spfsketch}
  For one-line proof sketches, we use the |\spfsketch| macro, which takes the same
  optional argument as |sproof| and another one: a natural language text that sketches
  the proof.
\end{function}

\begin{environment}{spfstep}
  Regular proof steps are marked up with the |step| environment, which takes an optional
  |KeyVal| argument for annotations. A proof step usually contains a local assertion
  (the text of the step) together with some kind of evidence that this can be derived
  from already established assertions.
\end{environment}

\begin{function}{\spfjust}
  This evidence is marked up with the |\spfjust| macro in the \pkg{stex-proofs}
  package. This environment totally invisible to the formatted result; it wraps the text
  in the proof step that corresponds to the evidence. The environment takes an optional
  |KeyVal| argument, which can have the |method| key, whose value is the name of a proof
  method (this will only need to mean something to the application that consumes the
  semantic annotations). Furthermore, the justification can contain ``premises''
  (specifications to assertions that were used justify the step) and ``arguments''
  (other information taken into account by the proof method).
\end{function}

\begin{function}{\premise}
  The |\premise| macro allows to mark up part of the text as reference to an assertion
  that is used in the argumentation. In the running example we have used the |\premise|
  macro to identify the inductive hypothesis.
\end{function}

\begin{function}{\justarg}
  The |\justarg| macro is very similar to |\premise| with the difference that it is used
  to mark up arguments to the proof method. Therefore the content of the first argument
  is interpreted as a mathematical object rather than as an identifier as in the case of
  |\premise|. In our example, we specified that the simplification should take place on
  the right hand side of the equation. Other examples include proof methods that
  instantiate. Here we would indicate the substituted object in a |\justarg| macro.
\end{function}

Note that both |\premise| and |\justarg| can be used with an empty second argument to
mark up premises and arguments that are not explicitly mentioned in the text.

\begin{environment}{subproof}
  The |spfcases| environment is used to mark up a subproof. This environment takes an
  optional |KeyVal| argument for semantic annotations and a second argument that allows
  to specify an introductory comment (just like in the |proof| environment). The
  |method| key can be used to give the name of the proof method
  executed to make this subproof.
\end{environment}

\begin{environment}{spfcases}
  The |spfcases| environment is used to mark up a proof by cases. Technically it is a
  variant of the |subproof| where the |method| is |by-cases|. Its contents are |spfcase|
  environments that mark up the cases one by one.
\end{environment}

\begin{environment}{spfcase}
  The content of a |spfcases| environment are a sequence of case proofs marked up in the
  |spfcase| environment, which takes an optional |KeyVal| argument for semantic
  annotations. The second argument is used to specify the the description of the case
  under consideration. The content of a |spfcase| environment is the same as that of a
  |sproof|, i.e. |spfstep|s, |spfcomment|s, and |spfcases| environments.
\end{environment}

\begin{function}{\spfcasesketch}
  |\spfcasesketch| is a variant of the |spfcase| environment that takes the same
  arguments, but instead of the |spfstep|s in the body uses a third argument for a proof
  sketch.
\end{function}

\begin{environment}{spfcomment}
  The |spfcomment| environment is much like a |step|, only that it does not have an
  object-level assertion of its own. Rather than asserting some fact that is relevant
  for the proof, it is used to explain where the proof is going, what we are attempting
  to to, or what we have achieved so far. As such, it cannot be the target of a
  |\premise|.
\end{environment}

\begin{function}{\sproofend}
  Traditionally, the end of a mathematical proof is marked with a little box at the end of
  the last line of the proof (if there is space and on the end of the next line if there
  isn't), like so:\sproofend

  The \pkg{stex-proofs} package provides the |\sproofend| macro for this.
\end{function}
  
\begin{variable}{\sProofEndSymbol}
  If a different symbol for the proof end is to be used (e.g. {\sl{q.e.d}}), then this can
  be obtained by specifying it using the |\sProofEndSymbol| configuration macro (e.g. by
  specifying |\sProofEndSymbol{q.e.d}|).
\end{variable}
  
Some of the proof structuring macros above will insert proof end symbols for sub-proofs,
in most cases, this is desirable to make the proof structure explicit, but sometimes this
wastes space (especially, if a proof ends in a case analysis which will supply its own
proof end marker). To suppress it locally, just set |proofend={}| in them or use use
|\sProofEndSymbol{}|.

%%% Local Variables:
%%% mode: latex
%%% TeX-master: "../stex-manual"
%%% End:

%  LocalWords:  hypothesis,id geq splitit arith:split-sum byindhyp rhs proofend

\end{sfragment}

\begin{sfragment}{Additional Packages}
  \ProvidesExplPackage{stex-tikzinput}{2021/08/31}{1.9}{bla}
\RequirePackage{stex}
\RequirePackage{tikzinput}

\newcommand\mhtikzinput[2][]{%
  \def\Gin@mhrepos{}\setkeys{Gin}{#1}%
  \stex_in_repository:nn\Gin@mhrepos{
    \tikzinput[#1]{\mhpath{##1}{#2}}
  }
}
\newcommand\cmhtikzinput[2][]{\begin{center}\mhgraphics[#1]{#2}\end{center}}

  \begin{sfragment}{Modular Document Structuring}
    This package supplies an infrastructure for writing {\omdoc} documents in {\LaTeX}.
This includes a simple structure sharing mechanism for \sTeX that allows to to move from
a copy-and-paste document development model to a copy-and-reference model, which
conserves space and simplifies document management. The augmented structure can be used
by MKM systems for added-value services, either directly from the \sTeX sources, or
after translation.

\begin{sfragment}[id=sec:STR]{Introduction}

 The |document-structure| package supplies macros and environments that allow to label document
 fragments and to reference them later in the same document or in other documents. In
 essence, this enhances the document-as-trees model to
 documents-as-directed-acyclic-graphs (DAG) model. This structure can be used by MKM
 systems for added-value services, either directly from the \sTeX sources, or after
 translation. Currently, trans-document referencing provided by this package can only be
 used in the \sTeX collection.

 DAG models of documents allow to replace the ``Copy and Paste'' in the source document
 with a label-and-reference model where document are shared in the document source and the
 formatter does the copying during document formatting/presentation.
\end{sfragment}

\begin{sfragment}[id=sec:user]{The User Interface}
  The \pkg{document-structure} package accepts the following options:
  \begin{center}
    \begin{tabular}{|l|p{10cm}|}\hline
      \texttt{class=\meta{name}} & load \meta{name}|.cls| instead of |article.cls|\\\hline 
      \texttt{topsect=\meta{sect}} & The top-level sectioning level; the default for
      \meta{sect} is \texttt{section}\\\hline 
    \end{tabular}
  \end{center}

\begin{sfragment}[id=sec:user:struct]{Document Structure}

  \begin{environment}{sfragment}
    The structure of the document is given by the |sfragment| environment just like in
    {\omdoc}. In the {\LaTeX} route, the |sfragment| environment is flexibly mapped to
    sectioning commands, inducing the proper sectioning level from the nesting of
    |sfragment| environments. Correspondingly, the |sfragment| environment takes an optional
    key/value argument for metadata followed by a regular argument for the (section) title
    of the sfragment. The optional metadata argument has the keys |id| for an identifier,
    |creators| and |contributors| for the Dublin Core metadata~\cite{DCMI:dmt03}. The
    option |short| allows to give a short title for the generated section. If the title
    contains semantic macros, they need to be protected by |\protect|\ednote{MK: still?},
    and we need to give the |loadmodules| key it needs no value. For instance we would
    have
\begin{latexcode}
\begin{smodule}{foo}
  \symdef{bar}{B^a_r}
   ...
   \begin{sfragment}[id=sec.barderiv,loadmodules]
     {Introducing $\protect\bar$ Derivations}
\end{latexcode}

\sTeX automatically computes the sectioning level, from the nesting of |sfragment|
environments.
\end{environment}

But sometimes, we want to skip levels (e.g. to use a subsection* as an introduction for a
chapter).

\begin{environment}{blindfragment}
  Therefore the |document-structure| package provides a variant |blindfragment| that does
  not produce markup, but increments the sectioning level and logically groups document
  parts that belong together, but where traditional document markup relies on convention
  rather than explicit markup. The |blindfragment| environment is useful e.g. for creating
  frontmatter at the correct level. The example below shows a typical setup for the outer
  document structure of a book with parts and chapters.
  
\begin{latexcode}
\begin{document}
\begin{blindfragment}
\begin{blindfragment}
\begin{frontmatter}
\maketitle\newpage
\begin{sfragment}{Preface}
... <<preface>> ...
\end{sfragment}
\clearpage\setcounter{tocdepth}{4}\tableofcontents\clearpage
\end{frontmatter}
\end{blindfragment}
... <<introductory remarks>> ...
\end{blindfragment}
\begin{sfragment}{Introduction}
... <<intro>> ...
\end{sfragment}
... <<more chapters>> ... 
\bibliographystyle{alpha}\bibliography{kwarc}
\end{document}
\end{latexcode}

Here we use two levels of |blindfragment|:
\begin{itemize}
\item The outer one groups the introductory parts of the book (which we assume to have a
  sectioning hierarchy topping at the part level). This |blindfragment| makes sure that
  the introductory remarks become a ``chapter'' instead of a ``part''.
\item Th inner one groups the frontmatter\footnote{We shied away from redefining the
    |frontmatter| to induce a blindfragment, but this may be the ``right'' way to go in
    the future.} and makes the preface of the book a section-level construct.\ednote{MK:
    We need a substitute for the ``Note that here the |display=flow| on the |sfragment|
    environment prevents numbering as is traditional for prefaces.''}
\end{itemize}
\end{environment}

\begin{function}{\skipfragment}
  The |\skipfragment| ``skips an |sfragment|'', i.e. it just steps the respective sectioning
  counter. This macro is useful, when we want to keep two documents in sync structurally,
  so that section numbers match up: Any section that is left out in one becomes a
  |\skipfragment|.
\end{function}

\begin{function}{\currentsectionlevel,\CurrentSectionLevel}
  The |\currentsectionlevel| macro supplies the name of the current sectioning level,
  e.g. ``chapter'', or ``subsection''. |\CurrentSectionLevel| is the capitalized
  variant. They are useful to write something like ``In this |\currentsectionlevel|, we
  will\ldots'' in an |sfragment| environment, where we do not know which sectioning level we
  will end up.
\end{function}
\end{sfragment}

\begin{sfragment}[id=sec:user:ignore]{Ignoring Inputs}

\begin{function}{\prematurestop,\afterprematurestop}
  For prematurely stopping the formatting of a document, \sTeX provides the
  |\prematurestop| macro. It can be used everywhere in a document and ignores all input
  after that -- backing out of the sfragment environment as needed. After that -- and
  before the implicit |\end{document}| it calls the internal |\afterprematurestop|, which
  can be customized to do additional cleanup or e.g. print the bibliography.

  |\prematurestop| is useful when one has a driver file, e.g. for a course taught multiple
  years and wants to generate course notes up to the current point in the lecture. Instead
  of commenting out the remaining parts, one can just move the |\prematurestop| macro.
  This is especially useful, if we need the rest of the file for processing, e.g. to
  generate a theory graph of the whole course with the already-covered parts marked up as
  an overview over the progress; see |import_graph.py| from the |lmhtools|
  utilities~\cite{lmhtools:github:on}.
\end{function}
\end{sfragment}

\begin{sfragment}[id=sec:user:gvars]{Global Variables}

  Text fragments and modules can be made more re-usable by the use of global
  variables. For instance, the admin section of a course can be made course-independent
  (and therefore re-usable) by using variables (actually token registers)
  |courseAcronym| and |courseTitle| instead of the text itself. The variables can then
  be set in the \sTeX preamble of the course notes file.
  
  \begin{function}{\setSGvar,\useSGvar}
    |\setSGvar{|\meta{vname}|}{|\meta{text}|}| to set the global variable \meta{vname} to
    \meta{text} and |\useSGvar{|\meta{vname}|}| to reference it.
  \end{function}
  
  \begin{function}{\ifSGvar}
    With|\ifSGvar| we can test for the contents of a global variable: the macro call
    |\ifSGvar{|\meta{vname}|}{|\meta{val}|}{|\meta{ctext}|}| tests the content of the
    global variable \meta{vname}, only if (after expansion) it is equal to \meta{val}, the
    conditional text \meta{ctext} is formatted.
  \end{function}
\end{sfragment}
\end{sfragment}

%%% Local Variables:
%%% mode: latex
%%% TeX-master: "../stex-manual"
%%% End:

%  LocalWords:  article.cls topsect DCMI:dmt03 loadmodules lmhtools
%  LocalWords:  prematurestop afterprematurestop import_graph.py STRlabel STRcopy vname
%  LocalWords:  STRsemantics setSGvar ifSGvar ctext

  \end{sfragment}
  \begin{sfragment}{Slides and Course Notes}
    \textcolor{red}{TODO: notesslides documentation}
  \end{sfragment}
  \begin{sfragment}{Homework, Problems and Exams}
    The \pkg{problem} package supplies an infrastructure that allows specify problem.  Problems
are text fragments that come with auxiliary functions: hints, notes, and
solutions\footnote{for the moment multiple choice problems are not supported, but may
  well be in a future version}. Furthermore, we can specify how long the solution to a
given problem is estimated to take and how many points will be awarded for a perfect
solution.

Finally, the \pkg{problem} package facilitates the management of problems in small files,
so that problems can be re-used in multiple environment. 

\begin{function}{solutions,notes,hints,gnotes,pts,min,boxed,test}
  The \pkg{problem} package takes the options |solutions| (should solutions be output?),
  |notes| (should the problem notes be presented?), |hints| (do we give the hints?),
  |gnotes| (do we show grading notes?), |pts| (do we display the points awarded for
  solving the problem?), |min| (do we display the estimated minutes for problem
  soling). If theses are specified, then the corresponding auxiliary parts of the problems
  are output, otherwise, they remain invisible.

  The |boxed| option specifies that problems should be formatted in framed boxes so that
  they are more visible in the text. Finally, the |test| option signifies that we are in a
  test situation, so this option does not show the solutions (of course), but leaves space
  for the students to solve them.
\end{function}

\begin{environment}{problem}
  The main environment provided by the \pkg{problem}package is (surprise surprise) the
  |problem| environment. It is used to mark up problems and exercises. The environment
  takes an optional KeyVal argument with the keys |id| as an identifier that can be
  reference later, |pts| for the points to be gained from this exercise in homework or
  quiz situations, |min| for the estimated minutes needed to solve the problem, and
  finally |title| for an informative title of the problem.
\end{environment}

\stexexample{%
\documentclass{article}
\usepackage[solutions,hints,pts,min]{problem}
\begin{document}
  \begin{sproblem}[id=elefants,pts=10,min=2,title=Fitting Elefants]
    How many Elefants can you fit into a Volkswagen beetle?
    \begin{hint}
      Think positively, this is simple!
    \end{hint}
    \begin{exnote}
      Justify your answer
    \end{exnote}
\begin{solution}[for=elefants,height=3cm]
  Four, two in the front seats, and two in the back.
  \begin{gnote}
    if they do not give the justification deduct 5 pts
  \end{gnote}
\end{solution}
\end{sproblem}
\end{document}
}

\begin{environment}{solution}
  The |solution| environment can be to specify a solution to a problem. If the package
  option |solutions| is set or |\solutionstrue| is set in the text, then the solution will
  be presented in the output. The |solution| environment takes an optional KeyVal argument
  with the keys |id| for an identifier that can be reference |for| to specify which
  problem this is a solution for, and |height| that allows to specify the amount of space
  to be left in test situations (i.e. if the |test| option is set in the |\usepackage|
  statement).
\end{environment}

\begin{environment}{hint,exnote,gnote}
  The |hint| and |exnote| environments can be used in a |problem| environment to give
  hints and to make notes that elaborate certain aspects of the problem.  The |gnote|
  (grading notes) environment can be used to document situtations that may arise in
  grading.
\end{environment}

\begin{function}{\startsolutions,\stopsolutions}
  Sometimes we would like to locally override the |solutions| option we have given to the
  package. To turn on solutions we use the |\startsolutions|, to turn them off,
  |\stopsolutions|. These two can be used at any point in the documents.
\end{function}

\begin{function}{\ifsolutions}
  Also, sometimes, we want content (e.g. in an exam with master solutions) conditional on
  whether solutions are shown. This can be done with the |\ifsolutions| conditional.
\end{function}

\begin{environment}{mcb}
  Multiple choice blocks can be formatted using the |mcb| environment, in which single
  choices are marked up with |\mcc| macro.
\end{environment}

\begin{function}{\mcc}
  |\mcc[|\meta{keyvals}|]{|\meta{text}|}| takes an optional key/value argument
  \meta{keyvals} for choice metadata and a required argument \meta{text} for the proposed
  answer text. The following keys are supported
  \begin{itemize}
  \item |T| for true answers, |F| for false ones,
  \item |Ttext| the verdict for true answers, |Ftext| for false ones, and
  \item |feedback| for a short feedback text given to the student.
  \end{itemize}
\end{function}

If we start the solutions, then we get

\stexexample{%
\startsolutions
\begin{sproblem}[title=Functions,name=functions1]
  What is the keyword to introduce a function definition in python?
  \begin{mcb}
    \mcc[T]{def}
    \mcc[F,feedback=that is for C and C++]{function}
    \mcc[F,feedback=that is for Standard ML]{fun}
    \mcc[F,Ftext=Nooooooooo,feedback=that is for Java]{public static void}
  \end{mcb}
\end{sproblem}
}
without solutions (that is what the students see during the exam/quiz)\ednote{MK: that did
not work!}
\stexexample{%
\stopsolutions
\begin{sproblem}[title=Functions,name=functions1]
  What is the keyword to introduce a function definition in python?
  \begin{mcb}
    \mcc[T]{def}
    \mcc[F,feedback=that is for C and C++]{function}
    \mcc[F,feedback=that is for Standard ML]{fun}
    \mcc[F,Ftext=Nooooooooo,feedback=that is for Java]{public static void}
  \end{mcb}
\end{sproblem}
}

\begin{function}{\includeproblem}
  The |\includeproblem| macro can be used to include a problem from another file. It takes
  an optional KeyVal argument and a second argument which is a path to the file containing
  the problem (the macro assumes that there is only one problem in the include file). The
  keys |title|, |min|, and |pts| specify the problem title, the estimated minutes for
  solving the problem and the points to be gained, and their values (if given) overwrite
  the ones specified in the |problem| environment in the included file.
\end{function}

The sum of the points and estimated minutes (that we specified in the |pts| and |min| keys
to the |problem| environment or the |\includeproblem| macro) to the log file and the
screen after each run. This is useful in preparing exams, where we want to make sure that
the students can indeed solve the problems in an allotted time period.

The |\min| and |\pts| macros allow to specify (i.e. to print to the margin) the
distribution of time and reward to parts of a problem, if the |pts| and |pts| options are
set. This allows to give students hints about the estimated time and the points to be
awarded.

%%% Local Variables:
%%% mode: latex
%%% TeX-master: "../stex-manual"
%%% End:

    
    
The \pkg{wexam} package and class supplies an infrastructure that allows to format
nice-looking assignment sheets by simply including problems from problem files marked up
with the \pkg{roblem} package.  It is designed to be compatible with |problems.sty|, and
inherits some of the functionality.

\begin{sfragment}[id=sec:user:options]{Package Options}

\begin{variable}{solutions,notes,hints,gnotes,pts,min}
The \pkg{wexam} package and class take the options |solutions|, |notes|, |hints|,
|gnotes|, |pts|, |min|, and |boxed| that are just passed on to the |problems| package
(cf. its documentation for a description of the intended behavior).
\end{variable}
\end{sfragment}

\begin{sfragment}{Assignments}
This package supplies the \DescribeEnv{assignment}|assignment| environment that groups
problems into assignment sheets. It takes an optional KeyVal argument with the keys
\DescribeMacro{number}|number| (for the assignment number; if none is given, 1 is
assumed as the default or --- in multi-assignment documents --- the ordinal of the
|assignment| environment), \DescribeMacro{title}|title| (for the assignment title; this
is referenced in the title of the assignment sheet), \DescribeMacro{type}|type| (for the
assignment type; e.g. ``quiz'', or ``homework''), \DescribeMacro{given}|given| (for the
date the assignment was given), and \DescribeMacro{due}|due| (for the date the
assignment is due).
\end{sfragment}

\begin{sfragment}{Typesetting Exams}

Furthermore, the |hwexam| package takes the option
\DescribeMacro{multiple}|multiple| that allows to combine multiple assignment sheets into
a compound document (the assignment sheets are treated as section, there is a table of
contents, etc.). 

Finally, there is the option \DescribeMacro{test}|test| that modifies the behavior to
facilitate formatting tests. Only in |test| mode, the macros |\testspace|,
|\testnewpage|, and |\testemptypage| have an effect: they generate space for the
students to solve the given problems. Thus they can be left in the {\LaTeX} source. 

\DescribeMacro{\testspace}|\testspace| takes an argument that expands to a dimension,
and leaves vertical space accordingly. \DescribeMacro{\testnewpage}|\testnewpage| makes
a new page in |test| mode, and \DescribeMacro{\testemptypage}|\testemptypage| generates
an empty page with the cautionary message that this page was intentionally left empty.

Finally, the \DescribeEnv{testheading}|\testheading| takes an optional keyword argument
where the keys \DescribeMacro{duration}|duration| specifies a string that specifies the
duration of the test, \DescribeMacro{min}|min| specifies the equivalent in number of
minutes, and \DescribeMacro{reqpts}|reqpts| the points that are required for a perfect
grade.

\begin{latexcode}
\title{320101 General Computer Science (Fall 2010)}
\begin{testheading}[duration=one hour,min=60,reqpts=27]
  Good luck to all students!
\end{testheading}
\end{latexcode}

Will result in
\begin{center}
  \begin{minipage}{.9\textwidth}
\makeatletter
\@problem{1.1}{4}{10}
\@problem{2.1}{4}{8}
\@problem{2.2}{6}{10}
\@problem{2.3}{6}{10}
\@problem{3.1}{4}{8}
\@problem{3.2}{4}{8}
\@problem{3.3}{2}{4}
\makeatother
\vspace*{-3ex}\hrule\vspace*{.5ex}  formats to\vspace*{1ex}
\hrule\par\noindent\vspace*{2ex}
\title{320101 General Computer Science (Fall 2010)}
\begin{testheading}[duration=one hour,min=60,reqpts=27]
  good luck
\end{testheading}
\end{minipage}
\end{center}
\end{sfragment}

\begin{sfragment}{Including Assignments}

The \DescribeMacro{\inputassignment}|\inputassignment| macro can be used to input
an assignment from another file. It takes an optional KeyVal argument and a second
argument which is a path to the file containing the problem (the macro assumes that
there is only one |assignment| environment in the included file).  The keys
\DescribeMacro{number}|number|, \DescribeMacro{title}|title|,
\DescribeMacro{type}|type|, \DescribeMacro{given}|given|, and \DescribeMacro{due}|due|
are just as for the |assignment| environment and (if given) overwrite the ones specified
in the |assignment| environment in the included file.
\end{sfragment}

%%% Local Variables:
%%% mode: latex
%%% TeX-master: "../stex-manual"
%%% End:

  \end{sfragment}

\end{sfragment}

\chapter{Stuff}

\begin{function}{\sTeX , \stex}
  Both print this \stex logo.
\end{function}

 \subsection{Semantic Macros and Notations}

 Semantic macros invoke a formally declared symbol.

 To declare a symbol (in a module), we use \cs{symdecl},
 which takes as argument the name of the corresponding
 semantic macro, e.g. |\symdecl{foo}| introduces the macro
 \cs{foo}. Additionally, \cs{symdecl} takes several options,
 the most important one being its arity. |foo| as declared above
 yields a \emph{constant} symbol. To introduce an \emph{operator}
 which takes arguments, we have to specify which arguments it takes.

 \begin{smodule}{SemanticMacrosExample}
   For example, to introduce binary multiplication,
   we can do |\symdecl{mult}[args=2]|. We can then supply
   the semantic macro with arbitrarily many notations, such as
   |\notation{mult}{#1 #2}|.
   
   \stexexample{
 \symdecl{mult}[args=2]
 \notation{mult}{#1 #2}
 $\mult{a}{b}$
}

 Since usually, a freshly introduced symbol also comes with a
 notation from the start, the \cs{symdef} command combines
 \cs{symdecl} and \cs{notation}. So instead of the above,
 we could have also written
 \begin{center} |\symdef{mult}[args=2]{#1 #2}| \end{center}

 \symdecl{mult}[args=2]
 \notation{mult}{#1 #2}

   \notation{mult}[cdot]{#1 \comp{\cdot} #2}
   \notation{mult}[times]{#1 \comp{\times} #2}
   Adding more notations like
   |\notation{mult}[cdot]{#1 \comp{\cdot} #2}| or 
   |\notation{mult}[times]{#1 \comp{\times} #2}|
   allows us to write |$\mult[cdot]{a}{b}$| and
   |$\mult[times]{a}{b}$|:
   \stexexample{
   \notation{mult}[cdot]{#1 \comp{\cdot} #2}
   \notation{mult}[times]{#1 \comp{\times} #2}
 $\mult[cdot]{a}{b}$ and $\mult[times]{a}{b}$
}
   \notation{mult}[cdot]{#1 \comp{\cdot} #2}
   \notation{mult}[times]{#1 \comp{\times} #2}

   Not using an explicit option with a semantic macro yields
   the first declared notation, unless changed\ednote{TODO}.

   Outside of math mode, or by using the starred variant
   |\foo*|, allows to provide a custom notation, where
   notational (or textual) components can be given
   explicitly in square brackets.
   \stexexample{
 $\mult*{\arg{a}\comp{\ast}\arg{b}}$ is the 
 \mult{\comp{product of} \arg{$a$} \comp{and} \arg{$b$}}
}

   In custom mode, prefixing an argument with a star will not
   print that argument, but still export it to \omdoc:
   \stexexample{
 \mult{\comp{Multiplying} \arg*{$\mult{a}{b}$} again by \arg{$b$}} yields...
}
   The syntax |*[|\meta{int}|]| allows switching
   the order of arguments. For example, given a 2-ary semantic
   macro |\forevery| with exemplary notation
   |\forall #1. #2|, we can write
   \stexexample{
     \symdecl{forevery}[args=2]
     \forevery{\arg[2]{The proposition $P$} \comp{holds for every} \arg[1]{$x\in A$}}
}

 When using |*[|$n$|]|, after reading the provided ($n$th) argument,
  the ``argument counter'' automatically 
 continues where we left off, so the |*[1]| in the above example
 can be omitted.

   For a macro with arity $>0$, we can refer to the operator
   \emph{itself} semantically by suffixing the semantic macro
   with an exclamation point |!| in either text or math mode.
   For that reason \cs{notation} (and thus \cs{symdef}) take an
   additional optional argument |op=|, which allows to assign
   a notation for the operator itself. e.g.
   \stexexample{
     \symdef{add}[args=2,op={+}]{#1 \comp+ #2}
     The operator $\add!$ adds two elements, as in $\add ab$.
   }

  |*| is composable with |!| for custom notations, as in:

   \stexexample{
 \mult!{\comp{Multiplication}} (denoted by $\mult!*{\comp\cdot}$) is defined by...
}

 The macro \cs{comp} as used everywhere above is responsible
 for highlighting, linking, and tooltips, and should be wrapped
 around the notation (or text) components that should be treated
 accordingly. While it is attractive to just wrap a whole notation,
 this would also wrap around e.g. the arguments themselves, so
 instead, the user is tasked with marking the notation components
 themself.

 The precise behaviour of \cs{comp} is governed by
 the macro \cs{@comp}, which takes two arguments: The tex code
 of the text
 (unexpanded) to highlight, and the URI of the current symbol.
 \cs{@comp} can be safely redefined to customize the behaviour.


 The starred variant |\symdecl*{foo}| does not introduce a semantic
 macro, but still declares a corresponding symbol. |foo| (like
 any other symbol, for that matter) can
 then be accessed via \cs{STEXsymbol}|{foo}| or (if |foo| was declared
 in a module |Foo|) via \cs{STEXModule}|{Foo}?{foo}|.

 both \cs{STEXsymbol} and \cs{STEXModule} take any
 arbitrary ending segment of a full URI to determine
 which symbol or module is meant. e.g.
 \cs{STEXsymbol}|{Foo?foo}| is also valid, as are e.g.
 \cs{STEXModule}|{path?Foo}?{foo}| or
 \cs{STEXsymbol}|{path?Foo?foo}|

 There's also a convient shortcut \cs{symref}|{?foo}{some text}| for
 \cs{STEXsymbol}|{?foo}![some text]|.

 \end{smodule}

 \subsubsection{Other Argument Types}

 So far, we have stated the arity of a semantic macro directly.
 This works if we only have ``normal'' (or more precisely: |i|-type) arguments.
  To make use of other argument types, instead of providing the arity
 numerically, we can provide it as a sequence of characters representing
 the argument types -- e.g. instead of writing |args=2|, we
 can equivalently write |args=ii|, indicating that the macro
 takes two |i|-type arguments.

 Besides |i|-type arguments, \sTeX has two other types, which we will
 discuss now.

 The first are \emph{binding} (|b|-type) arguments, representing
 variables that are \emph{bound} by the operator. This is the
 case for example in the above \cs{forevery}-macro:
 The first argument is not actually an argument that the
 |forevery| ``function'' is ``applied'' to; rather, the first argument
 is a new variable (e.g. $x$) that is \emph{bound} in the subsequent
 argument. More accurately, the macro should therefore have been
 implemented thusly:
   \begin{center}|\symdef{forevery}[args=bi]{\forall #1.\; #2}|\end{center}

 \begin{smodule}{OtherArgs}
 |b|-type arguments are indistinguishable from |i|-type arguments
 within \sTeX, but are treated very differently in \omdoc and by \mmt.
 More interesting \emph{within} \sTeX are |a|-type arguments,
 which represent (associative) arguments of flexible arity, which are
 provided as comma-separated lists.
 This allows e.g. better representing the \cs{mult}-macro above:
 
   \stexexample{
 \symdef{mult}[args=a]{#1}{##1 \comp\cdot ##2}
 $\mult{a,b,c,{d^e},f}$
}
 As the example above shows, notations get a little more complicated
 for associative arguments. For every |a|-type argument, the
 \cs{notation}-macro takes an additional argument that declares
 how individual entries in an |a|-type argument list are aggregated.
 The first notation argument then describes how the aggregated
 expression is combined into the full representation.

 For a more interesting example, consider a flexary operator
 for ordered sequences in ordered set, that taking 
 arguments |{a,b,c}| and |\mathbb{R}| prints
 $a \leq b \leq c\in \mathbb R$. This operator takes
 two arguments (an |a|-type argument and an |i|-type argument),
 aggregates the individuals of the associative argument using |\leq|,
 and combines the result with |\in| and the second argument thusly:

   \stexexample{
 \symdef{numseq}[args=ai]{#1 \comp\in #2}{##1 \comp\leq ##2}
 $\numseq{a,b,c}{\mathbb R}$
}

 Finally, |B|-type arguments combine the functionalities of |a|
 and |b|, i.e. they represent flexary binding operator arguments.

\ednote{what about e.g. \detokenize{\int_x\int_y\int_z f dx dy dz}?}
\ednote{``decompose'' a-type arguments into fixed-arity operators?}

 \end{smodule}

 \subsubsection{Precedences}

 Every notation has an (upwards) \emph{operator precedence} and
 for each argument a (downwards) \emph{argument precedence}
 used for automated bracketing. For example, a notation
 for a binary operator \cs{foo} could be declared like this:
 \begin{center} |\notation{foo}[prec=200;500x600]{#1 \comp{+} #2}| \end{center}
 assigning an operator precedence of 200, an argument precedence
 of 500 for the first argument, and an argument precedence of 600
 for the second argument.

 \sTeX insert brackets thusly: Upon encountering a semantic
 macro (such as \cs{foo}), its operator precedence (e.g. 200)
 is compared to the current downwards precedence (initially 
 \cs{neginfprec}). If the operator precedence is \emph{larger}
 than the current downwards precedence, parentheses are inserted
 around the semantic macro.

 Notations for symbols of arity 0 have a default precedence of \cs{infprec},
 i.e. by default, parentheses are never inserted around constants.
 Notations for symbols with arity $>0$ have a default operator
 precedence of $0$.
 If no argument precedences are explicitly provided, then by
 default they are equal to the operator precedence.

 Consequently, if some operator $A$ should bind stronger than
 some operator $B$, then $A$s operator precedence should be
 smaller than $B$s argument precedences.

 For example:
 \begin{smodule}{NotationsEx}
 \symdecl{plus}[args=2]
 \symdecl{times}[args=2]
 \stexexample{
\notation{plus}[prec=100]{#1 \comp{+} #2}
\notation{times}[prec=50]{#1 \comp{\cdot} #2}
$\plus{a}{\times{b}{c}}$ and $\times{a}{\plus{b}{c}}$
}


 \end{smodule}

 \subsection{Archives and Imports}

 \subsubsection{Namespaces}
   

 \subsubsection{Paths in Import-Statements}

 

	
	
\csname if@infulldoc\endcsname\else\end{document}\fi
