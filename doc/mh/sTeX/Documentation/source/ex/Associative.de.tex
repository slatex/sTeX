\documentclass[lang=de]{stex}
\libinput{docpreamble}
\begin{document}

\begin{smodule}[sig=en]{Associative}
  \begin{sdefinition}[for=associative]
    \vardef{vU}[name=U]{\comp{U}}
    \vardef{op}[args=2,type=\funspace{\vU,\vU}\vU,bind=forall,
      op=\comp{\circ}]
      {#1 \mathbin{\comp{\circ}} #2}
    \vardef{vx}[type=\vU,bind=forall]{\comp{x}}
    \vardef{vy}[type=\vU,bind=forall]{\comp{y}}
    \vardef{vz}[type=\vU,bind=forall]{\comp{z}}
  

    Eine \symref{function}{Funktion} $\fun{\op!}{\vU,\vU}\vU$
    mit
    \definiens{$
      \eq{
          \op{ (\op{\vx}{\vy}) }{\vz},
          \op{\vx}{ (\op{\vy}{\vz}) }
        }
    $}
    für alle $\inset{\vx,\vy,\vz}\vU$,
    wird \definiendum{associative}{assoziativ} genannt.
  \end{sdefinition}
\end{smodule}

\end{document}
