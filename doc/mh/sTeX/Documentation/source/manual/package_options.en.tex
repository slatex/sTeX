\documentclass{stex}
\libinput{docpreamble}

\begin{document}
\begin{sfragment}{Package and Class Options}
  
    \begin{itemize}
      \item|debug=|\meta{prefixes}: (see Developer Manual)
      \item|lang=|\meta{languages}: The languages to load the \pkg{babel}
        package with.
      \item|mathhub=|\meta{path}: See (somewhere).
      \item|usesms|/|writesms|:
        If |writesms| is set, content loaded from external math archives 
        is persisted in the file \cs{jobname}|.sms|.

        If |usesms| is set, the content of the
        |.sms|-file is loaded, obviating the need to reprocess
        the original files.

        The options are not mutually exclusive, but care should be taken
        if dependencies have changed between builds.

        This offers two advantages:
        \begin{enumerate}
          \item If a document has many (transitive) dependencies, |usesms|
            should significantly speed up the build process, and
          \item setting |usesms| allows for distributing the |.sms|-file
            to make the document \emph{standalone}, allowing for compilation
            without needing imported/used modules to be present. 
        \end{enumerate}

        Both package options can be overridden with the environment
        variables |STEX_USESMS| and |STEX_WRITESMS|.
      \item|checkterms|: If this option is set (or the environment variable
        |STEX_CHECKTERMS| is set to anything other than |false|), all
        types, definientia and notations are being typeset in a throwaway
        box when they are declared as a sanity check.

        It is recommended to set this option if you are primarily using
        \sTeX with |pdflatex| every time you add or change notations
        or typed/defined symbol declarations. Subsequently, this
        check can be safely turned off.
      \item|image|:
    %%^^A      \item |index=|\meta{name}: the \meta{csname} is indexed as if
    %%^^A        one had written \cs{cs}\Arg{name}.
    %%^^A      \item |no-index|: the \meta{csname} is not indexed.
    %%^^A      \item |module=|\meta{module}: the \meta{csname} is indexed in
    %%^^A        the list of commands from the \meta{module}; the \meta{module}
    %%^^A        can in particular be |TeX| for \enquote{\TeX{} and \LaTeXe{}}
    %%^^A        commands, or empty for commands which should be placed in the
    %%^^A        main index.  By default, the \meta{module} is deduced
    %%^^A        automatically from the command name.
    %%^^A      \item |replace| is a boolean key (\texttt{true} by default) which
    %%^^A        indicates whether to replace |@@| as \pkg{l3docstrip} does.
    \end{itemize}

\end{sfragment}
\end{document}