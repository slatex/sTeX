\documentclass{stex}
\libinput{docpreamble}
\begin{document}
  \begin{smodule}[title=Notations]{Notation}
    \importmodule{concepts?Symbol}
    \usemodule{concepts?SemanticMacro}
    \symdecl*{operator notation}

    \begin{sparagraph}[style={defibox,symdoc},name=notation]
        \stexvarmacro{macroname}
        A \sn{symbol} can be assigned arbitrarily many 
        \definame[post=s]{notation} that represent the
        \sn{symbol} in formulas. \Sns{notation} are invoked
        using a \sn{semantic macro} for a given \sn{symbol}.

        If a \sn{symbol} is a function taking arguments,
        its \sns{notation} represent the \emph{application}
        of the \sn{symbol}, and
        it can have additional \definame[post=s]{operator notation}
        that represents the \sn{symbol} \emph{itself}.

        The \sn{operator notation} for a \sn{semantic macro}
        \cs{macroname} can be invoked with the syntax
        \cs{macroname}|!|.

        \usemodule{macros?notation} A \sn{Notation?notation}
        is introduced with the \cs{notation}-macro.
    \end{sparagraph}

    \begin{sexample}[for={notation,operator notation}]
        The symbol ``$+$'' is an \sn{operator notation} for 
        \emph{addition}, whereas ``$a + b$'' is a \sn{notation}
        for addition \emph{applied to} two arguments $a$ and $b$.
    \end{sexample}
    
  \end{smodule}
\end{document}
