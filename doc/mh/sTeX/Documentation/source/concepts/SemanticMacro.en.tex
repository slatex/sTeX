\documentclass{stex}
\libinput{docpreamble}
\begin{document}
  \begin{smodule}[title=Semantic Macros]{SemanticMacro}
    \importmodule{concepts?Symbol}
    \usemodule{concepts?Notation}

    \begin{sparagraph}[style={defibox,symdoc},name=semantic macro]
        \stexvarmacro{macroname}

        A \definame{semantic macro} is a \LaTeX-macro that allows 
        for referencing a \sn{symbol} itself, or -- in the case of 
        e.g. a function -- the \emph{application} of a \sn{symbol} to 
        (one or multiple) \emph{arguments}; primarily by invoking a 
        \sn{symbol}'s \sn{notation} in \emph{math mode}.

        The command \cs{symdecl}|{macroname}| declares a new
        \sn{symbol} with name |macroname| and \sn{semantic macro}
        \cs{macroname}. In the case where we want the name and the 
        \sn{semantic macro} to be distinct,
        the command \cs{symdecl}|{macroname}[name=some name]|
        declares the name of the \sn{symbol} to be |some name|
        instead. 

        The starred variant \cs{symdecl}|*{name}| declares the concept
        with the given name, but does not generate a \sn{semantic macro}.
    \end{sparagraph}
    
  \end{smodule}
\end{document}
