\documentclass[lang={en,de}]{stex}
\libinput{docpreamble}
\setsectionlevel{section}
\begin{document}
  \begin{sfragment}{Importing Symbols}

    \usemodule{concepts?Symbol}
    \usemodule{concepts?Module}

    \usemodule[sTeX/MathBase/Functions]{mod?Function}
    \usemodule[sTeX/MathBase/Relations]{mod?Equal}

    There are still a lot of mathematical symbols used in our definition
    that are not semantically marked-up, like the function
    specification
    $\_:\_\times\_\rightarrow\_$, the equality, and the
    symbol $\in$. We can of course introduce them as new symbols,
    but chances are that someone else already has done the work
    for us, and we can reuse the symbols they have already implemented
    for us.

    Indeed, \sns{symbol} for \sns{function}, \sr{eq}{equality} and
    \sr{inset}{set elementhood} already exist. To import them, we need to know
    the math archive they live in (and of course have that
    archive available).

    \textcolor{red}{TODO: math archives, MathHub etc}

    \usemodule{macros?importmodule}

    The module |mod?Function| in the archive |sTeX/MathBase/Functions|
    gives use the semantic macros \cs{funspace} (for function spaces
    $A\to B$) and \cs{fun} (for function specifications $f:A\to B$),
    |mod?Equal| in |sTeX/MathBase/Relations| gives us equality
    \cs{eq},
    and |mod?Set| in |sTeX/MathBase/Sets| gives us elementhood 
    \cs{inset}.

    There are two ways of doing so; namely \cs{importmodule} and
    \cs{usemodule}. The syntax for both is the same:

    \cs{importmodule}|[|(archive name)|]{|(\sn{module} path)|}|
    makes all \sns{symbol} in the specified \sn{module} available to us,
    and also to all \sns{module} which later include the 
    current \sn{module} that imports it.
    \cs{usemodule} only makes the \sns{symbol} available locally, but
    does not export them further.
    
    Since currently, the notion of a monoid does not \emph{depend}
    on the \sns{symbol} we want to import, we will use \cs{usemodule}:

    \stexexamplefile{ex/monoid_4_usemodule.en.tex}[gobble=2,firstline=5,lastline=22,classoffset=1,morekeywords={
    \\mathstruct,\\monoid,\\universe,\\op,\\unit,\\fun,\\eq,\\inset
    }]

    The reason why we can use \cs{inset} in line 15 is that the \sn{module}
    |?Function| uses \cs{importmodule} to import the \sn{module}
    |mod?Set| in the archive |sTeX/MathBase/Sets|. If we had used
    \cs{usemodule} there instead, we would now have to explicitly
    import that \sn{module} as well.
  \end{sfragment}
\end{document}

    