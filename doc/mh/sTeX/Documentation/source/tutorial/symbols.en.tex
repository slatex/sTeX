\documentclass[lang={en,de}]{stex}
\libinput{docpreamble}
\setsectionlevel{section}
\begin{document}

  \begin{sfragment}{Symbol Declarations and References}

    \usemodule{concepts?Symbol}
    \usemodule{concepts?Module}

    The first thing to do is to let the system know that we introduce
    a new \emph{\sn{symbol}} ``monoid'' here. Symbols are always declared in
    \emph{\sns{module}}, which can be imported and reused elsewhere.

    \inputref{concepts/Symbol.en}
    \inputref{concepts/Module.en}

    For now, we simply
    wrap the whole definition in a \env{smodule}-environment, which
    we name |Monoid|.

    \Sn{symbol} declarations can get almost arbitrarily complex,
    but as a start, we will restrict ourselves to the starred
    variant \cs{symdecl}|*|, which will do nothing but declare
    a \sn{symbol} with the given name to exist.
    \usemodule{macros?symrefs}
    We can subsequently \emph{reference} a \sn{symbol} using
    \cs{symname} or \cs{symref}:

    \stexexamplefile{ex/monoid_2_symdecl.en.tex}[gobble=2,firstline=5,lastline=13]

    \cs{symname}|{some symbol}| simply prints the provided \emph{name}
    of the referenced \sn{symbol} and associates the text with that
    very \sn{symbol}. This is a special case for the more general macro
    \cs{symref}|{some symbol}{arbitrary text}|, which prints
    the provided text and associates \emph{that} with the \sn{symbol}. 
  \end{sfragment}

\end{document}