\documentclass[lang={en,de}]{stex}
\libinput{docpreamble}
\setsectionlevel{section}
\begin{document}

\iffalse
\begin{dangerbox} foo \end{dangerbox}
\begin{mmtbox} foo \end{mmtbox}
\begin{sparagraph}[style={defibox,symdoc}]
  foo
\end{sparagraph}
\fi

\ifinputref
  \begin{sfragment}{Tutorial}
\else
  \title{\sTeX Tutorial}
  \maketitle
  \tableofcontents\bigskip
\fi


    We will start with an exemplary \LaTeX\ snippet that introduces
    a new mathematical concept (namely a \emph{monoid}), and gradually
    augment this fragment by introducing \sTeX concepts of
    increasing complexity:

    \stexexamplefile{ex/monoid_1_plain.en.tex}[gobble=2,firstline=5,lastline=9]

    Our goal is to achieve the following instead:

    \begin{sexample}
      \begin{framed}\inputref{ex/Monoid.en.tex}\end{framed}
    \end{sexample}
    The highlighting is of course optional and customizable, but
    as the tooltips (hopefully, depending on your PDF viewer)
    show, this statement is now marked up with the (formal)
    semantics of the individual symbols, terms and even sentences.

    \ifstexhtml\else
      In this document, we use \compemph{this highlighting} for notation
      components, \symrefemph{this highlighting} for symbol references,
      \varemph{this highlighting} for (local) variables and
      and \defemph{this highlighting} for definienda; i.e.
      new concepts being introduced. 
    \fi
    
    \inputref{tutorial/symbols.en}
    \inputref{tutorial/notations.en}
    \inputref{tutorial/inheritance.en}
    \inputref{tutorial/variables.en}
    \inputref{tutorial/sequence-arguments.en}

    \stexexamplefile{ex/monoid_7_mathstructure.en.tex}[gobble=2,firstline=5,lastline=24,classoffset=1,morekeywords={
    \\mathstruct,\\monoid,\\universe,\\op,\\unit,\\fun,\\eq,\\inset
    },classoffset=9,morekeywords={
      \\vx
    }]
    \stexexamplefile{ex/monoid_instances_1.en.tex}[gobble=2,firstline=5,lastline=10,classoffset=1,morekeywords={
    \\mathstruct,\\monoid,\\universe,\\op,\\unit,\\fun,\\eq,\\inset
    },classoffset=9,morekeywords={
      \\vx,\\ma,\\mb
    }]

    \stexexamplefile{ex/monoid_instances_2.en.tex}[gobble=2,firstline=5,lastline=16,classoffset=1,morekeywords={
    \\mathstruct,\\monoid,\\universe,\\op,\\unit,\\fun,\\eq,\\inset,op
    },classoffset=9,morekeywords={
      \\vx,\\ma,\\mb,\\vy,\\vz,\\M
    }]

    \stexexamplefile{ex/monoid_7b_external.en.tex}[gobble=2,firstline=5,lastline=26,classoffset=1,morekeywords={
    \\mathstruct,\\monoid,\\universe,\\op,\\unit,\\fun,\\eq,\\inset
    },classoffset=9,morekeywords={
      \\vx
    }]

    \stexexamplefile{ex/monoid_8_types.en.tex}[gobble=2,firstline=5,lastline=28,classoffset=1,morekeywords={
    \\mathstruct,\\monoid,\\universe,\\op,\\unit,\\fun,\\eq,\\inset,
    \\funspace,\\collection
    },classoffset=9,morekeywords={
      \\vx
    }]

    \stexexamplefile{ex/monoid_9_sdefinition.en.tex}[gobble=2,firstline=5,lastline=28,classoffset=1,morekeywords={
    \\mathstruct,\\monoid,\\universe,\\op,\\unit,\\fun,\\eq,\\inset,
    \\funspace,\\collection
    },classoffset=9,morekeywords={
      \\vx
    }]

    \begin{sexample}
      \begin{mdframed}[linewidth=1pt,backgroundcolor=white]\small
        \hfill File \texttt{stexthm.sty}
        \lstinputlisting[language=sTeX,firstline=47,lastline=56]{stexthm.sty}
      \end{mdframed}
    \end{sexample}

    \begin{sexample}
      \begin{mdframed}[linewidth=1pt,backgroundcolor=white]\small
        \hfill File \texttt{stex-highlighting.sty}
        \lstinputlisting[language=sTeX,firstline=4,lastline=10]{stex-highlighting.sty}
        \lstinputlisting[language=sTeX,firstline=15,lastline=21]{stex-highlighting.sty}
      \end{mdframed}
    \end{sexample}

    \stexexamplefile{ex/Associative.en.tex}[gobble=2,firstline=5,lastline=31,classoffset=1,morekeywords={
    \\fun,\\eq,\\inset,
    \\funspace,\\collection,\\associative
    },classoffset=9,morekeywords={
      \\vx,\\op,\\vy,\\vz,\\vU
    }]

    \stexexamplefile{ex/Associative.de.tex}[gobble=2,firstline=5,lastline=27,classoffset=1,morekeywords={
    \\fun,\\eq,\\inset,
    \\funspace,\\collection,\\associative
    },classoffset=9,morekeywords={
      \\vx,\\op,\\vy,\\vz,\\vU
    }]

    \stexexamplefile{ex/monoid_10_inlineass.en.tex}[gobble=2,firstline=5,lastline=31,classoffset=1,morekeywords={
    \\mathstruct,\\monoid,\\universe,\\op,\\unit,\\fun,\\eq,\\inset,
    \\funspace,\\collection,\\associative
    },classoffset=9,morekeywords={
      \\vx
    }]

    \stexexamplefile{ex/monoid_sequence_1.en.tex}[gobble=2,firstline=5,lastline=9,classoffset=1,morekeywords={
    \\mathstruct,\\monoid,\\universe,\\op,\\unit,\\fun,\\eq,\\inset,
    \\funspace,\\collection,\\associative
    },classoffset=9,morekeywords={
      \\Ms
    }]

    \stexexamplefile{ex/Group.en.tex}[gobble=2,firstline=5,lastline=29,classoffset=1,morekeywords={
    \\mathstruct,\\monoid,\\universe,\\op,\\unit,\\fun,\\eq,\\inset,
    \\funspace,\\collection,\\associative,\\inverse,\\group
    },classoffset=9,morekeywords={
      \\Ms,\\vx
    }]

    \stexexamplefile{ex/Commutative.en.tex}[gobble=2,firstline=5,lastline=30,classoffset=1,morekeywords={
    \\fun,\\eq,\\inset,
    \\funspace,\\collection,\\commutative
    },classoffset=9,morekeywords={
      \\vx,\\op,\\vy,\\vz,\\vU
    }]

    \stexexamplefile{ex/AbelianGroup.en.tex}[gobble=2,firstline=5,lastline=19,classoffset=1,morekeywords={
    \\mathstruct,\\monoid,\\universe,\\op,\\unit,\\fun,\\eq,\\inset,
    \\funspace,\\collection,\\associative,\\inverse,\\group,
    \\commutative
    },classoffset=9,morekeywords={
      \\Ms,\\vx
    }]

    \stexexamplefile{ex/GroupDivision.en.tex}[gobble=2,firstline=5,lastline=20,classoffset=1,morekeywords={
    \\mathstruct,\\monoid,\\universe,\\op,\\unit,\\fun,\\eq,\\inset,
    \\funspace,\\collection,\\associative,\\inverse,\\group,\\notequal
    },classoffset=9,morekeywords={
      \\Ms,\\vx,\\vy
    }]

    \stexexamplefile{ex/Ring.en.tex}[gobble=2,firstline=5,lastline=52,classoffset=1,morekeywords={
    \\mathstruct,\\monoid,\\universe,\\op,\\unit,\\fun,\\eq,\\inset,
    \\funspace,\\collection,\\associative,\\inverse,\\group,\\notequal,
    \\plus,\\mult,\\ring
    },classoffset=9,morekeywords={
      \\Ms,\\vx,\\vy,\\vz
    }]

  \ifinputref\end{sfragment}\fi
  
\end{document}
