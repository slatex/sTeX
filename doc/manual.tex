\makeatletter
\ifcsname if@infulldoc\endcsname\else
    \expandafter\newif\csname if@infulldoc\endcsname\@infulldocfalse
\fi
\makeatother

\csname if@infulldoc\endcsname\else

\def\bibfolder{../lib/bib}

\RequirePackage{paralist}
\documentclass[full,kernel]{l3doc}
\usepackage[dvipsnames]{xcolor}
\usepackage[utf8]{inputenc}
\usepackage[T1]{fontenc}
\RequirePackage{morewrites}
\RequirePackage{tikzinput}
\usetikzlibrary{fit}

\usepackage[debug=all,lang=en, mathhub=./tests]{stex}
\usepackage{url,array,float,textcomp}
\usepackage[show]{ed}
\usepackage[hyperref=auto,style=alphabetic]{biblatex}
\addbibresource{\bibfolder/kwarcpubs.bib}
\addbibresource{\bibfolder/extpubs.bib}
\addbibresource{\bibfolder/kwarccrossrefs.bib}
\addbibresource{\bibfolder/extcrossrefs.bib}
\usepackage{amssymb}
\usepackage{amsfonts}
\usepackage{xspace}
\usepackage{hyperref}

\makeindex
\floatstyle{boxed}
\newfloat{exfig}{thp}{lop}
\floatname{exfig}{Example}

\usepackage{stex-tests}

\MakeShortVerb{\|}

\def\scsys#1{{{\sc #1}}\index{#1@{\sc #1}}\xspace}
\def\mmt{\textsc{Mmt}\xspace}
\def\xml{\scsys{Xml}}
\def\mathml{\scsys{MathML}}
\def\omdoc{\scsys{OMDoc}}
\def\openmath{\scsys{OpenMath}}
\def\latexml{\scsys{LaTeXML}}
\def\perl{\scsys{Perl}}
\def\cmathml{Content-{\sc MathML}\index{Content {\sc MathML}}\index{MathML@{\sc MathML}!content}}
\def\activemath{\scsys{ActiveMath}}
\def\twin#1#2{\index{#1!#2}\index{#2!#1}}
\def\twintoo#1#2{{#1 #2}\twin{#1}{#2}}
\def\atwin#1#2#3{\index{#1!#2!#3}\index{#3!#2 (#1)}}
\def\atwintoo#1#2#3{{#1 #2 #3}\atwin{#1}{#2}{#3}}
\def\cT{\mathcal{T}}\def\cD{\mathcal{D}}

\def\fileversion{3.0}
\def\filedate{\today}

\RequirePackage{pdfcomment}

\ExplSyntaxOn\makeatletter
\cs_set_protected:Npn \@comp #1 #2 {
  \pdftooltip {
    \textcolor{blue}{#1}
  } { #2 }
}

\cs_set_protected:Npn \@defemph #1 #2 {
  \pdftooltip { 
    \textbf{\textcolor{magenta}{#1}}
  } { #2 }
}

\def\__omtext_lec#1{#1}
\cs_new_protected:Npn \lec #1 {
  \strut\hfil\strut\null\hfill\__omtext_lec{#1}
}
\makeatother\ExplSyntaxOff

\makeatletter
\let\@stex@oldcomment\comment
\let\@stex@oldendcomment\endcomment

%\RequirePackage{comment}
\RequirePackage{document-structure}
\RequirePackage[hints,solutions,notes]{problem}
\RequirePackage{hwexam}

\let\comment\@stex@oldcomment
\let\endcomment\@stex@oldendcomment

\newif\ifinfulldoc\infulldocfalse
\makeatother

\def\basedocurl{https://github.com/slatex/sTeX/blob/latex3/doc}
\newcounter{module}

\NewDocumentEnvironment {module}{}{
  \stepcounter{module}
  \textbf{Module \themodule: \smoduletitle}
}{

}
\stexpatchmodule{\begin{module}}{\end{module}}

\def\compemph#1{\textcolor{blue}{#1}}
\def\symrefemph#1{\textcolor{green}{#1}}

\RequirePackage{pdfcomment}
\makeatletter
\protected\def\compemph@uri#1#2{%
  \pdftooltip{%
    \srefsymuri{#2}{\compemph{#1}}%
  }{%
    URI: \detokenize{#2}%
  }%
}
\protected\def\symrefemph@uri#1#2{%
  \pdftooltip{%
    \srefsymuri{#2}{\symrefemph{#1}}%
  }{%
    URI: \detokenize{#2}%
  }%
}
\makeatother

\infulldoctrue

\csname bool_set_true:N\expandafter\endcsname\csname stex_dtx_tests_bool\endcsname

\begin{document}
  \csname if@infulldoc\endcsname\else
	\title{
		The {\stex{3}} Manual
		\thanks{Version {\fileversion} (last revised {\filedate})}
 	}
	\author{Michael Kohlhase, Dennis Müller\\
		FAU Erlangen-Nürnberg\\
		\url{http://kwarc.info/}
	}
	\pagenumbering{roman}
	\maketitle
	
	\input{abstract}\bigskip

  This is the user manual for the \sTeX package and 
  associated software. It is primarily directed at end-users 
  who want to use \sTeX to author semantically
  enriched documents. For the full documentation, see
  \href{\basedocurl/stex.pdf}{the \sTeX documentation}
	
	\makeatletter
		\renewcommand\part{%
    		\clearpage
  			\thispagestyle{plain}%
  			\@tempswafalse
  			\null\vfil
  			\secdef\@part\@spart%
  		}
		\newcounter{chapter}
		\numberwithin{section}{chapter}
		\renewcommand\thechapter{\@arabic\c@chapter}
		\renewcommand\thesection{\thechapter.\@arabic\c@section}
		\newcommand*\chaptermark[1]{}
		\setcounter{secnumdepth}{2}
		\newcommand\@chapapp{\chaptername}
		%\newcommand\chaptername{Chapter}
  		\def\ps@headings{%
    		\let\@oddfoot\@empty
    		\def\@oddhead{{\slshape\rightmark}\hfil\thepage}%
    		\let\@mkboth\markboth
    		\def\chaptermark##1{%
      			\markright{\MakeUppercase{%
        			\ifnum \c@secnumdepth >\m@ne
            			\@chapapp\ \thechapter. \ %
        			\fi
        		##1}}%
        	}%
        }
		\newcommand\chapter{\clearpage
			\thispagestyle{plain}%
			\global\@topnum\z@
			\@afterindentfalse
			\secdef\@chapter\@schapter%
		}
		\def\@chapter[#1]#2{\refstepcounter{chapter}%
			\typeout{\@chapapp\space\thechapter.}%
			\addcontentsline{toc}{chapter}%
				{\protect\numberline{\thechapter}#1}%
			\chaptermark{#1}%
			\addtocontents{lof}{\protect\addvspace{10\p@}}%
			\addtocontents{lot}{\protect\addvspace{10\p@}}%
			\@makechapterhead{#2}%
			\@afterheading%
		}
		\def\@makechapterhead#1{%
			\vspace*{50\p@}%
			{\parindent \z@ \raggedright \normalfont
				\huge\bfseries \@chapapp\space \thechapter
				\par\nobreak
				\vskip 20\p@
				\interlinepenalty\@M
				\Huge \bfseries #1\par\nobreak
				\vskip 40\p@
			}%
		}
\newcommand*\l@chapter[2]{%
  \ifnum \c@tocdepth >\m@ne
    \addpenalty{-\@highpenalty}%
    \vskip 1.0em \@plus\p@
    \setlength\@tempdima{1.5em}%
    \begingroup
      \parindent \z@ \rightskip \@pnumwidth
      \parfillskip -\@pnumwidth
      \leavevmode \bfseries
      \advance\leftskip\@tempdima
      \hskip -\leftskip
      #1\nobreak\hfil
      \nobreak\hb@xt@\@pnumwidth{\hss #2%
                                 \kern-\p@\kern\p@}\par
      \penalty\@highpenalty
    \endgroup
  \fi}
\renewcommand*\l@section{\@dottedtocline{1}{1.5em}{2.8em}}
\renewcommand*\l@subsection{\@dottedtocline{2}{3.8em}{3.2em}}
\renewcommand*\l@subsubsection{\@dottedtocline{3}{7.0em}{4.1em}}
\def\partname{Part}
\def\toclevel@part{-1}
\def\maketitle{\chapter{\@title}}
\let\thanks\@gobble
\let\DelayPrintIndex\PrintIndex
\let\PrintIndex\@empty
\providecommand*{\hexnum}[1]{\text{\texttt{\char`\"}#1}}
\makeatother

\ExplSyntaxOn
\int_set:Nn \l_document_structure_section_level_int {1}
\ExplSyntaxOff

\clearpage

{%
  \def\\{:}% fix "newlines" in the ToC
  \tableofcontents
}

\clearpage
\pagenumbering{arabic}
	
\fi

\long\def\ignore#1{}

\begin{omgroup}{What is \sTeX?}
  
Formal systems for mathematics (such as interactive theorem provers)
have the potential to significantly increase both the accessibility
of published knowledge, as well as the confidence in its veracity,
by rendering the precise semantics of statements machine actionable.
This allows for a plurality of added-value services, from semantic
search up to verification and automated theorem proving.
Unfortunately, their usefulness is hidden behind severe barriers
to accessibility; primarily related to their surface languages
reminiscent of programming languages and very unlike informal
standards of presentation.

\sTeX minimizes this gap between informal and formal 
mathematics by integrating formal methods into established
and widespread authoring workflows, primarily \LaTeX, via 
non-intrusive semantic
annotations of arbitrary informal document fragments. That way
formal knowledge management services become available for informal
documents, accessible via an IDE for authors and via generated
\emph{active} documents for readers, while remaining fully compatible
with existing authoring workflows and publishing systems.

Additionally, an extensible library of reusable
document fragments is being developed, that serve as reference targets
for global disambiguation, intermediaries for content exchange
between systems and other services.

Every component of the system is designed modularly and extensibly,
and thus lay the groundwork for a potential full integration of
interactive theorem proving systems into established informal document
authoring workflows.

\paragraph{} The general \sTeX workflow combines functionalities
provided by several pieces of software:
\begin{itemize}
  \item The \sTeX package to use semantic annotations in
    {\LaTeX} documents,
  \item \RusTeX to convert |tex| sources to (semantically enriched)
    |xhtml|,
  \item The \mmt software, that extracts semantic information
    from the thus generated |xhtml| and provides semantically informed
    added value services.
\end{itemize}


% ----------------------------

\ignore{The objectives of this project will be achieved by developing a 
language and system 
that uses non-intrusive annotations
to augment informal documents with semantic information
(ranging from \textbf{fully formal} to \textbf{purely informal})
 without
impacting linguistic presentation or document layout. 
That way, the system
remains compatible with established publishing
pipelines and practices, while additionally providing flexiformal 
information that
enables formal knowledge management services, and hence produces 
\emph{rich active documents}, satisfying \textbf{R3}, \textbf{R4} and 
\textbf{R5}.
In particular, it will avoid commitment to a fixed logical foundation.
Instead, it will be designed as a modular pipeline of consecutive
and compositional
annotations, semantics extraction and translation steps, extensible
via new structuring mechanisms (\textbf{R1}), library content 
(\textbf{R2}),
NLP techniques, foundations, translation methods and 
end-user services.

Naturally, the benefits of formal knowledge management services scale 
with the amount of mathematics involved. Consequently I will primarily 
focus on those 
STEM fields in which mathematical methods are most prominently
used (e.g. mathematics, physics, computer science). Since in those fields
\LaTeX~is the most commonly used scientific writing tool, I will also
primarily focus on \LaTeX~as a development and evaluation target, but 
the system will be designed such that all components apart from
the surface language will be integrable with other writing tools 
(e.g. WYSIWYG word processors).

\paragraph{} The basic architecture of the proposed system is sketched in
\autoref{fig:architecture}.
\begin{figure}\centering
  \resizebox{0.95\textwidth}{!}{\tikzinput[]{diagram}}
  {\small (Note, that the syntax used
    in the box on the top right is prototypical and subject to change during the project.
    Details and open questions regarding the syntax are discussed here:
    \url{https://github.com/KWARC/FoMID/issues/1})}
  \caption{Basic Architecture of the Proposed System}\label{fig:architecture}
\end{figure}
A user can write their content using standard \LaTeX\ in an IDE;
ideally using semantic annotations provided by \sTeX
%and the library developed in \OBJref{smglom}
(as in the upper right of 
\autoref{fig:architecture}), but not necessarily so.

The document is converted to xhtml with \omdoc annotations
using \LaTeX ML in the background,
thus becoming actionable by the \mmt system. Both the source document
as well as the generated xhtml/\omdoc are accessible to a natural language
processing pripeline that can supply additional inferred semantic 
information or suggest annotations to the user, in the latter case 
augmenting the source document directly. This pipeline can use both 
classical NLP techniques using the GLIF system, as well as machine 
learning models such as \cite{own:fifom}.

A semiformal fragment is converted 
into an appropriate syntax tree (possibly containing opaque
informal nodes), 
thus becoming amenable
to flexiformal knowledge management services. In a consecutive step
-- if sufficiently annotated --, these are
additionally translated
to a fully formal foundation, e.g. using the techniques from 
\cite{DMueller:phd:19,own:translations}, allowing
more powerful services and conversion to established formal
systems. All three representations
are thus available from within the \mmt system for various
knowledge management services, interfaces for which can be
implemented in the IDE.

Importantly, every non-trivial arrow in the figure is 
composable and extensible -- 
translations to a foundation can be provided
by supplying an appropriate formalization and alignment-based
translations (or entirely new methods),
services can be implemented generically using the \mmt API,
NLP techniques can be implemented both inside and alongside of
GLIF, and the concrete syntax within \sTeX can be extended
by convenience macros in \LaTeX\ (enabling new
structuring mechanisms as in \textbf{R1} via 
\mmt extensions, see
\cite{MueRabRot:rslffml20}) as well as via additions to
the library, which will be extensible both from within the IDE
as well as on MathHub,
remaining backwards compatible with existing content in a surface 
language. Additionally, sufficiently disambiguated
statements can be translated to the syntax of 
external systems (such as interactive theorem prover systems
or computer algebra systems),
which can thus be integrated as additional services into the system.
}

\end{omgroup}

\begin{omgroup}{Quickstart}
	\begin{omgroup}{Setup}
		\begin{omgroup}{The \sTeX IDE}
      TODO: VSCode Plugin
    \end{omgroup}
    \begin{omgroup}{Manual Setup}
      Foregoing on the \sTeX IDE, we will need several
      pieces of software; namely:
      \begin{itemize}
        \item \textbf{The \sTeX-Package} available 
          \href{https://github.com/slatex/sTeX/blob/latex3/doc/stex.pdf}{here}%
          \ednote{For now, we require the \texttt{latex3}-branch}.
          Note, that the CTAN repository for \LaTeX{} packages
          may contain outdated versions of the \sTeX package, so
          make sure, that your |TEXMF| system variable is configured such
          that the packages available in the linked repository are prioritized
          over potential default packages that come with your \TeX{} distribution.

          %If you are only interested in using semantic macros in (ultimately)
          %|pdf|s generated by |pdflatex|, this is all you need.

        \item \textbf{The \mmt System} available
          \href{https://github.com/uniformal/MMT/tree/sTeX}{here}%
          \ednote{For now, we require the \texttt{sTeX}-branch, requiring manually
          compiling the MMT sources}. We recommend following
          the setup routine documented 
          \href{https://uniformal.github.io//doc/setup/}{here}.

          Following the setup routine (Step 3) will entail designating
          a |MathHub|-directory on your local file system, where
          the \mmt system will look for \sTeX/\mmt content archives.

        \item To make sure that \sTeX too knows where to find its
          archives, we need to set a global system variable |MATHHUB|,
          that points to your local |MathHub|-directory 
          (see \sref{sec.stexarchives}).
        \item \textbf{\sTeX Archives} If we only care about {\LaTeX} and generating |pdf|s, we do not
          technically need \mmt at all; however, we still need the |MATHHUB|
          system variable to be set. Furthermore, \mmt can make downloading
          content archives we might want to use significantly easier, since
          it makes sure that all dependencies of (often highly interrelated)
          \sTeX archives are cloned as well.

          Once set up, we can run |mmt| in a shell and download an archive along with
          all of its dependencies like this: |lmh install <name-of-repository>|,
          or a whole \emph{group} of archives; for example,
          |lmh install smglom| will download all smglom archives.
        \item \textbf{\RusTeX} The \mmt system will also set up \RusTeX for you,
          which is used to generate (semantically annotated)
          |xhtml| from tex sources. In lieu of using \mmt, you
          can also download and use \RusTeX directly
          \href{https://github.com/slatex/RusTeX}{here}.

      \end{itemize}
    \end{omgroup}
	\end{omgroup}
	\begin{omgroup}{A First \sTeX Document}
    Having set everything up, we can write a first
    \sTeX document. As an example, we will use the
    |smglom/calculus| and |smglom/arithmetics| archives, 
    which should be present in the designated |MathHub|-folder.

    The document we will consider is the following:
    \begin{framed}\begin{latexcode}
\documentclass{article}
\usepackage{stex}
\usepackage{xcolor}
\def\compemph#1{\textcolor{blue}{#1}}

\begin{document}
  \usemodule[smglom/calculus]{series}
  \usemodule[smglom/arithmetics]{realarith}

  The \symref{series}{series} $\infinitesum{n}{1}{
    \realdivide[frac]{1}{
      \realpower{2}{n}
    }
  }$ \symref{converges}{converges} towards $1$.
    
\end{document}
    \end{latexcode}\end{framed}

    Compiling this document with |pdflatex| should yield
    the output

    \begin{framed}
        The \textbf{series} 
        $\textcolor{blue}{\sum}_{n=1}^{\textcolor{blue}\infty} \frac{1}{2^n}$
        \textbf{converges} towards $1$.
    \end{framed}

    Note that the $\sum$ and $\infty$-symbols are highlighted in blue,
    and the words ``series'' and ``converges'' in bold.
    This signifies that these words and symbols 
    reference \sTeX \emph{symbols}
    formally declared somewhere; associating their 
    \emph{presentation} in the document with their (formal)
    definition - i.e. their semantics. The precise way
    in which they are highlighted (if at all) can of course
    be customized (see \ednote{somewhere later}).

    \begin{function}{\usemodule}
      The command |\usemodule[some/archive]{modulename}|
      finds some module in the appropriate archive -- in the first
      case (|\usemodule[smglom/calculus]{series}|), \sTeX
      looks for the archive |smglom/calculus| in our local
      MathHub-directory (see \sref{sec.stexarchives}), and
      in its source-folder for a file |series.tex|. Since no such
      file exists, and by default the document is assumed to be
      in \emph{english}, it picks the file |series.en.tex|, and
      indeed, in here we find a statement |\begin{module}{series}|.
      \iffalse\end{module}\fi
      
      \sTeX now reads this file and makes all semantic macros therein
      available to use, along with all its dependencies.
      This enables the usage of |\infinitesum| later on.

      Analogously, |\usemodule[smglom/arithmetics]{realarith}|
      opens the file |realarith.en.tex| in the |.../smglom/arithmetics/source|-folder
      and makes its contents available, e.g. |\realdivide| and |\realpower|.
    \end{function}

    \begin{function}{\symref,\symname}
      The command |\symref{symbolname}{text}| marks the |text|
      in the second argument as representing the |symbolname|
      in the first argument -- which is why the word ``series''
      is set in boldface. In the pdf, this is all that happens.
      In the |xhtml| (which we will investigate shortly) however,
      we will note that the word ``series'' is now annotated with the
      full URI of the symbol denoting the \emph{mathematical concept of
      a series}. In other words, the word is associated with an unambiguous
      semantics.

      Notably, in both cases above (\emph{series} and \emph{converges})
      the text that \emph{references} the symbol and the name of the symbol
      are identical. Since this occurs quite often, the shorthand
      |\symname{converges}| would have worked as well, where
      |\symname{foo-bar}| behaves exactly like |\symref{foo-bar}{foo bar}|
      - i.e. the text is simply the name of the symbol with ``|-|'' replaced by
      a space.
    \end{function}

    \begin{function}{\importmodule}
      If you investigated the contents of the imported modules 
      (|realarith| and |series|) more closely, you'll note that
      none of them contain a symbol ``|converges|''. Yet, we
      can use |\symref| to refer to ``converges''. That is because
      the symbol |converges| is found in 
      |smglom/calculus/source/sequenceConvergence.en.tex|, and
      |series.en.tex| contains the line
      |\importmodule{sequenceConvergence}|. The |\importmodule|-statement
      makes the module referenced available to all documents
      that include the current module. As such, a ``current module''
      has to exist for |\importmodule| to work, which is why the command
      is only allowed within a |module|-environment.
    \end{function}

    \textcolor{red}{TODO} explain |xhtml| conversion, MMT compilation
    (requires an archive...?).

	\end{omgroup}
\end{omgroup}

\begin{omgroup}{Using Semantic Macros}
	\textcolor{red}{TODO}
  % \iffalse meta-comment
% An Infrastructure for Semantic Macros and Module Scoping
% Copyright (c) 2019 Michael Kohlhase, all rights reserved
%                this file is released under the
%                LaTeX Project Public License (LPPL)
% 
% The original of this file is in the public repository at 
% http://github.com/sLaTeX/sTeX/
%
% TODO update copyright  
%
%<*driver>
\providecommand\bibfolder{../../lib/bib}
\RequirePackage{paralist}
\documentclass[full,kernel]{l3doc}
\usepackage[dvipsnames]{xcolor}
\usepackage[utf8]{inputenc}
\usepackage[T1]{fontenc}
\RequirePackage{morewrites}
\RequirePackage{tikzinput}
\usetikzlibrary{fit}

\usepackage[debug=all,lang=en, mathhub=./tests]{stex}
\usepackage{url,array,float,textcomp}
\usepackage[show]{ed}
\usepackage[hyperref=auto,style=alphabetic]{biblatex}
\addbibresource{\bibfolder/kwarcpubs.bib}
\addbibresource{\bibfolder/extpubs.bib}
\addbibresource{\bibfolder/kwarccrossrefs.bib}
\addbibresource{\bibfolder/extcrossrefs.bib}
\usepackage{amssymb}
\usepackage{amsfonts}
\usepackage{xspace}
\usepackage{hyperref}

\makeindex
\floatstyle{boxed}
\newfloat{exfig}{thp}{lop}
\floatname{exfig}{Example}

\usepackage{stex-tests}

\MakeShortVerb{\|}

\def\scsys#1{{{\sc #1}}\index{#1@{\sc #1}}\xspace}
\def\mmt{\textsc{Mmt}\xspace}
\def\xml{\scsys{Xml}}
\def\mathml{\scsys{MathML}}
\def\omdoc{\scsys{OMDoc}}
\def\openmath{\scsys{OpenMath}}
\def\latexml{\scsys{LaTeXML}}
\def\perl{\scsys{Perl}}
\def\cmathml{Content-{\sc MathML}\index{Content {\sc MathML}}\index{MathML@{\sc MathML}!content}}
\def\activemath{\scsys{ActiveMath}}
\def\twin#1#2{\index{#1!#2}\index{#2!#1}}
\def\twintoo#1#2{{#1 #2}\twin{#1}{#2}}
\def\atwin#1#2#3{\index{#1!#2!#3}\index{#3!#2 (#1)}}
\def\atwintoo#1#2#3{{#1 #2 #3}\atwin{#1}{#2}{#3}}
\def\cT{\mathcal{T}}\def\cD{\mathcal{D}}

\def\fileversion{3.0}
\def\filedate{\today}

\RequirePackage{pdfcomment}

\ExplSyntaxOn\makeatletter
\cs_set_protected:Npn \@comp #1 #2 {
  \pdftooltip {
    \textcolor{blue}{#1}
  } { #2 }
}

\cs_set_protected:Npn \@defemph #1 #2 {
  \pdftooltip { 
    \textbf{\textcolor{magenta}{#1}}
  } { #2 }
}

\def\__omtext_lec#1{#1}
\cs_new_protected:Npn \lec #1 {
  \strut\hfil\strut\null\hfill\__omtext_lec{#1}
}
\makeatother\ExplSyntaxOff

\makeatletter
\let\@stex@oldcomment\comment
\let\@stex@oldendcomment\endcomment

%\RequirePackage{comment}
\RequirePackage{document-structure}
\RequirePackage[hints,solutions,notes]{problem}
\RequirePackage{hwexam}

\let\comment\@stex@oldcomment
\let\endcomment\@stex@oldendcomment

\newif\ifinfulldoc\infulldocfalse
\makeatother

\def\basedocurl{https://github.com/slatex/sTeX/blob/latex3/doc}
\newcounter{module}

\NewDocumentEnvironment {module}{}{
  \stepcounter{module}
  \textbf{Module \themodule: \smoduletitle}
}{

}
\stexpatchmodule{\begin{module}}{\end{module}}

\def\compemph#1{\textcolor{blue}{#1}}
\def\symrefemph#1{\textcolor{green}{#1}}

\RequirePackage{pdfcomment}
\makeatletter
\protected\def\compemph@uri#1#2{%
  \pdftooltip{%
    \srefsymuri{#2}{\compemph{#1}}%
  }{%
    URI: \detokenize{#2}%
  }%
}
\protected\def\symrefemph@uri#1#2{%
  \pdftooltip{%
    \srefsymuri{#2}{\symrefemph{#1}}%
  }{%
    URI: \detokenize{#2}%
  }%
}
\makeatother

\begin{document}
  \DocInput{\jobname.dtx}
\end{document}
%</driver>
% \fi
%
% \title{ \sTeX-Terms
% 	\thanks{Version {\fileversion} (last revised {\filedate})} 
% }
%
% \author{Michael Kohlhase, Dennis Müller\\
% 	FAU Erlangen-Nürnberg\\
% 	\url{http://kwarc.info/}
% }
%
% \maketitle
%
%\ifinfulldoc\else
% This is the documentation for the \pkg{stex-terms} package.
% For a more high-level introduction, 
%  see \href{\basedocurl/manual.pdf}{the \sTeX Manual} or the
% \href{\basedocurl/stex.pdf}{full \sTeX documentation}.
%
% % \iffalse meta-comment
% An Infrastructure for Semantic Macros and Module Scoping
% Copyright (c) 2019 Michael Kohlhase, all rights reserved
%                this file is released under the
%                LaTeX Project Public License (LPPL)
% 
% The original of this file is in the public repository at 
% http://github.com/sLaTeX/sTeX/
%
% TODO update copyright  
%
%<*driver>
\providecommand\bibfolder{../../lib/bib}
\RequirePackage{paralist}
\documentclass[full,kernel]{l3doc}
\usepackage[dvipsnames]{xcolor}
\usepackage[utf8]{inputenc}
\usepackage[T1]{fontenc}
\RequirePackage{morewrites}
\RequirePackage{tikzinput}
\usetikzlibrary{fit}

\usepackage[debug=all,lang=en, mathhub=./tests]{stex}
\usepackage{url,array,float,textcomp}
\usepackage[show]{ed}
\usepackage[hyperref=auto,style=alphabetic]{biblatex}
\addbibresource{\bibfolder/kwarcpubs.bib}
\addbibresource{\bibfolder/extpubs.bib}
\addbibresource{\bibfolder/kwarccrossrefs.bib}
\addbibresource{\bibfolder/extcrossrefs.bib}
\usepackage{amssymb}
\usepackage{amsfonts}
\usepackage{xspace}
\usepackage{hyperref}

\makeindex
\floatstyle{boxed}
\newfloat{exfig}{thp}{lop}
\floatname{exfig}{Example}

\usepackage{stex-tests}

\MakeShortVerb{\|}

\def\scsys#1{{{\sc #1}}\index{#1@{\sc #1}}\xspace}
\def\mmt{\textsc{Mmt}\xspace}
\def\xml{\scsys{Xml}}
\def\mathml{\scsys{MathML}}
\def\omdoc{\scsys{OMDoc}}
\def\openmath{\scsys{OpenMath}}
\def\latexml{\scsys{LaTeXML}}
\def\perl{\scsys{Perl}}
\def\cmathml{Content-{\sc MathML}\index{Content {\sc MathML}}\index{MathML@{\sc MathML}!content}}
\def\activemath{\scsys{ActiveMath}}
\def\twin#1#2{\index{#1!#2}\index{#2!#1}}
\def\twintoo#1#2{{#1 #2}\twin{#1}{#2}}
\def\atwin#1#2#3{\index{#1!#2!#3}\index{#3!#2 (#1)}}
\def\atwintoo#1#2#3{{#1 #2 #3}\atwin{#1}{#2}{#3}}
\def\cT{\mathcal{T}}\def\cD{\mathcal{D}}

\def\fileversion{3.0}
\def\filedate{\today}

\RequirePackage{pdfcomment}

\ExplSyntaxOn\makeatletter
\cs_set_protected:Npn \@comp #1 #2 {
  \pdftooltip {
    \textcolor{blue}{#1}
  } { #2 }
}

\cs_set_protected:Npn \@defemph #1 #2 {
  \pdftooltip { 
    \textbf{\textcolor{magenta}{#1}}
  } { #2 }
}

\def\__omtext_lec#1{#1}
\cs_new_protected:Npn \lec #1 {
  \strut\hfil\strut\null\hfill\__omtext_lec{#1}
}
\makeatother\ExplSyntaxOff

\makeatletter
\let\@stex@oldcomment\comment
\let\@stex@oldendcomment\endcomment

%\RequirePackage{comment}
\RequirePackage{document-structure}
\RequirePackage[hints,solutions,notes]{problem}
\RequirePackage{hwexam}

\let\comment\@stex@oldcomment
\let\endcomment\@stex@oldendcomment

\newif\ifinfulldoc\infulldocfalse
\makeatother

\def\basedocurl{https://github.com/slatex/sTeX/blob/latex3/doc}
\newcounter{module}

\NewDocumentEnvironment {module}{}{
  \stepcounter{module}
  \textbf{Module \themodule: \smoduletitle}
}{

}
\stexpatchmodule{\begin{module}}{\end{module}}

\def\compemph#1{\textcolor{blue}{#1}}
\def\symrefemph#1{\textcolor{green}{#1}}

\RequirePackage{pdfcomment}
\makeatletter
\protected\def\compemph@uri#1#2{%
  \pdftooltip{%
    \srefsymuri{#2}{\compemph{#1}}%
  }{%
    URI: \detokenize{#2}%
  }%
}
\protected\def\symrefemph@uri#1#2{%
  \pdftooltip{%
    \srefsymuri{#2}{\symrefemph{#1}}%
  }{%
    URI: \detokenize{#2}%
  }%
}
\makeatother

\begin{document}
  \DocInput{\jobname.dtx}
\end{document}
%</driver>
% \fi
%
% \title{ \sTeX-Terms
% 	\thanks{Version {\fileversion} (last revised {\filedate})} 
% }
%
% \author{Michael Kohlhase, Dennis Müller\\
% 	FAU Erlangen-Nürnberg\\
% 	\url{http://kwarc.info/}
% }
%
% \maketitle
%
%\ifinfulldoc\else
% This is the documentation for the \pkg{stex-terms} package.
% For a more high-level introduction, 
%  see \href{\basedocurl/manual.pdf}{the \sTeX Manual} or the
% \href{\basedocurl/stex.pdf}{full \sTeX documentation}.
%
% % \iffalse meta-comment
% An Infrastructure for Semantic Macros and Module Scoping
% Copyright (c) 2019 Michael Kohlhase, all rights reserved
%                this file is released under the
%                LaTeX Project Public License (LPPL)
% 
% The original of this file is in the public repository at 
% http://github.com/sLaTeX/sTeX/
%
% TODO update copyright  
%
%<*driver>
\providecommand\bibfolder{../../lib/bib}
\input{../../doc/docheader}

\begin{document}
  \DocInput{\jobname.dtx}
\end{document}
%</driver>
% \fi
%
% \title{ \sTeX-Terms
% 	\thanks{Version {\fileversion} (last revised {\filedate})} 
% }
%
% \author{Michael Kohlhase, Dennis Müller\\
% 	FAU Erlangen-Nürnberg\\
% 	\url{http://kwarc.info/}
% }
%
% \maketitle
%
%\ifinfulldoc\else
% This is the documentation for the \pkg{stex-terms} package.
% For a more high-level introduction, 
%  see \href{\basedocurl/manual.pdf}{the \sTeX Manual} or the
% \href{\basedocurl/stex.pdf}{full \sTeX documentation}.
%
% \input{../../doc/packages/terms}
% \fi
%
% \begin{documentation}\label{pkg:terms:doc}
%
% Code related to symbolic expressions, typesetting notations,
% notation components, etc.
%
% \section{Macros and Environments}\label{pkg:terms:doc:macros}
%
% \begin{function}{\STEXsymbol}
%   Uses \cs{stex_get_symbol:n} to find the symbol denoted by
%   the first argument and passes the result on to
%   \cs{stex_invoke_symbol:n}
% \end{function}
%
% \begin{function}{\symref}
%   \begin{syntax} \cs{symref}\Arg{symbol}\Arg{text} \end{syntax}
%   shortcut for \cs{STEXsymbol}\Arg{symbol}|![|\meta{text}|]|
% \end{function}
%
% \begin{function}{\stex_invoke_symbol:n}
%   Executes a semantic macro. Outside of math mode or if followed by |*|,
%   it continues to \cs{stex_term_custom:nn}. In math mode,
%   it uses the default or optionally provided notation of
%   the associated symbol.
%
%   If followed by |!|, it will invoke the symbol \emph{itself}
%   rather than its application (and continue to
%   \cs{stex_term_custom:nn}), i.e. it allows to refer to
%   |\plus![addition]| as an operation, rather than
%   |\plus[addition of]{some}{terms}|.
% \end{function}
%
% \begin{function}{\_stex_term_math_oms:nnnn,\_stex_term_math_oma:nnnn,\_stex_term_math_omb:nnnn}
%   \begin{syntax} \meta{URI}\meta{fragment}\meta{precedence}\meta{body} \end{syntax}
%
% Annotates \meta{body} as an \omdoc-term (|OMID|, |OMA| or |OMBIND|, respectively) 
% with head symbol \meta{URI}, generated
% by the specific notation \meta{fragment} with (upwards) operator precedence
% \meta{precedence}. Inserts parentheses according to
% the current downwards precedence and operator precedence.
% \end{function}
%
% \begin{function}{\_stex_term_math_arg:nnn}
%   \begin{syntax} \cs{stex_term_arg:nnn}\meta{int}\meta{prec}\meta{body} \end{syntax}
% Annotates \meta{body} as the \meta{int}th argument of the current |OMA| or |OMBIND|,
% with (downwards) argument precedence \meta{prec}.
% \end{function}
%
% \begin{function}{\_stex_term_math_assoc_arg:nnnn}
%   \begin{syntax} \cs{stex_term_arg:nnn}\meta{int}\meta{prec}\meta{notation}\meta{body} \end{syntax}
% Annotates \meta{body} as the \meta{int}th (associative) \emph{sequence} argument
% (as comma-separated list of terms) of the current |OMA| or |OMBIND|,
% with (downwards) argument precedence \meta{prec} and associative
% notation \meta{notation}.
% 
% \end{function}
%
% \begin{variable}{\infprec, \neginfprec}
%   Maximal and minimal notation precedences.
% \end{variable}
%
% \begin{function}{\dobrackets}
%   \begin{syntax} \cs{dobrackets} \Arg{body} \end{syntax}
%   Puts \meta{body} in parentheses; scaled if in display mode
%   unscaled otherwise. Uses the current \sTeX brackets (by default |(| and |)|),
%   which can be changed temporarily using \cs{withbrackets}.
% \end{function}
%
% \begin{function}{\withbrackets}
%   \begin{syntax} \cs{withbrackets} \meta{left} \meta{right} \Arg{body} \end{syntax}
%   Temporarily (i.e. within \meta{body}) sets the brackets used by \sTeX for automated
%   bracketing (by default |(| and |)|) to \meta{left} and \meta{right}.
%
%   Note that \meta{left} and \meta{right} need to be allowed
%   after \cs{left} and \cs{right} in displaymode.
% \end{function}
%
% \begin{function}{\stex_term_custom:nn}
%   \begin{syntax} \cs{stex_term_custom:nn}\Arg{URI}\Arg{args}\end{syntax}
% Implements custom one-time notation.
% Invoked by \cs{stex_invoke_symbol:n} in text mode, or if
% followed by |*| in math mode, or whenever followed by |!|.
% \end{function}
%
% \begin{function}{\stex_highlight_term:nn}
%   \begin{syntax} \cs{stex_highlight_term:nn}\Arg{URI}\Arg{args}\end{syntax}
% Establishes a context for \cs{comp}. Stores the URI in a variable
% so that \cs{comp} knows which symbol governs the current notation.
% \end{function}
%
% \begin{function}{\comp, \compemph,\compemph@uri, \defemph, \defemph@uri, \symrefemph,\symrefemph@uri}
%   \begin{syntax} \cs{comp}\Arg{args}\end{syntax}
% Marks \meta{args} as a notation component of the current symbol for
% highlighting, linking, etc.
%
% The precise behavior is governed by \cs{@comp}, which takes as
% additional argument the URI of the current symbol. By default,
% \cs{@comp} adds the URI as a PDF tooltip and colors the highlighted part
% in blue.
%
% \cs{@defemph} behaves like \cs{@comp}, and can be similarly redefined,
% but marks an expression as \emph{definiendum} (used by \cs{definiendum})
% \end{function}
%
% \begin{function}{\STEXinvisible}
% Exports its argument as \omdoc (invisible), but does
% not produce PDF output. Useful e.g. for semantic macros
% that take arguments that are not part of the symbolic
% notation.
% \end{function}
%
% \begin{function}{\ellipses}
%   TODO
% \end{function}
%
% \end{documentation}
%
% \begin{implementation}\label{pkg:terms:impl}
%
% \section{\sTeX-Terms Implementation}
%
%    \begin{macrocode}
%<*package>

%%%%%%%%%%%%%   terms.dtx   %%%%%%%%%%%%%

%<@@=stex_terms>
%    \end{macrocode}
%
% Warnings and error messages
%
%    \begin{macrocode}
\msg_new:nnn{stex}{error/nonotation}{
  Symbol~#1~invoked,~but~has~no~notation#2!
}
\msg_new:nnn{stex}{error/notationarg}{
  Error~in~parsing~notation~#1
}
\msg_new:nnn{stex}{error/noop}{
  Symbol~#1~has~no~operator~notation~for~notation~#2
}
\msg_new:nnn{stex}{error/notallowed}{
  Symbol~invokation~#1~not~allowed~in~notation~component~of~#2
}

%    \end{macrocode}
% \subsection{Symbol Invokations}
%
%
% \begin{macro}{\stex_invoke_symbol:n}
%
%  Invokes a semantic macro
%
%    \begin{macrocode}
\keys_define:nn { stex / terms } {
  lang    .tl_set_x:N = \l_@@_lang_str ,
  variant .tl_set_x:N = \l_@@_variant_str ,
  unknown .code:n     = \str_set:Nx 
      \l_@@_variant_str \l_keys_key_str
}

\cs_new_protected:Nn \_@@_args:n {
  \str_clear:N \l_@@_lang_str
  \str_clear:N \l_@@_variant_str
  
  \keys_set:nn { stex / terms } { #1 }
}

\cs_new:Nn \_@@_reset:N {
  \tl_if_exist:NTF #1 {
    \def \exp_not:N #1 { \exp_args:No \exp_not:n #1 }
  }{
    \let \exp_not:N #1 \exp_not:N \undefined
  }
}

\bool_new:N \l_@@_allow_semantic_bool
\bool_set_true:N \l_@@_allow_semantic_bool

\cs_new_protected:Nn \stex_invoke_symbol:n {
  \bool_if:NTF \l_@@_allow_semantic_bool {
    \str_if_eq:eeF {
      \prop_item:cn {
        l_stex_symdecl_#1_prop
      }{ deprecate }
    }{}{
      \msg_warning:nnxx{stex}{warning/deprecated}{
        Symbol~#1
      }{
        \prop_item:cn {l_stex_symdecl_#1_prop}{ deprecate }
      }
    }
    \if_mode_math:
      \exp_after:wN \_@@_invoke_math:n
    \else:
      \exp_after:wN \_@@_invoke_text:n
    \fi: { #1 }
  }{
    \msg_error:nnxx{stex}{error/notallowed}{#1}{\l_stex_current_symbol_str}
  }
}

\cs_new_protected:Nn \_@@_invoke_text:n {
  \peek_charcode_remove:NTF ! {
    \_@@_invoke_op_custom:nn {#1}
  }{
    \_@@_invoke_custom:nn {#1}
  }
}

\cs_new_protected:Nn \_@@_invoke_math:n {
  \peek_charcode_remove:NTF ! {
    % operator
    \peek_charcode_remove:NTF * {
      % custom op
      \_@@_invoke_op_custom:nn {#1}
    }{
      % op notation
      \peek_charcode:NTF [ {
        \_@@_invoke_op_notation:nw {#1}
      }{
        \_@@_invoke_op_notation:nw {#1}[]
      }
    }
  }{
    \peek_charcode_remove:NTF * {
      \_@@_invoke_custom:nn {#1}
      % custom
    }{
      % normal
      \peek_charcode:NTF [ {
        \_@@_invoke_notation:nw {#1}
      }{
        \_@@_invoke_notation:nw {#1}[]
      }
    }
  }
}


\cs_new_protected:Nn \_@@_invoke_op_custom:nn {
  \exp_args:Nnx \use:nn {
    \str_set:Nn \l_stex_current_symbol_str { #1 }
    \bool_set_false:N \l_@@_allow_semantic_bool
    \_stex_term_oms:nnn {#1 \c_hash_str\c_hash_str}{#1}{
      \comp{ #2 }
    }
  }{
    \_@@_reset:N \l_stex_current_symbol_str
    \bool_set_true:N \l_@@_allow_semantic_bool
  }
}

\cs_new_protected:Nn \_@@_find_notation:nn {
  \str_set:Nn \l_stex_current_symbol_str { #1 }
  \_@@_args:n { #2 }
  \seq_if_empty:cTF {
    l_stex_symdecl_ #1 _notations 
  } {
    \msg_error:nnxx{stex}{error/nonotation}{#1}{s}
  } {
    \bool_lazy_all:nTF {
      {\str_if_empty_p:N \l_@@_variant_str}
      {\str_if_empty_p:N \l_@@_lang_str}
    }{
      \seq_get_left:cN {l_stex_symdecl_#1_notations}\l_@@_variant_str
    }{
      \seq_if_in:cxTF {l_stex_symdecl_#1_notations}{
        \l_@@_variant_str \c_hash_str \l_@@_lang_str
      }{
        \str_set:Nx \l_@@_variant_str { \l_@@_variant_str \c_hash_str \l_@@_lang_str }
      }{
        \msg_error:nnxx{stex}{error/nonotation}{#1}{
          ~\l_@@_variant_str \c_hash_str \l_@@_lang_str
        }
      }
    }
  }
}

\cs_new_protected:Npn \_@@_invoke_op_notation:nw #1 [#2] {
  \_@@_find_notation:nn { #1 }{ #2 }
  \bool_set_false:N \l_@@_allow_semantic_bool
  \cs_if_exist:cTF {
    stex_op_notation_ #1 \c_hash_str \l_@@_variant_str _cs
  }{
    \use:c{stex_op_notation_ #1 \c_hash_str \l_@@_variant_str _cs}
  }{
    \msg_error:nnxx{stex}{error/noop}{#1}{\l_@@_variant_str}
  }
  \bool_set_true:N \l_@@_allow_semantic_bool
}

\cs_new_protected:Npn \_@@_invoke_notation:nw #1 [#2] {
  \_@@_find_notation:nn { #1 }{ #2 }
  \cs_if_exist:cTF {
    stex_notation_ #1 \c_hash_str \l_@@_variant_str _cs
  }{
    \tl_set:Nx \stex_symbol_after_invokation_tl {
      \_@@_reset:N \stex_symbol_after_invokation_tl
      \_@@_reset:N \l_stex_current_symbol_str
      \bool_set_true:N \l_@@_allow_semantic_bool
    }
    \bool_set_false:N \l_@@_allow_semantic_bool
    \use:c{stex_notation_ #1 \c_hash_str \l_@@_variant_str _cs}
  }{
    \msg_error:nnxx{stex}{error/nonotation}{#1}{
      ~\l_@@_variant_str
    }
  }
}

\prop_new:N \l_@@_custom_args_prop

\cs_new_protected:Nn \_@@_invoke_custom:nn {
  \exp_args:Nnx \use:nn {
    \bool_set_false:N \l_@@_allow_semantic_bool
    \str_set:Nn \l_stex_current_symbol_str { #1 }
    \prop_clear:N \l_@@_custom_args_prop
    \prop_put:Nnn \l_@@_custom_args_prop {currnum} {1}
    \prop_put:Nnx \l_@@_custom_args_prop {args} {
      \prop_item:cn {
        l_stex_symdecl_\l_stex_get_symbol_uri_str _prop
      }{ args }
    }
    \tl_set:Nn \arg { \_@@_arg: }
    #2
    % TODO check that all arguments exist
  }{
    \_@@_reset:N \l_stex_current_symbol_str
    \_@@_reset:N \arg
    \_@@_reset:N \l_@@_custom_args_prop
    \bool_set_true:N \l_@@_allow_semantic_bool
  }
}

\NewDocumentCommand \_@@_arg: { s O{} m}{
  \tl_if_empty:nTF {#2}{
    \int_set:Nn \l_tmpa_int {\prop_item:Nn \l_@@_custom_args_prop {currnum}}
    \bool_set_true:N \l_tmpa_bool
    \bool_do_while:Nn \l_tmpa_bool {
      \exp_args:NNx \prop_if_in:NnTF \l_@@_custom_args_prop {\int_use:N \l_tmpa_int} {
        \int_incr:N \l_tmpa_int
      }{
        \bool_set_false:N \l_tmpa_bool
      }
    }
  }{
    \int_set:Nn \l_tmpa_int { #2 }
    \exp_args:NNx \prop_if_in:NnT \l_@@_custom_args_prop {\int_use:N \l_tmpa_int} {
      % TODO throw error
    }
  }
  \str_set:Nx \l_tmpa_str {\prop_item:Nn \l_@@_custom_args_prop {args} }
  \int_compare:nNnT \l_tmpa_int > {\str_count:N \l_tmpa_str} {
    % TODO throw error
  }
  \IfBooleanTF#1{
    \stex_annotate_invisible:n {
      \exp_args:No \_stex_term_arg:nn {\l_stex_current_symbol_str}{#3}
    }
  }{
    \exp_args:No \_stex_term_arg:nn {\l_stex_current_symbol_str}{#3}
  }
}


\cs_new_protected:Nn \_stex_term_arg:nn {
  \exp_args:Nnx \use:nn {
    \bool_set_true:N \l_@@_allow_semantic_bool
    \stex_annotate:nnn{ arg }{ #1 }{ #2 }
  }{
    \bool_set_false:N \l_@@_allow_semantic_bool
  }
}

\cs_new_protected:Nn \_stex_term_math_arg:nnn {
  \exp_args:Nnx \use:nn
    { \int_set:Nn \l_@@_downprec { #2 } 
        \_stex_term_arg:nn { #1 }{ #3 }
    }
    { \int_set:Nn \exp_not:N \l_@@_downprec { \int_use:N \l_@@_downprec } }
}


%    \end{macrocode}
% \end{macro}
%
% \subsection{Terms}
%
% Precedences:
% \begin{variable}{\infprec, \neginfprec, \l_@@_downprec}
%    \begin{macrocode}
\tl_const:Nx \infprec {\int_use:N \c_max_int}
\tl_const:Nx \neginfprec {-\int_use:N \c_max_int}
\int_new:N \l_@@_downprec
\int_set_eq:NN \l_@@_downprec \infprec
%    \end{macrocode}
% \end{variable}
%
% Bracketing:
%
% \begin{variable}{\l_@@_left_bracket_str, \l_@@_right_bracket_str}
%    \begin{macrocode}
\tl_set:Nn \l_@@_left_bracket_str (
\tl_set:Nn \l_@@_right_bracket_str )
%    \end{macrocode}
% \end{variable}
%
% \begin{macro}{\_@@_maybe_brackets:nn}
%
% Compares precedences and insert brackets accordingly
%
%    \begin{macrocode}
\cs_new_protected:Nn \_@@_maybe_brackets:nn {
  \bool_if:NTF \l_@@_brackets_done_bool {
    \bool_set_false:N \l_@@_brackets_done_bool
    #2
  } {
    \int_compare:nNnTF { #1 } > \l_@@_downprec {
      \bool_if:NTF \l_stex_inparray_bool { #2 }{
        \stex_debug:nn{dobrackets}{\number#1 > \number\l_@@_downprec; \detokenize{#2}}
        \dobrackets { #2 }
      }
    }{ #2 }
  }
}
%    \end{macrocode}
% \end{macro}
%
% \begin{macro}{\dobrackets}
%    \begin{macrocode}
\bool_new:N \l_@@_brackets_done_bool
%\RequirePackage{scalerel}
\cs_new_protected:Npn \dobrackets #1 {
  %\ThisStyle{\if D\m@switch
  %    \exp_args:Nnx \use:nn
  %    { \exp_after:wN \left\l_@@_left_bracket_str #1 } 
  %    { \exp_not:N\right\l_@@_right_bracket_str }
  %  \else
      \exp_args:Nnx \use:nn
      { 
        \bool_set_true:N \l_@@_brackets_done_bool
        \int_set:Nn \l_@@_downprec \infprec
        \l_@@_left_bracket_str 
        #1
      } 
      {
        \bool_set_false:N \l_@@_brackets_done_bool
        \l_@@_right_bracket_str 
        \int_set:Nn \l_@@_downprec { \int_use:N \l_@@_downprec }
      }
  %\fi}
}
%    \end{macrocode}
% \end{macro}
%
% \begin{macro}{\withbrackets}
%    \begin{macrocode}
\cs_new_protected:Npn \withbrackets #1 #2 #3 {
  \exp_args:Nnx \use:nn
  {  
    \tl_set:Nx \l_@@_left_bracket_str { #1 }
    \tl_set:Nx \l_@@_right_bracket_str { #2 }
    #3
  }
  {
    \tl_set:Nn \exp_not:N \l_@@_left_bracket_str 
      {\l_@@_left_bracket_str}
    \tl_set:Nn \exp_not:N \l_@@_right_bracket_str 
      {\l_@@_right_bracket_str}
  }
}
%    \end{macrocode}
% \end{macro}
%
% \begin{macro}{\STEXinvisible}
%    \begin{macrocode}
\cs_new_protected:Npn \STEXinvisible #1 {
  \stex_annotate_invisible:n { #1 }
}
%    \end{macrocode}
% \end{macro}
%
% \omdoc terms:
%
% \begin{macro}{\_stex_term_math_oms:nnnn}
%    \begin{macrocode}
\cs_new_protected:Nn \_stex_term_oms:nnn {
  \stex_annotate:nnn{ OMID }{ #2 }{
    \stex_highlight_term:nn { #1 } { #3 } 
  }
}

\cs_new_protected:Nn \_stex_term_math_oms:nnnn {
  \_@@_maybe_brackets:nn { #3 }{ 
    \_stex_term_oms:nnn { #1 } { #1\c_hash_str#2 } { #4 }
  }
}
%    \end{macrocode}
% \end{macro}
%
% \begin{macro}{\_stex_term_math_omv:nn}
%    \begin{macrocode}
\cs_new_protected:Nn \_stex_term_omv:nn {
  \stex_annotate:nnn{ OMID }{ #1 }{
    \stex_highlight_term:nn { #1 } { #2 } 
  }
}
%    \end{macrocode}
% \end{macro}
%
% \begin{macro}{\_stex_term_math_oma:nnnn}
%    \begin{macrocode}
\cs_new_protected:Nn \_stex_term_oma:nnn {
  \stex_annotate:nnn{ OMA }{ #2 }{
    \stex_highlight_term:nn { #1 } { #3 } 
  }
}

\cs_new_protected:Nn \_stex_term_math_oma:nnnn {
  \_@@_maybe_brackets:nn { #3 }{ 
    \_stex_term_oma:nnn { #1 } { #1\c_hash_str#2 } { #4 }
  }
}
%    \end{macrocode}
% \end{macro}
%
% \begin{macro}{\_stex_term_math_omb:nnnn}
%    \begin{macrocode}
\cs_new_protected:Nn \_stex_term_ombind:nnn {
  \stex_annotate:nnn{ OMBIND }{ #2 }{
    \stex_highlight_term:nn { #1 } { #3 }
  }
}

\cs_new_protected:Nn \_stex_term_math_omb:nnnn {
  \_@@_maybe_brackets:nn { #3 }{ 
    \_stex_term_ombind:nnn { #1 } { #1\c_hash_str#2 } { #4 }
  }
}
%    \end{macrocode}
% \end{macro}
%
% \begin{macro}{\_stex_term_math_assoc_arg:nnnn}
%    \begin{macrocode}
\cs_new_protected:Nn \_stex_term_math_assoc_arg:nnnn {
  % TODO sequences
  \clist_set:Nn \l_tmpa_clist{ #3 }
  \int_compare:nNnTF { \clist_count:N \l_tmpa_clist } < 2 {
    \tl_set:Nn \l_tmpa_tl { #3 }
  }{
    \cs_set:Npn \l_tmpa_cs ##1 ##2 { #4 }
    \clist_reverse:N \l_tmpa_clist
    \clist_pop:NN \l_tmpa_clist \l_tmpa_tl

    \clist_map_inline:Nn \l_tmpa_clist {
      \exp_args:NNNo \exp_args:NNo \tl_set:No \l_tmpa_tl {
        \exp_args:Nno 
        \l_tmpa_cs { ##1 } \l_tmpa_tl 
      }
    }
  }
  \exp_args:Nnno
   \_stex_term_math_arg:nnn{#1}{#2}\l_tmpa_tl
}
%    \end{macrocode}
% \end{macro}
%
% \begin{macro}{\stex_term_custom:nn}
%    \begin{macrocode}
\cs_new_protected:Nn \stex_term_custom:nn {
  \str_set:Nn \l_@@_custom_uri { #1 }
  \str_set:Nn \l_tmpa_str { #2 }
  \tl_clear:N \l_tmpa_tl
  \int_zero:N \l_tmpa_int
  \int_set:Nn \l_tmpb_int { \str_count:N \l_tmpa_str }
  \_@@_custom_loop:
}
%    \end{macrocode}
% \end{macro}
%
% \begin{macro}{\_@@_custom_loop:}
%    \begin{macrocode}
\cs_new_protected:Nn \_@@_custom_loop: {
  \bool_set_false:N \l_tmpa_bool
  \bool_while_do:nn {
    \str_if_eq_p:ee X {
      \str_item:Nn \l_tmpa_str { \l_tmpa_int + 1 }
    }
  }{
    \int_incr:N \l_tmpa_int
  }

  \peek_charcode:NTF [ {
    % notation/text component
    \_@@_custom_component:w
  } {
    \int_compare:nNnTF \l_tmpa_int = \l_tmpb_int {
      % all arguments read => finish
      \_@@_custom_final:
    } {
      % arguments missing
      \peek_charcode_remove:NTF * {
        % invisible, specific argument position or both
        \peek_charcode:NTF [ {
          % visible specific argument position
          \_@@_custom_arg:wn
        } {
          % invisible
          \peek_charcode_remove:NTF * {
            % invisible specific argument position
            \_@@_custom_arg_inv:wn
          } {
            % invisible next argument
            \_@@_custom_arg_inv:wn [ \l_tmpa_int + 1 ]
          }
        } 
      } {
        % next normal argument
        \_@@_custom_arg:wn [ \l_tmpa_int + 1 ]
      }
    }
  }
}
%    \end{macrocode}
% \end{macro}
%
% \begin{macro}{\_@@_custom_arg_inv:wn}
%    \begin{macrocode}
\cs_new_protected:Npn \_@@_custom_arg_inv:wn [ #1 ] #2 {
  \bool_set_true:N \l_tmpa_bool
  \_@@_custom_arg:wn [ #1 ] { #2 }
}
%    \end{macrocode}
% \end{macro}
%
% \begin{macro}{\_@@_custom_arg:wn}
%    \begin{macrocode}
\cs_new_protected:Npn \_@@_custom_arg:wn [ #1 ] #2 {
  \str_set:Nx \l_tmpb_str { 
    \str_item:Nn \l_tmpa_str { #1 }
  }
  \str_case:VnTF \l_tmpb_str {
    { X } {
      \msg_error:nnx{stex}{error/notationarg}{\l_@@_custom_uri}
    }
    { i } { \_@@_custom_set_X:n { #1 } }
    { b } { \_@@_custom_set_X:n { #1 } }
    { a } { \_@@_custom_set_X:n { #1 } } % TODO ?
    { B } { \_@@_custom_set_X:n { #1 } } % TODO ?
  }{}{
    \msg_error:nnx{stex}{error/notationarg}{\l_@@_custom_uri}
  }

  \bool_if:nTF \l_tmpa_bool {
    \tl_put_right:Nx \l_tmpa_tl {
      \stex_annotate_invisible:n {
        \_stex_term_arg:nn { \int_eval:n { #1 } } 
          \exp_not:n { { #2 } }
      }
    }
  } {
    \tl_put_right:Nx \l_tmpa_tl {
      \_stex_term_arg:nn { \int_eval:n { #1 } } 
        \exp_not:n { { #2 } }
    }
  }

  \_@@_custom_loop:
}
%    \end{macrocode}
% \end{macro}
%
% \begin{macro}{\_@@_custom_set_X:n}
%    \begin{macrocode}
\cs_new_protected:Nn \_@@_custom_set_X:n {
  \str_set:Nx \l_tmpa_str {
    \str_range:Nnn \l_tmpa_str 1 { #1 - 1 }
    X
    \str_range:Nnn \l_tmpa_str { #1 + 1 } { -1 }
  }
}
%    \end{macrocode}
% \end{macro}
%
% \begin{macro}{\_@@_custom_component:}
%    \begin{macrocode}
\cs_new_protected:Npn \_@@_custom_component:w [ #1 ] {
  \tl_put_right:Nn \l_tmpa_tl { \comp{ #1 } }
  \_@@_custom_loop:
}
%    \end{macrocode}
% \end{macro}
%
% \begin{macro}{\_@@_custom_final:}
%    \begin{macrocode}
\cs_new_protected:Nn \_@@_custom_final: {
  \int_compare:nNnTF \l_tmpb_int = 0 {
    \exp_args:Nnno \_stex_term_oms:nnn
  }{
    \str_if_in:NnTF \l_tmpa_str {b} {
      \exp_args:Nnno \_stex_term_ombind:nnn
    } {
      \exp_args:Nnno \_stex_term_oma:nnn
    }
  }
  { \l_@@_custom_uri } { \l_@@_custom_uri } { \l_tmpa_tl }
}
%    \end{macrocode}
% \end{macro}
%
% \begin{macro}{\symref,\symname}
%    \begin{macrocode}
\cs_new:Nn \stex_capitalize:n { \uppercase{#1} }

\keys_define:nn { stex / symname } {
  pre     .tl_set_x:N    = \l_@@_pre_tl ,
  post    .tl_set_x:N    = \l_@@_post_tl ,
  root    .tl_set_x:N    = \l_@@_root_tl
}

\cs_new_protected:Nn \stex_symname_args:n {
  \tl_clear:N \l_@@_post_tl
  \tl_clear:N \l_@@_pre_tl
  \tl_clear:N \l_@@_root_str
  \keys_set:nn { stex / symname } { #1 }
}

\NewDocumentCommand \symref { m m }{
  \let\compemph_uri_prev:\compemph@uri
  \let\compemph@uri\symrefemph@uri
  \STEXsymbol{#1}![ #2 ]
  \let\compemph@uri\compemph_uri_prev:
}

\NewDocumentCommand \synonym { O{} m m}{
  \stex_symname_args:n { #1 }
  \let\compemph_uri_prev:\compemph@uri
  \let\compemph@uri\symrefemph@uri
  % TODO
  \STEXsymbol{#2}![\l_@@_pre_tl #3 \l_@@_post_tl]
  \let\compemph@uri\compemph_uri_prev:
}

\NewDocumentCommand \symname { O{} m }{
  \stex_symname_args:n { #1 }
  \stex_get_symbol:n { #2 }
  \str_set:Nx \l_tmpa_str {
    \prop_item:cn { l_stex_symdecl_ \l_stex_get_symbol_uri_str _prop } { name }
  }
  \exp_args:NNno \str_replace_all:Nnn \l_tmpa_str {-} {~}
  
  \let\compemph_uri_prev:\compemph@uri
  \let\compemph@uri\symrefemph@uri
  \exp_args:NNx \use:nn
  \stex_invoke_symbol:n { { \l_stex_get_symbol_uri_str }![
    \l_@@_pre_tl \l_tmpa_str \l_@@_post_tl
  ] }
  \let\compemph@uri\compemph_uri_prev:
}

\NewDocumentCommand \Symname { O{} m }{
  \stex_symname_args:n { #1 }
  \stex_get_symbol:n { #2 }
  \str_set:Nx \l_tmpa_str {
    \prop_item:cn { l_stex_symdecl_ \l_stex_get_symbol_uri_str _prop } { name }
  }
  \exp_args:NNno \str_replace_all:Nnn \l_tmpa_str {-} {~}
  \let\compemph_uri_prev:\compemph@uri
  \let\compemph@uri\symrefemph@uri
  \exp_args:NNx \use:nn
  \stex_invoke_symbol:n { { \l_stex_get_symbol_uri_str }![
    \exp_after:wN \stex_capitalize:n \l_tmpa_str
      \l_@@_post_tl
  ] }
  \let\compemph@uri\compemph_uri_prev:
}
%    \end{macrocode}
% \end{macro}
%
%
% \subsection{Notation Components}
%    \begin{macrocode}
%<@@=stex_notationcomps>
%    \end{macrocode}
%
%
% \begin{macro}{\stex_highlight_term:nn}
%    \begin{macrocode}

\str_new:N \l_stex_current_symbol_str
\cs_new_protected:Nn \stex_highlight_term:nn {
  \exp_args:Nnx
  \use:nn {
    \str_set:Nx \l_stex_current_symbol_str { #1 }
    #2
  } {
    \str_set:Nx \exp_not:N \l_stex_current_symbol_str
      { \l_stex_current_symbol_str }
  }
}

\cs_new_protected:Nn \stex_unhighlight_term:n {
%  \latexml_if:TF {
%    #1
%  } {
%    \rustex_if:TF {
%      #1
%    } {
      #1 %\iffalse{{\fi}} #1 {{\iffalse}}\fi
%    }
%  }
}
%    \end{macrocode}
% \end{macro}
%
% \begin{macro}{\comp,\compemph@uri,\compemph,\defemph,\defemph@uri,\symrefemph,\symrefemph@uri}
%    \begin{macrocode}
\cs_new_protected:Npn \comp #1 {
  \str_if_empty:NF \l_stex_current_symbol_str {
    \rustex_if:TF {
      \stex_annotate:nnn { comp }{ \l_stex_current_symbol_str }{ #1 }
    }{
      \exp_args:Nnx \compemph@uri { #1 } { \l_stex_current_symbol_str }
    }
  }
}

\cs_new_protected:Npn \compemph@uri #1 #2 {
    \compemph{ #1 }
}


\cs_new_protected:Npn \compemph #1 {
    #1
}

\cs_new_protected:Npn \defemph@uri #1 #2 {
    \defemph{#1}
}

\cs_new_protected:Npn \defemph #1 {
    \textbf{#1}
}

\cs_new_protected:Npn \symrefemph@uri #1 #2 {
    \symrefemph{#1}
}

\cs_new_protected:Npn \symrefemph #1 {
    \textbf{#1}
}
%    \end{macrocode}
% \end{macro}
%
%
% \begin{macro}{\ellipses}
%    \begin{macrocode}
\NewDocumentCommand \ellipses {} { \ldots }
%    \end{macrocode}
% \end{macro}
%
%
% \begin{macro}{\parray,\prmatrix,\parrayline,\parraylineh,\parraycell}
%    \begin{macrocode}
\bool_new:N \l_stex_inparray_bool
\bool_set_false:N \l_stex_inparray_bool
\NewDocumentCommand \parray { m m } {
  \begingroup 
  \bool_set_true:N \l_stex_inparray_bool
  \begin{array}{#1}
    #2
  \end{array}
  \endgroup
}

\NewDocumentCommand \prmatrix { m } {
  \begingroup 
  \bool_set_true:N \l_stex_inparray_bool
  \begin{matrix}
    #1
  \end{matrix}
  \endgroup
}

\def \maybephline {
  \bool_if:NT \l_stex_inparray_bool {\hline}
}

\def \parrayline #1 #2 {
  #1 #2 \bool_if:NT \l_stex_inparray_bool {\\}
}

\def \pmrow #1 { \parrayline{}{ #1 } }

\def \parraylineh #1 #2 {
  #1 #2 \bool_if:NT \l_stex_inparray_bool {\\\hline}
}

\def \parraycell #1 {
  #1 \bool_if:NT \l_stex_inparray_bool {&}
}
%    \end{macrocode}
% \end{macro}
%
% \subsection{Variables}
%    \begin{macrocode}
%<@@=stex_variables>
%    \end{macrocode}
%
% \begin{macro}{\stex_invoke_variable:n}
%
%  Invokes a variable
%
%    \begin{macrocode}
\cs_new_protected:Nn \stex_invoke_variable:n {
  \if_mode_math:
    \exp_after:wN \_@@_invoke_math:n
  \else:
    \exp_after:wN \_@@_invoke_text:n
  \fi: {#1}
}

\cs_new_protected:Nn \_@@_invoke_text:n {
  %TODO
}


\cs_new_protected:Nn \_@@_invoke_math:n {
  \peek_charcode_remove:NTF ! {
    \peek_charcode_remove:NTF ! {
      \peek_charcode:NTF [ {
        \_@@_invoke_op_custom:nw
      }{
        % TODO throw error
      }
    }{
      \_@@_invoke_op:n { #1 }
    }
  }{
    \peek_charcode_remove:NTF * {
      \_@@_invoke_text:n { #1 }
    }{
      \_@@_invoke_math_ii:n { #1 }
    }
  }
}

\cs_new_protected:Nn \_@@_invoke_op:n {
  \cs_if_exist:cTF {
    stex_var_op_notation_ #1 _cs
  }{
    \use:c{stex_var_op_notation_ #1  _cs }
  }{
    \msg_error:nnxx{stex}{error/noop}{variable~#1}{}
  }
}

\cs_new_protected:Npn \_@@_invoke_math_ii:n  #1 {
  \cs_if_exist:cTF {
    stex_var_notation_#1_cs
  }{
    \str_set:Nn \l_stex_current_symbol_str { #1 }
    \use:c{stex_var_notation_#1_cs}
  }{
    \msg_error:nnxx{stex}{error/nonotation}{variable~#1}{s}
  }
}
%    \end{macrocode}
% \end{macro}
%
%
%    \begin{macrocode}
%</package>
%    \end{macrocode}
%
% \end{implementation}
%
% \PrintIndex

% \endinput
% Local Variables:
% mode: doctex
% TeX-master: t
% End:

% \fi
%
% \begin{documentation}\label{pkg:terms:doc}
%
% Code related to symbolic expressions, typesetting notations,
% notation components, etc.
%
% \section{Macros and Environments}\label{pkg:terms:doc:macros}
%
% \begin{function}{\STEXsymbol}
%   Uses \cs{stex_get_symbol:n} to find the symbol denoted by
%   the first argument and passes the result on to
%   \cs{stex_invoke_symbol:n}
% \end{function}
%
% \begin{function}{\symref}
%   \begin{syntax} \cs{symref}\Arg{symbol}\Arg{text} \end{syntax}
%   shortcut for \cs{STEXsymbol}\Arg{symbol}|![|\meta{text}|]|
% \end{function}
%
% \begin{function}{\stex_invoke_symbol:n}
%   Executes a semantic macro. Outside of math mode or if followed by |*|,
%   it continues to \cs{stex_term_custom:nn}. In math mode,
%   it uses the default or optionally provided notation of
%   the associated symbol.
%
%   If followed by |!|, it will invoke the symbol \emph{itself}
%   rather than its application (and continue to
%   \cs{stex_term_custom:nn}), i.e. it allows to refer to
%   |\plus![addition]| as an operation, rather than
%   |\plus[addition of]{some}{terms}|.
% \end{function}
%
% \begin{function}{\_stex_term_math_oms:nnnn,\_stex_term_math_oma:nnnn,\_stex_term_math_omb:nnnn}
%   \begin{syntax} \meta{URI}\meta{fragment}\meta{precedence}\meta{body} \end{syntax}
%
% Annotates \meta{body} as an \omdoc-term (|OMID|, |OMA| or |OMBIND|, respectively) 
% with head symbol \meta{URI}, generated
% by the specific notation \meta{fragment} with (upwards) operator precedence
% \meta{precedence}. Inserts parentheses according to
% the current downwards precedence and operator precedence.
% \end{function}
%
% \begin{function}{\_stex_term_math_arg:nnn}
%   \begin{syntax} \cs{stex_term_arg:nnn}\meta{int}\meta{prec}\meta{body} \end{syntax}
% Annotates \meta{body} as the \meta{int}th argument of the current |OMA| or |OMBIND|,
% with (downwards) argument precedence \meta{prec}.
% \end{function}
%
% \begin{function}{\_stex_term_math_assoc_arg:nnnn}
%   \begin{syntax} \cs{stex_term_arg:nnn}\meta{int}\meta{prec}\meta{notation}\meta{body} \end{syntax}
% Annotates \meta{body} as the \meta{int}th (associative) \emph{sequence} argument
% (as comma-separated list of terms) of the current |OMA| or |OMBIND|,
% with (downwards) argument precedence \meta{prec} and associative
% notation \meta{notation}.
% 
% \end{function}
%
% \begin{variable}{\infprec, \neginfprec}
%   Maximal and minimal notation precedences.
% \end{variable}
%
% \begin{function}{\dobrackets}
%   \begin{syntax} \cs{dobrackets} \Arg{body} \end{syntax}
%   Puts \meta{body} in parentheses; scaled if in display mode
%   unscaled otherwise. Uses the current \sTeX brackets (by default |(| and |)|),
%   which can be changed temporarily using \cs{withbrackets}.
% \end{function}
%
% \begin{function}{\withbrackets}
%   \begin{syntax} \cs{withbrackets} \meta{left} \meta{right} \Arg{body} \end{syntax}
%   Temporarily (i.e. within \meta{body}) sets the brackets used by \sTeX for automated
%   bracketing (by default |(| and |)|) to \meta{left} and \meta{right}.
%
%   Note that \meta{left} and \meta{right} need to be allowed
%   after \cs{left} and \cs{right} in displaymode.
% \end{function}
%
% \begin{function}{\stex_term_custom:nn}
%   \begin{syntax} \cs{stex_term_custom:nn}\Arg{URI}\Arg{args}\end{syntax}
% Implements custom one-time notation.
% Invoked by \cs{stex_invoke_symbol:n} in text mode, or if
% followed by |*| in math mode, or whenever followed by |!|.
% \end{function}
%
% \begin{function}{\stex_highlight_term:nn}
%   \begin{syntax} \cs{stex_highlight_term:nn}\Arg{URI}\Arg{args}\end{syntax}
% Establishes a context for \cs{comp}. Stores the URI in a variable
% so that \cs{comp} knows which symbol governs the current notation.
% \end{function}
%
% \begin{function}{\comp, \compemph,\compemph@uri, \defemph, \defemph@uri, \symrefemph,\symrefemph@uri}
%   \begin{syntax} \cs{comp}\Arg{args}\end{syntax}
% Marks \meta{args} as a notation component of the current symbol for
% highlighting, linking, etc.
%
% The precise behavior is governed by \cs{@comp}, which takes as
% additional argument the URI of the current symbol. By default,
% \cs{@comp} adds the URI as a PDF tooltip and colors the highlighted part
% in blue.
%
% \cs{@defemph} behaves like \cs{@comp}, and can be similarly redefined,
% but marks an expression as \emph{definiendum} (used by \cs{definiendum})
% \end{function}
%
% \begin{function}{\STEXinvisible}
% Exports its argument as \omdoc (invisible), but does
% not produce PDF output. Useful e.g. for semantic macros
% that take arguments that are not part of the symbolic
% notation.
% \end{function}
%
% \begin{function}{\ellipses}
%   TODO
% \end{function}
%
% \end{documentation}
%
% \begin{implementation}\label{pkg:terms:impl}
%
% \section{\sTeX-Terms Implementation}
%
%    \begin{macrocode}
%<*package>

%%%%%%%%%%%%%   terms.dtx   %%%%%%%%%%%%%

%<@@=stex_terms>
%    \end{macrocode}
%
% Warnings and error messages
%
%    \begin{macrocode}
\msg_new:nnn{stex}{error/nonotation}{
  Symbol~#1~invoked,~but~has~no~notation#2!
}
\msg_new:nnn{stex}{error/notationarg}{
  Error~in~parsing~notation~#1
}
\msg_new:nnn{stex}{error/noop}{
  Symbol~#1~has~no~operator~notation~for~notation~#2
}
\msg_new:nnn{stex}{error/notallowed}{
  Symbol~invokation~#1~not~allowed~in~notation~component~of~#2
}

%    \end{macrocode}
% \subsection{Symbol Invokations}
%
%
% \begin{macro}{\stex_invoke_symbol:n}
%
%  Invokes a semantic macro
%
%    \begin{macrocode}
\keys_define:nn { stex / terms } {
  lang    .tl_set_x:N = \l_@@_lang_str ,
  variant .tl_set_x:N = \l_@@_variant_str ,
  unknown .code:n     = \str_set:Nx 
      \l_@@_variant_str \l_keys_key_str
}

\cs_new_protected:Nn \_@@_args:n {
  \str_clear:N \l_@@_lang_str
  \str_clear:N \l_@@_variant_str
  
  \keys_set:nn { stex / terms } { #1 }
}

\cs_new:Nn \_@@_reset:N {
  \tl_if_exist:NTF #1 {
    \def \exp_not:N #1 { \exp_args:No \exp_not:n #1 }
  }{
    \let \exp_not:N #1 \exp_not:N \undefined
  }
}

\bool_new:N \l_@@_allow_semantic_bool
\bool_set_true:N \l_@@_allow_semantic_bool

\cs_new_protected:Nn \stex_invoke_symbol:n {
  \bool_if:NTF \l_@@_allow_semantic_bool {
    \str_if_eq:eeF {
      \prop_item:cn {
        l_stex_symdecl_#1_prop
      }{ deprecate }
    }{}{
      \msg_warning:nnxx{stex}{warning/deprecated}{
        Symbol~#1
      }{
        \prop_item:cn {l_stex_symdecl_#1_prop}{ deprecate }
      }
    }
    \if_mode_math:
      \exp_after:wN \_@@_invoke_math:n
    \else:
      \exp_after:wN \_@@_invoke_text:n
    \fi: { #1 }
  }{
    \msg_error:nnxx{stex}{error/notallowed}{#1}{\l_stex_current_symbol_str}
  }
}

\cs_new_protected:Nn \_@@_invoke_text:n {
  \peek_charcode_remove:NTF ! {
    \_@@_invoke_op_custom:nn {#1}
  }{
    \_@@_invoke_custom:nn {#1}
  }
}

\cs_new_protected:Nn \_@@_invoke_math:n {
  \peek_charcode_remove:NTF ! {
    % operator
    \peek_charcode_remove:NTF * {
      % custom op
      \_@@_invoke_op_custom:nn {#1}
    }{
      % op notation
      \peek_charcode:NTF [ {
        \_@@_invoke_op_notation:nw {#1}
      }{
        \_@@_invoke_op_notation:nw {#1}[]
      }
    }
  }{
    \peek_charcode_remove:NTF * {
      \_@@_invoke_custom:nn {#1}
      % custom
    }{
      % normal
      \peek_charcode:NTF [ {
        \_@@_invoke_notation:nw {#1}
      }{
        \_@@_invoke_notation:nw {#1}[]
      }
    }
  }
}


\cs_new_protected:Nn \_@@_invoke_op_custom:nn {
  \exp_args:Nnx \use:nn {
    \str_set:Nn \l_stex_current_symbol_str { #1 }
    \bool_set_false:N \l_@@_allow_semantic_bool
    \_stex_term_oms:nnn {#1 \c_hash_str\c_hash_str}{#1}{
      \comp{ #2 }
    }
  }{
    \_@@_reset:N \l_stex_current_symbol_str
    \bool_set_true:N \l_@@_allow_semantic_bool
  }
}

\cs_new_protected:Nn \_@@_find_notation:nn {
  \str_set:Nn \l_stex_current_symbol_str { #1 }
  \_@@_args:n { #2 }
  \seq_if_empty:cTF {
    l_stex_symdecl_ #1 _notations 
  } {
    \msg_error:nnxx{stex}{error/nonotation}{#1}{s}
  } {
    \bool_lazy_all:nTF {
      {\str_if_empty_p:N \l_@@_variant_str}
      {\str_if_empty_p:N \l_@@_lang_str}
    }{
      \seq_get_left:cN {l_stex_symdecl_#1_notations}\l_@@_variant_str
    }{
      \seq_if_in:cxTF {l_stex_symdecl_#1_notations}{
        \l_@@_variant_str \c_hash_str \l_@@_lang_str
      }{
        \str_set:Nx \l_@@_variant_str { \l_@@_variant_str \c_hash_str \l_@@_lang_str }
      }{
        \msg_error:nnxx{stex}{error/nonotation}{#1}{
          ~\l_@@_variant_str \c_hash_str \l_@@_lang_str
        }
      }
    }
  }
}

\cs_new_protected:Npn \_@@_invoke_op_notation:nw #1 [#2] {
  \_@@_find_notation:nn { #1 }{ #2 }
  \bool_set_false:N \l_@@_allow_semantic_bool
  \cs_if_exist:cTF {
    stex_op_notation_ #1 \c_hash_str \l_@@_variant_str _cs
  }{
    \use:c{stex_op_notation_ #1 \c_hash_str \l_@@_variant_str _cs}
  }{
    \msg_error:nnxx{stex}{error/noop}{#1}{\l_@@_variant_str}
  }
  \bool_set_true:N \l_@@_allow_semantic_bool
}

\cs_new_protected:Npn \_@@_invoke_notation:nw #1 [#2] {
  \_@@_find_notation:nn { #1 }{ #2 }
  \cs_if_exist:cTF {
    stex_notation_ #1 \c_hash_str \l_@@_variant_str _cs
  }{
    \tl_set:Nx \stex_symbol_after_invokation_tl {
      \_@@_reset:N \stex_symbol_after_invokation_tl
      \_@@_reset:N \l_stex_current_symbol_str
      \bool_set_true:N \l_@@_allow_semantic_bool
    }
    \bool_set_false:N \l_@@_allow_semantic_bool
    \use:c{stex_notation_ #1 \c_hash_str \l_@@_variant_str _cs}
  }{
    \msg_error:nnxx{stex}{error/nonotation}{#1}{
      ~\l_@@_variant_str
    }
  }
}

\prop_new:N \l_@@_custom_args_prop

\cs_new_protected:Nn \_@@_invoke_custom:nn {
  \exp_args:Nnx \use:nn {
    \bool_set_false:N \l_@@_allow_semantic_bool
    \str_set:Nn \l_stex_current_symbol_str { #1 }
    \prop_clear:N \l_@@_custom_args_prop
    \prop_put:Nnn \l_@@_custom_args_prop {currnum} {1}
    \prop_put:Nnx \l_@@_custom_args_prop {args} {
      \prop_item:cn {
        l_stex_symdecl_\l_stex_get_symbol_uri_str _prop
      }{ args }
    }
    \tl_set:Nn \arg { \_@@_arg: }
    #2
    % TODO check that all arguments exist
  }{
    \_@@_reset:N \l_stex_current_symbol_str
    \_@@_reset:N \arg
    \_@@_reset:N \l_@@_custom_args_prop
    \bool_set_true:N \l_@@_allow_semantic_bool
  }
}

\NewDocumentCommand \_@@_arg: { s O{} m}{
  \tl_if_empty:nTF {#2}{
    \int_set:Nn \l_tmpa_int {\prop_item:Nn \l_@@_custom_args_prop {currnum}}
    \bool_set_true:N \l_tmpa_bool
    \bool_do_while:Nn \l_tmpa_bool {
      \exp_args:NNx \prop_if_in:NnTF \l_@@_custom_args_prop {\int_use:N \l_tmpa_int} {
        \int_incr:N \l_tmpa_int
      }{
        \bool_set_false:N \l_tmpa_bool
      }
    }
  }{
    \int_set:Nn \l_tmpa_int { #2 }
    \exp_args:NNx \prop_if_in:NnT \l_@@_custom_args_prop {\int_use:N \l_tmpa_int} {
      % TODO throw error
    }
  }
  \str_set:Nx \l_tmpa_str {\prop_item:Nn \l_@@_custom_args_prop {args} }
  \int_compare:nNnT \l_tmpa_int > {\str_count:N \l_tmpa_str} {
    % TODO throw error
  }
  \IfBooleanTF#1{
    \stex_annotate_invisible:n {
      \exp_args:No \_stex_term_arg:nn {\l_stex_current_symbol_str}{#3}
    }
  }{
    \exp_args:No \_stex_term_arg:nn {\l_stex_current_symbol_str}{#3}
  }
}


\cs_new_protected:Nn \_stex_term_arg:nn {
  \exp_args:Nnx \use:nn {
    \bool_set_true:N \l_@@_allow_semantic_bool
    \stex_annotate:nnn{ arg }{ #1 }{ #2 }
  }{
    \bool_set_false:N \l_@@_allow_semantic_bool
  }
}

\cs_new_protected:Nn \_stex_term_math_arg:nnn {
  \exp_args:Nnx \use:nn
    { \int_set:Nn \l_@@_downprec { #2 } 
        \_stex_term_arg:nn { #1 }{ #3 }
    }
    { \int_set:Nn \exp_not:N \l_@@_downprec { \int_use:N \l_@@_downprec } }
}


%    \end{macrocode}
% \end{macro}
%
% \subsection{Terms}
%
% Precedences:
% \begin{variable}{\infprec, \neginfprec, \l_@@_downprec}
%    \begin{macrocode}
\tl_const:Nx \infprec {\int_use:N \c_max_int}
\tl_const:Nx \neginfprec {-\int_use:N \c_max_int}
\int_new:N \l_@@_downprec
\int_set_eq:NN \l_@@_downprec \infprec
%    \end{macrocode}
% \end{variable}
%
% Bracketing:
%
% \begin{variable}{\l_@@_left_bracket_str, \l_@@_right_bracket_str}
%    \begin{macrocode}
\tl_set:Nn \l_@@_left_bracket_str (
\tl_set:Nn \l_@@_right_bracket_str )
%    \end{macrocode}
% \end{variable}
%
% \begin{macro}{\_@@_maybe_brackets:nn}
%
% Compares precedences and insert brackets accordingly
%
%    \begin{macrocode}
\cs_new_protected:Nn \_@@_maybe_brackets:nn {
  \bool_if:NTF \l_@@_brackets_done_bool {
    \bool_set_false:N \l_@@_brackets_done_bool
    #2
  } {
    \int_compare:nNnTF { #1 } > \l_@@_downprec {
      \bool_if:NTF \l_stex_inparray_bool { #2 }{
        \stex_debug:nn{dobrackets}{\number#1 > \number\l_@@_downprec; \detokenize{#2}}
        \dobrackets { #2 }
      }
    }{ #2 }
  }
}
%    \end{macrocode}
% \end{macro}
%
% \begin{macro}{\dobrackets}
%    \begin{macrocode}
\bool_new:N \l_@@_brackets_done_bool
%\RequirePackage{scalerel}
\cs_new_protected:Npn \dobrackets #1 {
  %\ThisStyle{\if D\m@switch
  %    \exp_args:Nnx \use:nn
  %    { \exp_after:wN \left\l_@@_left_bracket_str #1 } 
  %    { \exp_not:N\right\l_@@_right_bracket_str }
  %  \else
      \exp_args:Nnx \use:nn
      { 
        \bool_set_true:N \l_@@_brackets_done_bool
        \int_set:Nn \l_@@_downprec \infprec
        \l_@@_left_bracket_str 
        #1
      } 
      {
        \bool_set_false:N \l_@@_brackets_done_bool
        \l_@@_right_bracket_str 
        \int_set:Nn \l_@@_downprec { \int_use:N \l_@@_downprec }
      }
  %\fi}
}
%    \end{macrocode}
% \end{macro}
%
% \begin{macro}{\withbrackets}
%    \begin{macrocode}
\cs_new_protected:Npn \withbrackets #1 #2 #3 {
  \exp_args:Nnx \use:nn
  {  
    \tl_set:Nx \l_@@_left_bracket_str { #1 }
    \tl_set:Nx \l_@@_right_bracket_str { #2 }
    #3
  }
  {
    \tl_set:Nn \exp_not:N \l_@@_left_bracket_str 
      {\l_@@_left_bracket_str}
    \tl_set:Nn \exp_not:N \l_@@_right_bracket_str 
      {\l_@@_right_bracket_str}
  }
}
%    \end{macrocode}
% \end{macro}
%
% \begin{macro}{\STEXinvisible}
%    \begin{macrocode}
\cs_new_protected:Npn \STEXinvisible #1 {
  \stex_annotate_invisible:n { #1 }
}
%    \end{macrocode}
% \end{macro}
%
% \omdoc terms:
%
% \begin{macro}{\_stex_term_math_oms:nnnn}
%    \begin{macrocode}
\cs_new_protected:Nn \_stex_term_oms:nnn {
  \stex_annotate:nnn{ OMID }{ #2 }{
    \stex_highlight_term:nn { #1 } { #3 } 
  }
}

\cs_new_protected:Nn \_stex_term_math_oms:nnnn {
  \_@@_maybe_brackets:nn { #3 }{ 
    \_stex_term_oms:nnn { #1 } { #1\c_hash_str#2 } { #4 }
  }
}
%    \end{macrocode}
% \end{macro}
%
% \begin{macro}{\_stex_term_math_omv:nn}
%    \begin{macrocode}
\cs_new_protected:Nn \_stex_term_omv:nn {
  \stex_annotate:nnn{ OMID }{ #1 }{
    \stex_highlight_term:nn { #1 } { #2 } 
  }
}
%    \end{macrocode}
% \end{macro}
%
% \begin{macro}{\_stex_term_math_oma:nnnn}
%    \begin{macrocode}
\cs_new_protected:Nn \_stex_term_oma:nnn {
  \stex_annotate:nnn{ OMA }{ #2 }{
    \stex_highlight_term:nn { #1 } { #3 } 
  }
}

\cs_new_protected:Nn \_stex_term_math_oma:nnnn {
  \_@@_maybe_brackets:nn { #3 }{ 
    \_stex_term_oma:nnn { #1 } { #1\c_hash_str#2 } { #4 }
  }
}
%    \end{macrocode}
% \end{macro}
%
% \begin{macro}{\_stex_term_math_omb:nnnn}
%    \begin{macrocode}
\cs_new_protected:Nn \_stex_term_ombind:nnn {
  \stex_annotate:nnn{ OMBIND }{ #2 }{
    \stex_highlight_term:nn { #1 } { #3 }
  }
}

\cs_new_protected:Nn \_stex_term_math_omb:nnnn {
  \_@@_maybe_brackets:nn { #3 }{ 
    \_stex_term_ombind:nnn { #1 } { #1\c_hash_str#2 } { #4 }
  }
}
%    \end{macrocode}
% \end{macro}
%
% \begin{macro}{\_stex_term_math_assoc_arg:nnnn}
%    \begin{macrocode}
\cs_new_protected:Nn \_stex_term_math_assoc_arg:nnnn {
  % TODO sequences
  \clist_set:Nn \l_tmpa_clist{ #3 }
  \int_compare:nNnTF { \clist_count:N \l_tmpa_clist } < 2 {
    \tl_set:Nn \l_tmpa_tl { #3 }
  }{
    \cs_set:Npn \l_tmpa_cs ##1 ##2 { #4 }
    \clist_reverse:N \l_tmpa_clist
    \clist_pop:NN \l_tmpa_clist \l_tmpa_tl

    \clist_map_inline:Nn \l_tmpa_clist {
      \exp_args:NNNo \exp_args:NNo \tl_set:No \l_tmpa_tl {
        \exp_args:Nno 
        \l_tmpa_cs { ##1 } \l_tmpa_tl 
      }
    }
  }
  \exp_args:Nnno
   \_stex_term_math_arg:nnn{#1}{#2}\l_tmpa_tl
}
%    \end{macrocode}
% \end{macro}
%
% \begin{macro}{\stex_term_custom:nn}
%    \begin{macrocode}
\cs_new_protected:Nn \stex_term_custom:nn {
  \str_set:Nn \l_@@_custom_uri { #1 }
  \str_set:Nn \l_tmpa_str { #2 }
  \tl_clear:N \l_tmpa_tl
  \int_zero:N \l_tmpa_int
  \int_set:Nn \l_tmpb_int { \str_count:N \l_tmpa_str }
  \_@@_custom_loop:
}
%    \end{macrocode}
% \end{macro}
%
% \begin{macro}{\_@@_custom_loop:}
%    \begin{macrocode}
\cs_new_protected:Nn \_@@_custom_loop: {
  \bool_set_false:N \l_tmpa_bool
  \bool_while_do:nn {
    \str_if_eq_p:ee X {
      \str_item:Nn \l_tmpa_str { \l_tmpa_int + 1 }
    }
  }{
    \int_incr:N \l_tmpa_int
  }

  \peek_charcode:NTF [ {
    % notation/text component
    \_@@_custom_component:w
  } {
    \int_compare:nNnTF \l_tmpa_int = \l_tmpb_int {
      % all arguments read => finish
      \_@@_custom_final:
    } {
      % arguments missing
      \peek_charcode_remove:NTF * {
        % invisible, specific argument position or both
        \peek_charcode:NTF [ {
          % visible specific argument position
          \_@@_custom_arg:wn
        } {
          % invisible
          \peek_charcode_remove:NTF * {
            % invisible specific argument position
            \_@@_custom_arg_inv:wn
          } {
            % invisible next argument
            \_@@_custom_arg_inv:wn [ \l_tmpa_int + 1 ]
          }
        } 
      } {
        % next normal argument
        \_@@_custom_arg:wn [ \l_tmpa_int + 1 ]
      }
    }
  }
}
%    \end{macrocode}
% \end{macro}
%
% \begin{macro}{\_@@_custom_arg_inv:wn}
%    \begin{macrocode}
\cs_new_protected:Npn \_@@_custom_arg_inv:wn [ #1 ] #2 {
  \bool_set_true:N \l_tmpa_bool
  \_@@_custom_arg:wn [ #1 ] { #2 }
}
%    \end{macrocode}
% \end{macro}
%
% \begin{macro}{\_@@_custom_arg:wn}
%    \begin{macrocode}
\cs_new_protected:Npn \_@@_custom_arg:wn [ #1 ] #2 {
  \str_set:Nx \l_tmpb_str { 
    \str_item:Nn \l_tmpa_str { #1 }
  }
  \str_case:VnTF \l_tmpb_str {
    { X } {
      \msg_error:nnx{stex}{error/notationarg}{\l_@@_custom_uri}
    }
    { i } { \_@@_custom_set_X:n { #1 } }
    { b } { \_@@_custom_set_X:n { #1 } }
    { a } { \_@@_custom_set_X:n { #1 } } % TODO ?
    { B } { \_@@_custom_set_X:n { #1 } } % TODO ?
  }{}{
    \msg_error:nnx{stex}{error/notationarg}{\l_@@_custom_uri}
  }

  \bool_if:nTF \l_tmpa_bool {
    \tl_put_right:Nx \l_tmpa_tl {
      \stex_annotate_invisible:n {
        \_stex_term_arg:nn { \int_eval:n { #1 } } 
          \exp_not:n { { #2 } }
      }
    }
  } {
    \tl_put_right:Nx \l_tmpa_tl {
      \_stex_term_arg:nn { \int_eval:n { #1 } } 
        \exp_not:n { { #2 } }
    }
  }

  \_@@_custom_loop:
}
%    \end{macrocode}
% \end{macro}
%
% \begin{macro}{\_@@_custom_set_X:n}
%    \begin{macrocode}
\cs_new_protected:Nn \_@@_custom_set_X:n {
  \str_set:Nx \l_tmpa_str {
    \str_range:Nnn \l_tmpa_str 1 { #1 - 1 }
    X
    \str_range:Nnn \l_tmpa_str { #1 + 1 } { -1 }
  }
}
%    \end{macrocode}
% \end{macro}
%
% \begin{macro}{\_@@_custom_component:}
%    \begin{macrocode}
\cs_new_protected:Npn \_@@_custom_component:w [ #1 ] {
  \tl_put_right:Nn \l_tmpa_tl { \comp{ #1 } }
  \_@@_custom_loop:
}
%    \end{macrocode}
% \end{macro}
%
% \begin{macro}{\_@@_custom_final:}
%    \begin{macrocode}
\cs_new_protected:Nn \_@@_custom_final: {
  \int_compare:nNnTF \l_tmpb_int = 0 {
    \exp_args:Nnno \_stex_term_oms:nnn
  }{
    \str_if_in:NnTF \l_tmpa_str {b} {
      \exp_args:Nnno \_stex_term_ombind:nnn
    } {
      \exp_args:Nnno \_stex_term_oma:nnn
    }
  }
  { \l_@@_custom_uri } { \l_@@_custom_uri } { \l_tmpa_tl }
}
%    \end{macrocode}
% \end{macro}
%
% \begin{macro}{\symref,\symname}
%    \begin{macrocode}
\cs_new:Nn \stex_capitalize:n { \uppercase{#1} }

\keys_define:nn { stex / symname } {
  pre     .tl_set_x:N    = \l_@@_pre_tl ,
  post    .tl_set_x:N    = \l_@@_post_tl ,
  root    .tl_set_x:N    = \l_@@_root_tl
}

\cs_new_protected:Nn \stex_symname_args:n {
  \tl_clear:N \l_@@_post_tl
  \tl_clear:N \l_@@_pre_tl
  \tl_clear:N \l_@@_root_str
  \keys_set:nn { stex / symname } { #1 }
}

\NewDocumentCommand \symref { m m }{
  \let\compemph_uri_prev:\compemph@uri
  \let\compemph@uri\symrefemph@uri
  \STEXsymbol{#1}![ #2 ]
  \let\compemph@uri\compemph_uri_prev:
}

\NewDocumentCommand \synonym { O{} m m}{
  \stex_symname_args:n { #1 }
  \let\compemph_uri_prev:\compemph@uri
  \let\compemph@uri\symrefemph@uri
  % TODO
  \STEXsymbol{#2}![\l_@@_pre_tl #3 \l_@@_post_tl]
  \let\compemph@uri\compemph_uri_prev:
}

\NewDocumentCommand \symname { O{} m }{
  \stex_symname_args:n { #1 }
  \stex_get_symbol:n { #2 }
  \str_set:Nx \l_tmpa_str {
    \prop_item:cn { l_stex_symdecl_ \l_stex_get_symbol_uri_str _prop } { name }
  }
  \exp_args:NNno \str_replace_all:Nnn \l_tmpa_str {-} {~}
  
  \let\compemph_uri_prev:\compemph@uri
  \let\compemph@uri\symrefemph@uri
  \exp_args:NNx \use:nn
  \stex_invoke_symbol:n { { \l_stex_get_symbol_uri_str }![
    \l_@@_pre_tl \l_tmpa_str \l_@@_post_tl
  ] }
  \let\compemph@uri\compemph_uri_prev:
}

\NewDocumentCommand \Symname { O{} m }{
  \stex_symname_args:n { #1 }
  \stex_get_symbol:n { #2 }
  \str_set:Nx \l_tmpa_str {
    \prop_item:cn { l_stex_symdecl_ \l_stex_get_symbol_uri_str _prop } { name }
  }
  \exp_args:NNno \str_replace_all:Nnn \l_tmpa_str {-} {~}
  \let\compemph_uri_prev:\compemph@uri
  \let\compemph@uri\symrefemph@uri
  \exp_args:NNx \use:nn
  \stex_invoke_symbol:n { { \l_stex_get_symbol_uri_str }![
    \exp_after:wN \stex_capitalize:n \l_tmpa_str
      \l_@@_post_tl
  ] }
  \let\compemph@uri\compemph_uri_prev:
}
%    \end{macrocode}
% \end{macro}
%
%
% \subsection{Notation Components}
%    \begin{macrocode}
%<@@=stex_notationcomps>
%    \end{macrocode}
%
%
% \begin{macro}{\stex_highlight_term:nn}
%    \begin{macrocode}

\str_new:N \l_stex_current_symbol_str
\cs_new_protected:Nn \stex_highlight_term:nn {
  \exp_args:Nnx
  \use:nn {
    \str_set:Nx \l_stex_current_symbol_str { #1 }
    #2
  } {
    \str_set:Nx \exp_not:N \l_stex_current_symbol_str
      { \l_stex_current_symbol_str }
  }
}

\cs_new_protected:Nn \stex_unhighlight_term:n {
%  \latexml_if:TF {
%    #1
%  } {
%    \rustex_if:TF {
%      #1
%    } {
      #1 %\iffalse{{\fi}} #1 {{\iffalse}}\fi
%    }
%  }
}
%    \end{macrocode}
% \end{macro}
%
% \begin{macro}{\comp,\compemph@uri,\compemph,\defemph,\defemph@uri,\symrefemph,\symrefemph@uri}
%    \begin{macrocode}
\cs_new_protected:Npn \comp #1 {
  \str_if_empty:NF \l_stex_current_symbol_str {
    \rustex_if:TF {
      \stex_annotate:nnn { comp }{ \l_stex_current_symbol_str }{ #1 }
    }{
      \exp_args:Nnx \compemph@uri { #1 } { \l_stex_current_symbol_str }
    }
  }
}

\cs_new_protected:Npn \compemph@uri #1 #2 {
    \compemph{ #1 }
}


\cs_new_protected:Npn \compemph #1 {
    #1
}

\cs_new_protected:Npn \defemph@uri #1 #2 {
    \defemph{#1}
}

\cs_new_protected:Npn \defemph #1 {
    \textbf{#1}
}

\cs_new_protected:Npn \symrefemph@uri #1 #2 {
    \symrefemph{#1}
}

\cs_new_protected:Npn \symrefemph #1 {
    \textbf{#1}
}
%    \end{macrocode}
% \end{macro}
%
%
% \begin{macro}{\ellipses}
%    \begin{macrocode}
\NewDocumentCommand \ellipses {} { \ldots }
%    \end{macrocode}
% \end{macro}
%
%
% \begin{macro}{\parray,\prmatrix,\parrayline,\parraylineh,\parraycell}
%    \begin{macrocode}
\bool_new:N \l_stex_inparray_bool
\bool_set_false:N \l_stex_inparray_bool
\NewDocumentCommand \parray { m m } {
  \begingroup 
  \bool_set_true:N \l_stex_inparray_bool
  \begin{array}{#1}
    #2
  \end{array}
  \endgroup
}

\NewDocumentCommand \prmatrix { m } {
  \begingroup 
  \bool_set_true:N \l_stex_inparray_bool
  \begin{matrix}
    #1
  \end{matrix}
  \endgroup
}

\def \maybephline {
  \bool_if:NT \l_stex_inparray_bool {\hline}
}

\def \parrayline #1 #2 {
  #1 #2 \bool_if:NT \l_stex_inparray_bool {\\}
}

\def \pmrow #1 { \parrayline{}{ #1 } }

\def \parraylineh #1 #2 {
  #1 #2 \bool_if:NT \l_stex_inparray_bool {\\\hline}
}

\def \parraycell #1 {
  #1 \bool_if:NT \l_stex_inparray_bool {&}
}
%    \end{macrocode}
% \end{macro}
%
% \subsection{Variables}
%    \begin{macrocode}
%<@@=stex_variables>
%    \end{macrocode}
%
% \begin{macro}{\stex_invoke_variable:n}
%
%  Invokes a variable
%
%    \begin{macrocode}
\cs_new_protected:Nn \stex_invoke_variable:n {
  \if_mode_math:
    \exp_after:wN \_@@_invoke_math:n
  \else:
    \exp_after:wN \_@@_invoke_text:n
  \fi: {#1}
}

\cs_new_protected:Nn \_@@_invoke_text:n {
  %TODO
}


\cs_new_protected:Nn \_@@_invoke_math:n {
  \peek_charcode_remove:NTF ! {
    \peek_charcode_remove:NTF ! {
      \peek_charcode:NTF [ {
        \_@@_invoke_op_custom:nw
      }{
        % TODO throw error
      }
    }{
      \_@@_invoke_op:n { #1 }
    }
  }{
    \peek_charcode_remove:NTF * {
      \_@@_invoke_text:n { #1 }
    }{
      \_@@_invoke_math_ii:n { #1 }
    }
  }
}

\cs_new_protected:Nn \_@@_invoke_op:n {
  \cs_if_exist:cTF {
    stex_var_op_notation_ #1 _cs
  }{
    \use:c{stex_var_op_notation_ #1  _cs }
  }{
    \msg_error:nnxx{stex}{error/noop}{variable~#1}{}
  }
}

\cs_new_protected:Npn \_@@_invoke_math_ii:n  #1 {
  \cs_if_exist:cTF {
    stex_var_notation_#1_cs
  }{
    \str_set:Nn \l_stex_current_symbol_str { #1 }
    \use:c{stex_var_notation_#1_cs}
  }{
    \msg_error:nnxx{stex}{error/nonotation}{variable~#1}{s}
  }
}
%    \end{macrocode}
% \end{macro}
%
%
%    \begin{macrocode}
%</package>
%    \end{macrocode}
%
% \end{implementation}
%
% \PrintIndex

% \endinput
% Local Variables:
% mode: doctex
% TeX-master: t
% End:

% \fi
%
% \begin{documentation}\label{pkg:terms:doc}
%
% Code related to symbolic expressions, typesetting notations,
% notation components, etc.
%
% \section{Macros and Environments}\label{pkg:terms:doc:macros}
%
% \begin{function}{\STEXsymbol}
%   Uses \cs{stex_get_symbol:n} to find the symbol denoted by
%   the first argument and passes the result on to
%   \cs{stex_invoke_symbol:n}
% \end{function}
%
% \begin{function}{\symref}
%   \begin{syntax} \cs{symref}\Arg{symbol}\Arg{text} \end{syntax}
%   shortcut for \cs{STEXsymbol}\Arg{symbol}|![|\meta{text}|]|
% \end{function}
%
% \begin{function}{\stex_invoke_symbol:n}
%   Executes a semantic macro. Outside of math mode or if followed by |*|,
%   it continues to \cs{stex_term_custom:nn}. In math mode,
%   it uses the default or optionally provided notation of
%   the associated symbol.
%
%   If followed by |!|, it will invoke the symbol \emph{itself}
%   rather than its application (and continue to
%   \cs{stex_term_custom:nn}), i.e. it allows to refer to
%   |\plus![addition]| as an operation, rather than
%   |\plus[addition of]{some}{terms}|.
% \end{function}
%
% \begin{function}{\_stex_term_math_oms:nnnn,\_stex_term_math_oma:nnnn,\_stex_term_math_omb:nnnn}
%   \begin{syntax} \meta{URI}\meta{fragment}\meta{precedence}\meta{body} \end{syntax}
%
% Annotates \meta{body} as an \omdoc-term (|OMID|, |OMA| or |OMBIND|, respectively) 
% with head symbol \meta{URI}, generated
% by the specific notation \meta{fragment} with (upwards) operator precedence
% \meta{precedence}. Inserts parentheses according to
% the current downwards precedence and operator precedence.
% \end{function}
%
% \begin{function}{\_stex_term_math_arg:nnn}
%   \begin{syntax} \cs{stex_term_arg:nnn}\meta{int}\meta{prec}\meta{body} \end{syntax}
% Annotates \meta{body} as the \meta{int}th argument of the current |OMA| or |OMBIND|,
% with (downwards) argument precedence \meta{prec}.
% \end{function}
%
% \begin{function}{\_stex_term_math_assoc_arg:nnnn}
%   \begin{syntax} \cs{stex_term_arg:nnn}\meta{int}\meta{prec}\meta{notation}\meta{body} \end{syntax}
% Annotates \meta{body} as the \meta{int}th (associative) \emph{sequence} argument
% (as comma-separated list of terms) of the current |OMA| or |OMBIND|,
% with (downwards) argument precedence \meta{prec} and associative
% notation \meta{notation}.
% 
% \end{function}
%
% \begin{variable}{\infprec, \neginfprec}
%   Maximal and minimal notation precedences.
% \end{variable}
%
% \begin{function}{\dobrackets}
%   \begin{syntax} \cs{dobrackets} \Arg{body} \end{syntax}
%   Puts \meta{body} in parentheses; scaled if in display mode
%   unscaled otherwise. Uses the current \sTeX brackets (by default |(| and |)|),
%   which can be changed temporarily using \cs{withbrackets}.
% \end{function}
%
% \begin{function}{\withbrackets}
%   \begin{syntax} \cs{withbrackets} \meta{left} \meta{right} \Arg{body} \end{syntax}
%   Temporarily (i.e. within \meta{body}) sets the brackets used by \sTeX for automated
%   bracketing (by default |(| and |)|) to \meta{left} and \meta{right}.
%
%   Note that \meta{left} and \meta{right} need to be allowed
%   after \cs{left} and \cs{right} in displaymode.
% \end{function}
%
% \begin{function}{\stex_term_custom:nn}
%   \begin{syntax} \cs{stex_term_custom:nn}\Arg{URI}\Arg{args}\end{syntax}
% Implements custom one-time notation.
% Invoked by \cs{stex_invoke_symbol:n} in text mode, or if
% followed by |*| in math mode, or whenever followed by |!|.
% \end{function}
%
% \begin{function}{\stex_highlight_term:nn}
%   \begin{syntax} \cs{stex_highlight_term:nn}\Arg{URI}\Arg{args}\end{syntax}
% Establishes a context for \cs{comp}. Stores the URI in a variable
% so that \cs{comp} knows which symbol governs the current notation.
% \end{function}
%
% \begin{function}{\comp, \compemph,\compemph@uri, \defemph, \defemph@uri, \symrefemph,\symrefemph@uri}
%   \begin{syntax} \cs{comp}\Arg{args}\end{syntax}
% Marks \meta{args} as a notation component of the current symbol for
% highlighting, linking, etc.
%
% The precise behavior is governed by \cs{@comp}, which takes as
% additional argument the URI of the current symbol. By default,
% \cs{@comp} adds the URI as a PDF tooltip and colors the highlighted part
% in blue.
%
% \cs{@defemph} behaves like \cs{@comp}, and can be similarly redefined,
% but marks an expression as \emph{definiendum} (used by \cs{definiendum})
% \end{function}
%
% \begin{function}{\STEXinvisible}
% Exports its argument as \omdoc (invisible), but does
% not produce PDF output. Useful e.g. for semantic macros
% that take arguments that are not part of the symbolic
% notation.
% \end{function}
%
% \begin{function}{\ellipses}
%   TODO
% \end{function}
%
% \end{documentation}
%
% \begin{implementation}\label{pkg:terms:impl}
%
% \section{\sTeX-Terms Implementation}
%
%    \begin{macrocode}
%<*package>

%%%%%%%%%%%%%   terms.dtx   %%%%%%%%%%%%%

%<@@=stex_terms>
%    \end{macrocode}
%
% Warnings and error messages
%
%    \begin{macrocode}
\msg_new:nnn{stex}{error/nonotation}{
  Symbol~#1~invoked,~but~has~no~notation#2!
}
\msg_new:nnn{stex}{error/notationarg}{
  Error~in~parsing~notation~#1
}
\msg_new:nnn{stex}{error/noop}{
  Symbol~#1~has~no~operator~notation~for~notation~#2
}
\msg_new:nnn{stex}{error/notallowed}{
  Symbol~invokation~#1~not~allowed~in~notation~component~of~#2
}

%    \end{macrocode}
% \subsection{Symbol Invokations}
%
%
% \begin{macro}{\stex_invoke_symbol:n}
%
%  Invokes a semantic macro
%
%    \begin{macrocode}
\keys_define:nn { stex / terms } {
  lang    .tl_set_x:N = \l_@@_lang_str ,
  variant .tl_set_x:N = \l_@@_variant_str ,
  unknown .code:n     = \str_set:Nx 
      \l_@@_variant_str \l_keys_key_str
}

\cs_new_protected:Nn \_@@_args:n {
  \str_clear:N \l_@@_lang_str
  \str_clear:N \l_@@_variant_str
  
  \keys_set:nn { stex / terms } { #1 }
}

\cs_new:Nn \_@@_reset:N {
  \tl_if_exist:NTF #1 {
    \def \exp_not:N #1 { \exp_args:No \exp_not:n #1 }
  }{
    \let \exp_not:N #1 \exp_not:N \undefined
  }
}

\bool_new:N \l_@@_allow_semantic_bool
\bool_set_true:N \l_@@_allow_semantic_bool

\cs_new_protected:Nn \stex_invoke_symbol:n {
  \bool_if:NTF \l_@@_allow_semantic_bool {
    \str_if_eq:eeF {
      \prop_item:cn {
        l_stex_symdecl_#1_prop
      }{ deprecate }
    }{}{
      \msg_warning:nnxx{stex}{warning/deprecated}{
        Symbol~#1
      }{
        \prop_item:cn {l_stex_symdecl_#1_prop}{ deprecate }
      }
    }
    \if_mode_math:
      \exp_after:wN \_@@_invoke_math:n
    \else:
      \exp_after:wN \_@@_invoke_text:n
    \fi: { #1 }
  }{
    \msg_error:nnxx{stex}{error/notallowed}{#1}{\l_stex_current_symbol_str}
  }
}

\cs_new_protected:Nn \_@@_invoke_text:n {
  \peek_charcode_remove:NTF ! {
    \_@@_invoke_op_custom:nn {#1}
  }{
    \_@@_invoke_custom:nn {#1}
  }
}

\cs_new_protected:Nn \_@@_invoke_math:n {
  \peek_charcode_remove:NTF ! {
    % operator
    \peek_charcode_remove:NTF * {
      % custom op
      \_@@_invoke_op_custom:nn {#1}
    }{
      % op notation
      \peek_charcode:NTF [ {
        \_@@_invoke_op_notation:nw {#1}
      }{
        \_@@_invoke_op_notation:nw {#1}[]
      }
    }
  }{
    \peek_charcode_remove:NTF * {
      \_@@_invoke_custom:nn {#1}
      % custom
    }{
      % normal
      \peek_charcode:NTF [ {
        \_@@_invoke_notation:nw {#1}
      }{
        \_@@_invoke_notation:nw {#1}[]
      }
    }
  }
}


\cs_new_protected:Nn \_@@_invoke_op_custom:nn {
  \exp_args:Nnx \use:nn {
    \str_set:Nn \l_stex_current_symbol_str { #1 }
    \bool_set_false:N \l_@@_allow_semantic_bool
    \_stex_term_oms:nnn {#1 \c_hash_str\c_hash_str}{#1}{
      \comp{ #2 }
    }
  }{
    \_@@_reset:N \l_stex_current_symbol_str
    \bool_set_true:N \l_@@_allow_semantic_bool
  }
}

\cs_new_protected:Nn \_@@_find_notation:nn {
  \str_set:Nn \l_stex_current_symbol_str { #1 }
  \_@@_args:n { #2 }
  \seq_if_empty:cTF {
    l_stex_symdecl_ #1 _notations 
  } {
    \msg_error:nnxx{stex}{error/nonotation}{#1}{s}
  } {
    \bool_lazy_all:nTF {
      {\str_if_empty_p:N \l_@@_variant_str}
      {\str_if_empty_p:N \l_@@_lang_str}
    }{
      \seq_get_left:cN {l_stex_symdecl_#1_notations}\l_@@_variant_str
    }{
      \seq_if_in:cxTF {l_stex_symdecl_#1_notations}{
        \l_@@_variant_str \c_hash_str \l_@@_lang_str
      }{
        \str_set:Nx \l_@@_variant_str { \l_@@_variant_str \c_hash_str \l_@@_lang_str }
      }{
        \msg_error:nnxx{stex}{error/nonotation}{#1}{
          ~\l_@@_variant_str \c_hash_str \l_@@_lang_str
        }
      }
    }
  }
}

\cs_new_protected:Npn \_@@_invoke_op_notation:nw #1 [#2] {
  \_@@_find_notation:nn { #1 }{ #2 }
  \bool_set_false:N \l_@@_allow_semantic_bool
  \cs_if_exist:cTF {
    stex_op_notation_ #1 \c_hash_str \l_@@_variant_str _cs
  }{
    \use:c{stex_op_notation_ #1 \c_hash_str \l_@@_variant_str _cs}
  }{
    \msg_error:nnxx{stex}{error/noop}{#1}{\l_@@_variant_str}
  }
  \bool_set_true:N \l_@@_allow_semantic_bool
}

\cs_new_protected:Npn \_@@_invoke_notation:nw #1 [#2] {
  \_@@_find_notation:nn { #1 }{ #2 }
  \cs_if_exist:cTF {
    stex_notation_ #1 \c_hash_str \l_@@_variant_str _cs
  }{
    \tl_set:Nx \stex_symbol_after_invokation_tl {
      \_@@_reset:N \stex_symbol_after_invokation_tl
      \_@@_reset:N \l_stex_current_symbol_str
      \bool_set_true:N \l_@@_allow_semantic_bool
    }
    \bool_set_false:N \l_@@_allow_semantic_bool
    \use:c{stex_notation_ #1 \c_hash_str \l_@@_variant_str _cs}
  }{
    \msg_error:nnxx{stex}{error/nonotation}{#1}{
      ~\l_@@_variant_str
    }
  }
}

\prop_new:N \l_@@_custom_args_prop

\cs_new_protected:Nn \_@@_invoke_custom:nn {
  \exp_args:Nnx \use:nn {
    \bool_set_false:N \l_@@_allow_semantic_bool
    \str_set:Nn \l_stex_current_symbol_str { #1 }
    \prop_clear:N \l_@@_custom_args_prop
    \prop_put:Nnn \l_@@_custom_args_prop {currnum} {1}
    \prop_put:Nnx \l_@@_custom_args_prop {args} {
      \prop_item:cn {
        l_stex_symdecl_\l_stex_get_symbol_uri_str _prop
      }{ args }
    }
    \tl_set:Nn \arg { \_@@_arg: }
    #2
    % TODO check that all arguments exist
  }{
    \_@@_reset:N \l_stex_current_symbol_str
    \_@@_reset:N \arg
    \_@@_reset:N \l_@@_custom_args_prop
    \bool_set_true:N \l_@@_allow_semantic_bool
  }
}

\NewDocumentCommand \_@@_arg: { s O{} m}{
  \tl_if_empty:nTF {#2}{
    \int_set:Nn \l_tmpa_int {\prop_item:Nn \l_@@_custom_args_prop {currnum}}
    \bool_set_true:N \l_tmpa_bool
    \bool_do_while:Nn \l_tmpa_bool {
      \exp_args:NNx \prop_if_in:NnTF \l_@@_custom_args_prop {\int_use:N \l_tmpa_int} {
        \int_incr:N \l_tmpa_int
      }{
        \bool_set_false:N \l_tmpa_bool
      }
    }
  }{
    \int_set:Nn \l_tmpa_int { #2 }
    \exp_args:NNx \prop_if_in:NnT \l_@@_custom_args_prop {\int_use:N \l_tmpa_int} {
      % TODO throw error
    }
  }
  \str_set:Nx \l_tmpa_str {\prop_item:Nn \l_@@_custom_args_prop {args} }
  \int_compare:nNnT \l_tmpa_int > {\str_count:N \l_tmpa_str} {
    % TODO throw error
  }
  \IfBooleanTF#1{
    \stex_annotate_invisible:n {
      \exp_args:No \_stex_term_arg:nn {\l_stex_current_symbol_str}{#3}
    }
  }{
    \exp_args:No \_stex_term_arg:nn {\l_stex_current_symbol_str}{#3}
  }
}


\cs_new_protected:Nn \_stex_term_arg:nn {
  \exp_args:Nnx \use:nn {
    \bool_set_true:N \l_@@_allow_semantic_bool
    \stex_annotate:nnn{ arg }{ #1 }{ #2 }
  }{
    \bool_set_false:N \l_@@_allow_semantic_bool
  }
}

\cs_new_protected:Nn \_stex_term_math_arg:nnn {
  \exp_args:Nnx \use:nn
    { \int_set:Nn \l_@@_downprec { #2 } 
        \_stex_term_arg:nn { #1 }{ #3 }
    }
    { \int_set:Nn \exp_not:N \l_@@_downprec { \int_use:N \l_@@_downprec } }
}


%    \end{macrocode}
% \end{macro}
%
% \subsection{Terms}
%
% Precedences:
% \begin{variable}{\infprec, \neginfprec, \l_@@_downprec}
%    \begin{macrocode}
\tl_const:Nx \infprec {\int_use:N \c_max_int}
\tl_const:Nx \neginfprec {-\int_use:N \c_max_int}
\int_new:N \l_@@_downprec
\int_set_eq:NN \l_@@_downprec \infprec
%    \end{macrocode}
% \end{variable}
%
% Bracketing:
%
% \begin{variable}{\l_@@_left_bracket_str, \l_@@_right_bracket_str}
%    \begin{macrocode}
\tl_set:Nn \l_@@_left_bracket_str (
\tl_set:Nn \l_@@_right_bracket_str )
%    \end{macrocode}
% \end{variable}
%
% \begin{macro}{\_@@_maybe_brackets:nn}
%
% Compares precedences and insert brackets accordingly
%
%    \begin{macrocode}
\cs_new_protected:Nn \_@@_maybe_brackets:nn {
  \bool_if:NTF \l_@@_brackets_done_bool {
    \bool_set_false:N \l_@@_brackets_done_bool
    #2
  } {
    \int_compare:nNnTF { #1 } > \l_@@_downprec {
      \bool_if:NTF \l_stex_inparray_bool { #2 }{
        \stex_debug:nn{dobrackets}{\number#1 > \number\l_@@_downprec; \detokenize{#2}}
        \dobrackets { #2 }
      }
    }{ #2 }
  }
}
%    \end{macrocode}
% \end{macro}
%
% \begin{macro}{\dobrackets}
%    \begin{macrocode}
\bool_new:N \l_@@_brackets_done_bool
%\RequirePackage{scalerel}
\cs_new_protected:Npn \dobrackets #1 {
  %\ThisStyle{\if D\m@switch
  %    \exp_args:Nnx \use:nn
  %    { \exp_after:wN \left\l_@@_left_bracket_str #1 } 
  %    { \exp_not:N\right\l_@@_right_bracket_str }
  %  \else
      \exp_args:Nnx \use:nn
      { 
        \bool_set_true:N \l_@@_brackets_done_bool
        \int_set:Nn \l_@@_downprec \infprec
        \l_@@_left_bracket_str 
        #1
      } 
      {
        \bool_set_false:N \l_@@_brackets_done_bool
        \l_@@_right_bracket_str 
        \int_set:Nn \l_@@_downprec { \int_use:N \l_@@_downprec }
      }
  %\fi}
}
%    \end{macrocode}
% \end{macro}
%
% \begin{macro}{\withbrackets}
%    \begin{macrocode}
\cs_new_protected:Npn \withbrackets #1 #2 #3 {
  \exp_args:Nnx \use:nn
  {  
    \tl_set:Nx \l_@@_left_bracket_str { #1 }
    \tl_set:Nx \l_@@_right_bracket_str { #2 }
    #3
  }
  {
    \tl_set:Nn \exp_not:N \l_@@_left_bracket_str 
      {\l_@@_left_bracket_str}
    \tl_set:Nn \exp_not:N \l_@@_right_bracket_str 
      {\l_@@_right_bracket_str}
  }
}
%    \end{macrocode}
% \end{macro}
%
% \begin{macro}{\STEXinvisible}
%    \begin{macrocode}
\cs_new_protected:Npn \STEXinvisible #1 {
  \stex_annotate_invisible:n { #1 }
}
%    \end{macrocode}
% \end{macro}
%
% \omdoc terms:
%
% \begin{macro}{\_stex_term_math_oms:nnnn}
%    \begin{macrocode}
\cs_new_protected:Nn \_stex_term_oms:nnn {
  \stex_annotate:nnn{ OMID }{ #2 }{
    \stex_highlight_term:nn { #1 } { #3 } 
  }
}

\cs_new_protected:Nn \_stex_term_math_oms:nnnn {
  \_@@_maybe_brackets:nn { #3 }{ 
    \_stex_term_oms:nnn { #1 } { #1\c_hash_str#2 } { #4 }
  }
}
%    \end{macrocode}
% \end{macro}
%
% \begin{macro}{\_stex_term_math_omv:nn}
%    \begin{macrocode}
\cs_new_protected:Nn \_stex_term_omv:nn {
  \stex_annotate:nnn{ OMID }{ #1 }{
    \stex_highlight_term:nn { #1 } { #2 } 
  }
}
%    \end{macrocode}
% \end{macro}
%
% \begin{macro}{\_stex_term_math_oma:nnnn}
%    \begin{macrocode}
\cs_new_protected:Nn \_stex_term_oma:nnn {
  \stex_annotate:nnn{ OMA }{ #2 }{
    \stex_highlight_term:nn { #1 } { #3 } 
  }
}

\cs_new_protected:Nn \_stex_term_math_oma:nnnn {
  \_@@_maybe_brackets:nn { #3 }{ 
    \_stex_term_oma:nnn { #1 } { #1\c_hash_str#2 } { #4 }
  }
}
%    \end{macrocode}
% \end{macro}
%
% \begin{macro}{\_stex_term_math_omb:nnnn}
%    \begin{macrocode}
\cs_new_protected:Nn \_stex_term_ombind:nnn {
  \stex_annotate:nnn{ OMBIND }{ #2 }{
    \stex_highlight_term:nn { #1 } { #3 }
  }
}

\cs_new_protected:Nn \_stex_term_math_omb:nnnn {
  \_@@_maybe_brackets:nn { #3 }{ 
    \_stex_term_ombind:nnn { #1 } { #1\c_hash_str#2 } { #4 }
  }
}
%    \end{macrocode}
% \end{macro}
%
% \begin{macro}{\_stex_term_math_assoc_arg:nnnn}
%    \begin{macrocode}
\cs_new_protected:Nn \_stex_term_math_assoc_arg:nnnn {
  % TODO sequences
  \clist_set:Nn \l_tmpa_clist{ #3 }
  \int_compare:nNnTF { \clist_count:N \l_tmpa_clist } < 2 {
    \tl_set:Nn \l_tmpa_tl { #3 }
  }{
    \cs_set:Npn \l_tmpa_cs ##1 ##2 { #4 }
    \clist_reverse:N \l_tmpa_clist
    \clist_pop:NN \l_tmpa_clist \l_tmpa_tl

    \clist_map_inline:Nn \l_tmpa_clist {
      \exp_args:NNNo \exp_args:NNo \tl_set:No \l_tmpa_tl {
        \exp_args:Nno 
        \l_tmpa_cs { ##1 } \l_tmpa_tl 
      }
    }
  }
  \exp_args:Nnno
   \_stex_term_math_arg:nnn{#1}{#2}\l_tmpa_tl
}
%    \end{macrocode}
% \end{macro}
%
% \begin{macro}{\stex_term_custom:nn}
%    \begin{macrocode}
\cs_new_protected:Nn \stex_term_custom:nn {
  \str_set:Nn \l_@@_custom_uri { #1 }
  \str_set:Nn \l_tmpa_str { #2 }
  \tl_clear:N \l_tmpa_tl
  \int_zero:N \l_tmpa_int
  \int_set:Nn \l_tmpb_int { \str_count:N \l_tmpa_str }
  \_@@_custom_loop:
}
%    \end{macrocode}
% \end{macro}
%
% \begin{macro}{\_@@_custom_loop:}
%    \begin{macrocode}
\cs_new_protected:Nn \_@@_custom_loop: {
  \bool_set_false:N \l_tmpa_bool
  \bool_while_do:nn {
    \str_if_eq_p:ee X {
      \str_item:Nn \l_tmpa_str { \l_tmpa_int + 1 }
    }
  }{
    \int_incr:N \l_tmpa_int
  }

  \peek_charcode:NTF [ {
    % notation/text component
    \_@@_custom_component:w
  } {
    \int_compare:nNnTF \l_tmpa_int = \l_tmpb_int {
      % all arguments read => finish
      \_@@_custom_final:
    } {
      % arguments missing
      \peek_charcode_remove:NTF * {
        % invisible, specific argument position or both
        \peek_charcode:NTF [ {
          % visible specific argument position
          \_@@_custom_arg:wn
        } {
          % invisible
          \peek_charcode_remove:NTF * {
            % invisible specific argument position
            \_@@_custom_arg_inv:wn
          } {
            % invisible next argument
            \_@@_custom_arg_inv:wn [ \l_tmpa_int + 1 ]
          }
        } 
      } {
        % next normal argument
        \_@@_custom_arg:wn [ \l_tmpa_int + 1 ]
      }
    }
  }
}
%    \end{macrocode}
% \end{macro}
%
% \begin{macro}{\_@@_custom_arg_inv:wn}
%    \begin{macrocode}
\cs_new_protected:Npn \_@@_custom_arg_inv:wn [ #1 ] #2 {
  \bool_set_true:N \l_tmpa_bool
  \_@@_custom_arg:wn [ #1 ] { #2 }
}
%    \end{macrocode}
% \end{macro}
%
% \begin{macro}{\_@@_custom_arg:wn}
%    \begin{macrocode}
\cs_new_protected:Npn \_@@_custom_arg:wn [ #1 ] #2 {
  \str_set:Nx \l_tmpb_str { 
    \str_item:Nn \l_tmpa_str { #1 }
  }
  \str_case:VnTF \l_tmpb_str {
    { X } {
      \msg_error:nnx{stex}{error/notationarg}{\l_@@_custom_uri}
    }
    { i } { \_@@_custom_set_X:n { #1 } }
    { b } { \_@@_custom_set_X:n { #1 } }
    { a } { \_@@_custom_set_X:n { #1 } } % TODO ?
    { B } { \_@@_custom_set_X:n { #1 } } % TODO ?
  }{}{
    \msg_error:nnx{stex}{error/notationarg}{\l_@@_custom_uri}
  }

  \bool_if:nTF \l_tmpa_bool {
    \tl_put_right:Nx \l_tmpa_tl {
      \stex_annotate_invisible:n {
        \_stex_term_arg:nn { \int_eval:n { #1 } } 
          \exp_not:n { { #2 } }
      }
    }
  } {
    \tl_put_right:Nx \l_tmpa_tl {
      \_stex_term_arg:nn { \int_eval:n { #1 } } 
        \exp_not:n { { #2 } }
    }
  }

  \_@@_custom_loop:
}
%    \end{macrocode}
% \end{macro}
%
% \begin{macro}{\_@@_custom_set_X:n}
%    \begin{macrocode}
\cs_new_protected:Nn \_@@_custom_set_X:n {
  \str_set:Nx \l_tmpa_str {
    \str_range:Nnn \l_tmpa_str 1 { #1 - 1 }
    X
    \str_range:Nnn \l_tmpa_str { #1 + 1 } { -1 }
  }
}
%    \end{macrocode}
% \end{macro}
%
% \begin{macro}{\_@@_custom_component:}
%    \begin{macrocode}
\cs_new_protected:Npn \_@@_custom_component:w [ #1 ] {
  \tl_put_right:Nn \l_tmpa_tl { \comp{ #1 } }
  \_@@_custom_loop:
}
%    \end{macrocode}
% \end{macro}
%
% \begin{macro}{\_@@_custom_final:}
%    \begin{macrocode}
\cs_new_protected:Nn \_@@_custom_final: {
  \int_compare:nNnTF \l_tmpb_int = 0 {
    \exp_args:Nnno \_stex_term_oms:nnn
  }{
    \str_if_in:NnTF \l_tmpa_str {b} {
      \exp_args:Nnno \_stex_term_ombind:nnn
    } {
      \exp_args:Nnno \_stex_term_oma:nnn
    }
  }
  { \l_@@_custom_uri } { \l_@@_custom_uri } { \l_tmpa_tl }
}
%    \end{macrocode}
% \end{macro}
%
% \begin{macro}{\symref,\symname}
%    \begin{macrocode}
\cs_new:Nn \stex_capitalize:n { \uppercase{#1} }

\keys_define:nn { stex / symname } {
  pre     .tl_set_x:N    = \l_@@_pre_tl ,
  post    .tl_set_x:N    = \l_@@_post_tl ,
  root    .tl_set_x:N    = \l_@@_root_tl
}

\cs_new_protected:Nn \stex_symname_args:n {
  \tl_clear:N \l_@@_post_tl
  \tl_clear:N \l_@@_pre_tl
  \tl_clear:N \l_@@_root_str
  \keys_set:nn { stex / symname } { #1 }
}

\NewDocumentCommand \symref { m m }{
  \let\compemph_uri_prev:\compemph@uri
  \let\compemph@uri\symrefemph@uri
  \STEXsymbol{#1}![ #2 ]
  \let\compemph@uri\compemph_uri_prev:
}

\NewDocumentCommand \synonym { O{} m m}{
  \stex_symname_args:n { #1 }
  \let\compemph_uri_prev:\compemph@uri
  \let\compemph@uri\symrefemph@uri
  % TODO
  \STEXsymbol{#2}![\l_@@_pre_tl #3 \l_@@_post_tl]
  \let\compemph@uri\compemph_uri_prev:
}

\NewDocumentCommand \symname { O{} m }{
  \stex_symname_args:n { #1 }
  \stex_get_symbol:n { #2 }
  \str_set:Nx \l_tmpa_str {
    \prop_item:cn { l_stex_symdecl_ \l_stex_get_symbol_uri_str _prop } { name }
  }
  \exp_args:NNno \str_replace_all:Nnn \l_tmpa_str {-} {~}
  
  \let\compemph_uri_prev:\compemph@uri
  \let\compemph@uri\symrefemph@uri
  \exp_args:NNx \use:nn
  \stex_invoke_symbol:n { { \l_stex_get_symbol_uri_str }![
    \l_@@_pre_tl \l_tmpa_str \l_@@_post_tl
  ] }
  \let\compemph@uri\compemph_uri_prev:
}

\NewDocumentCommand \Symname { O{} m }{
  \stex_symname_args:n { #1 }
  \stex_get_symbol:n { #2 }
  \str_set:Nx \l_tmpa_str {
    \prop_item:cn { l_stex_symdecl_ \l_stex_get_symbol_uri_str _prop } { name }
  }
  \exp_args:NNno \str_replace_all:Nnn \l_tmpa_str {-} {~}
  \let\compemph_uri_prev:\compemph@uri
  \let\compemph@uri\symrefemph@uri
  \exp_args:NNx \use:nn
  \stex_invoke_symbol:n { { \l_stex_get_symbol_uri_str }![
    \exp_after:wN \stex_capitalize:n \l_tmpa_str
      \l_@@_post_tl
  ] }
  \let\compemph@uri\compemph_uri_prev:
}
%    \end{macrocode}
% \end{macro}
%
%
% \subsection{Notation Components}
%    \begin{macrocode}
%<@@=stex_notationcomps>
%    \end{macrocode}
%
%
% \begin{macro}{\stex_highlight_term:nn}
%    \begin{macrocode}

\str_new:N \l_stex_current_symbol_str
\cs_new_protected:Nn \stex_highlight_term:nn {
  \exp_args:Nnx
  \use:nn {
    \str_set:Nx \l_stex_current_symbol_str { #1 }
    #2
  } {
    \str_set:Nx \exp_not:N \l_stex_current_symbol_str
      { \l_stex_current_symbol_str }
  }
}

\cs_new_protected:Nn \stex_unhighlight_term:n {
%  \latexml_if:TF {
%    #1
%  } {
%    \rustex_if:TF {
%      #1
%    } {
      #1 %\iffalse{{\fi}} #1 {{\iffalse}}\fi
%    }
%  }
}
%    \end{macrocode}
% \end{macro}
%
% \begin{macro}{\comp,\compemph@uri,\compemph,\defemph,\defemph@uri,\symrefemph,\symrefemph@uri}
%    \begin{macrocode}
\cs_new_protected:Npn \comp #1 {
  \str_if_empty:NF \l_stex_current_symbol_str {
    \rustex_if:TF {
      \stex_annotate:nnn { comp }{ \l_stex_current_symbol_str }{ #1 }
    }{
      \exp_args:Nnx \compemph@uri { #1 } { \l_stex_current_symbol_str }
    }
  }
}

\cs_new_protected:Npn \compemph@uri #1 #2 {
    \compemph{ #1 }
}


\cs_new_protected:Npn \compemph #1 {
    #1
}

\cs_new_protected:Npn \defemph@uri #1 #2 {
    \defemph{#1}
}

\cs_new_protected:Npn \defemph #1 {
    \textbf{#1}
}

\cs_new_protected:Npn \symrefemph@uri #1 #2 {
    \symrefemph{#1}
}

\cs_new_protected:Npn \symrefemph #1 {
    \textbf{#1}
}
%    \end{macrocode}
% \end{macro}
%
%
% \begin{macro}{\ellipses}
%    \begin{macrocode}
\NewDocumentCommand \ellipses {} { \ldots }
%    \end{macrocode}
% \end{macro}
%
%
% \begin{macro}{\parray,\prmatrix,\parrayline,\parraylineh,\parraycell}
%    \begin{macrocode}
\bool_new:N \l_stex_inparray_bool
\bool_set_false:N \l_stex_inparray_bool
\NewDocumentCommand \parray { m m } {
  \begingroup 
  \bool_set_true:N \l_stex_inparray_bool
  \begin{array}{#1}
    #2
  \end{array}
  \endgroup
}

\NewDocumentCommand \prmatrix { m } {
  \begingroup 
  \bool_set_true:N \l_stex_inparray_bool
  \begin{matrix}
    #1
  \end{matrix}
  \endgroup
}

\def \maybephline {
  \bool_if:NT \l_stex_inparray_bool {\hline}
}

\def \parrayline #1 #2 {
  #1 #2 \bool_if:NT \l_stex_inparray_bool {\\}
}

\def \pmrow #1 { \parrayline{}{ #1 } }

\def \parraylineh #1 #2 {
  #1 #2 \bool_if:NT \l_stex_inparray_bool {\\\hline}
}

\def \parraycell #1 {
  #1 \bool_if:NT \l_stex_inparray_bool {&}
}
%    \end{macrocode}
% \end{macro}
%
% \subsection{Variables}
%    \begin{macrocode}
%<@@=stex_variables>
%    \end{macrocode}
%
% \begin{macro}{\stex_invoke_variable:n}
%
%  Invokes a variable
%
%    \begin{macrocode}
\cs_new_protected:Nn \stex_invoke_variable:n {
  \if_mode_math:
    \exp_after:wN \_@@_invoke_math:n
  \else:
    \exp_after:wN \_@@_invoke_text:n
  \fi: {#1}
}

\cs_new_protected:Nn \_@@_invoke_text:n {
  %TODO
}


\cs_new_protected:Nn \_@@_invoke_math:n {
  \peek_charcode_remove:NTF ! {
    \peek_charcode_remove:NTF ! {
      \peek_charcode:NTF [ {
        \_@@_invoke_op_custom:nw
      }{
        % TODO throw error
      }
    }{
      \_@@_invoke_op:n { #1 }
    }
  }{
    \peek_charcode_remove:NTF * {
      \_@@_invoke_text:n { #1 }
    }{
      \_@@_invoke_math_ii:n { #1 }
    }
  }
}

\cs_new_protected:Nn \_@@_invoke_op:n {
  \cs_if_exist:cTF {
    stex_var_op_notation_ #1 _cs
  }{
    \use:c{stex_var_op_notation_ #1  _cs }
  }{
    \msg_error:nnxx{stex}{error/noop}{variable~#1}{}
  }
}

\cs_new_protected:Npn \_@@_invoke_math_ii:n  #1 {
  \cs_if_exist:cTF {
    stex_var_notation_#1_cs
  }{
    \str_set:Nn \l_stex_current_symbol_str { #1 }
    \use:c{stex_var_notation_#1_cs}
  }{
    \msg_error:nnxx{stex}{error/nonotation}{variable~#1}{s}
  }
}
%    \end{macrocode}
% \end{macro}
%
%
%    \begin{macrocode}
%</package>
%    \end{macrocode}
%
% \end{implementation}
%
% \PrintIndex

% \endinput
% Local Variables:
% mode: doctex
% TeX-master: t
% End:

  % \iffalse meta-comment
% An Infrastructure for Semantic Macros and Module Scoping
% Copyright (c) 2019 Michael Kohlhase, all rights reserved
%                this file is released under the
%                LaTeX Project Public License (LPPL)
% 
% The original of this file is in the public repository at 
% http://github.com/sLaTeX/sTeX/
%
% TODO update copyright  
%
%<*driver>
\providecommand\bibfolder{../../lib/bib}
\RequirePackage{paralist}
\documentclass[full,kernel]{l3doc}
\usepackage[dvipsnames]{xcolor}
\usepackage[utf8]{inputenc}
\usepackage[T1]{fontenc}
\RequirePackage{morewrites}
\RequirePackage{tikzinput}
\usetikzlibrary{fit}

\usepackage[debug=all,lang=en, mathhub=./tests]{stex}
\usepackage{url,array,float,textcomp}
\usepackage[show]{ed}
\usepackage[hyperref=auto,style=alphabetic]{biblatex}
\addbibresource{\bibfolder/kwarcpubs.bib}
\addbibresource{\bibfolder/extpubs.bib}
\addbibresource{\bibfolder/kwarccrossrefs.bib}
\addbibresource{\bibfolder/extcrossrefs.bib}
\usepackage{amssymb}
\usepackage{amsfonts}
\usepackage{xspace}
\usepackage{hyperref}

\makeindex
\floatstyle{boxed}
\newfloat{exfig}{thp}{lop}
\floatname{exfig}{Example}

\usepackage{stex-tests}

\MakeShortVerb{\|}

\def\scsys#1{{{\sc #1}}\index{#1@{\sc #1}}\xspace}
\def\mmt{\textsc{Mmt}\xspace}
\def\xml{\scsys{Xml}}
\def\mathml{\scsys{MathML}}
\def\omdoc{\scsys{OMDoc}}
\def\openmath{\scsys{OpenMath}}
\def\latexml{\scsys{LaTeXML}}
\def\perl{\scsys{Perl}}
\def\cmathml{Content-{\sc MathML}\index{Content {\sc MathML}}\index{MathML@{\sc MathML}!content}}
\def\activemath{\scsys{ActiveMath}}
\def\twin#1#2{\index{#1!#2}\index{#2!#1}}
\def\twintoo#1#2{{#1 #2}\twin{#1}{#2}}
\def\atwin#1#2#3{\index{#1!#2!#3}\index{#3!#2 (#1)}}
\def\atwintoo#1#2#3{{#1 #2 #3}\atwin{#1}{#2}{#3}}
\def\cT{\mathcal{T}}\def\cD{\mathcal{D}}

\def\fileversion{3.0}
\def\filedate{\today}

\RequirePackage{pdfcomment}

\ExplSyntaxOn\makeatletter
\cs_set_protected:Npn \@comp #1 #2 {
  \pdftooltip {
    \textcolor{blue}{#1}
  } { #2 }
}

\cs_set_protected:Npn \@defemph #1 #2 {
  \pdftooltip { 
    \textbf{\textcolor{magenta}{#1}}
  } { #2 }
}

\def\__omtext_lec#1{#1}
\cs_new_protected:Npn \lec #1 {
  \strut\hfil\strut\null\hfill\__omtext_lec{#1}
}
\makeatother\ExplSyntaxOff

\makeatletter
\let\@stex@oldcomment\comment
\let\@stex@oldendcomment\endcomment

%\RequirePackage{comment}
\RequirePackage{document-structure}
\RequirePackage[hints,solutions,notes]{problem}
\RequirePackage{hwexam}

\let\comment\@stex@oldcomment
\let\endcomment\@stex@oldendcomment

\newif\ifinfulldoc\infulldocfalse
\makeatother

\def\basedocurl{https://github.com/slatex/sTeX/blob/latex3/doc}
\newcounter{module}

\NewDocumentEnvironment {module}{}{
  \stepcounter{module}
  \textbf{Module \themodule: \smoduletitle}
}{

}
\stexpatchmodule{\begin{module}}{\end{module}}

\def\compemph#1{\textcolor{blue}{#1}}
\def\symrefemph#1{\textcolor{green}{#1}}

\RequirePackage{pdfcomment}
\makeatletter
\protected\def\compemph@uri#1#2{%
  \pdftooltip{%
    \srefsymuri{#2}{\compemph{#1}}%
  }{%
    URI: \detokenize{#2}%
  }%
}
\protected\def\symrefemph@uri#1#2{%
  \pdftooltip{%
    \srefsymuri{#2}{\symrefemph{#1}}%
  }{%
    URI: \detokenize{#2}%
  }%
}
\makeatother

\begin{document}
  \DocInput{\jobname.dtx}
\end{document}
%</driver>
% \fi
%
% \title{ \sTeX-References
% 	\thanks{Version {\fileversion} (last revised {\filedate})} 
% }
%
% \author{Michael Kohlhase, Dennis Müller\\
% 	FAU Erlangen-Nürnberg\\
% 	\url{http://kwarc.info/}
% }
%
% \maketitle
%
%\ifinfulldoc\else
% This is the documentation for the \pkg{stex-references} package.
% For a more high-level introduction, 
%  see \href{\basedocurl/manual.pdf}{the \sTeX Manual} or the
% \href{\basedocurl/stex.pdf}{full \sTeX documentation}.
%
% \textcolor{red}{TODO: references documentation}
% \fi
%
% \begin{documentation}\label{pkg:sref:doc}
%
% Code related to links and cross-references
%
% \section{Macros and Environments}\label{pkg:sref:doc:macros}
%
% \end{documentation}
%
% \begin{implementation}\label{pkg:sref:impl}
%
% \section{\sTeX-References Implementation}
%
%    \begin{macrocode}
%<*package>

%%%%%%%%%%%%%   references.dtx   %%%%%%%%%%%%%

%\RequirePackage{hyperref}
%\RequirePackage{cleveref}
%<@@=stex_refs>
%    \end{macrocode}
%
% Warnings and error messages
%
%    \begin{macrocode}

%    \end{macrocode}
%
%    \begin{macrocode}
\iow_new:N \c_@@_refs_iow
\AddToHook{begindocument}{
  \iow_open:Nn \c_@@_refs_iow {\jobname.sref}
}
\AddToHook{enddocument}{
  \iow_close:N \c_@@_refs_iow
}

\str_set:Nn \g_@@_title_tl {Unnamed~Document}

\NewDocumentCommand \STEXreftitle { m } {
  \tl_gset:Nx \g_@@_title_tl { #1 }
}
%    \end{macrocode}
%
% \subsection{Document URIs and URLs}
%
%    \begin{macrocode}
\seq_new:N \g_@@_all_refs_seq

\str_new:N \l_stex_current_docns_str

\cs_new_protected:Nn \stex_get_document_uri: {
  \seq_set_eq:NN \l_tmpa_seq \g_stex_currentfile_seq
  \seq_pop_right:NN \l_tmpa_seq \l_tmpb_str
  \exp_args:NNno \seq_set_split:Nnn \l_tmpb_seq . \l_tmpb_str
  \seq_get_left:NN \l_tmpb_seq \l_tmpb_str
  \seq_put_right:No \l_tmpa_seq \l_tmpb_str

  \str_clear:N \l_tmpa_str
  \prop_if_exist:NT \l_stex_current_repository_prop {
    \prop_get:NnNF \l_stex_current_repository_prop { narr } \l_tmpa_str {
      \prop_get:NnNF \l_stex_current_repository_prop { ns } \l_tmpa_str {}
    }
  }

  \str_if_empty:NTF \l_tmpa_str {
    \str_set:Nx \l_stex_current_docns_str { 
      file:/\stex_path_to_string:N \l_tmpa_seq
    }
  }{
    \bool_set_true:N \l_tmpa_bool
    \bool_while_do:Nn \l_tmpa_bool {
      \seq_pop_left:NN \l_tmpa_seq \l_tmpb_str
      \exp_args:No \str_case:nnTF { \l_tmpb_str } {
        {source} { \bool_set_false:N \l_tmpa_bool }
      }{}{
        \seq_if_empty:NT \l_tmpa_seq {
          \bool_set_false:N \l_tmpa_bool
        }
      }
    }
  
    \seq_if_empty:NTF \l_tmpa_seq {
      \str_set_eq:NN \l_stex_current_docns_str \l_tmpa_str
    }{
      \str_set:Nx \l_stex_current_docns_str { 
        \l_tmpa_str/\stex_path_to_string:N \l_tmpa_seq
      }
    }
  }
}
%    \end{macrocode}
%
%
%    \begin{macrocode}
\str_new:N \l_stex_current_docurl_str
\cs_new_protected:Nn \stex_get_document_url: {
  \seq_set_eq:NN \l_tmpa_seq \g_stex_currentfile_seq
  \seq_pop_right:NN \l_tmpa_seq \l_tmpb_str
  \exp_args:NNno \seq_set_split:Nnn \l_tmpb_seq . \l_tmpb_str
  \seq_get_left:NN \l_tmpb_seq \l_tmpb_str
  \seq_put_right:No \l_tmpa_seq \l_tmpb_str

  \str_clear:N \l_tmpa_str
  \prop_if_exist:NT \l_stex_current_repository_prop {
    \prop_get:NnNF \l_stex_current_repository_prop { docurl } \l_tmpa_str {
      \prop_get:NnNF \l_stex_current_repository_prop { narr } \l_tmpa_str {
        \prop_get:NnNF \l_stex_current_repository_prop { ns } \l_tmpa_str {}
      }
    }
  }

  \str_if_empty:NTF \l_tmpa_str {
    \str_set:Nx \l_stex_current_docurl_str { 
      file:/\stex_path_to_string:N \l_tmpa_seq
    }
  }{
    \bool_set_true:N \l_tmpa_bool
    \bool_while_do:Nn \l_tmpa_bool {
      \seq_pop_left:NN \l_tmpa_seq \l_tmpb_str
      \exp_args:No \str_case:nnTF { \l_tmpb_str } {
        {source} { \bool_set_false:N \l_tmpa_bool }
      }{}{
        \seq_if_empty:NT \l_tmpa_seq {
          \bool_set_false:N \l_tmpa_bool
        }
      }
    }
  
    \seq_if_empty:NTF \l_tmpa_seq {
      \str_set_eq:NN \l_stex_current_docurl_str \l_tmpa_str
    }{
      \str_set:Nx \l_stex_current_docurl_str { 
        \l_tmpa_str/\stex_path_to_string:N \l_tmpa_seq
      }
    }
  }
}
%    \end{macrocode}
%
% \subsection{Setting Reference Targets}
%
%    \begin{macrocode}
\str_const:Nn \c_@@_url_str{URL}
\str_const:Nn \c_@@_ref_str{REF}
% @currentlabel -> number
% @currentlabelname -> title
% @currentHref -> name.number <- id of some kind
% \theH# -> \arabic{section}
% \the#  -> number
% \hyper@makecurrent{#}
\cs_new_protected:Nn \stex_ref_new_doc_target:n {
  \stex_get_document_uri:
  \str_set:Nx \l_tmpa_str { #1 }
  \str_if_empty:NT \l_tmpa_str {
    \int_zero:N \l_tmpa_int
    \bool_set_true:N \l_tmpa_bool
    \bool_while_do:Nn \l_tmpa_bool {
      \cs_if_exist:cTF {
        sref_\l_stex_current_docns_str?? REF_\int_use:N \l_tmpa_int _type
      }{
        \int_incr:N \l_tmpa_int
      }{
        \str_set:Nx \l_tmpa_str { REF_\int_use:N \l_tmpa_int }
        \bool_set_false:N \l_tmpa_bool
      }
    }
  }
  \str_set:Nx \l_tmpa_str {
    \l_stex_current_docns_str??\l_tmpa_str
  }
  \seq_gput_right:No \g_@@_all_refs_seq \l_tmpa_str
  \stex_if_smsmode:TF {
    \stex_get_document_url:
    \str_gset_eq:cN {sref_url_\l_tmpa_str _str}\l_stex_current_docurl_str
    \str_gset_eq:cN {sref_\l_tmpa_str _type}\c_@@_url_str
  }{
    \iow_now:Nx \c_@@_refs_iow { \l_tmpa_str~=~\expandafter{\@currentlabel\iffalse}{\fi~in~\exp_args:No\unexpanded\g_@@_title_tl},}
    \exp_args:Nx\label{sref_\l_tmpa_str}

    \exp_args:NNNx\immediate\write\@auxout{\stexauxadddocref{\l_tmpa_str}}
    \str_gset:cx {sref_\l_tmpa_str _type}\c_@@_ref_str
  }
}
\cs_new_protected:Npn \stexauxadddocref #1 {
  \str_set:Nx \l_tmpa_str {#1}
  \str_gset_eq:cN{sref_\l_tmpa_str _type}\c_@@_ref_str
  \seq_gput_right:Nx \g_@@_all_refs_seq {\l_tmpa_str}
}
%    \end{macrocode}
%
%    \begin{macrocode}
\cs_new_protected:Nn \stex_ref_new_sym_target:n {
  \str_gset_eq:cN {sref_sym_#1_uri} \l_stex_current_docns_str
}
%    \end{macrocode}
%
% \subsection{Using References}
%
%    \begin{macrocode}
\str_new:N \l_@@_indocument_str
\keys_define:nn { stex / sref } {
  linktext      .tl_set:N  = \l_@@_linktext_tl ,
  fallback      .tl_set:N  = \l_@@_fallback_tl ,
  pre           .tl_set:N  = \l_@@_pre_tl ,
  post          .tl_set:N  = \l_@@_post_tl ,
  %indoc         .str_set_x:N  = \l_@@_repo_str ,
}

\bool_new:N \c_@@_hyperref_bool
\bool_set_false:N \c_@@_hyperref_bool
\AddToHook{begindocument}{
  \@ifpackageloaded{hyperref}{
    \bool_set_true:N \c_@@_hyperref_bool
  }{}
}


\cs_new_protected:Nn \_@@_args:n {
  \tl_clear:N \l_@@_linktext_tl
  \tl_clear:N \l_@@_fallback_tl
  \tl_clear:N \l_@@_pre_tl
  \tl_clear:N \l_@@_post_tl
  \str_clear:N \l_@@_repo_str
  \keys_set:nn { stex / sref } { #1 }
}

\NewDocumentCommand \sref { O{} m}{
  \_@@_args:n { #1 }
  \str_if_empty:NTF \l_@@_indocument_str {
    \str_set:Nn \l_tmpa_str { #2 }
    \int_set:Nn \l_tmpa_int { \str_count:N \l_tmpa_str }
    \tl_set:Nn \l_tmpa_tl {
      \l_@@_fallback_tl
    }
    \seq_map_inline:Nn \g_@@_all_refs_seq {
      \str_set:Nn \l_tmpb_str { ##1 }
      \str_if_eq:eeT { \l_tmpa_str } {
        \str_range:Nnn \l_tmpb_str { -\l_tmpa_int }{ -1 }
      } {
        \seq_map_break:n {
          \tl_set:Nn \l_tmpa_tl {
            % doc uri in \l_tmpb_str
            \str_set:Nx \l_tmpa_str {\use:c{sref_\l_tmpb_str _type}}
            \str_if_eq:NNTF \l_tmpa_str \c_@@_ref_str {
              % reference
              \cs_if_exist:cTF{autoref}{
                \l_@@_pre_tl\autoref{sref_\l_tmpb_str}\l_@@_post_tl
              }{
                \l_@@_pre_tl\ref{sref_\l_tmpb_str}\l_@@_post_tl
              }
            }{
              % URL
              \if_bool:N \c_@@_hyperref_bool {
                \exp_args:Nx \href{\use:c{sref_url_\l_tmpb_str _str}}{\l_@@_fallback_tl}
              }{
                \l_@@_fallback_tl
              }
            }
          }
        }
      }
    }
    \l_tmpa_tl
  }{
    % TODO
  }
}

%    \end{macrocode}
%
%
%    \begin{macrocode}
%</package>
%    \end{macrocode}
%
% \end{implementation}
%
% \PrintIndex

\end{omgroup}

\begin{omgroup}[id=sec.stexarchives]{\sTeX Archives}
  %%
%% This is file `mathhub.sty',
%% generated with the docstrip utility.
%%
%% The original source files were:
%%
%% mathhub.dtx  (with options: `package')
%% 
\NeedsTeXFormat{LaTeX2e}[1999/12/01]
\RequirePackage{stex-base}
\RequirePackage{keyval}
\if@latexml\else\RequirePackage{xparse}\fi
\if@latexml\else\RequirePackage{xstring}\fi
\if@latexml\else\RequirePackage[abspath]{currfile}\fi
\RequirePackage{pathsuris}
\newcommand\mhcurrentrepos[1]{%
  \edef\@test{#1}%
  \ifx\@test\mh@currentrepos% if new dir = old dir
    \relax% no need to change
  \else%
    \protected@write\@auxout{}{\string\@mhcurrentrepos{#1}}%
  \fi%
  \@mhcurrentrepos{#1}% define mh@currentrepos
}%
\newcommand\@mhcurrentrepos[1]{\edef\mh@currentrepos{#1}}%
\def\modules@@first#1/#2;{#1}
\newcommand\libinput[1]{%
\edef\@mh@group{\expandafter\modules@@first\mh@currentrepos;}
\let\orig@inffile\mh@inffile\let\orig@libfile\mh@libfile
\def\mh@inffile{\MathHub{\@mh@group/meta-inf/lib/#1}}
\def\mh@libfile{\MathHub{\mh@currentrepos/lib/#1}}%
\IfFileExists\mh@inffile{\input\mh@inffile}{}%
\IfFileExists\mh@inffile{}{\IfFileExists\mh@libfile{}{%
  {\PackageError{mathhub}
    {Library file missing; cannot input #1.tex\MessageBreak%
    Both \mh@libfile.tex\MessageBreak and \mh@inffile.tex\MessageBreak%
    do not exist}%
  {Check whether the file name is correct}}}}
\IfFileExists\mh@libfile{\input\mh@libfile\relax}{}
\let\mh@inffile\orig@inffile\let\mh@libfile\orig@libfile}
\newcommand\libusepackage[2][]{%
\edef\@mh@group{\expandafter\modules@@first\mh@currentrepos;}
\let\orig@inffile\mh@inffile\let\orig@libfile\mh@libfile
\edef\mh@inffile{\MathHub{\@mh@group/meta-inf/lib/#2}}
\edef\mh@libfile{\MathHub{\mh@currentrepos/lib/#2}}%
\IfFileExists{\mh@inffile.sty}{\usepackage[#1]{\mh@inffile}}{}%
\IfFileExists {\mh@inffile.sty}{}{\IfFileExists{\mh@libfile.sty}{}{%
  {\PackageError{mathhub}
    {Library file missing; cannot use package #2.sty\MessageBreak%
    Both \mh@libfile.sty\MessageBreak and \mh@inffile.sty\MessageBreak%
    do not exist}%
  {Check whether the file name is correct}}}}
\IfFileExists{\@libfile.sty}{\usepackage[#1]{\@libfile}}{}}

% adapted from
% https://tex.stackexchange.com/questions/62010/can-i-access-system-environment-variables-from-latex-for-instance-home, check there if it breaks.

\ExplSyntaxOn
\sys_get_shell:nnN{kpsewhich ~ --var-value ~ MATHHUB} { } \MATHHUB
\tl_trim_spaces:N \l_tmpa_tl
\ifx\MATHHUB\empty\else
\def\temp_def_path#1{\defpath{MathHub}{#1}}
\expandafter\temp_def_path\expandafter{\MATHHUB}
\fi
\ExplSyntaxOff
%\mhcurrentrepos{\StrBetween\PWD\MATHHUB{/current/}}
\endinput
%%
%% End of file `mathhub.sty'.

\end{omgroup}

\begin{omgroup}{Creating New Modules and Symbols}
	\textcolor{red}{TODO}
  %%
%% This is file `modules.sty',
%% generated with the docstrip utility.
%%
%% The original source files were:
%%
%% modules.dtx  (with options: `package')
%% 
\NeedsTeXFormat{LaTeX2e}[1999/12/01]
\ProvidesPackage{modules}[2020/10/14 v1.6 Semantic Markup]
\newif\if@modules@html@\@modules@html@true
\DeclareOption{omdocmode}{\@modules@html@false}
\newif\if@modules@mh@\@modules@mh@false
\DeclareOption{mh}{\@modules@mh@true}
\newif\ifmod@show\mod@showfalse
\DeclareOption{showmods}{\mod@showtrue}
\newif\ifaux@req\aux@reqtrue
\DeclareOption{noauxreq}{\aux@reqfalse}
\newif\ifmod@qualified\mod@qualifiedfalse
\DeclareOption{qualifiedimports}{\mod@qualifiedtrue}
\newif\if@trwarn\@trwarnfalse
\DeclareOption{trwarn}{\@trwarntrue}
\DeclareOption*{\PassOptionsToPackage{\CurrentOption}{sref}}
\ProcessOptions
\RequirePackage{stex-base}
\RequirePackage{sref}
\RequirePackage{pathsuris}
\RequirePackage[abspath]{currfile}
\RequirePackage{standalone}
\if@modules@mh@\RequirePackage{modules-mh}\fi
\RequirePackage{xspace}
\if@latexml\else\ifmod@show\RequirePackage{mdframed}\fi\fi
\addmetakey*{module}{title}
\addmetakey*{module}{id}
\addmetakey*{module}{creators}
\addmetakey*{module}{contributors}
\addmetakey*{module}{srccite}
\addmetakey*{module}{align}[WithTheModuleOfTheSameName]
\addmetakey*{module}{ns}
\addmetakey*{module}{narr}
\addmetakey*{module}{noalign}[true]
\ifdef{\thesection}{\newcounter{module}[section]}{\newcounter{module}}%
\newrobustcmd\module@heading{%
  \stepcounter{module}%
  \ifmod@show%
  \noindent{\textbf{Module} \thesection.\themodule [\module@id]}%
  \sref@label@id{Module \thesection.\themodule [\module@id]}%
    \ifx\module@title\@empty :\quad\else\quad(\module@title)\hfill\\\fi%
  \fi%
}% mod@show
\newenvironment{module}[1][]{%
  \begin{@module}[#1]%
    \ifcsundef{mod@path}{}{\csxdef{module@\module@id @path}{\mod@path}}%
    \ifcsundef{mod@ext}{}{\csxdef{module@\module@id @ext}{\mod@ext}}%
  \module@heading% make the headings
  \ignorespacesandpars\usemodule@maybesetcodes}{%
  \end{@module}%
  \ignorespacesafterend%
}%
\ifmod@show\surroundwithmdframed{module@om@common}\fi%
\newif\ifarchive@ns@empty@\archive@ns@empty@false
\def\set@default@ns{
  \edef\@module@ns@temp{\currfileabspath}
  \edef\@module@ns@temp{\if@iswindows@\windows@to@path\@module@ns@temp\else\@module@ns@temp\fi}
  \path@dropextension\@module@ns@temp{@module@ns@temp}
  \archive@ns@empty@false
  \unless\ifcsname mathhub@archive@ns\endcsname
    \archive@ns@empty@true
  \else
    \ifx\mathhub@archive@ns\@empty\archive@ns@empty@true\fi
  \fi
  \ifarchive@ns@empty@
    \asuri{@module@ns@temp}{file\@Colon\@Slash\@Slash\@module@ns@temp}
    \@module@ns@temp{drop extension}
    \setkeys{module}{ns=\@module@ns@tempuri}
  \else
    \asuri{@module@filepath@temp}{file\@Colon\@Slash\@Slash\@module@ns@temp}
    \asuri{@module@ns@temp}{\mathhub@archive@ns}
    \asuri{@module@archivedir}{file\@Colon\@Slash\@Slash\mathhub@archive@dir\@Slash source}
    \IfBeginWith\@module@filepath@temppath\@module@archivedirpath{
      \StrLen\@module@archivediruri[\ns@temp@length]
      \StrGobbleLeft\@module@filepath@tempuri\ns@temp@length[\@module@filepath@tempuri]
      \edef\@module@ns@tempuri{\@module@ns@tempuri\@module@filepath@tempuri}
    }{}
    \setkeys{module}{ns=\@module@ns@tempuri}
  \fi
}

\def\set@next@moduleid{
  \unless\ifcsname namespace@\module@ns @unnamedmodules\endcsname
      \csgdef{namespace@\module@ns @unnamedmodules}{0}
  \fi
  \edef\namespace@currnum{\csname namespace@\module@ns @unnamedmodules\endcsname}
  \edef\module@temp@setidname{\noexpand\setkeys{module}{id=module\namespace@currnum}}
  \module@temp@setidname
  \csxdef{namespace@\module@ns @unnamedmodules}{\the\numexpr\namespace@currnum+1}
}

\newenvironment{@module}[1][]{
  \metasetkeys{module}{#1}
  \ifx\module@ns\@empty\set@default@ns\fi
  \ifx\module@narr\@empty
    \setkeys{module}{narr=\module@ns}
  \fi
  \ifcsname module@id\endcsname
    \ifx\module@id\@empty\set@next@moduleid\fi
  \else\set@next@moduleid\fi
  \seturi[module@uri@]{\module@ns\@QuestionMark\module@id}
  \csxdef{\module@uri@uri}{\noexpand\@invoke@module{\module@uri@uri}}
  \csxdef{moduleid@\module@id}{\noexpand\@invoke@module{\module@uri@uri}}
  \csxdef{Module\module@id}{\noexpand\@invoke@module{\module@uri@uri}}
  \edef\this@module{%
    \expandafter\noexpand\csname module@defs@\module@uri@uri\endcsname%
  }%
  \csxdef{module@defs@\module@uri@uri}{%
    \expandafter\def\expandafter\noexpand\csname Module\module@id\endcsname%
      {\noexpand\@invoke@module{\module@uri@uri}}%
  }%
  \ifmod@qualified%
    \edef\this@qualified@module{%
      \expandafter\noexpand\csname module@defs@\module@uri@uri\endcsname%
    }%
    \csxdef{module@defs@qualified@\module@uri@uri}{%
      \expandafter\def\expandafter\noexpand\csname Module\module@id\endcsname%
      {\noexpand\@invoke@module{\module@uri@uri}}%
    }%
  \fi%
}{}%

\def\@URI{uri}

\def\@invoke@module#1#2{%
  \ifx\@URI#2%
    #1%
  \else%
    % TODO something else
    #2%
  \fi%
}

\def\activate@defs#1{%
  \edef\activate@defs@uri{\csname moduleid@#1\endcsname\@URI}%
  \ifcsundef{module@\activate@defs@uri @activated}{\csname module@defs@\activate@defs@uri\endcsname}{}%
  \@namedef{module@\activate@defs@uri @activated}{true}%
}%
\def\g@addto@macro@safe#1#2{\ifx#1\relax\def#1{}\fi\g@addto@macro#1{#2}}
\def\export@defs#1{\@ifundefined{module@id}{}{%
\expandafter\expandafter\expandafter\g@addto@macro@safe%
\expandafter\this@module\expandafter{\activate@defs{#1}}}}%
\newif\if@importing\@importingfalse
\newcommand\update@used@modules[1]{%
  \ifx\used@modules\@empty%
    \edef\used@modules{#1}%
  \else%
    \edef\used@modules{\used@modules,#1}%
  \fi}
\gdef\used@modules{}
\srefaddidkey{importmodule}
\addmetakey{importmodule}{load}
\addmetakey{importmodule}{dir}
\addmetakey[false]{importmodule}{conservative}[true]
\newcommand\importmodule[2][]{%
\metasetkeys{importmodule}{#1}%
\usemodule@maybesetcodes
\update@used@modules{#2}%
\ifx\importmodule@dir\@empty
\@importmodule[\importmodule@load]{#2}{export}%
\else\@importmodule[\importmodule@dir/#2]{#2}{export}\fi%
\ignorespacesandpars}
\newcommand\@importmodule[3][]{%
{\@importingtrue% to shut up macros while in the group opened here
\edef\@load{#1}%
\edef\@load{\expandafter\detokenize\expandafter{\@load}}%
\ifx\@load\@empty\relax\else%
\ifcsundef{module@#2@path}{\requiremodules{#1}}%
{\edef\@path{\csname module@#2@path\endcsname}%
\IfStrEq\@load\@path{\relax}% if the known path is the same as the requested one do nothing
{\PackageError{modules}% else signal an error
{Module Name Clash\MessageBreak
A module with name #2 was already loaded under the path "\@path"\MessageBreak
The imported path "\@load" is probably a different module with the\MessageBreak
same name; this is dangerous -- not importing}%
{Check whether the Module name is correct}}}%
\fi}%
\activate@defs{#2}% activate the module
\edef\@export{#3}\def\@@export{export}%prepare comparison
\ifx\@export\@@export\export@defs{#2}\fi% export the module
}%
\newcommand\usemodule[2][]{%
\metasetkeys{importmodule}{#1}%
\update@used@modules{#2}%
\ifx\importmodule@dir\@empty
\@importmodule[\importmodule@load]{#2}{noexport}%
\else\@importmodule[\importmodule@dir/#2]{#2}{noexport}\fi%
\ignorespacesandpars}
\newcommand\withusedmodules[2]{{\@for\@I:=#1\do{\activate@defs\@I}{#2}}}%
\newrobustcmd\importOMDocmodule[3][]{\PackageError{modules}%
  {The \protect\importOMDocmodule macro is deprecated}
  {use \protect\importmodule instead!}}%
\let\metalanguage=\importmodule%
\let\mod@newcommand=\providerobustcmd%
\srefaddidkey{conceptdef}%
\addmetakey*{conceptdef}{title}%
\addmetakey{conceptdef}{subject}%
\addmetakey*{conceptdef}{display}%
\def\conceptdef@type{Symbol}%
\newrobustcmd\conceptdef[2][]{%
  \metasetkeys{conceptdef}{#1}%
  \ifx\conceptdef@display\st@flow\else{\stDMemph{\conceptdef@type} #2:}\fi%
  \ifx\conceptdef@title\@empty~\else~(\stDMemph{\conceptdef@title})\par\fi%
}%
\newif\if@symdeflocal%
\srefaddidkey{symdef}%
\define@key{symdef}{local}[true]{\@symdeflocaltrue}%
\define@key{symdef}{noverb}[all]{}%
\define@key{symdef}{align}[WithTheSymbolOfTheSameName]{}%
\define@key{symdef}{specializes}{}%
\addmetakey*{symdef}{noalign}[true]
\define@key{symdef}{primary}[true]{}%
\define@key{symdef}{assocarg}{}%
\define@key{symdef}{bvars}{}%
\define@key{symdef}{bargs}{}%
\addmetakey{symdef}{ns}%
\addmetakey{symdef}{name}%
\addmetakey*{symdef}{title}%
\addmetakey*{symdef}{description}%
\addmetakey{symdef}{subject}%
\addmetakey*{symdef}{display}%
\def\symdef{\@ifnextchar[{\@symdef}{\@symdef[]}}%
\def\@symdef[#1]#2{\@ifnextchar[{\@@symdef[#1]{#2}}{\@@symdef[#1]{#2}[0]}}%
\def\@mod@nc#1{\mod@newcommand{#1}[1]}%
\def\ignorespacesandpars{\begingroup\catcode13=10\@ifnextchar\relax{\endgroup}{\endgroup}}
\def\ignorespacesandparsafterend#1\ignorespaces\fi{#1\fi\ignorespacesandpars}
\def\ignorespacesandpars{\ifhmode\unskip\fi\@ifnextchar\par{\expandafter\ignorespacesandpars\@gobble}{}}
\def\@@symdef[#1]#2[#3]#4{%
  \@symdeflocalfalse%
  \metasetkeys{symdef}{#1}%
  \usemodule@maybesetcodes%
  \expandafter\mod@newcommand\csname modules@#2@pres@\endcsname[#3]{#4}%
  \expandafter\mod@newcommand\csname #2\endcsname[1][]%
  {\csname modules@#2@pres@##1\endcsname}%
\expandafter\@mod@nc\csname mod@symref@#2\expandafter\endcsname\expandafter%
{\expandafter\mod@termref\expandafter{\module@uri@uri}{#2}{##1}}%
  \if@symdeflocal%
  \else%
    \ifcsundef{module@id}{}{%
      \expandafter\g@addto@macro@safe\this@module%
      {\expandafter\mod@newcommand\csname modules@#2@pres@\endcsname[#3]{#4}}%
      \expandafter\g@addto@macro@safe\this@module%
      {\expandafter\mod@newcommand\csname #2\endcsname[1][]%
      {\csname modules@#2@pres@##1\endcsname}}%
        \expandafter\expandafter\expandafter\g@addto@macro@safe\expandafter\this@module\expandafter%
        {\expandafter\@mod@nc\csname mod@symref@#2\expandafter\endcsname\expandafter%
        {\expandafter\mod@termref\expandafter{\module@uri@uri}{#2}{##1}}}%
      \ifmod@qualified%
        \expandafter\g@addto@macro@safe\this@qualified@module%
        {\expandafter\mod@newcommand\csname modules@#2@pres@qualified\endcsname[#3]{#4}}%
        \expandafter\g@addto@macro@safe\this@qualified@module%
        {\expandafter\def\csname#2@qualified\endcsname{\csname modules@#2@pres@qualified\endcsname}}%
      \fi%
    }% mod@qualified
\fi% symdeflocal
  \ifmod@show%
    \ifx\symdef@display\st@flow\else{\noindent\stDMemph{\symdef@type} #2:}\fi%
    \ifx\symdef@title\@empty~\else~(\stDMemph{\symdef@title})\par\fi%
  \fi%
  \ignorespacesandpars%
}% mod@show
\def\symdef@type{Symbol}%
\providecommand{\stDMemph}[1]{\textbf{#1}}
\def\symvariant#1{%
  \@ifnextchar[{\@symvariant{#1}}{\@symvariant{#1}[0]}%
  }%
\def\@symvariant#1[#2]#3#4{%
  \usemodule@maybesetcodes
  \expandafter\mod@newcommand\csname modules@#1@pres@#3\endcsname[#2]{#4}%
  \ifcsundef{module@id}{}{%
    \expandafter\g@addto@macro\this@module%
    {\expandafter\mod@newcommand\csname modules@#1@pres@#3\endcsname[#2]{#4}}%
  }%
\ignorespacesandpars}%
\def\resymdef{%
  \@ifnextchar[{\@resymdef}{\@resymdef[]}%
}%
\def\@resymdef[#1]#2{%
  \@ifnextchar[{\@@resymdef[#1]{#2}}{\@@resymdef[#1]{#2}[0]}%
}%
\def\@@resymdef[#1]#2[#3]#4{%
  \PackageError{modules}%
  {The \protect\resymdef macro is deprecated}{use the \protect\symvariant instead!}%
}%
\let\abbrdef\symdef%
\define@key{DefMathOp}{name}{%
  \def\defmathop@name{#1}%
}%
\newrobustcmd\DefMathOp[2][]{%
  \setkeys{DefMathOp}{#1}%
  \symdef[#1]{\defmathop@name}{#2}%
}%
\newcommand\assdef[2][]{#2}
\let\vardef\abbrdef
\addmetakey{symtest}{name}%
\addmetakey{symtest}{variant}%
\newrobustcmd\symtest[3][]{%
  \if@importing%
  \else%
    \metasetkeys{symtest}{#1}%
    \par\noindent \textbf{Symbol}~%
    \ifx\symtest@name\@empty\texttt{#2}\else\texttt{\symtest@name}\fi%
    \ifx\symtest@variant\@empty\else\ (variant \texttt{\symtest@variant})\fi%
    \ with semantic macro %
    \texttt{\textbackslash #2\ifx\symtest@variant\@empty\else[\symtest@variant]\fi}%
    : used e.g. in \ensuremath{#3}%
  \fi%
  \ignorespacesandpars%
}%
\addmetakey{abbrtest}{name}%
\newrobustcmd\abbrtest[3][]{%
  \if@importing%
  \else%
    \metasetkeys{abbrtest}{#1}%
    \par\noindent \textbf{Abbreviation}~%
    \ifx\abbrtest@name\@empty\texttt{#2}\else\texttt{\abbrtest@name}\fi%
    : used e.g. in \ensuremath{#3}%
  \fi%
  \ignorespacesandpars}%
\def\mod@true{true}%
\addmetakey[false]{termdef}{local}%
\addmetakey{termdef}{name}%
\newrobustcmd\termdef[3][]{%
  \metasetkeys{termdef}{#1}%
  \expandafter\mod@newcommand\csname#2\endcsname[0]{#3\xspace}%
  \ifx\termdef@local\mod@true%
  \else%
    \ifcsundef{module@id}{}{%
      \expandafter\g@addto@macro\this@module%
      {\expandafter\mod@newcommand\csname#2\endcsname[0]{#3\xspace}}%
    }%
  \fi%
}%
\def\@capitalize#1{\uppercase{#1}}%
\newrobustcmd\capitalize[1]{\expandafter\@capitalize #1}%
\newcommand\mod@component[1]{}
\newcommand\mod@termref[3]{\def\@test{#3}%
  \@ifundefined{module@defs@#1}{%
    \protect\G@refundefinedtrue%
    \if@trwarn
      \PackageWarning{modules}{`\protect\termref' with unidentified cd "#1":\MessageBreak
        the cd key must reference an active module}%
    \else
      \PackageError{modules}{`\protect\termref' with unidentified cd "#1"}
      {the cd key must reference an active module}%
    \fi}%
  {\def\@label{sref@#2@#1\mod@component{#1}@target}%
    \@ifundefined{module@#1@path}% local reference
    {\sref@hlink@ifh{\@label}{\ifx\@test\@empty #2\else #3\fi}%
    }%
    {\def\@uri{\csname module@#1@path\endcsname\mod@component{#1}.pdf\#\@label}%
      \sref@href@ifh{\@uri}{\ifx\@test\@empty #2\else #3\fi}%
}%
  }}%
\newif\if@smsmode\@smsmodefalse
\def\usemodule@escapechar@allowed{true}
\def\usemodule@allow#1{
  \expandafter\let\csname usemodule@allowedmacro@#1\endcsname\usemodule@escapechar@allowed
}
\def\usemodule@allowenv#1{
  \expandafter\let\csname usemodule@allowedenv@#1\endcsname\usemodule@escapechar@allowed
}
\def\usemodule@escapechar@beginstring{begin}
\def\usemodule@escapechar@endstring{end}
\usemodule@allow{symdef}
\usemodule@allow{abbrdef}
\usemodule@allow{importmodule}
\usemodule@allowenv{module}
\usemodule@allow{importmhmodule}
\usemodule@allow{gimport}
\usemodule@allowenv{modsig}
\usemodule@allowenv{mhmodsig}
\usemodule@allowenv{mhmodnl}
\usemodule@allowenv{modnl}
\usemodule@allow{symvariant}
\usemodule@allow{symi}
\usemodule@allow{symii}
\usemodule@allow{symiii}
\usemodule@allow{symiv}
\catcode`\.=0
.catcode`.\=13
.def.@active@slash{\}
.catcode`.<=1
.catcode`.>=2
.catcode`.{=12
.catcode`.}=12
.def.@open@brace<{>
.def.@close@brace<}>
.catcode`.\=0
\catcode`\.=12
\catcode`\{=1
\catcode`\}=2
\catcode`\<=12
\catcode`\>=12
  \def\set@usemodule@catcodes{%
      \global\catcode`\\=13%
      \global\catcode`\#=12%
      \global\catcode`\{=12%
      \global\catcode`\}=12%
      \global\catcode`\$=12%$
      \global\catcode`\^=12%
      \global\catcode`\_=12%
      \global\catcode`\&=12%
      \expandafter\let\@active@slash\usemodule@escapechar%
  }
  \def\reset@usemodule@catcodes{%
      \global\catcode`\\=0%
      \global\catcode`\#=6%
      \global\catcode`\{=1%
      \global\catcode`\}=2%
      \global\catcode`\$=3%$
      \global\catcode`\^=7%
      \global\catcode`\_=8%
      \global\catcode`\&=4%
  }
  \def\usemodule@maybesetcodes{%
    \if@smsmode\set@usemodule@catcodes\fi%
  }
  \newrobustcmd\requiremodules[1]{%
    \mod@showfalse%
    \edef\mod@path{#1}%
    \edef\mod@path{\expandafter\detokenize\expandafter{\mod@path}}%
    \requiremodules@smsmode{#1}%
  }%
  \newbox\modules@import@tempbox
  \def\requiremodules@smsmode#1{
    \setbox\modules@import@tempbox\vbox{%
      \@smsmodetrue%
      \set@usemodule@catcodes%
      \hbadness=100000\relax%
      \hsize=10pt%
      \hfuzz=10000pt\relax%
      \input{#1.tex}%
      \reset@usemodule@catcodes%
      }%
      \usemodule@maybesetcodes
  }

\def\usemodule@escapechar{%
    \def\usemodule@escape@currcs{}%
    \usemodule@escape@parse@nextchar@%
}%
\long\def\usemodule@escape@parse@nextchar@#1{%
    \ifcat a#1 %
        \edef\usemodule@escape@currcs{\usemodule@escape@currcs#1}%
        \let\usemodule@do@next\usemodule@escape@parse@nextchar@%
    \else%
      \def\usemodule@last@char{#1}%
      \def\usemodule@do@next{\usemodule@escapechar@checkcs}%
    \fi%
    \usemodule@do@next%
}
\def\usemodule@escapechar@checkcs{
    \ifx\usemodule@escape@currcs\usemodule@escapechar@beginstring%
        \edef\usemodule@do@next{\noexpand\usemodule@escapechar@checkbeginenv\usemodule@last@char}%
    \else%
        \ifx\usemodule@escape@currcs\usemodule@escapechar@endstring%
          \edef\usemodule@do@next{\noexpand\usemodule@escapechar@checkendenv\usemodule@last@char}%
        \else%
            \expandafter\ifx\csname usemodule@allowedmacro@\usemodule@escape@currcs\endcsname%
                \usemodule@escapechar@allowed%
              \ifx\usemodule@last@char\@open@brace%
                \expandafter\let\expandafter\usemodule@do@next@ii\csname\usemodule@escape@currcs\endcsname
                \edef\usemodule@do@next{\noexpand\usemodule@converttoproperbraces\@open@brace}
              \else%
                \reset@usemodule@catcodes%
                \edef\usemodule@do@next{\expandafter\noexpand\csname\usemodule@escape@currcs\endcsname\usemodule@last@char}
              \fi%
            \else\def\usemodule@do@next{\relax\usemodule@last@char}\fi%
        \fi%
    \fi%
    \usemodule@do@next%
}
\expandafter\expandafter\expandafter\def%
\expandafter\expandafter\expandafter\usemodule@converttoproperbraces%
\expandafter\@open@brace\expandafter#\expandafter1\@close@brace{%
  \reset@usemodule@catcodes%
  \usemodule@do@next@ii{#1}%
}
\expandafter\expandafter\expandafter\def%
\expandafter\expandafter\expandafter\usemodule@escapechar@checkbeginenv%
\expandafter\@open@brace\expandafter#\expandafter1\@close@brace{%
    \expandafter\ifx\csname usemodule@allowedenv@#1\endcsname\usemodule@escapechar@allowed%
        \reset@usemodule@catcodes%
        \def\usemodule@do@next{\begin{#1}}%
    \else%
        \def\usemodule@do@next{#1}%
    \fi%
    \usemodule@do@next%
}
\expandafter\expandafter\expandafter\def%
\expandafter\expandafter\expandafter\usemodule@escapechar@checkendenv%
\expandafter\@open@brace\expandafter#\expandafter1\@close@brace{%
    \expandafter\ifx\csname usemodule@allowedenv@#1\endcsname\usemodule@escapechar@allowed%
        %\reset@usemodule@catcodes%
        \def\usemodule@do@next{\end{#1}}%
    \else%
      \def\usemodule@do@next{#1}%
    \fi%
    \usemodule@do@next%
}
\newrobustcmd\@requiremodules[1]{%
  \if@tempswa\requiremodules{#1}\fi%
}%
\def\inputref@preskip{}
\def\inputref@postskip{}
\newrobustcmd\inputref[1]{%
  \def\@Slash{/}
  \edef\@load{#1}%
  \StrChar{\@load}{1}[\@testchar]
  \inputref@preskip%
  \ifx\@testchar\@Slash%
    \edef\mod@path{#1}%
    \edef\mod@path{\expandafter\detokenize\expandafter{\mod@path}}%
    \input{#1}%
  \else%
    \@cpath{#1}\input{\@CanPath.tex}%
  \fi%
  \inputref@postskip%
}%
\def\requirepackage#1#2{\makeatletter\input{#1.sty}\makeatother}%
\newcommand\namespace[2][]{\ignorespacesandpars}
\newrobustcmd\sinput[1]{%
  \PackageError{modules}%
  {The `\protect\sinput' macro is deprecated}{use the \protect\input instead!}%
}%
\newrobustcmd\sinputref[1]{%
  \PackageError{modules}%
  {The \protect\sinputref macro is deprecated}{use the \protect\inputref instead!}%
}%
\define@key{module}{uses}{\PackageError{modules}%
  {The 'uses' key on {module} macro is deprecated}{}}
\define@key{module}{usesqualified}{\PackageError{modules}%
  {The 'usesqualified' key on {module} macro is deprecated}{}}
\def\coolurion{\PackageWarning{modules}{coolurion is obsolete, please remove}}%
\def\coolurioff{\PackageWarning{modules}{coolurioff is obsolete, please remove}}%
\def\csymdef{\@ifnextchar[{\@csymdef}{\@csymdef[]}}%
\def\@csymdef[#1]#2{%
  \@ifnextchar[{\@@csymdef[#1]{#2}}{\@@csymdef[#1]{#2}[0]}%
}%
\def\@@csymdef[#1]#2[#3]#4#5{%
  \@@symdef[#1]{#2}[#3]{#4}%
}%
\def\notationdef[#1]#2#3{}
\newrobustcmd\reqmodules[2]{%
  \ifinlist{#1}{\@register}{}{\listadd\@register{#1}\input{#1.#2}}%
}%
\newcounter{@pl}
\DeclareListParser*{\forpathlist}{/}
\def\file@name#1{%
  \setcounter{@pl}{0}%
  \forpathlist{\stepcounter{@pl}\listadd\@pathlist}{#1}
  \def\do##1{%
    \ifnumequal{\value{@pl}}{1}{##1}{\addtocounter{@pl}{-1}}
  }%
  \dolistloop{\@pathlist}%
}%
\def\file@path#1{%
  \setcounter{@pl}{0}%
  \forpathlist{\stepcounter{@pl}\listadd\@pathlist}{#1}%
  \def\do##1{%
    \ifnumequal{\value{@pl}}{1}{}{%
      \addtocounter{@pl}{-1}%
      \ifnumequal{\value{@pl}}{1}{##1}{##1/}%
    }%
  }%
  \dolistloop{\@pathlist}%
}%
\def\@NEWcurrentprefix{}
\def\NEWrequiremodules#1{%
  \def\@pref{\file@path{#1}}%
  \ifx\@pref\@empty%
  \else%
    \xdef\@NEWcurrentprefix{\@NEWcurrentprefix/\@pref}%
  \fi%
  \edef\@input@me{\@NEWcurrentprefix/\file@name{#1}}%
  \message{requiring \@input@me}\reqmodule{\@input@me}%
}%
\endinput
%%
%% End of file `modules.sty'.

  \textcolor{red}{TODO: symbols documentation}
  % \iffalse meta-comment
% An Infrastructure for Semantic Macros and Module Scoping
% Copyright (c) 2019 Michael Kohlhase, all rights reserved
%                this file is released under the
%                LaTeX Project Public License (LPPL)
% 
% The original of this file is in the public repository at 
% http://github.com/sLaTeX/sTeX/
%
% TODO update copyright  
%
%<*driver>
\providecommand\bibfolder{../../lib/bib}
\RequirePackage{paralist}
\documentclass[full,kernel]{l3doc}
\usepackage[dvipsnames]{xcolor}
\usepackage[utf8]{inputenc}
\usepackage[T1]{fontenc}
\RequirePackage{morewrites}
\RequirePackage{tikzinput}
\usetikzlibrary{fit}

\usepackage[debug=all,lang=en, mathhub=./tests]{stex}
\usepackage{url,array,float,textcomp}
\usepackage[show]{ed}
\usepackage[hyperref=auto,style=alphabetic]{biblatex}
\addbibresource{\bibfolder/kwarcpubs.bib}
\addbibresource{\bibfolder/extpubs.bib}
\addbibresource{\bibfolder/kwarccrossrefs.bib}
\addbibresource{\bibfolder/extcrossrefs.bib}
\usepackage{amssymb}
\usepackage{amsfonts}
\usepackage{xspace}
\usepackage{hyperref}

\makeindex
\floatstyle{boxed}
\newfloat{exfig}{thp}{lop}
\floatname{exfig}{Example}

\usepackage{stex-tests}

\MakeShortVerb{\|}

\def\scsys#1{{{\sc #1}}\index{#1@{\sc #1}}\xspace}
\def\mmt{\textsc{Mmt}\xspace}
\def\xml{\scsys{Xml}}
\def\mathml{\scsys{MathML}}
\def\omdoc{\scsys{OMDoc}}
\def\openmath{\scsys{OpenMath}}
\def\latexml{\scsys{LaTeXML}}
\def\perl{\scsys{Perl}}
\def\cmathml{Content-{\sc MathML}\index{Content {\sc MathML}}\index{MathML@{\sc MathML}!content}}
\def\activemath{\scsys{ActiveMath}}
\def\twin#1#2{\index{#1!#2}\index{#2!#1}}
\def\twintoo#1#2{{#1 #2}\twin{#1}{#2}}
\def\atwin#1#2#3{\index{#1!#2!#3}\index{#3!#2 (#1)}}
\def\atwintoo#1#2#3{{#1 #2 #3}\atwin{#1}{#2}{#3}}
\def\cT{\mathcal{T}}\def\cD{\mathcal{D}}

\def\fileversion{3.0}
\def\filedate{\today}

\RequirePackage{pdfcomment}

\ExplSyntaxOn\makeatletter
\cs_set_protected:Npn \@comp #1 #2 {
  \pdftooltip {
    \textcolor{blue}{#1}
  } { #2 }
}

\cs_set_protected:Npn \@defemph #1 #2 {
  \pdftooltip { 
    \textbf{\textcolor{magenta}{#1}}
  } { #2 }
}

\def\__omtext_lec#1{#1}
\cs_new_protected:Npn \lec #1 {
  \strut\hfil\strut\null\hfill\__omtext_lec{#1}
}
\makeatother\ExplSyntaxOff

\makeatletter
\let\@stex@oldcomment\comment
\let\@stex@oldendcomment\endcomment

%\RequirePackage{comment}
\RequirePackage{document-structure}
\RequirePackage[hints,solutions,notes]{problem}
\RequirePackage{hwexam}

\let\comment\@stex@oldcomment
\let\endcomment\@stex@oldendcomment

\newif\ifinfulldoc\infulldocfalse
\makeatother

\def\basedocurl{https://github.com/slatex/sTeX/blob/latex3/doc}
\newcounter{module}

\NewDocumentEnvironment {module}{}{
  \stepcounter{module}
  \textbf{Module \themodule: \smoduletitle}
}{

}
\stexpatchmodule{\begin{module}}{\end{module}}

\def\compemph#1{\textcolor{blue}{#1}}
\def\symrefemph#1{\textcolor{green}{#1}}

\RequirePackage{pdfcomment}
\makeatletter
\protected\def\compemph@uri#1#2{%
  \pdftooltip{%
    \srefsymuri{#2}{\compemph{#1}}%
  }{%
    URI: \detokenize{#2}%
  }%
}
\protected\def\symrefemph@uri#1#2{%
  \pdftooltip{%
    \srefsymuri{#2}{\symrefemph{#1}}%
  }{%
    URI: \detokenize{#2}%
  }%
}
\makeatother

\begin{document}
  \DocInput{\jobname.dtx}
\end{document}
%</driver>
% \fi
%
% \title{ \sTeX-Module Inheritance
% 	\thanks{Version {\fileversion} (last revised {\filedate})} 
% }
%
% \author{Michael Kohlhase, Dennis Müller\\
% 	FAU Erlangen-Nürnberg\\
% 	\url{http://kwarc.info/}
% }
%
% \maketitle
%
%\ifinfulldoc\else
% This is the documentation for the \pkg{stex-inheritance} package.
% For a more high-level introduction, 
%  see \href{\basedocurl/manual.pdf}{the \sTeX Manual} or the
% \href{\basedocurl/stex.pdf}{full \sTeX documentation}.
%
% % \iffalse meta-comment
% An Infrastructure for Semantic Macros and Module Scoping
% Copyright (c) 2019 Michael Kohlhase, all rights reserved
%                this file is released under the
%                LaTeX Project Public License (LPPL)
% 
% The original of this file is in the public repository at 
% http://github.com/sLaTeX/sTeX/
%
% TODO update copyright  
%
%<*driver>
\providecommand\bibfolder{../../lib/bib}
\RequirePackage{paralist}
\documentclass[full,kernel]{l3doc}
\usepackage[dvipsnames]{xcolor}
\usepackage[utf8]{inputenc}
\usepackage[T1]{fontenc}
\RequirePackage{morewrites}
\RequirePackage{tikzinput}
\usetikzlibrary{fit}

\usepackage[debug=all,lang=en, mathhub=./tests]{stex}
\usepackage{url,array,float,textcomp}
\usepackage[show]{ed}
\usepackage[hyperref=auto,style=alphabetic]{biblatex}
\addbibresource{\bibfolder/kwarcpubs.bib}
\addbibresource{\bibfolder/extpubs.bib}
\addbibresource{\bibfolder/kwarccrossrefs.bib}
\addbibresource{\bibfolder/extcrossrefs.bib}
\usepackage{amssymb}
\usepackage{amsfonts}
\usepackage{xspace}
\usepackage{hyperref}

\makeindex
\floatstyle{boxed}
\newfloat{exfig}{thp}{lop}
\floatname{exfig}{Example}

\usepackage{stex-tests}

\MakeShortVerb{\|}

\def\scsys#1{{{\sc #1}}\index{#1@{\sc #1}}\xspace}
\def\mmt{\textsc{Mmt}\xspace}
\def\xml{\scsys{Xml}}
\def\mathml{\scsys{MathML}}
\def\omdoc{\scsys{OMDoc}}
\def\openmath{\scsys{OpenMath}}
\def\latexml{\scsys{LaTeXML}}
\def\perl{\scsys{Perl}}
\def\cmathml{Content-{\sc MathML}\index{Content {\sc MathML}}\index{MathML@{\sc MathML}!content}}
\def\activemath{\scsys{ActiveMath}}
\def\twin#1#2{\index{#1!#2}\index{#2!#1}}
\def\twintoo#1#2{{#1 #2}\twin{#1}{#2}}
\def\atwin#1#2#3{\index{#1!#2!#3}\index{#3!#2 (#1)}}
\def\atwintoo#1#2#3{{#1 #2 #3}\atwin{#1}{#2}{#3}}
\def\cT{\mathcal{T}}\def\cD{\mathcal{D}}

\def\fileversion{3.0}
\def\filedate{\today}

\RequirePackage{pdfcomment}

\ExplSyntaxOn\makeatletter
\cs_set_protected:Npn \@comp #1 #2 {
  \pdftooltip {
    \textcolor{blue}{#1}
  } { #2 }
}

\cs_set_protected:Npn \@defemph #1 #2 {
  \pdftooltip { 
    \textbf{\textcolor{magenta}{#1}}
  } { #2 }
}

\def\__omtext_lec#1{#1}
\cs_new_protected:Npn \lec #1 {
  \strut\hfil\strut\null\hfill\__omtext_lec{#1}
}
\makeatother\ExplSyntaxOff

\makeatletter
\let\@stex@oldcomment\comment
\let\@stex@oldendcomment\endcomment

%\RequirePackage{comment}
\RequirePackage{document-structure}
\RequirePackage[hints,solutions,notes]{problem}
\RequirePackage{hwexam}

\let\comment\@stex@oldcomment
\let\endcomment\@stex@oldendcomment

\newif\ifinfulldoc\infulldocfalse
\makeatother

\def\basedocurl{https://github.com/slatex/sTeX/blob/latex3/doc}
\newcounter{module}

\NewDocumentEnvironment {module}{}{
  \stepcounter{module}
  \textbf{Module \themodule: \smoduletitle}
}{

}
\stexpatchmodule{\begin{module}}{\end{module}}

\def\compemph#1{\textcolor{blue}{#1}}
\def\symrefemph#1{\textcolor{green}{#1}}

\RequirePackage{pdfcomment}
\makeatletter
\protected\def\compemph@uri#1#2{%
  \pdftooltip{%
    \srefsymuri{#2}{\compemph{#1}}%
  }{%
    URI: \detokenize{#2}%
  }%
}
\protected\def\symrefemph@uri#1#2{%
  \pdftooltip{%
    \srefsymuri{#2}{\symrefemph{#1}}%
  }{%
    URI: \detokenize{#2}%
  }%
}
\makeatother

\begin{document}
  \DocInput{\jobname.dtx}
\end{document}
%</driver>
% \fi
%
% \title{ \sTeX-Module Inheritance
% 	\thanks{Version {\fileversion} (last revised {\filedate})} 
% }
%
% \author{Michael Kohlhase, Dennis Müller\\
% 	FAU Erlangen-Nürnberg\\
% 	\url{http://kwarc.info/}
% }
%
% \maketitle
%
%\ifinfulldoc\else
% This is the documentation for the \pkg{stex-inheritance} package.
% For a more high-level introduction, 
%  see \href{\basedocurl/manual.pdf}{the \sTeX Manual} or the
% \href{\basedocurl/stex.pdf}{full \sTeX documentation}.
%
% % \iffalse meta-comment
% An Infrastructure for Semantic Macros and Module Scoping
% Copyright (c) 2019 Michael Kohlhase, all rights reserved
%                this file is released under the
%                LaTeX Project Public License (LPPL)
% 
% The original of this file is in the public repository at 
% http://github.com/sLaTeX/sTeX/
%
% TODO update copyright  
%
%<*driver>
\providecommand\bibfolder{../../lib/bib}
\input{../../doc/docheader}

\begin{document}
  \DocInput{\jobname.dtx}
\end{document}
%</driver>
% \fi
%
% \title{ \sTeX-Module Inheritance
% 	\thanks{Version {\fileversion} (last revised {\filedate})} 
% }
%
% \author{Michael Kohlhase, Dennis Müller\\
% 	FAU Erlangen-Nürnberg\\
% 	\url{http://kwarc.info/}
% }
%
% \maketitle
%
%\ifinfulldoc\else
% This is the documentation for the \pkg{stex-inheritance} package.
% For a more high-level introduction, 
%  see \href{\basedocurl/manual.pdf}{the \sTeX Manual} or the
% \href{\basedocurl/stex.pdf}{full \sTeX documentation}.
%
% \input{../../doc/packages/inheritance}
% \fi
%
% \begin{documentation}\label{pkg:inheritance:doc}
%
% Code related to Module Inheritance, in particular \emph{sms mode}.
%
% \section{Macros and Environments}\label{pkg:inheritance:doc:macros}
%
%    \subsection{SMS Mode}
% ``SMS Mode'' is used when loading modules from external tex files.
% It deactivates any output and ignores all \TeX\ commands
% not explicitly allowed via the following lists:
%
% \begin{variable}{\g_stex_smsmode_allowedmacros_tl}
%   Macros that are executed as is; i.e. with the category code scheme
%   used in SMS mode.
% \end{variable}
%
% \begin{variable}{\g_stex_smsmode_allowedmacros_escape_tl}
%   Macros that are executed with the category codes restored.
%
%   Importantly, these macros need to call \cs{stex_smsmode_set_codes:}
%   after reading all arguments. Note, that 
%   \cs{stex_smsmode_set_codes:} takes
%   care of checking whether we are in SMS mode in the first place, 
%   so calling this function eagerly is unproblematic.
% \end{variable}
%
% \begin{variable}{\g_stex_smsmode_allowedenvs_seq}
%   The names of environments that should be allowed in SMS mode.
%   The corresponding \cs{begin}-statements are treated like
%   the macros in \cs{g_stex_smsmode_allowedmacros_escape_tl}, so
%   \cs{stex_smsmode_set_codes:} should be called at the end of the
%   \cs{begin}-code. Since \cs{end}-statements take no arguments anyway,
%   those are called with the SMS mode category code scheme active.
% \end{variable}
%
% \begin{function}[pTF]{\stex_if_smsmode:}
%   Tests whether SMS mode is currently active.
% \end{function}
%
% \begin{function}{\stex_smsmode_set_codes:}
%   Sets the current category code scheme to that of the SMS mode, if
%   SMS mode is currently active and if necessary.
%
%   This method should be called at the end of every macro or 
%   \cs{begin} environment code that are allowed in SMS mode.
%
% \end{function}
%
% \begin{function}{\stex_in_smsmode:nn}
%   \begin{syntax} \cs{stex_in_smsmode:nn} \Arg{name} \Arg{code} \end{syntax}
%   Executes \meta{code} in SMS mode. \meta{name} can be arbitrary,
%   but should be distinct, since it allows for nesting 
%   \cs{stex_in_smsmode:nn} without spuriously terminating SMS mode.
% \end{function}
%
%\stextest{
% \immediate\openout\testfile=./tests/sometest.tex
% \immediate\write\testfile{\detokenize{\this is \a test}^^J}
% \immediate\write\testfile{\detokenize{this \is a \test}}
% \immediate\closeout\testfile
% \ExplSyntaxOn
% \stex_in_smsmode:nn { foo } { 
%   \input{tests/sometest.tex} 
% }
% \ExplSyntaxOff
%}
%
%
%    \subsection{Imports and Inheritance}
%
% \begin{function}{\importmodule}
%   \begin{syntax} \cs{importmodule}|[|\meta{archive-ID}|]|\Arg{module-path} \end{syntax}
%   Imports a module by reading it from a file and ``activating'' it.
%   \sTeX determines the module and its containing file by passing its
%   arguments on to \cs{stex_import_module_path:nn}.
%   
% \end{function}
%
%\stextest{
%   \begin{module}{Foo}
%      \symdecl[name=foo, args=3]{bar}
%      \symdecl[args=bai]{foobar}
%      Meaning:~\present\bar\\
%   \end{module}
%      Meaning:~\present\bar\\
%   \begin{module}{Importtest}
%     \importmodule{Foo}
%      Meaning:~\present\bar\\
%   \end{module}
%   \begin{module}{Importtest2}
%     \importmodule{Importtest}
%      Meaning:~\present\bar\\
%   \end{module}
%}
%
% \begin{function}{\usemodule}
%   \begin{syntax} \cs{importmodule}|[|\meta{archive-ID}|]|\Arg{module-path} \end{syntax}
%   Like \cs{importmodule}, but does not export its contents;
%   i.e. including the current module will not activate the used module
%   
% \end{function}
%
%\stextest{
%   \begin{module}{UseTest1}
%      \symdecl{foo}
%   \end{module}
%   \begin{module}{UseTest2}
%     \usemodule{UseTest1}
%      \symdecl{bar}
%      Meaning:~\present\foo\\
%   \end{module}
%   \begin{module}{UseTest3}
%     \importmodule{UseTest2}
%     Meaning:~\present\foo\\
%     Meaning:~\present\bar\\
%
%     All modules: \ExplSyntaxOn
%       \seq_use:Nn \l_stex_all_modules_seq {,~} \\
%      All~symbols:~
%        \seq_use:Nn \l_stex_all_symbols_seq {,~}
%     \ExplSyntaxOff 
%   \end{module}
%}
%
%
%\stextest{
% Circular dependencies:
% \begin{module}{CircDep1}
%   \importmodule[Foo/Bar]{circular1?Circular1}
%   \importmodule[Bar/Foo]{circular2?Circular2}
%   \present\fooA\\
%   \present\fooB
% \end{module}
%}
%
% \begin{function}{\stex_import_module_uri:nn}
%   \begin{syntax} \cs{stex_import_module_uri:nn} \Arg{archive-ID} \Arg{module-path} \end{syntax}
%   Determines the URI of a module by splitting
%   \meta{module-path} into \meta{path}|?|\meta{name}. If \meta{module-path}
%   does \emph{not} contain a |?|-character, we consider it to be the \meta{name},
%   and \meta{path} to be empty.
%
%   If \meta{archive-ID} is empty, it is automatically set to the
%   ID of the current archive (if one exists).
%
%   \begin{enumerate}
%   \item If \meta{archive-ID} is empty:
%     \begin{enumerate}
%     \item If \meta{path} is empty, then
%         \meta{name} must have been declared earlier in the same file
%         and retrievable from \cs{g_stex_modules_in_file_seq}, or
%         a file with name \meta{name}|.|\meta{lang}|.tex| must exist
%         in the same folder, containing a module \meta{name}.
%
%         That module should have the same namespace as the current one.
%     \item If \meta{path} is not empty, it must point to the relative
%         path of the containing file as well as the namespace.
%     \end{enumerate}
%   \item Otherwise:
%      \begin{enumerate}
%         \item If \meta{path} is empty, then
%         \meta{name} must have been declared earlier in the same file
%         and retrievable from \cs{g_stex_modules_in_file_seq}, or
%         a file with name \meta{name}|.|\meta{lang}|.tex| must exist
%         in the top |source| folder of the archive, 
%         containing a module \meta{name}.
%
%         That module should lie directly in the namespace 
%         of the archive.
%     \item If \meta{path} is not empty, it must point to the
%         path of the containing file as well as the namespace,
%         relative to the namespace of the archive.
%
%         If a module by that namespace exists, it is returned.
%         Otherwise, we call \cs{stex_require_module:nn} 
%         on the |source| directory of the archive to find the
%         file.
%       \end{enumerate}
%   \end{enumerate}
% \end{function}
%
% \begin{function}{\stex_import_require_module:nnnn}
%    \begin{syntax} \Arg{ns} \Arg{archive-ID} \Arg{path} \Arg{name} \end{syntax}
%     Checks whether a module with URI \meta{ns}|?|\meta{name} already
%     exists. If not, it looks for a plausible file that declares
%     a module with that URI.
%
%     Finally, activates that module by executing its |content|-field.
% \end{function}
%
%
% \end{documentation}
%
% \begin{implementation}\label{pkg:inheritance:impl}
%
% \section{\sTeX-Module Inheritance Implementation}
%
%    \begin{macrocode}
%<*package>

%%%%%%%%%%%%%   inheritance.dtx   %%%%%%%%%%%%%

%    \end{macrocode}
%
% \subsection{SMS Mode}
%    \begin{macrocode}
%<@@=stex_smsmode>
%    \end{macrocode}
%
% \begin{variable}{
%   \g_stex_smsmode_allowedmacros_tl,
%   \g_stex_smsmode_allowedmacros_escape_tl,
%   \g_stex_smsmode_allowedenvs_seq
% }
%    \begin{macrocode}
\tl_new:N \g_stex_smsmode_allowedmacros_tl
\tl_new:N \g_stex_smsmode_allowedmacros_escape_tl
\seq_new:N \g_stex_smsmode_allowedenvs_seq

\tl_set:Nn \g_stex_smsmode_allowedmacros_tl {
  \makeatletter
  \makeatother
  \ExplSyntaxOn
  \ExplSyntaxOff
  \rustexBREAK
}

\tl_set:Nn \g_stex_smsmode_allowedmacros_escape_tl {
  \symdef
  \importmodule
  \notation
  \symdecl
  \STEXexport
  \inlineass
  \inlinedef
  \inlineex
}

\exp_args:NNx \seq_set_from_clist:Nn \g_stex_smsmode_allowedenvs_seq {
  \tl_to_str:n {
    module,
    @module,
    sdefinition,
    sexample,
    sassertion,
    sparagraph
  }
}
%    \end{macrocode}
% \end{variable}
%
% \begin{macro}[pTF]{\stex_if_smsmode:}
%    \begin{macrocode}
\bool_new:N \g_@@_bool
\bool_set_false:N \g_@@_bool
\prg_new_conditional:Nnn \stex_if_smsmode: { p, T, F, TF } {
  \bool_if:NTF \g_@@_bool \prg_return_true: \prg_return_false:
}
%    \end{macrocode}
% \end{macro}
%
%
% \begin{macro}[pTF]{\_@@_if_catcodes:}
% Checks whether the SMS mode category code scheme is active.
%    \begin{macrocode}
\bool_new:N \g_@@_catcode_bool
\bool_set_false:N \g_@@_catcode_bool
\prg_new_conditional:Nnn \_@@_if_catcodes: { p, T, F, TF } {
  \bool_if:NTF \g_@@_catcode_bool 
    \prg_return_true: \prg_return_false:
}
%    \end{macrocode}
% \end{macro}
%
% \begin{macro}{\stex_smsmode_set_codes:}
%    \begin{macrocode}
\cs_new_protected:Nn \stex_smsmode_set_codes: {
  \stex_if_smsmode:T {
    \_@@_if_catcodes:F {
      \bool_gset_true:N \g_@@_catcode_bool
      \exp_after:wN \char_gset_active_eq:NN 
        \c_backslash_str \_@@_cs:
      \tex_global:D \char_set_catcode_active:N \\
      \tex_global:D \char_set_catcode_other:N $
      \tex_global:D \char_set_catcode_other:N ^
      \tex_global:D \char_set_catcode_other:N _
      \tex_global:D \char_set_catcode_other:N &
      \tex_global:D \char_set_catcode_other:N ##
    }
  }
} \iffalse $ \fi % to make syntax highlighting work again
%    \end{macrocode}
% \end{macro}
%
% \begin{macro}{\_@@_unset_codes:}
%   Sets category code scheme back from the one used in SMS mode.
%    \begin{macrocode}
\cs_new_protected:Nn \_@@_unset_codes: {
  \_@@_if_catcodes:T {
    \bool_gset_false:N \g_@@_catcode_bool
    \exp_after:wN \tex_global:D \exp_after:wN 
      \char_set_catcode_escape:N \c_backslash_str
    \tex_global:D \char_set_catcode_math_toggle:N $
    \tex_global:D \char_set_catcode_math_superscript:N ^
    \tex_global:D \char_set_catcode_math_subscript:N _
    \tex_global:D \char_set_catcode_alignment:N &
    \tex_global:D \char_set_catcode_parameter:N ##
  }
} \iffalse $ \fi % to make syntax highlighting work again
%    \end{macrocode}
% \end{macro}
%
% \begin{macro}{\stex_in_smsmode:nn}
%    \begin{macrocode}
\cs_new_protected:Nn \stex_in_smsmode:nn {
  \vbox_set:Nn \l_tmpa_box {
    \bool_set_eq:cN { l_@@_#1_bool } \g_@@_bool
    \bool_gset_true:N \g_@@_bool
    \stex_smsmode_set_codes:
    #2
    \bool_gset_eq:Nc \g_@@_bool { l_@@_#1_bool }
    \stex_if_smsmode:F {
      \_@@_unset_codes:
    }
  }
  \box_clear:N \l_tmpa_box
}
%    \end{macrocode}
% \end{macro}
%
% \begin{macro}{\_@@_cs:}
% is executed on encountering |\| in smsmode.
% It checks whether the corresponding command is allowed and executes
% or ignores it accordingly:
%    \begin{macrocode}
\cs_new_protected:Nn \_@@_cs: {
  \str_clear:N \l_tmpa_str
  \peek_analysis_map_inline:n {
    % #1: token (one expansion)
    % #2: charcode
    % #3 catcode
    \token_if_eq_charcode:NNTF ##3 B {
      % token is a letter
      \exp_args:NNo \str_put_right:Nn \l_tmpa_str { ##1 }
    } {
      \str_if_empty:NTF \l_tmpa_str {
        % we don't allow (or need) single non-letter CSs
        % for now
        \peek_analysis_map_break: 
      }{
        \str_if_eq:onTF \l_tmpa_str { begin } {
          \peek_analysis_map_break:n { 
            \exp_after:wN \_@@_checkbegin:n ##1
          }
        } {
          \str_if_eq:onTF \l_tmpa_str { end } {
            \peek_analysis_map_break:n { 
              \exp_after:wN \_@@_checkend:n ##1
            }
          } {
          \tl_set:Nn \l_tmpa_tl { \use:c{\l_tmpa_str} }
          \exp_args:NNNo \exp_args:NNo \tl_if_in:NnTF 
            \g_stex_smsmode_allowedmacros_tl 
              { \use:c{\l_tmpa_str} } {
              \stex_debug:nn{modules}{Executing~1:~\l_tmpa_str}
              \peek_analysis_map_break:n { 
                \exp_after:wN \l_tmpa_tl ##1
              }
            } {
              \exp_args:NNNo \exp_args:NNo \tl_if_in:NnTF 
              \g_stex_smsmode_allowedmacros_escape_tl 
                { \use:c{\l_tmpa_str} } {
                \_@@_unset_codes:
                \stex_debug:nn{modules}{Executing~2:~\l_tmpa_str}
                % TODO \_@@_rescan_cs:
%                \int_compare:nNnTF {##2} = {92} {
%                  \peek_analysis_map_break:n {
%                    \_@@_unset_codes:
%                    \_@@_rescan_cs:
%                  }
%                } {
                  \peek_analysis_map_break:n {
                    \exp_after:wN \l_tmpa_tl ##1
                  }
%                }
              } {
                  \int_compare:nNnTF {##2} = {92} {
                    \peek_analysis_map_break:n { \_@@_cs: }
                  }{
                    \peek_analysis_map_break:n { \exp_after:wN\relax ##1 }
                  }
              }
            }
          }
        }
      }
    }
  }
}
%    \end{macrocode}
% \end{macro}
%
%
% \begin{macro}{\_@@_rescan_cs:}
% If the last token gobbled by |\stex_smsmode_cs:| happened to be
% a |\|, we need to rescan the cs name and reinsert it into the input
% stream:
%    \begin{macrocode}
\cs_new_protected:Nn \_@@_rescan_cs: {
  \str_clear:N \l_tmpb_str
  \peek_analysis_map_inline:n {
    \token_if_eq_charcode:NNTF ##3 B {
      % token is a letter
      \exp_args:NNo \str_put_right:Nn \l_tmpb_str { ##1 }
    } {
      \peek_analysis_map_break:n {
        \exp_after:wN \use:c \exp_after:wN { 
          \exp_after:wN \l_tmpa_str\exp_after:wN 
        } \use:c { \l_tmpb_str \exp_after:wN } ##1
      }
    }
  }
}
%    \end{macrocode}
% \end{macro}
%
% \begin{macro}{\_@@_checkbegin:n}
% called on |\begin|; checks whether the environment being opened
% is allowed in SMS mode. 
%    \begin{macrocode}
\cs_new_protected:Nn \_@@_checkbegin:n {
  \str_set:Nn \l_tmpa_str { #1 }
  \seq_if_in:NoT \g_stex_smsmode_allowedenvs_seq \l_tmpa_str {
    \_@@_unset_codes:
    \begin{#1}
  }
}
%    \end{macrocode}
% \end{macro}
%
% \begin{macro}{\_@@_checkend:n}
% called on |\end|; checks whether the environment being opened
% is allowed in SMS mode. 
%    \begin{macrocode}
\cs_new_protected:Nn \_@@_checkend:n {
  \str_set:Nn \l_tmpa_str { #1 }
  \seq_if_in:NoT \g_stex_smsmode_allowedenvs_seq \l_tmpa_str {
    \end{#1}
  }
}
%    \end{macrocode}
% \end{macro}
%
% \subsection{Inheritance}
%    \begin{macrocode}
%<@@=stex_importmodule>
%    \end{macrocode}
%
% \begin{macro}{\stex_import_module_uri:nn}
%    \begin{macrocode}
\cs_new_protected:Nn \stex_import_module_uri:nn {
  \str_set:Nx \l_stex_import_archive_str { #1 }
  \str_set:Nn \l_stex_import_path_str { #2 }

  \exp_args:NNNo \seq_set_split:Nnn \l_tmpb_seq ? { \l_stex_import_path_str }
  \seq_pop_right:NN \l_tmpb_seq \l_stex_import_name_str
  \str_set:Nx \l_stex_import_path_str { \seq_use:Nn \l_tmpb_seq ? }

  \stex_modules_current_namespace:
  \bool_lazy_all:nTF {
    {\str_if_empty_p:N \l_stex_import_archive_str}
    {\str_if_empty_p:N \l_stex_import_path_str}
    {\stex_if_module_exists_p:n { \l_stex_module_ns_str ? \l_stex_import_name_str } }
  }{
    \str_set_eq:NN \l_stex_import_path_str \l_stex_modules_subpath_str
    \str_set_eq:NN \l_stex_import_ns_str \l_stex_module_ns_str
  }{
    \str_if_empty:NT \l_stex_import_archive_str {
      \prop_if_exist:NT \l_stex_current_repository_prop {
        \prop_get:NnN \l_stex_current_repository_prop { id } \l_stex_import_archive_str
      }
    }  
    \str_if_empty:NTF \l_stex_import_archive_str {
      \str_if_empty:NF \l_stex_import_path_str {
        \str_set:Nx \l_stex_import_ns_str {
          \l_stex_module_ns_str / \l_stex_import_path_str
        }
      }
    }{
      \stex_require_repository:n \l_stex_import_archive_str
      \prop_get:cnN { c_stex_mathhub_\l_stex_import_archive_str _manifest_prop } { ns }
        \l_stex_import_ns_str
      \str_if_empty:NF \l_stex_import_path_str {
        \str_set:Nx \l_stex_import_ns_str {
          \l_stex_import_ns_str / \l_stex_import_path_str
        }
      }
    }
  }
}
%    \end{macrocode}
% \end{macro}
%
% \begin{variable}{
%   \l_stex_import_name_str,\l_stex_import_archive_str,\l_stex_import_path_str,\l_stex_import_ns_str
% }
%   Store the return values of \cs{stex_import_module_uri:nn}.
%    \begin{macrocode}
\str_new:N \l_stex_import_name_str
\str_new:N \l_stex_import_archive_str
\str_new:N \l_stex_import_path_str
\str_new:N \l_stex_import_ns_str
%    \end{macrocode}
% \end{variable}
%
% \begin{macro}{\stex_import_require_module:nnnn}
%    \begin{syntax} \Arg{ns} \Arg{archive-ID} \Arg{path} \Arg{name} \end{syntax}
%    \begin{macrocode}
\cs_new_protected:Nn \stex_import_require_module:nnnn {
  \exp_args:Nx \stex_if_module_exists:nF { #1 ? #4 } {

    % archive
    \str_set:Nx \l_tmpa_str { #2 }
    \str_if_empty:NTF \l_tmpa_str {
      \seq_set_eq:NN \l_tmpa_seq \g_stex_currentfile_seq
    } {
      \stex_path_from_string:Nn \l_tmpb_seq { \l_tmpa_str }
      \seq_concat:NNN \l_tmpa_seq \c_stex_mathhub_seq \l_tmpb_seq
      \seq_put_right:Nn \l_tmpa_seq { source }
    }

    % path
    \str_set:Nx \l_tmpb_str { #3 }
    \str_if_empty:NTF \l_tmpb_str {
      \str_set:Nx \l_tmpa_str { \stex_path_to_string:N \l_tmpa_seq / #4 }
      
      \ltx@ifpackageloaded{babel} { 
        \exp_args:NNx \prop_get:NnNF \c_stex_language_abbrevs_prop 
            { \languagename } \l_tmpb_str {
              \msg_error:nnx{stex}{error/unknownlanguage}{\languagename}
            }
      } {
        \str_clear:N \l_tmpb_str
      }

      \stex_debug:nn{modules}{Checking~\l_tmpa_str.\l_tmpb_str.tex}
      \IfFileExists{ \l_tmpa_str.\l_tmpb_str.tex }{
        \str_gset:Nx \g_@@_file_str { \l_tmpa_str.\l_tmpb_str.tex }
      }{
        \stex_debug:nn{modules}{Checking~\l_tmpa_str.tex}
        \IfFileExists{ \l_tmpa_str.tex }{
          \str_gset:Nx \g_@@_file_str { \l_tmpa_str.tex }
        }{
          % try english as default
          \stex_debug:nn{modules}{Checking~\l_tmpa_str.en.tex}
          \IfFileExists{ \l_tmpa_str.en.tex }{
            \str_gset:Nx \g_@@_file_str { \l_tmpa_str.en.tex }
          }{
            \msg_error:nnx{stex}{error/unknownmodule}{#1?#4}
          }
        }
      }

    } {
      \seq_set_split:NnV \l_tmpb_seq / \l_tmpb_str
      \seq_concat:NNN \l_tmpa_seq \l_tmpa_seq \l_tmpb_seq
      
      \ltx@ifpackageloaded{babel} { 
        \exp_args:NNx \prop_get:NnNF \c_stex_language_abbrevs_prop 
            { \languagename } \l_tmpb_str {
              \msg_error:nnx{stex}{error/unknownlanguage}{\languagename}
            }
      } {
        \str_clear:N \l_tmpb_str
      }

      \stex_path_to_string:NN \l_tmpa_seq \l_tmpa_str

      \stex_debug:nn{modules}{Checking~\l_tmpa_str/#4.\l_tmpb_str.tex}
      \IfFileExists{ \l_tmpa_str/#4.\l_tmpb_str.tex }{
        \str_gset:Nx \g_@@_file_str { \l_tmpa_str/#4.\l_tmpb_str.tex }
      }{
        \stex_debug:nn{modules}{Checking~\l_tmpa_str/#4.tex}
        \IfFileExists{ \l_tmpa_str/#4.tex }{
          \str_gset:Nx \g_@@_file_str { \l_tmpa_str/#4.tex }
        }{
          % try english as default
          \stex_debug:nn{modules}{Checking~\l_tmpa_str/#4.en.tex}
          \IfFileExists{ \l_tmpa_str/#4.en.tex }{
            \str_gset:Nx \g_@@_file_str { \l_tmpa_str/#4.en.tex }
          }{
            \stex_debug:nn{modules}{Checking~\l_tmpa_str.\l_tmpb_str.tex}
            \IfFileExists{ \l_tmpa_str.\l_tmpb_str.tex }{
              \str_gset:Nx \g_@@_file_str { \l_tmpa_str.\l_tmpb_str.tex }
            }{
              \stex_debug:nn{modules}{Checking~\l_tmpa_str.tex}
              \IfFileExists{ \l_tmpa_str.tex }{
                \str_gset:Nx \g_@@_file_str { \l_tmpa_str.tex }
              }{
                % try english as default
                \stex_debug:nn{modules}{Checking~\l_tmpa_str.en.tex}
                \IfFileExists{ \l_tmpa_str.en.tex }{
                  \str_gset:Nx \g_@@_file_str { \l_tmpa_str.en.tex }
                }{
                  \msg_error:nnx{stex}{error/unknownmodule}{#1?#4}
                }
              }
            }
          }
        }
      }
    }

    \exp_args:No \stex_in_smsmode:nn { \g_@@_file_str } {
      \seq_clear:N \l_stex_all_modules_seq
      \str_clear:N \l_stex_current_module_str
      \str_set:Nx \l_tmpb_str { #2 }
      \str_if_empty:NF \l_tmpb_str {
        \stex_set_current_repository:n { #2 }
      }
      \stex_debug:nn{modules}{Loading~\g_@@_file_str}
      \input { \g_@@_file_str }
    }
    
    \stex_if_module_exists:nF { #1 ? #4 } {
      \msg_error:nnx{stex}{error/unknownmodule}{
        #1?#4~(in~file~\g_@@_file_str)
      }
    }
  }
  \stex_activate_module:n { #1 ? #4 }
}
%    \end{macrocode}
% \end{macro}
%
% \begin{macro}{\importmodule}
%    \begin{macrocode}
\NewDocumentCommand \importmodule { O{} m } {
  \stex_import_module_uri:nn { #1 } { #2 }
  \stex_debug:nn{modules}{Importing~module:~
    \l_stex_import_ns_str ? \l_stex_import_name_str
  }
  \stex_if_smsmode:F {
    \stex_import_require_module:nnnn 
    { \l_stex_import_ns_str } { \l_stex_import_archive_str } 
    { \l_stex_import_path_str } { \l_stex_import_name_str }
    \stex_annotate_invisible:nnn 
      {import} {\l_stex_import_ns_str ? \l_stex_import_name_str} {}
  }
  \exp_args:Nx \stex_add_to_current_module:n {
    \stex_import_require_module:nnnn 
    { \l_stex_import_ns_str } { \l_stex_import_archive_str } 
    { \l_stex_import_path_str } { \l_stex_import_name_str }
  }
  \exp_args:Nx \stex_add_import_to_current_module:n {
    \l_stex_import_ns_str ? \l_stex_import_name_str
  }
  \stex_smsmode_set_codes:
}
\stex_deactivate_macro:Nn \importmodule {module~environments}
%    \end{macrocode}
% \end{macro}
%
% \begin{macro}{\usemodule}
%    \begin{macrocode}
\NewDocumentCommand \usemodule { O{} m } {
  \stex_if_smsmode:F {
    \stex_import_module_uri:nn { #1 } { #2 }
    \stex_import_require_module:nnnn 
    { \l_stex_import_ns_str } { \l_stex_import_archive_str }
    { \l_stex_import_path_str } { \l_stex_import_name_str }
    \stex_annotate_invisible:nnn 
      {usemodule} {\l_stex_import_ns_str ? \l_stex_import_name_str} {}
  }
  \stex_smsmode_set_codes:
}
%    \end{macrocode}
% \end{macro}
%
%    \begin{macrocode}
%</package>
%    \end{macrocode}
%
% \end{implementation}
%
% \PrintIndex

% \fi
%
% \begin{documentation}\label{pkg:inheritance:doc}
%
% Code related to Module Inheritance, in particular \emph{sms mode}.
%
% \section{Macros and Environments}\label{pkg:inheritance:doc:macros}
%
%    \subsection{SMS Mode}
% ``SMS Mode'' is used when loading modules from external tex files.
% It deactivates any output and ignores all \TeX\ commands
% not explicitly allowed via the following lists:
%
% \begin{variable}{\g_stex_smsmode_allowedmacros_tl}
%   Macros that are executed as is; i.e. with the category code scheme
%   used in SMS mode.
% \end{variable}
%
% \begin{variable}{\g_stex_smsmode_allowedmacros_escape_tl}
%   Macros that are executed with the category codes restored.
%
%   Importantly, these macros need to call \cs{stex_smsmode_set_codes:}
%   after reading all arguments. Note, that 
%   \cs{stex_smsmode_set_codes:} takes
%   care of checking whether we are in SMS mode in the first place, 
%   so calling this function eagerly is unproblematic.
% \end{variable}
%
% \begin{variable}{\g_stex_smsmode_allowedenvs_seq}
%   The names of environments that should be allowed in SMS mode.
%   The corresponding \cs{begin}-statements are treated like
%   the macros in \cs{g_stex_smsmode_allowedmacros_escape_tl}, so
%   \cs{stex_smsmode_set_codes:} should be called at the end of the
%   \cs{begin}-code. Since \cs{end}-statements take no arguments anyway,
%   those are called with the SMS mode category code scheme active.
% \end{variable}
%
% \begin{function}[pTF]{\stex_if_smsmode:}
%   Tests whether SMS mode is currently active.
% \end{function}
%
% \begin{function}{\stex_smsmode_set_codes:}
%   Sets the current category code scheme to that of the SMS mode, if
%   SMS mode is currently active and if necessary.
%
%   This method should be called at the end of every macro or 
%   \cs{begin} environment code that are allowed in SMS mode.
%
% \end{function}
%
% \begin{function}{\stex_in_smsmode:nn}
%   \begin{syntax} \cs{stex_in_smsmode:nn} \Arg{name} \Arg{code} \end{syntax}
%   Executes \meta{code} in SMS mode. \meta{name} can be arbitrary,
%   but should be distinct, since it allows for nesting 
%   \cs{stex_in_smsmode:nn} without spuriously terminating SMS mode.
% \end{function}
%
%\stextest{
% \immediate\openout\testfile=./tests/sometest.tex
% \immediate\write\testfile{\detokenize{\this is \a test}^^J}
% \immediate\write\testfile{\detokenize{this \is a \test}}
% \immediate\closeout\testfile
% \ExplSyntaxOn
% \stex_in_smsmode:nn { foo } { 
%   \input{tests/sometest.tex} 
% }
% \ExplSyntaxOff
%}
%
%
%    \subsection{Imports and Inheritance}
%
% \begin{function}{\importmodule}
%   \begin{syntax} \cs{importmodule}|[|\meta{archive-ID}|]|\Arg{module-path} \end{syntax}
%   Imports a module by reading it from a file and ``activating'' it.
%   \sTeX determines the module and its containing file by passing its
%   arguments on to \cs{stex_import_module_path:nn}.
%   
% \end{function}
%
%\stextest{
%   \begin{module}{Foo}
%      \symdecl[name=foo, args=3]{bar}
%      \symdecl[args=bai]{foobar}
%      Meaning:~\present\bar\\
%   \end{module}
%      Meaning:~\present\bar\\
%   \begin{module}{Importtest}
%     \importmodule{Foo}
%      Meaning:~\present\bar\\
%   \end{module}
%   \begin{module}{Importtest2}
%     \importmodule{Importtest}
%      Meaning:~\present\bar\\
%   \end{module}
%}
%
% \begin{function}{\usemodule}
%   \begin{syntax} \cs{importmodule}|[|\meta{archive-ID}|]|\Arg{module-path} \end{syntax}
%   Like \cs{importmodule}, but does not export its contents;
%   i.e. including the current module will not activate the used module
%   
% \end{function}
%
%\stextest{
%   \begin{module}{UseTest1}
%      \symdecl{foo}
%   \end{module}
%   \begin{module}{UseTest2}
%     \usemodule{UseTest1}
%      \symdecl{bar}
%      Meaning:~\present\foo\\
%   \end{module}
%   \begin{module}{UseTest3}
%     \importmodule{UseTest2}
%     Meaning:~\present\foo\\
%     Meaning:~\present\bar\\
%
%     All modules: \ExplSyntaxOn
%       \seq_use:Nn \l_stex_all_modules_seq {,~} \\
%      All~symbols:~
%        \seq_use:Nn \l_stex_all_symbols_seq {,~}
%     \ExplSyntaxOff 
%   \end{module}
%}
%
%
%\stextest{
% Circular dependencies:
% \begin{module}{CircDep1}
%   \importmodule[Foo/Bar]{circular1?Circular1}
%   \importmodule[Bar/Foo]{circular2?Circular2}
%   \present\fooA\\
%   \present\fooB
% \end{module}
%}
%
% \begin{function}{\stex_import_module_uri:nn}
%   \begin{syntax} \cs{stex_import_module_uri:nn} \Arg{archive-ID} \Arg{module-path} \end{syntax}
%   Determines the URI of a module by splitting
%   \meta{module-path} into \meta{path}|?|\meta{name}. If \meta{module-path}
%   does \emph{not} contain a |?|-character, we consider it to be the \meta{name},
%   and \meta{path} to be empty.
%
%   If \meta{archive-ID} is empty, it is automatically set to the
%   ID of the current archive (if one exists).
%
%   \begin{enumerate}
%   \item If \meta{archive-ID} is empty:
%     \begin{enumerate}
%     \item If \meta{path} is empty, then
%         \meta{name} must have been declared earlier in the same file
%         and retrievable from \cs{g_stex_modules_in_file_seq}, or
%         a file with name \meta{name}|.|\meta{lang}|.tex| must exist
%         in the same folder, containing a module \meta{name}.
%
%         That module should have the same namespace as the current one.
%     \item If \meta{path} is not empty, it must point to the relative
%         path of the containing file as well as the namespace.
%     \end{enumerate}
%   \item Otherwise:
%      \begin{enumerate}
%         \item If \meta{path} is empty, then
%         \meta{name} must have been declared earlier in the same file
%         and retrievable from \cs{g_stex_modules_in_file_seq}, or
%         a file with name \meta{name}|.|\meta{lang}|.tex| must exist
%         in the top |source| folder of the archive, 
%         containing a module \meta{name}.
%
%         That module should lie directly in the namespace 
%         of the archive.
%     \item If \meta{path} is not empty, it must point to the
%         path of the containing file as well as the namespace,
%         relative to the namespace of the archive.
%
%         If a module by that namespace exists, it is returned.
%         Otherwise, we call \cs{stex_require_module:nn} 
%         on the |source| directory of the archive to find the
%         file.
%       \end{enumerate}
%   \end{enumerate}
% \end{function}
%
% \begin{function}{\stex_import_require_module:nnnn}
%    \begin{syntax} \Arg{ns} \Arg{archive-ID} \Arg{path} \Arg{name} \end{syntax}
%     Checks whether a module with URI \meta{ns}|?|\meta{name} already
%     exists. If not, it looks for a plausible file that declares
%     a module with that URI.
%
%     Finally, activates that module by executing its |content|-field.
% \end{function}
%
%
% \end{documentation}
%
% \begin{implementation}\label{pkg:inheritance:impl}
%
% \section{\sTeX-Module Inheritance Implementation}
%
%    \begin{macrocode}
%<*package>

%%%%%%%%%%%%%   inheritance.dtx   %%%%%%%%%%%%%

%    \end{macrocode}
%
% \subsection{SMS Mode}
%    \begin{macrocode}
%<@@=stex_smsmode>
%    \end{macrocode}
%
% \begin{variable}{
%   \g_stex_smsmode_allowedmacros_tl,
%   \g_stex_smsmode_allowedmacros_escape_tl,
%   \g_stex_smsmode_allowedenvs_seq
% }
%    \begin{macrocode}
\tl_new:N \g_stex_smsmode_allowedmacros_tl
\tl_new:N \g_stex_smsmode_allowedmacros_escape_tl
\seq_new:N \g_stex_smsmode_allowedenvs_seq

\tl_set:Nn \g_stex_smsmode_allowedmacros_tl {
  \makeatletter
  \makeatother
  \ExplSyntaxOn
  \ExplSyntaxOff
  \rustexBREAK
}

\tl_set:Nn \g_stex_smsmode_allowedmacros_escape_tl {
  \symdef
  \importmodule
  \notation
  \symdecl
  \STEXexport
  \inlineass
  \inlinedef
  \inlineex
}

\exp_args:NNx \seq_set_from_clist:Nn \g_stex_smsmode_allowedenvs_seq {
  \tl_to_str:n {
    module,
    @module,
    sdefinition,
    sexample,
    sassertion,
    sparagraph
  }
}
%    \end{macrocode}
% \end{variable}
%
% \begin{macro}[pTF]{\stex_if_smsmode:}
%    \begin{macrocode}
\bool_new:N \g_@@_bool
\bool_set_false:N \g_@@_bool
\prg_new_conditional:Nnn \stex_if_smsmode: { p, T, F, TF } {
  \bool_if:NTF \g_@@_bool \prg_return_true: \prg_return_false:
}
%    \end{macrocode}
% \end{macro}
%
%
% \begin{macro}[pTF]{\_@@_if_catcodes:}
% Checks whether the SMS mode category code scheme is active.
%    \begin{macrocode}
\bool_new:N \g_@@_catcode_bool
\bool_set_false:N \g_@@_catcode_bool
\prg_new_conditional:Nnn \_@@_if_catcodes: { p, T, F, TF } {
  \bool_if:NTF \g_@@_catcode_bool 
    \prg_return_true: \prg_return_false:
}
%    \end{macrocode}
% \end{macro}
%
% \begin{macro}{\stex_smsmode_set_codes:}
%    \begin{macrocode}
\cs_new_protected:Nn \stex_smsmode_set_codes: {
  \stex_if_smsmode:T {
    \_@@_if_catcodes:F {
      \bool_gset_true:N \g_@@_catcode_bool
      \exp_after:wN \char_gset_active_eq:NN 
        \c_backslash_str \_@@_cs:
      \tex_global:D \char_set_catcode_active:N \\
      \tex_global:D \char_set_catcode_other:N $
      \tex_global:D \char_set_catcode_other:N ^
      \tex_global:D \char_set_catcode_other:N _
      \tex_global:D \char_set_catcode_other:N &
      \tex_global:D \char_set_catcode_other:N ##
    }
  }
} \iffalse $ \fi % to make syntax highlighting work again
%    \end{macrocode}
% \end{macro}
%
% \begin{macro}{\_@@_unset_codes:}
%   Sets category code scheme back from the one used in SMS mode.
%    \begin{macrocode}
\cs_new_protected:Nn \_@@_unset_codes: {
  \_@@_if_catcodes:T {
    \bool_gset_false:N \g_@@_catcode_bool
    \exp_after:wN \tex_global:D \exp_after:wN 
      \char_set_catcode_escape:N \c_backslash_str
    \tex_global:D \char_set_catcode_math_toggle:N $
    \tex_global:D \char_set_catcode_math_superscript:N ^
    \tex_global:D \char_set_catcode_math_subscript:N _
    \tex_global:D \char_set_catcode_alignment:N &
    \tex_global:D \char_set_catcode_parameter:N ##
  }
} \iffalse $ \fi % to make syntax highlighting work again
%    \end{macrocode}
% \end{macro}
%
% \begin{macro}{\stex_in_smsmode:nn}
%    \begin{macrocode}
\cs_new_protected:Nn \stex_in_smsmode:nn {
  \vbox_set:Nn \l_tmpa_box {
    \bool_set_eq:cN { l_@@_#1_bool } \g_@@_bool
    \bool_gset_true:N \g_@@_bool
    \stex_smsmode_set_codes:
    #2
    \bool_gset_eq:Nc \g_@@_bool { l_@@_#1_bool }
    \stex_if_smsmode:F {
      \_@@_unset_codes:
    }
  }
  \box_clear:N \l_tmpa_box
}
%    \end{macrocode}
% \end{macro}
%
% \begin{macro}{\_@@_cs:}
% is executed on encountering |\| in smsmode.
% It checks whether the corresponding command is allowed and executes
% or ignores it accordingly:
%    \begin{macrocode}
\cs_new_protected:Nn \_@@_cs: {
  \str_clear:N \l_tmpa_str
  \peek_analysis_map_inline:n {
    % #1: token (one expansion)
    % #2: charcode
    % #3 catcode
    \token_if_eq_charcode:NNTF ##3 B {
      % token is a letter
      \exp_args:NNo \str_put_right:Nn \l_tmpa_str { ##1 }
    } {
      \str_if_empty:NTF \l_tmpa_str {
        % we don't allow (or need) single non-letter CSs
        % for now
        \peek_analysis_map_break: 
      }{
        \str_if_eq:onTF \l_tmpa_str { begin } {
          \peek_analysis_map_break:n { 
            \exp_after:wN \_@@_checkbegin:n ##1
          }
        } {
          \str_if_eq:onTF \l_tmpa_str { end } {
            \peek_analysis_map_break:n { 
              \exp_after:wN \_@@_checkend:n ##1
            }
          } {
          \tl_set:Nn \l_tmpa_tl { \use:c{\l_tmpa_str} }
          \exp_args:NNNo \exp_args:NNo \tl_if_in:NnTF 
            \g_stex_smsmode_allowedmacros_tl 
              { \use:c{\l_tmpa_str} } {
              \stex_debug:nn{modules}{Executing~1:~\l_tmpa_str}
              \peek_analysis_map_break:n { 
                \exp_after:wN \l_tmpa_tl ##1
              }
            } {
              \exp_args:NNNo \exp_args:NNo \tl_if_in:NnTF 
              \g_stex_smsmode_allowedmacros_escape_tl 
                { \use:c{\l_tmpa_str} } {
                \_@@_unset_codes:
                \stex_debug:nn{modules}{Executing~2:~\l_tmpa_str}
                % TODO \_@@_rescan_cs:
%                \int_compare:nNnTF {##2} = {92} {
%                  \peek_analysis_map_break:n {
%                    \_@@_unset_codes:
%                    \_@@_rescan_cs:
%                  }
%                } {
                  \peek_analysis_map_break:n {
                    \exp_after:wN \l_tmpa_tl ##1
                  }
%                }
              } {
                  \int_compare:nNnTF {##2} = {92} {
                    \peek_analysis_map_break:n { \_@@_cs: }
                  }{
                    \peek_analysis_map_break:n { \exp_after:wN\relax ##1 }
                  }
              }
            }
          }
        }
      }
    }
  }
}
%    \end{macrocode}
% \end{macro}
%
%
% \begin{macro}{\_@@_rescan_cs:}
% If the last token gobbled by |\stex_smsmode_cs:| happened to be
% a |\|, we need to rescan the cs name and reinsert it into the input
% stream:
%    \begin{macrocode}
\cs_new_protected:Nn \_@@_rescan_cs: {
  \str_clear:N \l_tmpb_str
  \peek_analysis_map_inline:n {
    \token_if_eq_charcode:NNTF ##3 B {
      % token is a letter
      \exp_args:NNo \str_put_right:Nn \l_tmpb_str { ##1 }
    } {
      \peek_analysis_map_break:n {
        \exp_after:wN \use:c \exp_after:wN { 
          \exp_after:wN \l_tmpa_str\exp_after:wN 
        } \use:c { \l_tmpb_str \exp_after:wN } ##1
      }
    }
  }
}
%    \end{macrocode}
% \end{macro}
%
% \begin{macro}{\_@@_checkbegin:n}
% called on |\begin|; checks whether the environment being opened
% is allowed in SMS mode. 
%    \begin{macrocode}
\cs_new_protected:Nn \_@@_checkbegin:n {
  \str_set:Nn \l_tmpa_str { #1 }
  \seq_if_in:NoT \g_stex_smsmode_allowedenvs_seq \l_tmpa_str {
    \_@@_unset_codes:
    \begin{#1}
  }
}
%    \end{macrocode}
% \end{macro}
%
% \begin{macro}{\_@@_checkend:n}
% called on |\end|; checks whether the environment being opened
% is allowed in SMS mode. 
%    \begin{macrocode}
\cs_new_protected:Nn \_@@_checkend:n {
  \str_set:Nn \l_tmpa_str { #1 }
  \seq_if_in:NoT \g_stex_smsmode_allowedenvs_seq \l_tmpa_str {
    \end{#1}
  }
}
%    \end{macrocode}
% \end{macro}
%
% \subsection{Inheritance}
%    \begin{macrocode}
%<@@=stex_importmodule>
%    \end{macrocode}
%
% \begin{macro}{\stex_import_module_uri:nn}
%    \begin{macrocode}
\cs_new_protected:Nn \stex_import_module_uri:nn {
  \str_set:Nx \l_stex_import_archive_str { #1 }
  \str_set:Nn \l_stex_import_path_str { #2 }

  \exp_args:NNNo \seq_set_split:Nnn \l_tmpb_seq ? { \l_stex_import_path_str }
  \seq_pop_right:NN \l_tmpb_seq \l_stex_import_name_str
  \str_set:Nx \l_stex_import_path_str { \seq_use:Nn \l_tmpb_seq ? }

  \stex_modules_current_namespace:
  \bool_lazy_all:nTF {
    {\str_if_empty_p:N \l_stex_import_archive_str}
    {\str_if_empty_p:N \l_stex_import_path_str}
    {\stex_if_module_exists_p:n { \l_stex_module_ns_str ? \l_stex_import_name_str } }
  }{
    \str_set_eq:NN \l_stex_import_path_str \l_stex_modules_subpath_str
    \str_set_eq:NN \l_stex_import_ns_str \l_stex_module_ns_str
  }{
    \str_if_empty:NT \l_stex_import_archive_str {
      \prop_if_exist:NT \l_stex_current_repository_prop {
        \prop_get:NnN \l_stex_current_repository_prop { id } \l_stex_import_archive_str
      }
    }  
    \str_if_empty:NTF \l_stex_import_archive_str {
      \str_if_empty:NF \l_stex_import_path_str {
        \str_set:Nx \l_stex_import_ns_str {
          \l_stex_module_ns_str / \l_stex_import_path_str
        }
      }
    }{
      \stex_require_repository:n \l_stex_import_archive_str
      \prop_get:cnN { c_stex_mathhub_\l_stex_import_archive_str _manifest_prop } { ns }
        \l_stex_import_ns_str
      \str_if_empty:NF \l_stex_import_path_str {
        \str_set:Nx \l_stex_import_ns_str {
          \l_stex_import_ns_str / \l_stex_import_path_str
        }
      }
    }
  }
}
%    \end{macrocode}
% \end{macro}
%
% \begin{variable}{
%   \l_stex_import_name_str,\l_stex_import_archive_str,\l_stex_import_path_str,\l_stex_import_ns_str
% }
%   Store the return values of \cs{stex_import_module_uri:nn}.
%    \begin{macrocode}
\str_new:N \l_stex_import_name_str
\str_new:N \l_stex_import_archive_str
\str_new:N \l_stex_import_path_str
\str_new:N \l_stex_import_ns_str
%    \end{macrocode}
% \end{variable}
%
% \begin{macro}{\stex_import_require_module:nnnn}
%    \begin{syntax} \Arg{ns} \Arg{archive-ID} \Arg{path} \Arg{name} \end{syntax}
%    \begin{macrocode}
\cs_new_protected:Nn \stex_import_require_module:nnnn {
  \exp_args:Nx \stex_if_module_exists:nF { #1 ? #4 } {

    % archive
    \str_set:Nx \l_tmpa_str { #2 }
    \str_if_empty:NTF \l_tmpa_str {
      \seq_set_eq:NN \l_tmpa_seq \g_stex_currentfile_seq
    } {
      \stex_path_from_string:Nn \l_tmpb_seq { \l_tmpa_str }
      \seq_concat:NNN \l_tmpa_seq \c_stex_mathhub_seq \l_tmpb_seq
      \seq_put_right:Nn \l_tmpa_seq { source }
    }

    % path
    \str_set:Nx \l_tmpb_str { #3 }
    \str_if_empty:NTF \l_tmpb_str {
      \str_set:Nx \l_tmpa_str { \stex_path_to_string:N \l_tmpa_seq / #4 }
      
      \ltx@ifpackageloaded{babel} { 
        \exp_args:NNx \prop_get:NnNF \c_stex_language_abbrevs_prop 
            { \languagename } \l_tmpb_str {
              \msg_error:nnx{stex}{error/unknownlanguage}{\languagename}
            }
      } {
        \str_clear:N \l_tmpb_str
      }

      \stex_debug:nn{modules}{Checking~\l_tmpa_str.\l_tmpb_str.tex}
      \IfFileExists{ \l_tmpa_str.\l_tmpb_str.tex }{
        \str_gset:Nx \g_@@_file_str { \l_tmpa_str.\l_tmpb_str.tex }
      }{
        \stex_debug:nn{modules}{Checking~\l_tmpa_str.tex}
        \IfFileExists{ \l_tmpa_str.tex }{
          \str_gset:Nx \g_@@_file_str { \l_tmpa_str.tex }
        }{
          % try english as default
          \stex_debug:nn{modules}{Checking~\l_tmpa_str.en.tex}
          \IfFileExists{ \l_tmpa_str.en.tex }{
            \str_gset:Nx \g_@@_file_str { \l_tmpa_str.en.tex }
          }{
            \msg_error:nnx{stex}{error/unknownmodule}{#1?#4}
          }
        }
      }

    } {
      \seq_set_split:NnV \l_tmpb_seq / \l_tmpb_str
      \seq_concat:NNN \l_tmpa_seq \l_tmpa_seq \l_tmpb_seq
      
      \ltx@ifpackageloaded{babel} { 
        \exp_args:NNx \prop_get:NnNF \c_stex_language_abbrevs_prop 
            { \languagename } \l_tmpb_str {
              \msg_error:nnx{stex}{error/unknownlanguage}{\languagename}
            }
      } {
        \str_clear:N \l_tmpb_str
      }

      \stex_path_to_string:NN \l_tmpa_seq \l_tmpa_str

      \stex_debug:nn{modules}{Checking~\l_tmpa_str/#4.\l_tmpb_str.tex}
      \IfFileExists{ \l_tmpa_str/#4.\l_tmpb_str.tex }{
        \str_gset:Nx \g_@@_file_str { \l_tmpa_str/#4.\l_tmpb_str.tex }
      }{
        \stex_debug:nn{modules}{Checking~\l_tmpa_str/#4.tex}
        \IfFileExists{ \l_tmpa_str/#4.tex }{
          \str_gset:Nx \g_@@_file_str { \l_tmpa_str/#4.tex }
        }{
          % try english as default
          \stex_debug:nn{modules}{Checking~\l_tmpa_str/#4.en.tex}
          \IfFileExists{ \l_tmpa_str/#4.en.tex }{
            \str_gset:Nx \g_@@_file_str { \l_tmpa_str/#4.en.tex }
          }{
            \stex_debug:nn{modules}{Checking~\l_tmpa_str.\l_tmpb_str.tex}
            \IfFileExists{ \l_tmpa_str.\l_tmpb_str.tex }{
              \str_gset:Nx \g_@@_file_str { \l_tmpa_str.\l_tmpb_str.tex }
            }{
              \stex_debug:nn{modules}{Checking~\l_tmpa_str.tex}
              \IfFileExists{ \l_tmpa_str.tex }{
                \str_gset:Nx \g_@@_file_str { \l_tmpa_str.tex }
              }{
                % try english as default
                \stex_debug:nn{modules}{Checking~\l_tmpa_str.en.tex}
                \IfFileExists{ \l_tmpa_str.en.tex }{
                  \str_gset:Nx \g_@@_file_str { \l_tmpa_str.en.tex }
                }{
                  \msg_error:nnx{stex}{error/unknownmodule}{#1?#4}
                }
              }
            }
          }
        }
      }
    }

    \exp_args:No \stex_in_smsmode:nn { \g_@@_file_str } {
      \seq_clear:N \l_stex_all_modules_seq
      \str_clear:N \l_stex_current_module_str
      \str_set:Nx \l_tmpb_str { #2 }
      \str_if_empty:NF \l_tmpb_str {
        \stex_set_current_repository:n { #2 }
      }
      \stex_debug:nn{modules}{Loading~\g_@@_file_str}
      \input { \g_@@_file_str }
    }
    
    \stex_if_module_exists:nF { #1 ? #4 } {
      \msg_error:nnx{stex}{error/unknownmodule}{
        #1?#4~(in~file~\g_@@_file_str)
      }
    }
  }
  \stex_activate_module:n { #1 ? #4 }
}
%    \end{macrocode}
% \end{macro}
%
% \begin{macro}{\importmodule}
%    \begin{macrocode}
\NewDocumentCommand \importmodule { O{} m } {
  \stex_import_module_uri:nn { #1 } { #2 }
  \stex_debug:nn{modules}{Importing~module:~
    \l_stex_import_ns_str ? \l_stex_import_name_str
  }
  \stex_if_smsmode:F {
    \stex_import_require_module:nnnn 
    { \l_stex_import_ns_str } { \l_stex_import_archive_str } 
    { \l_stex_import_path_str } { \l_stex_import_name_str }
    \stex_annotate_invisible:nnn 
      {import} {\l_stex_import_ns_str ? \l_stex_import_name_str} {}
  }
  \exp_args:Nx \stex_add_to_current_module:n {
    \stex_import_require_module:nnnn 
    { \l_stex_import_ns_str } { \l_stex_import_archive_str } 
    { \l_stex_import_path_str } { \l_stex_import_name_str }
  }
  \exp_args:Nx \stex_add_import_to_current_module:n {
    \l_stex_import_ns_str ? \l_stex_import_name_str
  }
  \stex_smsmode_set_codes:
}
\stex_deactivate_macro:Nn \importmodule {module~environments}
%    \end{macrocode}
% \end{macro}
%
% \begin{macro}{\usemodule}
%    \begin{macrocode}
\NewDocumentCommand \usemodule { O{} m } {
  \stex_if_smsmode:F {
    \stex_import_module_uri:nn { #1 } { #2 }
    \stex_import_require_module:nnnn 
    { \l_stex_import_ns_str } { \l_stex_import_archive_str }
    { \l_stex_import_path_str } { \l_stex_import_name_str }
    \stex_annotate_invisible:nnn 
      {usemodule} {\l_stex_import_ns_str ? \l_stex_import_name_str} {}
  }
  \stex_smsmode_set_codes:
}
%    \end{macrocode}
% \end{macro}
%
%    \begin{macrocode}
%</package>
%    \end{macrocode}
%
% \end{implementation}
%
% \PrintIndex

% \fi
%
% \begin{documentation}\label{pkg:inheritance:doc}
%
% Code related to Module Inheritance, in particular \emph{sms mode}.
%
% \section{Macros and Environments}\label{pkg:inheritance:doc:macros}
%
%    \subsection{SMS Mode}
% ``SMS Mode'' is used when loading modules from external tex files.
% It deactivates any output and ignores all \TeX\ commands
% not explicitly allowed via the following lists:
%
% \begin{variable}{\g_stex_smsmode_allowedmacros_tl}
%   Macros that are executed as is; i.e. with the category code scheme
%   used in SMS mode.
% \end{variable}
%
% \begin{variable}{\g_stex_smsmode_allowedmacros_escape_tl}
%   Macros that are executed with the category codes restored.
%
%   Importantly, these macros need to call \cs{stex_smsmode_set_codes:}
%   after reading all arguments. Note, that 
%   \cs{stex_smsmode_set_codes:} takes
%   care of checking whether we are in SMS mode in the first place, 
%   so calling this function eagerly is unproblematic.
% \end{variable}
%
% \begin{variable}{\g_stex_smsmode_allowedenvs_seq}
%   The names of environments that should be allowed in SMS mode.
%   The corresponding \cs{begin}-statements are treated like
%   the macros in \cs{g_stex_smsmode_allowedmacros_escape_tl}, so
%   \cs{stex_smsmode_set_codes:} should be called at the end of the
%   \cs{begin}-code. Since \cs{end}-statements take no arguments anyway,
%   those are called with the SMS mode category code scheme active.
% \end{variable}
%
% \begin{function}[pTF]{\stex_if_smsmode:}
%   Tests whether SMS mode is currently active.
% \end{function}
%
% \begin{function}{\stex_smsmode_set_codes:}
%   Sets the current category code scheme to that of the SMS mode, if
%   SMS mode is currently active and if necessary.
%
%   This method should be called at the end of every macro or 
%   \cs{begin} environment code that are allowed in SMS mode.
%
% \end{function}
%
% \begin{function}{\stex_in_smsmode:nn}
%   \begin{syntax} \cs{stex_in_smsmode:nn} \Arg{name} \Arg{code} \end{syntax}
%   Executes \meta{code} in SMS mode. \meta{name} can be arbitrary,
%   but should be distinct, since it allows for nesting 
%   \cs{stex_in_smsmode:nn} without spuriously terminating SMS mode.
% \end{function}
%
%\stextest{
% \immediate\openout\testfile=./tests/sometest.tex
% \immediate\write\testfile{\detokenize{\this is \a test}^^J}
% \immediate\write\testfile{\detokenize{this \is a \test}}
% \immediate\closeout\testfile
% \ExplSyntaxOn
% \stex_in_smsmode:nn { foo } { 
%   \input{tests/sometest.tex} 
% }
% \ExplSyntaxOff
%}
%
%
%    \subsection{Imports and Inheritance}
%
% \begin{function}{\importmodule}
%   \begin{syntax} \cs{importmodule}|[|\meta{archive-ID}|]|\Arg{module-path} \end{syntax}
%   Imports a module by reading it from a file and ``activating'' it.
%   \sTeX determines the module and its containing file by passing its
%   arguments on to \cs{stex_import_module_path:nn}.
%   
% \end{function}
%
%\stextest{
%   \begin{module}{Foo}
%      \symdecl[name=foo, args=3]{bar}
%      \symdecl[args=bai]{foobar}
%      Meaning:~\present\bar\\
%   \end{module}
%      Meaning:~\present\bar\\
%   \begin{module}{Importtest}
%     \importmodule{Foo}
%      Meaning:~\present\bar\\
%   \end{module}
%   \begin{module}{Importtest2}
%     \importmodule{Importtest}
%      Meaning:~\present\bar\\
%   \end{module}
%}
%
% \begin{function}{\usemodule}
%   \begin{syntax} \cs{importmodule}|[|\meta{archive-ID}|]|\Arg{module-path} \end{syntax}
%   Like \cs{importmodule}, but does not export its contents;
%   i.e. including the current module will not activate the used module
%   
% \end{function}
%
%\stextest{
%   \begin{module}{UseTest1}
%      \symdecl{foo}
%   \end{module}
%   \begin{module}{UseTest2}
%     \usemodule{UseTest1}
%      \symdecl{bar}
%      Meaning:~\present\foo\\
%   \end{module}
%   \begin{module}{UseTest3}
%     \importmodule{UseTest2}
%     Meaning:~\present\foo\\
%     Meaning:~\present\bar\\
%
%     All modules: \ExplSyntaxOn
%       \seq_use:Nn \l_stex_all_modules_seq {,~} \\
%      All~symbols:~
%        \seq_use:Nn \l_stex_all_symbols_seq {,~}
%     \ExplSyntaxOff 
%   \end{module}
%}
%
%
%\stextest{
% Circular dependencies:
% \begin{module}{CircDep1}
%   \importmodule[Foo/Bar]{circular1?Circular1}
%   \importmodule[Bar/Foo]{circular2?Circular2}
%   \present\fooA\\
%   \present\fooB
% \end{module}
%}
%
% \begin{function}{\stex_import_module_uri:nn}
%   \begin{syntax} \cs{stex_import_module_uri:nn} \Arg{archive-ID} \Arg{module-path} \end{syntax}
%   Determines the URI of a module by splitting
%   \meta{module-path} into \meta{path}|?|\meta{name}. If \meta{module-path}
%   does \emph{not} contain a |?|-character, we consider it to be the \meta{name},
%   and \meta{path} to be empty.
%
%   If \meta{archive-ID} is empty, it is automatically set to the
%   ID of the current archive (if one exists).
%
%   \begin{enumerate}
%   \item If \meta{archive-ID} is empty:
%     \begin{enumerate}
%     \item If \meta{path} is empty, then
%         \meta{name} must have been declared earlier in the same file
%         and retrievable from \cs{g_stex_modules_in_file_seq}, or
%         a file with name \meta{name}|.|\meta{lang}|.tex| must exist
%         in the same folder, containing a module \meta{name}.
%
%         That module should have the same namespace as the current one.
%     \item If \meta{path} is not empty, it must point to the relative
%         path of the containing file as well as the namespace.
%     \end{enumerate}
%   \item Otherwise:
%      \begin{enumerate}
%         \item If \meta{path} is empty, then
%         \meta{name} must have been declared earlier in the same file
%         and retrievable from \cs{g_stex_modules_in_file_seq}, or
%         a file with name \meta{name}|.|\meta{lang}|.tex| must exist
%         in the top |source| folder of the archive, 
%         containing a module \meta{name}.
%
%         That module should lie directly in the namespace 
%         of the archive.
%     \item If \meta{path} is not empty, it must point to the
%         path of the containing file as well as the namespace,
%         relative to the namespace of the archive.
%
%         If a module by that namespace exists, it is returned.
%         Otherwise, we call \cs{stex_require_module:nn} 
%         on the |source| directory of the archive to find the
%         file.
%       \end{enumerate}
%   \end{enumerate}
% \end{function}
%
% \begin{function}{\stex_import_require_module:nnnn}
%    \begin{syntax} \Arg{ns} \Arg{archive-ID} \Arg{path} \Arg{name} \end{syntax}
%     Checks whether a module with URI \meta{ns}|?|\meta{name} already
%     exists. If not, it looks for a plausible file that declares
%     a module with that URI.
%
%     Finally, activates that module by executing its |content|-field.
% \end{function}
%
%
% \end{documentation}
%
% \begin{implementation}\label{pkg:inheritance:impl}
%
% \section{\sTeX-Module Inheritance Implementation}
%
%    \begin{macrocode}
%<*package>

%%%%%%%%%%%%%   inheritance.dtx   %%%%%%%%%%%%%

%    \end{macrocode}
%
% \subsection{SMS Mode}
%    \begin{macrocode}
%<@@=stex_smsmode>
%    \end{macrocode}
%
% \begin{variable}{
%   \g_stex_smsmode_allowedmacros_tl,
%   \g_stex_smsmode_allowedmacros_escape_tl,
%   \g_stex_smsmode_allowedenvs_seq
% }
%    \begin{macrocode}
\tl_new:N \g_stex_smsmode_allowedmacros_tl
\tl_new:N \g_stex_smsmode_allowedmacros_escape_tl
\seq_new:N \g_stex_smsmode_allowedenvs_seq

\tl_set:Nn \g_stex_smsmode_allowedmacros_tl {
  \makeatletter
  \makeatother
  \ExplSyntaxOn
  \ExplSyntaxOff
  \rustexBREAK
}

\tl_set:Nn \g_stex_smsmode_allowedmacros_escape_tl {
  \symdef
  \importmodule
  \notation
  \symdecl
  \STEXexport
  \inlineass
  \inlinedef
  \inlineex
}

\exp_args:NNx \seq_set_from_clist:Nn \g_stex_smsmode_allowedenvs_seq {
  \tl_to_str:n {
    module,
    @module,
    sdefinition,
    sexample,
    sassertion,
    sparagraph
  }
}
%    \end{macrocode}
% \end{variable}
%
% \begin{macro}[pTF]{\stex_if_smsmode:}
%    \begin{macrocode}
\bool_new:N \g_@@_bool
\bool_set_false:N \g_@@_bool
\prg_new_conditional:Nnn \stex_if_smsmode: { p, T, F, TF } {
  \bool_if:NTF \g_@@_bool \prg_return_true: \prg_return_false:
}
%    \end{macrocode}
% \end{macro}
%
%
% \begin{macro}[pTF]{\_@@_if_catcodes:}
% Checks whether the SMS mode category code scheme is active.
%    \begin{macrocode}
\bool_new:N \g_@@_catcode_bool
\bool_set_false:N \g_@@_catcode_bool
\prg_new_conditional:Nnn \_@@_if_catcodes: { p, T, F, TF } {
  \bool_if:NTF \g_@@_catcode_bool 
    \prg_return_true: \prg_return_false:
}
%    \end{macrocode}
% \end{macro}
%
% \begin{macro}{\stex_smsmode_set_codes:}
%    \begin{macrocode}
\cs_new_protected:Nn \stex_smsmode_set_codes: {
  \stex_if_smsmode:T {
    \_@@_if_catcodes:F {
      \bool_gset_true:N \g_@@_catcode_bool
      \exp_after:wN \char_gset_active_eq:NN 
        \c_backslash_str \_@@_cs:
      \tex_global:D \char_set_catcode_active:N \\
      \tex_global:D \char_set_catcode_other:N $
      \tex_global:D \char_set_catcode_other:N ^
      \tex_global:D \char_set_catcode_other:N _
      \tex_global:D \char_set_catcode_other:N &
      \tex_global:D \char_set_catcode_other:N ##
    }
  }
} \iffalse $ \fi % to make syntax highlighting work again
%    \end{macrocode}
% \end{macro}
%
% \begin{macro}{\_@@_unset_codes:}
%   Sets category code scheme back from the one used in SMS mode.
%    \begin{macrocode}
\cs_new_protected:Nn \_@@_unset_codes: {
  \_@@_if_catcodes:T {
    \bool_gset_false:N \g_@@_catcode_bool
    \exp_after:wN \tex_global:D \exp_after:wN 
      \char_set_catcode_escape:N \c_backslash_str
    \tex_global:D \char_set_catcode_math_toggle:N $
    \tex_global:D \char_set_catcode_math_superscript:N ^
    \tex_global:D \char_set_catcode_math_subscript:N _
    \tex_global:D \char_set_catcode_alignment:N &
    \tex_global:D \char_set_catcode_parameter:N ##
  }
} \iffalse $ \fi % to make syntax highlighting work again
%    \end{macrocode}
% \end{macro}
%
% \begin{macro}{\stex_in_smsmode:nn}
%    \begin{macrocode}
\cs_new_protected:Nn \stex_in_smsmode:nn {
  \vbox_set:Nn \l_tmpa_box {
    \bool_set_eq:cN { l_@@_#1_bool } \g_@@_bool
    \bool_gset_true:N \g_@@_bool
    \stex_smsmode_set_codes:
    #2
    \bool_gset_eq:Nc \g_@@_bool { l_@@_#1_bool }
    \stex_if_smsmode:F {
      \_@@_unset_codes:
    }
  }
  \box_clear:N \l_tmpa_box
}
%    \end{macrocode}
% \end{macro}
%
% \begin{macro}{\_@@_cs:}
% is executed on encountering |\| in smsmode.
% It checks whether the corresponding command is allowed and executes
% or ignores it accordingly:
%    \begin{macrocode}
\cs_new_protected:Nn \_@@_cs: {
  \str_clear:N \l_tmpa_str
  \peek_analysis_map_inline:n {
    % #1: token (one expansion)
    % #2: charcode
    % #3 catcode
    \token_if_eq_charcode:NNTF ##3 B {
      % token is a letter
      \exp_args:NNo \str_put_right:Nn \l_tmpa_str { ##1 }
    } {
      \str_if_empty:NTF \l_tmpa_str {
        % we don't allow (or need) single non-letter CSs
        % for now
        \peek_analysis_map_break: 
      }{
        \str_if_eq:onTF \l_tmpa_str { begin } {
          \peek_analysis_map_break:n { 
            \exp_after:wN \_@@_checkbegin:n ##1
          }
        } {
          \str_if_eq:onTF \l_tmpa_str { end } {
            \peek_analysis_map_break:n { 
              \exp_after:wN \_@@_checkend:n ##1
            }
          } {
          \tl_set:Nn \l_tmpa_tl { \use:c{\l_tmpa_str} }
          \exp_args:NNNo \exp_args:NNo \tl_if_in:NnTF 
            \g_stex_smsmode_allowedmacros_tl 
              { \use:c{\l_tmpa_str} } {
              \stex_debug:nn{modules}{Executing~1:~\l_tmpa_str}
              \peek_analysis_map_break:n { 
                \exp_after:wN \l_tmpa_tl ##1
              }
            } {
              \exp_args:NNNo \exp_args:NNo \tl_if_in:NnTF 
              \g_stex_smsmode_allowedmacros_escape_tl 
                { \use:c{\l_tmpa_str} } {
                \_@@_unset_codes:
                \stex_debug:nn{modules}{Executing~2:~\l_tmpa_str}
                % TODO \_@@_rescan_cs:
%                \int_compare:nNnTF {##2} = {92} {
%                  \peek_analysis_map_break:n {
%                    \_@@_unset_codes:
%                    \_@@_rescan_cs:
%                  }
%                } {
                  \peek_analysis_map_break:n {
                    \exp_after:wN \l_tmpa_tl ##1
                  }
%                }
              } {
                  \int_compare:nNnTF {##2} = {92} {
                    \peek_analysis_map_break:n { \_@@_cs: }
                  }{
                    \peek_analysis_map_break:n { \exp_after:wN\relax ##1 }
                  }
              }
            }
          }
        }
      }
    }
  }
}
%    \end{macrocode}
% \end{macro}
%
%
% \begin{macro}{\_@@_rescan_cs:}
% If the last token gobbled by |\stex_smsmode_cs:| happened to be
% a |\|, we need to rescan the cs name and reinsert it into the input
% stream:
%    \begin{macrocode}
\cs_new_protected:Nn \_@@_rescan_cs: {
  \str_clear:N \l_tmpb_str
  \peek_analysis_map_inline:n {
    \token_if_eq_charcode:NNTF ##3 B {
      % token is a letter
      \exp_args:NNo \str_put_right:Nn \l_tmpb_str { ##1 }
    } {
      \peek_analysis_map_break:n {
        \exp_after:wN \use:c \exp_after:wN { 
          \exp_after:wN \l_tmpa_str\exp_after:wN 
        } \use:c { \l_tmpb_str \exp_after:wN } ##1
      }
    }
  }
}
%    \end{macrocode}
% \end{macro}
%
% \begin{macro}{\_@@_checkbegin:n}
% called on |\begin|; checks whether the environment being opened
% is allowed in SMS mode. 
%    \begin{macrocode}
\cs_new_protected:Nn \_@@_checkbegin:n {
  \str_set:Nn \l_tmpa_str { #1 }
  \seq_if_in:NoT \g_stex_smsmode_allowedenvs_seq \l_tmpa_str {
    \_@@_unset_codes:
    \begin{#1}
  }
}
%    \end{macrocode}
% \end{macro}
%
% \begin{macro}{\_@@_checkend:n}
% called on |\end|; checks whether the environment being opened
% is allowed in SMS mode. 
%    \begin{macrocode}
\cs_new_protected:Nn \_@@_checkend:n {
  \str_set:Nn \l_tmpa_str { #1 }
  \seq_if_in:NoT \g_stex_smsmode_allowedenvs_seq \l_tmpa_str {
    \end{#1}
  }
}
%    \end{macrocode}
% \end{macro}
%
% \subsection{Inheritance}
%    \begin{macrocode}
%<@@=stex_importmodule>
%    \end{macrocode}
%
% \begin{macro}{\stex_import_module_uri:nn}
%    \begin{macrocode}
\cs_new_protected:Nn \stex_import_module_uri:nn {
  \str_set:Nx \l_stex_import_archive_str { #1 }
  \str_set:Nn \l_stex_import_path_str { #2 }

  \exp_args:NNNo \seq_set_split:Nnn \l_tmpb_seq ? { \l_stex_import_path_str }
  \seq_pop_right:NN \l_tmpb_seq \l_stex_import_name_str
  \str_set:Nx \l_stex_import_path_str { \seq_use:Nn \l_tmpb_seq ? }

  \stex_modules_current_namespace:
  \bool_lazy_all:nTF {
    {\str_if_empty_p:N \l_stex_import_archive_str}
    {\str_if_empty_p:N \l_stex_import_path_str}
    {\stex_if_module_exists_p:n { \l_stex_module_ns_str ? \l_stex_import_name_str } }
  }{
    \str_set_eq:NN \l_stex_import_path_str \l_stex_modules_subpath_str
    \str_set_eq:NN \l_stex_import_ns_str \l_stex_module_ns_str
  }{
    \str_if_empty:NT \l_stex_import_archive_str {
      \prop_if_exist:NT \l_stex_current_repository_prop {
        \prop_get:NnN \l_stex_current_repository_prop { id } \l_stex_import_archive_str
      }
    }  
    \str_if_empty:NTF \l_stex_import_archive_str {
      \str_if_empty:NF \l_stex_import_path_str {
        \str_set:Nx \l_stex_import_ns_str {
          \l_stex_module_ns_str / \l_stex_import_path_str
        }
      }
    }{
      \stex_require_repository:n \l_stex_import_archive_str
      \prop_get:cnN { c_stex_mathhub_\l_stex_import_archive_str _manifest_prop } { ns }
        \l_stex_import_ns_str
      \str_if_empty:NF \l_stex_import_path_str {
        \str_set:Nx \l_stex_import_ns_str {
          \l_stex_import_ns_str / \l_stex_import_path_str
        }
      }
    }
  }
}
%    \end{macrocode}
% \end{macro}
%
% \begin{variable}{
%   \l_stex_import_name_str,\l_stex_import_archive_str,\l_stex_import_path_str,\l_stex_import_ns_str
% }
%   Store the return values of \cs{stex_import_module_uri:nn}.
%    \begin{macrocode}
\str_new:N \l_stex_import_name_str
\str_new:N \l_stex_import_archive_str
\str_new:N \l_stex_import_path_str
\str_new:N \l_stex_import_ns_str
%    \end{macrocode}
% \end{variable}
%
% \begin{macro}{\stex_import_require_module:nnnn}
%    \begin{syntax} \Arg{ns} \Arg{archive-ID} \Arg{path} \Arg{name} \end{syntax}
%    \begin{macrocode}
\cs_new_protected:Nn \stex_import_require_module:nnnn {
  \exp_args:Nx \stex_if_module_exists:nF { #1 ? #4 } {

    % archive
    \str_set:Nx \l_tmpa_str { #2 }
    \str_if_empty:NTF \l_tmpa_str {
      \seq_set_eq:NN \l_tmpa_seq \g_stex_currentfile_seq
    } {
      \stex_path_from_string:Nn \l_tmpb_seq { \l_tmpa_str }
      \seq_concat:NNN \l_tmpa_seq \c_stex_mathhub_seq \l_tmpb_seq
      \seq_put_right:Nn \l_tmpa_seq { source }
    }

    % path
    \str_set:Nx \l_tmpb_str { #3 }
    \str_if_empty:NTF \l_tmpb_str {
      \str_set:Nx \l_tmpa_str { \stex_path_to_string:N \l_tmpa_seq / #4 }
      
      \ltx@ifpackageloaded{babel} { 
        \exp_args:NNx \prop_get:NnNF \c_stex_language_abbrevs_prop 
            { \languagename } \l_tmpb_str {
              \msg_error:nnx{stex}{error/unknownlanguage}{\languagename}
            }
      } {
        \str_clear:N \l_tmpb_str
      }

      \stex_debug:nn{modules}{Checking~\l_tmpa_str.\l_tmpb_str.tex}
      \IfFileExists{ \l_tmpa_str.\l_tmpb_str.tex }{
        \str_gset:Nx \g_@@_file_str { \l_tmpa_str.\l_tmpb_str.tex }
      }{
        \stex_debug:nn{modules}{Checking~\l_tmpa_str.tex}
        \IfFileExists{ \l_tmpa_str.tex }{
          \str_gset:Nx \g_@@_file_str { \l_tmpa_str.tex }
        }{
          % try english as default
          \stex_debug:nn{modules}{Checking~\l_tmpa_str.en.tex}
          \IfFileExists{ \l_tmpa_str.en.tex }{
            \str_gset:Nx \g_@@_file_str { \l_tmpa_str.en.tex }
          }{
            \msg_error:nnx{stex}{error/unknownmodule}{#1?#4}
          }
        }
      }

    } {
      \seq_set_split:NnV \l_tmpb_seq / \l_tmpb_str
      \seq_concat:NNN \l_tmpa_seq \l_tmpa_seq \l_tmpb_seq
      
      \ltx@ifpackageloaded{babel} { 
        \exp_args:NNx \prop_get:NnNF \c_stex_language_abbrevs_prop 
            { \languagename } \l_tmpb_str {
              \msg_error:nnx{stex}{error/unknownlanguage}{\languagename}
            }
      } {
        \str_clear:N \l_tmpb_str
      }

      \stex_path_to_string:NN \l_tmpa_seq \l_tmpa_str

      \stex_debug:nn{modules}{Checking~\l_tmpa_str/#4.\l_tmpb_str.tex}
      \IfFileExists{ \l_tmpa_str/#4.\l_tmpb_str.tex }{
        \str_gset:Nx \g_@@_file_str { \l_tmpa_str/#4.\l_tmpb_str.tex }
      }{
        \stex_debug:nn{modules}{Checking~\l_tmpa_str/#4.tex}
        \IfFileExists{ \l_tmpa_str/#4.tex }{
          \str_gset:Nx \g_@@_file_str { \l_tmpa_str/#4.tex }
        }{
          % try english as default
          \stex_debug:nn{modules}{Checking~\l_tmpa_str/#4.en.tex}
          \IfFileExists{ \l_tmpa_str/#4.en.tex }{
            \str_gset:Nx \g_@@_file_str { \l_tmpa_str/#4.en.tex }
          }{
            \stex_debug:nn{modules}{Checking~\l_tmpa_str.\l_tmpb_str.tex}
            \IfFileExists{ \l_tmpa_str.\l_tmpb_str.tex }{
              \str_gset:Nx \g_@@_file_str { \l_tmpa_str.\l_tmpb_str.tex }
            }{
              \stex_debug:nn{modules}{Checking~\l_tmpa_str.tex}
              \IfFileExists{ \l_tmpa_str.tex }{
                \str_gset:Nx \g_@@_file_str { \l_tmpa_str.tex }
              }{
                % try english as default
                \stex_debug:nn{modules}{Checking~\l_tmpa_str.en.tex}
                \IfFileExists{ \l_tmpa_str.en.tex }{
                  \str_gset:Nx \g_@@_file_str { \l_tmpa_str.en.tex }
                }{
                  \msg_error:nnx{stex}{error/unknownmodule}{#1?#4}
                }
              }
            }
          }
        }
      }
    }

    \exp_args:No \stex_in_smsmode:nn { \g_@@_file_str } {
      \seq_clear:N \l_stex_all_modules_seq
      \str_clear:N \l_stex_current_module_str
      \str_set:Nx \l_tmpb_str { #2 }
      \str_if_empty:NF \l_tmpb_str {
        \stex_set_current_repository:n { #2 }
      }
      \stex_debug:nn{modules}{Loading~\g_@@_file_str}
      \input { \g_@@_file_str }
    }
    
    \stex_if_module_exists:nF { #1 ? #4 } {
      \msg_error:nnx{stex}{error/unknownmodule}{
        #1?#4~(in~file~\g_@@_file_str)
      }
    }
  }
  \stex_activate_module:n { #1 ? #4 }
}
%    \end{macrocode}
% \end{macro}
%
% \begin{macro}{\importmodule}
%    \begin{macrocode}
\NewDocumentCommand \importmodule { O{} m } {
  \stex_import_module_uri:nn { #1 } { #2 }
  \stex_debug:nn{modules}{Importing~module:~
    \l_stex_import_ns_str ? \l_stex_import_name_str
  }
  \stex_if_smsmode:F {
    \stex_import_require_module:nnnn 
    { \l_stex_import_ns_str } { \l_stex_import_archive_str } 
    { \l_stex_import_path_str } { \l_stex_import_name_str }
    \stex_annotate_invisible:nnn 
      {import} {\l_stex_import_ns_str ? \l_stex_import_name_str} {}
  }
  \exp_args:Nx \stex_add_to_current_module:n {
    \stex_import_require_module:nnnn 
    { \l_stex_import_ns_str } { \l_stex_import_archive_str } 
    { \l_stex_import_path_str } { \l_stex_import_name_str }
  }
  \exp_args:Nx \stex_add_import_to_current_module:n {
    \l_stex_import_ns_str ? \l_stex_import_name_str
  }
  \stex_smsmode_set_codes:
}
\stex_deactivate_macro:Nn \importmodule {module~environments}
%    \end{macrocode}
% \end{macro}
%
% \begin{macro}{\usemodule}
%    \begin{macrocode}
\NewDocumentCommand \usemodule { O{} m } {
  \stex_if_smsmode:F {
    \stex_import_module_uri:nn { #1 } { #2 }
    \stex_import_require_module:nnnn 
    { \l_stex_import_ns_str } { \l_stex_import_archive_str }
    { \l_stex_import_path_str } { \l_stex_import_name_str }
    \stex_annotate_invisible:nnn 
      {usemodule} {\l_stex_import_ns_str ? \l_stex_import_name_str} {}
  }
  \stex_smsmode_set_codes:
}
%    \end{macrocode}
% \end{macro}
%
%    \begin{macrocode}
%</package>
%    \end{macrocode}
%
% \end{implementation}
%
% \PrintIndex

  % \iffalse meta-comment
% An Infrastructure for Semantic Macros and Module Scoping
% Copyright (c) 2019 Michael Kohlhase, all rights reserved
%                this file is released under the
%                LaTeX Project Public License (LPPL)
% 
% The original of this file is in the public repository at 
% http://github.com/sLaTeX/sTeX/
%
% TODO update copyright  
%
%<*driver>
\providecommand\bibfolder{../../lib/bib}
\RequirePackage{paralist}
\documentclass[full,kernel]{l3doc}
\usepackage[dvipsnames]{xcolor}
\usepackage[utf8]{inputenc}
\usepackage[T1]{fontenc}
\RequirePackage{morewrites}
\RequirePackage{tikzinput}
\usetikzlibrary{fit}

\usepackage[debug=all,lang=en, mathhub=./tests]{stex}
\usepackage{url,array,float,textcomp}
\usepackage[show]{ed}
\usepackage[hyperref=auto,style=alphabetic]{biblatex}
\addbibresource{\bibfolder/kwarcpubs.bib}
\addbibresource{\bibfolder/extpubs.bib}
\addbibresource{\bibfolder/kwarccrossrefs.bib}
\addbibresource{\bibfolder/extcrossrefs.bib}
\usepackage{amssymb}
\usepackage{amsfonts}
\usepackage{xspace}
\usepackage{hyperref}

\makeindex
\floatstyle{boxed}
\newfloat{exfig}{thp}{lop}
\floatname{exfig}{Example}

\usepackage{stex-tests}

\MakeShortVerb{\|}

\def\scsys#1{{{\sc #1}}\index{#1@{\sc #1}}\xspace}
\def\mmt{\textsc{Mmt}\xspace}
\def\xml{\scsys{Xml}}
\def\mathml{\scsys{MathML}}
\def\omdoc{\scsys{OMDoc}}
\def\openmath{\scsys{OpenMath}}
\def\latexml{\scsys{LaTeXML}}
\def\perl{\scsys{Perl}}
\def\cmathml{Content-{\sc MathML}\index{Content {\sc MathML}}\index{MathML@{\sc MathML}!content}}
\def\activemath{\scsys{ActiveMath}}
\def\twin#1#2{\index{#1!#2}\index{#2!#1}}
\def\twintoo#1#2{{#1 #2}\twin{#1}{#2}}
\def\atwin#1#2#3{\index{#1!#2!#3}\index{#3!#2 (#1)}}
\def\atwintoo#1#2#3{{#1 #2 #3}\atwin{#1}{#2}{#3}}
\def\cT{\mathcal{T}}\def\cD{\mathcal{D}}

\def\fileversion{3.0}
\def\filedate{\today}

\RequirePackage{pdfcomment}

\ExplSyntaxOn\makeatletter
\cs_set_protected:Npn \@comp #1 #2 {
  \pdftooltip {
    \textcolor{blue}{#1}
  } { #2 }
}

\cs_set_protected:Npn \@defemph #1 #2 {
  \pdftooltip { 
    \textbf{\textcolor{magenta}{#1}}
  } { #2 }
}

\def\__omtext_lec#1{#1}
\cs_new_protected:Npn \lec #1 {
  \strut\hfil\strut\null\hfill\__omtext_lec{#1}
}
\makeatother\ExplSyntaxOff

\makeatletter
\let\@stex@oldcomment\comment
\let\@stex@oldendcomment\endcomment

%\RequirePackage{comment}
\RequirePackage{document-structure}
\RequirePackage[hints,solutions,notes]{problem}
\RequirePackage{hwexam}

\let\comment\@stex@oldcomment
\let\endcomment\@stex@oldendcomment

\newif\ifinfulldoc\infulldocfalse
\makeatother

\def\basedocurl{https://github.com/slatex/sTeX/blob/latex3/doc}
\newcounter{module}

\NewDocumentEnvironment {module}{}{
  \stepcounter{module}
  \textbf{Module \themodule: \smoduletitle}
}{

}
\stexpatchmodule{\begin{module}}{\end{module}}

\def\compemph#1{\textcolor{blue}{#1}}
\def\symrefemph#1{\textcolor{green}{#1}}

\RequirePackage{pdfcomment}
\makeatletter
\protected\def\compemph@uri#1#2{%
  \pdftooltip{%
    \srefsymuri{#2}{\compemph{#1}}%
  }{%
    URI: \detokenize{#2}%
  }%
}
\protected\def\symrefemph@uri#1#2{%
  \pdftooltip{%
    \srefsymuri{#2}{\symrefemph{#1}}%
  }{%
    URI: \detokenize{#2}%
  }%
}
\makeatother

\begin{document}
  \DocInput{\jobname.dtx}
\end{document}
%</driver>
% \fi
%
% \title{ \sTeX-Structural Features
% 	\thanks{Version {\fileversion} (last revised {\filedate})} 
% }
%
% \author{Michael Kohlhase, Dennis Müller\\
% 	FAU Erlangen-Nürnberg\\
% 	\url{http://kwarc.info/}
% }
%
% \maketitle
%
%\ifinfulldoc\elseGiven modules:

\stexexample{
    \begin{smodule}{magma}
        \symdef{universe}{\comp{\mathcal U}}
        \symdef[args=2,op=\circ]{operation}{#1 \comp\circ #2}
    \end{smodule}
    \begin{smodule}{monoid}
        \importmodule{magma}
        \symdef{unit}{\comp e}
    \end{smodule}
    \begin{smodule}{group}
        \importmodule{monoid}
        \symdef[args=1]{inverse}{{#1}^{\comp{-1}}}
    \end{smodule}
}

We can form a module for \emph{rings} by ``cloning''
an instance of |group| (for addition) and |monoid| (for multiplication),
respectively, and ``glueing them together'' to ensure they share the
same universe:

\stexexample{
    \begin{smodule}{ring}
        \begin{copymodule}{group}{addition}
            \renamedecl[name=universe]{universe}{runiverse}
            \renamedecl[name=plus]{operation}{rplus}
            \renamedecl[name=zero]{unit}{rzero}
            \renamedecl[name=uminus]{inverse}{ruminus}
        \end{copymodule}
        \notation[plus,op=+,prec=60]{rplus}{#1 \comp+ #2}
        %\setnotation{rplus}{plus}
        \notation[zero]{rzero}{\comp0}
        %\setnotation{rzero}{zero}
        \notation[uminus,op=-]{ruminus}{\comp- #1}
        %\setnotation{ruminus}{uminus}
        \begin{copymodule}{monoid}{multiplication}
            \assign{universe}{\runiverse}
            \renamedecl[name=times]{operation}{rtimes}
            \renamedecl[name=one]{unit}{rone}
        \end{copymodule}
        \notation[cdot,op=\cdot,prec=50]{rtimes}{#1 \comp\cdot #2}
        %\setnotation{rtimes}{cdot}
        \notation[one]{rone}{\comp1}
        %\setnotation{rone}{one}
        Test: $\rtimes a{\rplus c{\rtimes de}}$
    \end{smodule}
}

\textcolor{red}{TODO: explain donotclone}


\stexexample{
    \begin{smodule}{int}
        \symdef{Integers}{\comp{\mathbb Z}}
        \symdef[args=2,op=+]{plus}{#1 \comp+ #2}
        \symdef{zero}{\comp0}
        \symdef[args=1,op=-]{uminus}{\comp-#1}

        \begin{interpretmodule}{group}{intisgroup}
            \assign{universe}{\Integers}
            \assign{operation}{\plus!}
            \assign{unit}{\zero}
            \assign{inverse}{\uminus!}
        \end{interpretmodule}
    \end{smodule}
}\fi
%
% \begin{documentation}\label{pkg:features:doc}
%
% Code related to structural features
%
% \section{Macros and Environments}\label{pkg:features:doc:macros}
%
% \subsection{Structures}
%
% \begin{environment}{mathstructure}
%   TODO
% \end{environment}
%
%^^A\stextest{
%^^A \begin{module}{StructureTest1}
%^^A   \begin{mathstructure}[name=Magma]{magma}
%^^A     \symdef{universe}{\comp M}
%^^A     \symdef[args=2]{op}{#1 \comp\circ #2}
%^^A       $\isa{\op ab}\universe$
%^^A   \end{mathstructure}
%^^A
%^^A     \ExplSyntaxOn
%^^A     \prop_get:NnN \g_stex_last_feature_prop {fields} \l_tmpa_seq
%^^A     \seq_use:Nn \l_tmpa_seq {,}
%^^A     \ExplSyntaxOff
%^^A
%^^A     \present\magma
%^^A
%^^A     \instantiate{magma}[
%^^A       universe ! {{\comp U}},
%^^A       op ! {{#1 \comp+ #2 }}
%^^A     ]{mM}
%^^A     \notation[op = U]{mM/universe}{\comp U}
%^^A     \notation[op = +]{mM/op}{#1 \comp+ #2}
%^^A
%^^A     Test: $\mM{op}ab$
%^^A
%^^A     Test2: $\mM{}$
%^^A \end{module}
%^^A}
%
%
% \end{documentation}
%
% \begin{implementation}\label{pkg:features:impl}
%
% \section{\sTeX-Structural Features Implementation}
%
%    \begin{macrocode}
%<*package>

%%%%%%%%%%%%%   features.dtx   %%%%%%%%%%%%%

%<@@=stex_features>
%    \end{macrocode}
%
% Warnings and error messages
%
%    \begin{macrocode}

%    \end{macrocode}
%
% \subsection{Imports with modification}
%
%
%    \begin{macrocode}
\seq_new:N \l_stex_implicit_morphisms_seq
\NewDocumentCommand \implicitmorphism { O{} m m}{
  \stex_import_module_uri:nn { #1 } { #2 }
  \stex_debug:nn{implicits}{
    Implicit~morphism:~
    \l_stex_module_ns_str ? \l_@@_name_str
  }
  \exp_args:NNx \seq_if_in:NnT \l_stex_all_modules_seq {
    \l_stex_module_ns_str ? \l_@@_name_str
  }{
    \msg_error:nnn{stex}{error/conflictingmodules}{
      \l_stex_module_ns_str ? \l_@@_name_str
    }
  }

  % TODO
  


  \seq_put_right:Nx \l_stex_implicit_morphisms_seq {
    \l_stex_module_ns_str ? \l_@@_name_str
  }
}

%    \end{macrocode}
%
%
% \subsection{The feature environment}
%
% \begin{environment}{structural@feature}
%    \begin{macrocode}

\NewDocumentEnvironment{structural@feature}{ m m m }{
  \stex_if_in_module:F {
    \msg_set:nnn{stex}{error/nomodule}{
      Structural~Feature~has~to~occur~in~a~module:\\
      Feature~#2~of~type~#1\\
      In~File:~\stex_path_to_string:N \g_stex_currentfile_seq
    }
    \msg_error:nn{stex}{error/nomodule}
  }

  \str_set:Nx \l_stex_module_name_str {
    \prop_item:Nn \l_stex_current_module_prop
      { name } / #2 - feature
  }
  
  \str_set:Nx \l_stex_module_ns_str {
    \prop_item:Nn \l_stex_current_module_prop
      { ns }
  }

  
  \str_clear:N \l_tmpa_str
  \seq_clear:N \l_tmpa_seq
  \tl_clear:N \l_tmpa_tl
  \exp_args:NNx \prop_set_from_keyval:Nn \l_stex_current_module_prop {
    origname  = #2,
    name      = \l_stex_module_name_str ,
    ns        = \l_stex_module_ns_str ,
    imports   = \exp_not:o { \l_tmpa_seq } ,
    constants = \exp_not:o { \l_tmpa_seq } ,
    content   = \exp_not:o { \l_tmpa_tl }  ,
    file      = \exp_not:o { \g_stex_currentfile_seq } ,
    lang      = \l_stex_module_lang_str ,
    sig       = \l_tmpa_str ,
    meta      = \l_tmpa_str ,
    feature   = #1 ,
  }

  \stex_if_smsmode:TF {
    \stex_smsmode_set_codes:
  } {
    \begin{stex_annotate_env}{ feature:#1 }{}
      \stex_annotate_invisible:nnn{header}{}{ #3 }
  }
}{  
  \str_set:Nx \l_tmpa_str {
    c_stex_feature_
    \prop_item:Nn \l_stex_current_module_prop { ns } ?
    \prop_item:Nn \l_stex_current_module_prop { name }
    _prop
  }
  \prop_gset_eq:cN { \l_tmpa_str } \l_stex_current_module_prop
  \prop_gset_eq:NN \g_stex_last_feature_prop \l_stex_current_module_prop
  \stex_if_smsmode:TF {
    \exp_args:Nx \stex_add_to_sms:n {
      \prop_gset_from_keyval:cn {
        c_stex_feature_
        \prop_item:Nn \l_stex_current_module_prop { ns } ?
        \prop_item:Nn \l_stex_current_module_prop { name }
        _prop
      } {
        origname  = #2,
        name      = \prop_item:cn { \l_tmpa_str } { name } ,
        ns        = \prop_item:cn { \l_tmpa_str } { ns } ,
        imports   = \prop_item:cn { \l_tmpa_str } { imports } ,
        constants = \prop_item:cn { \l_tmpa_str } { constants } ,
        content   = \prop_item:cn { \l_tmpa_str } { content } ,
        file      = \prop_item:cn { \l_tmpa_str } { file } ,
        lang      = \prop_item:cn { \l_tmpa_str } { lang } ,
        sig       = \prop_item:cn { \l_tmpa_str } { sig } ,
        meta      = \prop_item:cn { \l_tmpa_str } { meta } ,
        feature   = \prop_item:cn { \l_tmpa_str } { feature }
      }
    }
  } {
      \end{stex_annotate_env}
  }
}

%    \end{macrocode}
% \end{environment}
%
%
% \subsection{Features}
%
% \begin{environment}{structure}
%    \begin{macrocode}

\prop_new:N \l_stex_all_structures_prop

\keys_define:nn { stex / features / structure } {
  name         .str_set_x:N  = \l_@@_structure_name_str ,
}

\cs_new_protected:Nn \_@@_structure_args:n {
  \str_clear:N \l_@@_structure_name_str
  \keys_set:nn { stex / features / structure } { #1 }
}

%\stex_new_feature:nnnn { structure } { O{} m } {
%  \_@@_structure_args:n { ##1 }
%  \str_if_empty:NT \l_@@_structure_name_str {
%    \str_set:Nx \l_@@_structure_name_str { ##2 }
%  }
%} {
%
%}

\NewDocumentEnvironment{mathstructure}{ O{} m }{
  \_@@_structure_args:n { #1 }
  \str_if_empty:NT \l_@@_structure_name_str {
    \str_set:Nx \l_@@_structure_name_str { #2 }
  }
  \exp_args:Nnnx
  \begin{structural@feature}{ structure }
    { \l_@@_structure_name_str }{}
    \seq_clear:N \l_tmpa_seq
    \prop_put:Nno \l_stex_current_module_prop { fields } \l_tmpa_seq

}{
    \prop_get:NnN \l_stex_current_module_prop { constants } \l_tmpa_seq
    \prop_get:NnN \l_stex_current_module_prop { fields } \l_tmpb_seq
    \str_set:Nx \l_tmpa_str {
      \prop_item:Nn \l_stex_current_module_prop { ns } ?
      \prop_item:Nn \l_stex_current_module_prop { name }
    }
    \seq_map_inline:Nn \l_tmpa_seq {
      \exp_args:NNx \seq_put_right:Nn \l_tmpb_seq { \l_tmpa_str ? ##1 }
    }
    \prop_put:Nno \l_stex_current_module_prop { fields } { \l_tmpb_seq }
    \exp_args:Nnx
    \AddToHookNext { env / mathstructure / after }{
      \symdecl[type = \exp_not:N\collection,def={\STEXsymbol{module-type}{
        \_stex_term_math_oms:nnnn { \l_tmpa_str }{}{0}{}
      }}, name = \prop_item:Nn \l_stex_current_module_prop { origname }]{ #2 }
      \STEXexport {
        \prop_put:Nno \exp_not:N \l_stex_all_structures_prop 
          {\prop_item:Nn \l_stex_current_module_prop { origname }}
          {\l_tmpa_str}
          \prop_put:Nno \exp_not:N \l_stex_all_structures_prop 
            {#2}{\l_tmpa_str}
%        \seq_put_right:Nn \exp_not:N \l_stex_all_structures_seq {
%          \prop_item:Nn \l_stex_current_module_prop { origname },
%          \l_tmpa_str
%        }
%        \seq_put_right:Nn \exp_not:N \l_stex_all_structures_seq {
%          #2,\l_tmpa_str
%        }
%        \tl_set:cx { #2 } { 
%          \stex_invoke_structure:n { \l_tmpa_str }
      }
    }
    
  \end{structural@feature}
  % \g_stex_last_feature_prop
}
%    \end{macrocode}
% \end{environment}
%
%
% \begin{macro}{\instantiate}
%    \begin{macrocode}
\seq_new:N \l_@@_structure_field_seq
\str_new:N \l_@@_structure_field_str
\str_new:N \l_@@_structure_def_tl
\prop_new:N \l_@@_structure_prop
\NewDocumentCommand \instantiate { m O{} m }{
  \stex_smsmode_set_codes:
  \prop_get:NnN \l_stex_all_structures_prop {#1} \l_tmpa_str
  \prop_set_eq:Nc \l_@@_structure_prop {
    c_stex_feature_\l_tmpa_str _prop
  }
  \seq_set_from_clist:Nn \l_@@_structure_field_seq { #2 }
  \seq_map_inline:Nn \l_@@_structure_field_seq {
    \seq_set_split:Nnn \l_tmpa_seq{=}{ ##1 }
    \int_compare:nNnTF {\seq_count:N \l_tmpa_seq} > 1 {
      \seq_get_left:NN \l_tmpa_seq \l_tmpa_tl
      \exp_args:NNno \seq_set_split:Nnn \l_tmpb_seq
        {!} \l_tmpa_tl 
      \int_compare:nNnTF {\seq_count:N \l_tmpb_seq} > 1 {
        \str_set:Nx \l_@@_structure_field_str {\seq_item:Nn \l_tmpb_seq 1}
        \seq_get_right:NN \l_tmpb_seq \l_tmpb_tl
        \seq_get_right:NN \l_tmpa_seq \l_tmpa_tl
      }{
        \str_set:Nx \l_@@_structure_field_str \l_tmpa_tl
        \seq_get_right:NN \l_tmpa_seq \l_tmpa_tl
        \exp_args:NNno \seq_set_split:Nnn \l_tmpb_seq{!}
          \l_tmpa_tl 
        \int_compare:nNnTF {\seq_count:N \l_tmpb_seq} > 1 {
          \seq_get_left:NN \l_tmpb_seq \l_tmpa_tl
          \seq_get_right:NN \l_tmpb_seq \l_tmpb_tl
        }{
          \tl_clear:N \l_tmpb_tl
        }
      }
    }{
      \seq_set_split:Nnn \l_tmpa_seq{!}{ ##1 }
      \int_compare:nNnTF {\seq_count:N \l_tmpa_seq} > 1 {
        \str_set:Nx \l_@@_structure_field_str {\seq_item:Nn \l_tmpa_seq 1}
        \seq_get_right:NN \l_tmpa_seq \l_tmpb_tl
        \tl_clear:N \l_tmpa_tl
      }{
        % TODO throw error
      }
    }
    % \l_tmpa_str: name
    % \l_tmpa_tl: definiens
    % \l_tmpb_tl: notation
    \tl_if_empty:NT \l_@@_structure_field_str { 
      % TODO throw error
    }
    \str_clear:N \l_tmpb_str
    
    \prop_get:NnN \l_@@_structure_prop { fields } \l_tmpa_seq
    \seq_map_inline:Nn \l_tmpa_seq {
      \seq_set_split:Nnn \l_tmpb_seq ? { ####1 }
      \seq_get_right:NN \l_tmpb_seq \l_tmpb_str
      \str_if_eq:NNT \l_@@_structure_field_str \l_tmpb_str {
        \seq_map_break:n {
          \str_set:Nn \l_tmpb_str { ####1 }
        }
      }
    }
    \prop_get:cnN { l_stex_symdecl_ \l_tmpb_str _prop } {args}
      \l_tmpb_str

    \tl_if_empty:NTF \l_tmpb_tl {
      \tl_if_empty:NF \l_tmpa_tl {
        \exp_args:Nx \use:n {
          \symdecl[args=\l_tmpb_str,def={\exp_args:No\exp_not:n{\l_tmpa_tl}}]{#3/\l_@@_structure_field_str}
        }
      }
    }{
      \tl_if_empty:NTF \l_tmpa_tl {
        \exp_args:Nx \use:n {
          \symdef[args=\l_tmpb_str]{#3/\l_@@_structure_field_str}\exp_after:wN\exp_not:n\exp_after:wN{\l_tmpb_tl}
        }

      }{
        \exp_args:Nx \use:n {
          \symdef[args=\l_tmpb_str,def={\exp_args:No\exp_not:n{\l_tmpa_tl}}]{#3/\l_@@_structure_field_str}
          \exp_after:wN\exp_not:n\exp_after:wN{\l_tmpb_tl}
        }
      }
    }
%    \par \prop_item:Nn \l_stex_current_module_prop {ns} ?
%    \prop_item:Nn \l_stex_current_module_prop {name} ?
%    #3/\l_@@_structure_field_str
%    \par
%    \expandafter\present\csname
%      l_stex_symdecl_
%      \prop_item:Nn \l_stex_current_module_prop {ns} ?
%      \prop_item:Nn \l_stex_current_module_prop {name} ?
%      #3/\l_@@_structure_field_str
%      _prop
%    \endcsname
  }

  \tl_clear:N \l_@@_structure_def_tl

  \prop_get:NnN \l_@@_structure_prop { fields } \l_tmpa_seq
  \seq_map_inline:Nn \l_tmpa_seq {
    \seq_set_split:Nnn \l_tmpb_seq ? { ##1 }
    \seq_get_right:NN \l_tmpb_seq \l_tmpa_str
    \exp_args:Nx \use:n {
      \tl_put_right:Nn \exp_not:N \l_@@_structure_def_tl {

      }
    }

    \prop_if_exist:cF {
      l_stex_symdecl_
      \prop_item:Nn \l_stex_current_module_prop {ns} ?
      \prop_item:Nn \l_stex_current_module_prop {name} ?
      #3/\l_tmpa_str
      _prop
    }{
      \prop_get:cnN { l_stex_symdecl_ ##1 _prop } {args}
        \l_tmpb_str
      \exp_args:Nx \use:n {
        \symdecl[args=\l_tmpb_str]{#3/\l_tmpa_str}
      }
    }
  }

  \symdecl*[type={\STEXsymbol{module-type}{
    \_stex_term_math_oms:nnnn {
      \prop_item:Nn \l_@@_structure_prop {ns} ?
      \prop_item:Nn \l_@@_structure_prop {name}
      }{}{0}{}
  }}]{#3}
  
  % TODO: -> sms file

  \tl_set:cx{ #3 }{
    \stex_invoke_structure:nnn {
      \prop_item:Nn \l_stex_current_module_prop {ns} ?
      \prop_item:Nn \l_stex_current_module_prop {name} ? #3
    } {
      \prop_item:Nn \l_@@_structure_prop {ns} ?
      \prop_item:Nn \l_@@_structure_prop {name}
    }
  }

}
%    \end{macrocode}
% \end{macro}
%
%
%
% \begin{macro}{\stex_invoke_structure:nnn}
%    \begin{macrocode}
% #1: URI of the instance
% #2: URI of the instantiated module
\cs_new_protected:Nn \stex_invoke_structure:nnn {
  \tl_if_empty:nTF{ #3 }{  
    \prop_set_eq:Nc \l_@@_structure_prop {
      c_stex_feature_ #2 _prop
    }
    \tl_clear:N \l_tmpa_tl
    \prop_get:NnN \l_@@_structure_prop { fields } \l_tmpa_seq
    \seq_map_inline:Nn \l_tmpa_seq {
      \seq_set_split:Nnn \l_tmpb_seq ? { ##1 }
      \seq_get_right:NN \l_tmpb_seq \l_tmpa_str
      \cs_if_exist:cT {
        stex_notation_ #1/\l_tmpa_str \c_hash_str\c_hash_str _cs
      }{
        \tl_if_empty:NF \l_tmpa_tl {
          \tl_put_right:Nn \l_tmpa_tl {,}
        }
        \tl_put_right:Nx \l_tmpa_tl {
          \stex_invoke_symbol:n {#1/\l_tmpa_str}!
        }
      }
    }
    \exp_args:No \mathstruct \l_tmpa_tl
  }{
    \stex_invoke_symbol:n{#1/#3}
  }
}
%    \end{macrocode}
% \end{macro}
%
%    \begin{macrocode}
%</package>
%    \end{macrocode}
%
% \end{implementation}
%
% \PrintIndex

  \begin{omgroup}{Primitive Symbols (The \sTeX Metatheory)}
    % \iffalse meta-comment
% An Infrastructure for Semantic Macros and Module Scoping
% Copyright (c) 2019 Michael Kohlhase, all rights reserved
%                this file is released under the
%                LaTeX Project Public License (LPPL)
% 
% The original of this file is in the public repository at 
% http://github.com/sLaTeX/sTeX/
%
% TODO update copyright  
%
%<*driver>
\providecommand\bibfolder{../../lib/bib}
\RequirePackage{paralist}
\documentclass[full,kernel]{l3doc}
\usepackage[dvipsnames]{xcolor}
\usepackage[utf8]{inputenc}
\usepackage[T1]{fontenc}
\RequirePackage{morewrites}
\RequirePackage{tikzinput}
\usetikzlibrary{fit}

\usepackage[debug=all,lang=en, mathhub=./tests]{stex}
\usepackage{url,array,float,textcomp}
\usepackage[show]{ed}
\usepackage[hyperref=auto,style=alphabetic]{biblatex}
\addbibresource{\bibfolder/kwarcpubs.bib}
\addbibresource{\bibfolder/extpubs.bib}
\addbibresource{\bibfolder/kwarccrossrefs.bib}
\addbibresource{\bibfolder/extcrossrefs.bib}
\usepackage{amssymb}
\usepackage{amsfonts}
\usepackage{xspace}
\usepackage{hyperref}

\makeindex
\floatstyle{boxed}
\newfloat{exfig}{thp}{lop}
\floatname{exfig}{Example}

\usepackage{stex-tests}

\MakeShortVerb{\|}

\def\scsys#1{{{\sc #1}}\index{#1@{\sc #1}}\xspace}
\def\mmt{\textsc{Mmt}\xspace}
\def\xml{\scsys{Xml}}
\def\mathml{\scsys{MathML}}
\def\omdoc{\scsys{OMDoc}}
\def\openmath{\scsys{OpenMath}}
\def\latexml{\scsys{LaTeXML}}
\def\perl{\scsys{Perl}}
\def\cmathml{Content-{\sc MathML}\index{Content {\sc MathML}}\index{MathML@{\sc MathML}!content}}
\def\activemath{\scsys{ActiveMath}}
\def\twin#1#2{\index{#1!#2}\index{#2!#1}}
\def\twintoo#1#2{{#1 #2}\twin{#1}{#2}}
\def\atwin#1#2#3{\index{#1!#2!#3}\index{#3!#2 (#1)}}
\def\atwintoo#1#2#3{{#1 #2 #3}\atwin{#1}{#2}{#3}}
\def\cT{\mathcal{T}}\def\cD{\mathcal{D}}

\def\fileversion{3.0}
\def\filedate{\today}

\RequirePackage{pdfcomment}

\ExplSyntaxOn\makeatletter
\cs_set_protected:Npn \@comp #1 #2 {
  \pdftooltip {
    \textcolor{blue}{#1}
  } { #2 }
}

\cs_set_protected:Npn \@defemph #1 #2 {
  \pdftooltip { 
    \textbf{\textcolor{magenta}{#1}}
  } { #2 }
}

\def\__omtext_lec#1{#1}
\cs_new_protected:Npn \lec #1 {
  \strut\hfil\strut\null\hfill\__omtext_lec{#1}
}
\makeatother\ExplSyntaxOff

\makeatletter
\let\@stex@oldcomment\comment
\let\@stex@oldendcomment\endcomment

%\RequirePackage{comment}
\RequirePackage{document-structure}
\RequirePackage[hints,solutions,notes]{problem}
\RequirePackage{hwexam}

\let\comment\@stex@oldcomment
\let\endcomment\@stex@oldendcomment

\newif\ifinfulldoc\infulldocfalse
\makeatother

\def\basedocurl{https://github.com/slatex/sTeX/blob/latex3/doc}
\newcounter{module}

\NewDocumentEnvironment {module}{}{
  \stepcounter{module}
  \textbf{Module \themodule: \smoduletitle}
}{

}
\stexpatchmodule{\begin{module}}{\end{module}}

\def\compemph#1{\textcolor{blue}{#1}}
\def\symrefemph#1{\textcolor{green}{#1}}

\RequirePackage{pdfcomment}
\makeatletter
\protected\def\compemph@uri#1#2{%
  \pdftooltip{%
    \srefsymuri{#2}{\compemph{#1}}%
  }{%
    URI: \detokenize{#2}%
  }%
}
\protected\def\symrefemph@uri#1#2{%
  \pdftooltip{%
    \srefsymuri{#2}{\symrefemph{#1}}%
  }{%
    URI: \detokenize{#2}%
  }%
}
\makeatother

\begin{document}
  \DocInput{\jobname.dtx}
\end{document}
%</driver>
% \fi
%
% \title{ \sTeX-Metatheory
% 	\thanks{Version {\fileversion} (last revised {\filedate})} 
% }
%
% \author{Michael Kohlhase, Dennis Müller\\
% 	FAU Erlangen-Nürnberg\\
% 	\url{http://kwarc.info/}
% }
%
% \maketitle
%
%\ifinfulldoc\else
% This is the documentation for the \pkg{stex-metatheory} package.
% For a more high-level introduction, 
%  see \href{\basedocurl/manual.pdf}{the \sTeX Manual} or the
% \href{\basedocurl/stex.pdf}{full \sTeX documentation}.
%
% % \iffalse meta-comment
% An Infrastructure for Semantic Macros and Module Scoping
% Copyright (c) 2019 Michael Kohlhase, all rights reserved
%                this file is released under the
%                LaTeX Project Public License (LPPL)
% 
% The original of this file is in the public repository at 
% http://github.com/sLaTeX/sTeX/
%
% TODO update copyright  
%
%<*driver>
\providecommand\bibfolder{../../lib/bib}
\RequirePackage{paralist}
\documentclass[full,kernel]{l3doc}
\usepackage[dvipsnames]{xcolor}
\usepackage[utf8]{inputenc}
\usepackage[T1]{fontenc}
\RequirePackage{morewrites}
\RequirePackage{tikzinput}
\usetikzlibrary{fit}

\usepackage[debug=all,lang=en, mathhub=./tests]{stex}
\usepackage{url,array,float,textcomp}
\usepackage[show]{ed}
\usepackage[hyperref=auto,style=alphabetic]{biblatex}
\addbibresource{\bibfolder/kwarcpubs.bib}
\addbibresource{\bibfolder/extpubs.bib}
\addbibresource{\bibfolder/kwarccrossrefs.bib}
\addbibresource{\bibfolder/extcrossrefs.bib}
\usepackage{amssymb}
\usepackage{amsfonts}
\usepackage{xspace}
\usepackage{hyperref}

\makeindex
\floatstyle{boxed}
\newfloat{exfig}{thp}{lop}
\floatname{exfig}{Example}

\usepackage{stex-tests}

\MakeShortVerb{\|}

\def\scsys#1{{{\sc #1}}\index{#1@{\sc #1}}\xspace}
\def\mmt{\textsc{Mmt}\xspace}
\def\xml{\scsys{Xml}}
\def\mathml{\scsys{MathML}}
\def\omdoc{\scsys{OMDoc}}
\def\openmath{\scsys{OpenMath}}
\def\latexml{\scsys{LaTeXML}}
\def\perl{\scsys{Perl}}
\def\cmathml{Content-{\sc MathML}\index{Content {\sc MathML}}\index{MathML@{\sc MathML}!content}}
\def\activemath{\scsys{ActiveMath}}
\def\twin#1#2{\index{#1!#2}\index{#2!#1}}
\def\twintoo#1#2{{#1 #2}\twin{#1}{#2}}
\def\atwin#1#2#3{\index{#1!#2!#3}\index{#3!#2 (#1)}}
\def\atwintoo#1#2#3{{#1 #2 #3}\atwin{#1}{#2}{#3}}
\def\cT{\mathcal{T}}\def\cD{\mathcal{D}}

\def\fileversion{3.0}
\def\filedate{\today}

\RequirePackage{pdfcomment}

\ExplSyntaxOn\makeatletter
\cs_set_protected:Npn \@comp #1 #2 {
  \pdftooltip {
    \textcolor{blue}{#1}
  } { #2 }
}

\cs_set_protected:Npn \@defemph #1 #2 {
  \pdftooltip { 
    \textbf{\textcolor{magenta}{#1}}
  } { #2 }
}

\def\__omtext_lec#1{#1}
\cs_new_protected:Npn \lec #1 {
  \strut\hfil\strut\null\hfill\__omtext_lec{#1}
}
\makeatother\ExplSyntaxOff

\makeatletter
\let\@stex@oldcomment\comment
\let\@stex@oldendcomment\endcomment

%\RequirePackage{comment}
\RequirePackage{document-structure}
\RequirePackage[hints,solutions,notes]{problem}
\RequirePackage{hwexam}

\let\comment\@stex@oldcomment
\let\endcomment\@stex@oldendcomment

\newif\ifinfulldoc\infulldocfalse
\makeatother

\def\basedocurl{https://github.com/slatex/sTeX/blob/latex3/doc}
\newcounter{module}

\NewDocumentEnvironment {module}{}{
  \stepcounter{module}
  \textbf{Module \themodule: \smoduletitle}
}{

}
\stexpatchmodule{\begin{module}}{\end{module}}

\def\compemph#1{\textcolor{blue}{#1}}
\def\symrefemph#1{\textcolor{green}{#1}}

\RequirePackage{pdfcomment}
\makeatletter
\protected\def\compemph@uri#1#2{%
  \pdftooltip{%
    \srefsymuri{#2}{\compemph{#1}}%
  }{%
    URI: \detokenize{#2}%
  }%
}
\protected\def\symrefemph@uri#1#2{%
  \pdftooltip{%
    \srefsymuri{#2}{\symrefemph{#1}}%
  }{%
    URI: \detokenize{#2}%
  }%
}
\makeatother

\begin{document}
  \DocInput{\jobname.dtx}
\end{document}
%</driver>
% \fi
%
% \title{ \sTeX-Metatheory
% 	\thanks{Version {\fileversion} (last revised {\filedate})} 
% }
%
% \author{Michael Kohlhase, Dennis Müller\\
% 	FAU Erlangen-Nürnberg\\
% 	\url{http://kwarc.info/}
% }
%
% \maketitle
%
%\ifinfulldoc\else
% This is the documentation for the \pkg{stex-metatheory} package.
% For a more high-level introduction, 
%  see \href{\basedocurl/manual.pdf}{the \sTeX Manual} or the
% \href{\basedocurl/stex.pdf}{full \sTeX documentation}.
%
% % \iffalse meta-comment
% An Infrastructure for Semantic Macros and Module Scoping
% Copyright (c) 2019 Michael Kohlhase, all rights reserved
%                this file is released under the
%                LaTeX Project Public License (LPPL)
% 
% The original of this file is in the public repository at 
% http://github.com/sLaTeX/sTeX/
%
% TODO update copyright  
%
%<*driver>
\providecommand\bibfolder{../../lib/bib}
\input{../../doc/docheader}

\begin{document}
  \DocInput{\jobname.dtx}
\end{document}
%</driver>
% \fi
%
% \title{ \sTeX-Metatheory
% 	\thanks{Version {\fileversion} (last revised {\filedate})} 
% }
%
% \author{Michael Kohlhase, Dennis Müller\\
% 	FAU Erlangen-Nürnberg\\
% 	\url{http://kwarc.info/}
% }
%
% \maketitle
%
%\ifinfulldoc\else
% This is the documentation for the \pkg{stex-metatheory} package.
% For a more high-level introduction, 
%  see \href{\basedocurl/manual.pdf}{the \sTeX Manual} or the
% \href{\basedocurl/stex.pdf}{full \sTeX documentation}.
%
% \input{../../doc/packages/metatheory}
% \fi
%
% \begin{documentation}\label{pkg:metatheory:doc}
%
% The default meta theory for an \sTeX module. Contains
% symbols so ubiquitous, that it is virtually impossible
% to describe any flexiformal content without them, or
% that are required to annotate even the most primitive symbols
% with meaningful (foundation-independent) ``type''-annotations,
% or required for basic structuring principles (theorems, definitions).
%
% Foundations should ideally instantiate these symbols
% with their formal counterparts, e.g. |isa| corresponds
% to a typing operation in typed setting, or the $\in$-operator
% in set-theoretic contexts; |bind| corresponds to a universal
% quantifier in ($n$th-order) logic, or a $\Pi$ in dependent type
% theories.
%
% \section{Symbols}\label{pkg:metatheory:symbols}
%
% \end{documentation}
%
% \begin{implementation}\label{pkg:metatheory:impl}
%
% \section{\sTeX-Metatheory Implementation}
%
%    \begin{macrocode}
%<*package>
%<@@=stex_modules>

%%%%%%%%%%%%%   metatheory.dtx   %%%%%%%%%%%%%

\str_const:Nn \c_stex_metatheory_ns_str {http://mathhub.info/sTeX}
\begingroup
\stex_module_setup:nn{
  ns=\c_stex_metatheory_ns_str,
  meta=NONE
}{Metatheory}
\stex_reactivate_macro:N \symdecl
\stex_reactivate_macro:N \notation
\stex_reactivate_macro:N \symdef
\ExplSyntaxOff
\csname stex_suppress_html:n\endcsname{
  % is-a (a:A, a \in A, a is an A, etc.)
  \symdecl[args=ai]{isa}
  \notation[typed]{isa}{#1 \comp{:} #2}{##1 \comp, ##2}
  \notation[in]{isa}{#1 \comp\in #2}{##1 \comp, ##2}
  \notation[pred]{isa}{#2\comp(#1 \comp)}{##1 \comp, ##2}

  % bind (\forall, \Pi, \lambda etc.)
  \symdecl[args=Bi]{bind}
  \notation[forall]{bind}{\comp\forall #1.\;#2}{##1 \comp, ##2}
  \notation[Pi]{bind}{\comp\prod_{#1}#2}{##1 \comp, ##2}
  \notation[depfun]{bind}{\comp( #1 \comp{)\;\to\;} #2}{##1 \comp, ##2}

  % dummy variable
  \symdecl{dummyvar}
  \notation[underscore]{dummyvar}{\comp\_}
  \notation[dot]{dummyvar}{\comp\cdot}
  \notation[dash]{dummyvar}{\comp{{\rm --}}}

  %fromto (function space, Hom-set, implication etc.)
  \symdecl[args=ai]{fromto}
  \notation[xarrow]{fromto}{#1 \comp\to #2}{##1 \comp\times ##2}
  \notation[arrow]{fromto}{#1 \comp\to #2}{##1 \comp\to ##2}

  % mapto (lambda etc.)
  %\symdecl[args=Bi]{mapto}
  %\notation[mapsto]{mapto}{#1 \comp\mapsto #2}{#1 \comp, #2}
  %\notation[lambda]{mapto}{\comp\lambda #1 \comp.\; #2}{#1 \comp, #2}
  %\notation[lambdau]{mapto}{\comp\lambda_{#1} \comp.\; #2}{#1 \comp, #2}

  % function/operator application
  \symdecl[args=ia]{apply}
  \notation[prec=0;0x\infprec,parens]{apply}{#1 \comp( #2 \comp)}{##1 \comp, ##2}
  \notation[prec=0;0x\infprec,lambda]{apply}{#1 \; #2 }{##1 \; ##2}

  % ``type'' of all collections (sets,classes,types,kinds)
  \symdecl{collection}
  \notation[U]{collection}{\comp{\mathcal{U}}}
  \notation[set]{collection}{\comp{\textsf{Set}}}

  % sequences
  \symdecl[args=1]{seqtype}
  \notation[kleene]{seqtype}{#1^{\comp\ast}}

  \symdef[args=2,li,prec=nobrackets]{sequence-index}{{#1}_{#2}}
  \notation[ui,prec=nobrackets]{sequence-index}{{#1}^{#2}}

  \symdef[args=a,prec=nobrackets]{aseqdots}{#1\comp{,\ellipses}}{##1\comp,##2}
  \symdef[args=ai,prec=nobrackets]{aseqfromto}{#1\comp{,\ellipses,}#2}{##1\comp,##2}
  \symdef[args=aii,prec=nobrackets]{aseqfromtovia}{#1\comp{,\ellipses,}#2\comp{,\ellipses,}#3}{##1\comp,##2}

  % letin (``let'', local definitions, variable substitution)
  \symdecl[args=bii]{letin}
  \notation[let]{letin}{\comp{{\rm let}}\;#1\comp{=}#2\;\comp{{\rm in}}\;#3}
  \notation[subst]{letin}{#3 \comp[ #1 \comp/ #2 \comp]}
  \notation[frac]{letin}{#3 \comp[ \frac{#2}{#1} \comp]}

  % structures
  \symdecl*[args=1]{module-type}
  \notation{module-type}{\mathtt{MOD} #1}
  \symdecl[name=mathematical-structure,args=a]{mathstruct} % TODO
  \notation[angle,prec=nobrackets]{mathstruct}{\comp\langle #1 \comp\rangle}{##1 \comp, ##2}

}
  \ExplSyntaxOn
  \stex_add_to_current_module:n{
    \let\nappa\apply
    \def\nappli#1#2#3#4{\apply{#1}{\naseqli{#2}{#3}{#4}}}
    \def\nappui#1#2#3#4{\apply{#1}{\nasequi{#2}{#3}{#4}}}
    \def\livar{\csname sequence-index\endcsname[li]}
    \def\uivar{\csname sequence-index\endcsname[ui]}
    \def\naseqli#1#2#3{\aseqfromto{\livar{#1}{#2}}{\livar{#1}{#3}}}
    \def\nasequi#1#2#3{\aseqfromto{\uivar{#1}{#2}}{\uivar{#1}{#3}}}
    \def\nappe#1#2#3{\apply{#1}{\aseqfromto{#2}{#3}}}
  }
\_@@_end_module:
\endgroup
%    \end{macrocode}
%
%
%    \begin{macrocode}
%</package>
%    \end{macrocode}
%
% \end{implementation}
%
% \PrintIndex

% \endinput
% Local Variables:
% mode: doctex
% TeX-master: t
% End:

% \fi
%
% \begin{documentation}\label{pkg:metatheory:doc}
%
% The default meta theory for an \sTeX module. Contains
% symbols so ubiquitous, that it is virtually impossible
% to describe any flexiformal content without them, or
% that are required to annotate even the most primitive symbols
% with meaningful (foundation-independent) ``type''-annotations,
% or required for basic structuring principles (theorems, definitions).
%
% Foundations should ideally instantiate these symbols
% with their formal counterparts, e.g. |isa| corresponds
% to a typing operation in typed setting, or the $\in$-operator
% in set-theoretic contexts; |bind| corresponds to a universal
% quantifier in ($n$th-order) logic, or a $\Pi$ in dependent type
% theories.
%
% \section{Symbols}\label{pkg:metatheory:symbols}
%
% \end{documentation}
%
% \begin{implementation}\label{pkg:metatheory:impl}
%
% \section{\sTeX-Metatheory Implementation}
%
%    \begin{macrocode}
%<*package>
%<@@=stex_modules>

%%%%%%%%%%%%%   metatheory.dtx   %%%%%%%%%%%%%

\str_const:Nn \c_stex_metatheory_ns_str {http://mathhub.info/sTeX}
\begingroup
\stex_module_setup:nn{
  ns=\c_stex_metatheory_ns_str,
  meta=NONE
}{Metatheory}
\stex_reactivate_macro:N \symdecl
\stex_reactivate_macro:N \notation
\stex_reactivate_macro:N \symdef
\ExplSyntaxOff
\csname stex_suppress_html:n\endcsname{
  % is-a (a:A, a \in A, a is an A, etc.)
  \symdecl[args=ai]{isa}
  \notation[typed]{isa}{#1 \comp{:} #2}{##1 \comp, ##2}
  \notation[in]{isa}{#1 \comp\in #2}{##1 \comp, ##2}
  \notation[pred]{isa}{#2\comp(#1 \comp)}{##1 \comp, ##2}

  % bind (\forall, \Pi, \lambda etc.)
  \symdecl[args=Bi]{bind}
  \notation[forall]{bind}{\comp\forall #1.\;#2}{##1 \comp, ##2}
  \notation[Pi]{bind}{\comp\prod_{#1}#2}{##1 \comp, ##2}
  \notation[depfun]{bind}{\comp( #1 \comp{)\;\to\;} #2}{##1 \comp, ##2}

  % dummy variable
  \symdecl{dummyvar}
  \notation[underscore]{dummyvar}{\comp\_}
  \notation[dot]{dummyvar}{\comp\cdot}
  \notation[dash]{dummyvar}{\comp{{\rm --}}}

  %fromto (function space, Hom-set, implication etc.)
  \symdecl[args=ai]{fromto}
  \notation[xarrow]{fromto}{#1 \comp\to #2}{##1 \comp\times ##2}
  \notation[arrow]{fromto}{#1 \comp\to #2}{##1 \comp\to ##2}

  % mapto (lambda etc.)
  %\symdecl[args=Bi]{mapto}
  %\notation[mapsto]{mapto}{#1 \comp\mapsto #2}{#1 \comp, #2}
  %\notation[lambda]{mapto}{\comp\lambda #1 \comp.\; #2}{#1 \comp, #2}
  %\notation[lambdau]{mapto}{\comp\lambda_{#1} \comp.\; #2}{#1 \comp, #2}

  % function/operator application
  \symdecl[args=ia]{apply}
  \notation[prec=0;0x\infprec,parens]{apply}{#1 \comp( #2 \comp)}{##1 \comp, ##2}
  \notation[prec=0;0x\infprec,lambda]{apply}{#1 \; #2 }{##1 \; ##2}

  % ``type'' of all collections (sets,classes,types,kinds)
  \symdecl{collection}
  \notation[U]{collection}{\comp{\mathcal{U}}}
  \notation[set]{collection}{\comp{\textsf{Set}}}

  % sequences
  \symdecl[args=1]{seqtype}
  \notation[kleene]{seqtype}{#1^{\comp\ast}}

  \symdef[args=2,li,prec=nobrackets]{sequence-index}{{#1}_{#2}}
  \notation[ui,prec=nobrackets]{sequence-index}{{#1}^{#2}}

  \symdef[args=a,prec=nobrackets]{aseqdots}{#1\comp{,\ellipses}}{##1\comp,##2}
  \symdef[args=ai,prec=nobrackets]{aseqfromto}{#1\comp{,\ellipses,}#2}{##1\comp,##2}
  \symdef[args=aii,prec=nobrackets]{aseqfromtovia}{#1\comp{,\ellipses,}#2\comp{,\ellipses,}#3}{##1\comp,##2}

  % letin (``let'', local definitions, variable substitution)
  \symdecl[args=bii]{letin}
  \notation[let]{letin}{\comp{{\rm let}}\;#1\comp{=}#2\;\comp{{\rm in}}\;#3}
  \notation[subst]{letin}{#3 \comp[ #1 \comp/ #2 \comp]}
  \notation[frac]{letin}{#3 \comp[ \frac{#2}{#1} \comp]}

  % structures
  \symdecl*[args=1]{module-type}
  \notation{module-type}{\mathtt{MOD} #1}
  \symdecl[name=mathematical-structure,args=a]{mathstruct} % TODO
  \notation[angle,prec=nobrackets]{mathstruct}{\comp\langle #1 \comp\rangle}{##1 \comp, ##2}

}
  \ExplSyntaxOn
  \stex_add_to_current_module:n{
    \let\nappa\apply
    \def\nappli#1#2#3#4{\apply{#1}{\naseqli{#2}{#3}{#4}}}
    \def\nappui#1#2#3#4{\apply{#1}{\nasequi{#2}{#3}{#4}}}
    \def\livar{\csname sequence-index\endcsname[li]}
    \def\uivar{\csname sequence-index\endcsname[ui]}
    \def\naseqli#1#2#3{\aseqfromto{\livar{#1}{#2}}{\livar{#1}{#3}}}
    \def\nasequi#1#2#3{\aseqfromto{\uivar{#1}{#2}}{\uivar{#1}{#3}}}
    \def\nappe#1#2#3{\apply{#1}{\aseqfromto{#2}{#3}}}
  }
\_@@_end_module:
\endgroup
%    \end{macrocode}
%
%
%    \begin{macrocode}
%</package>
%    \end{macrocode}
%
% \end{implementation}
%
% \PrintIndex

% \endinput
% Local Variables:
% mode: doctex
% TeX-master: t
% End:

% \fi
%
% \begin{documentation}\label{pkg:metatheory:doc}
%
% The default meta theory for an \sTeX module. Contains
% symbols so ubiquitous, that it is virtually impossible
% to describe any flexiformal content without them, or
% that are required to annotate even the most primitive symbols
% with meaningful (foundation-independent) ``type''-annotations,
% or required for basic structuring principles (theorems, definitions).
%
% Foundations should ideally instantiate these symbols
% with their formal counterparts, e.g. |isa| corresponds
% to a typing operation in typed setting, or the $\in$-operator
% in set-theoretic contexts; |bind| corresponds to a universal
% quantifier in ($n$th-order) logic, or a $\Pi$ in dependent type
% theories.
%
% \section{Symbols}\label{pkg:metatheory:symbols}
%
% \end{documentation}
%
% \begin{implementation}\label{pkg:metatheory:impl}
%
% \section{\sTeX-Metatheory Implementation}
%
%    \begin{macrocode}
%<*package>
%<@@=stex_modules>

%%%%%%%%%%%%%   metatheory.dtx   %%%%%%%%%%%%%

\str_const:Nn \c_stex_metatheory_ns_str {http://mathhub.info/sTeX}
\begingroup
\stex_module_setup:nn{
  ns=\c_stex_metatheory_ns_str,
  meta=NONE
}{Metatheory}
\stex_reactivate_macro:N \symdecl
\stex_reactivate_macro:N \notation
\stex_reactivate_macro:N \symdef
\ExplSyntaxOff
\csname stex_suppress_html:n\endcsname{
  % is-a (a:A, a \in A, a is an A, etc.)
  \symdecl[args=ai]{isa}
  \notation[typed]{isa}{#1 \comp{:} #2}{##1 \comp, ##2}
  \notation[in]{isa}{#1 \comp\in #2}{##1 \comp, ##2}
  \notation[pred]{isa}{#2\comp(#1 \comp)}{##1 \comp, ##2}

  % bind (\forall, \Pi, \lambda etc.)
  \symdecl[args=Bi]{bind}
  \notation[forall]{bind}{\comp\forall #1.\;#2}{##1 \comp, ##2}
  \notation[Pi]{bind}{\comp\prod_{#1}#2}{##1 \comp, ##2}
  \notation[depfun]{bind}{\comp( #1 \comp{)\;\to\;} #2}{##1 \comp, ##2}

  % dummy variable
  \symdecl{dummyvar}
  \notation[underscore]{dummyvar}{\comp\_}
  \notation[dot]{dummyvar}{\comp\cdot}
  \notation[dash]{dummyvar}{\comp{{\rm --}}}

  %fromto (function space, Hom-set, implication etc.)
  \symdecl[args=ai]{fromto}
  \notation[xarrow]{fromto}{#1 \comp\to #2}{##1 \comp\times ##2}
  \notation[arrow]{fromto}{#1 \comp\to #2}{##1 \comp\to ##2}

  % mapto (lambda etc.)
  %\symdecl[args=Bi]{mapto}
  %\notation[mapsto]{mapto}{#1 \comp\mapsto #2}{#1 \comp, #2}
  %\notation[lambda]{mapto}{\comp\lambda #1 \comp.\; #2}{#1 \comp, #2}
  %\notation[lambdau]{mapto}{\comp\lambda_{#1} \comp.\; #2}{#1 \comp, #2}

  % function/operator application
  \symdecl[args=ia]{apply}
  \notation[prec=0;0x\infprec,parens]{apply}{#1 \comp( #2 \comp)}{##1 \comp, ##2}
  \notation[prec=0;0x\infprec,lambda]{apply}{#1 \; #2 }{##1 \; ##2}

  % ``type'' of all collections (sets,classes,types,kinds)
  \symdecl{collection}
  \notation[U]{collection}{\comp{\mathcal{U}}}
  \notation[set]{collection}{\comp{\textsf{Set}}}

  % sequences
  \symdecl[args=1]{seqtype}
  \notation[kleene]{seqtype}{#1^{\comp\ast}}

  \symdef[args=2,li,prec=nobrackets]{sequence-index}{{#1}_{#2}}
  \notation[ui,prec=nobrackets]{sequence-index}{{#1}^{#2}}

  \symdef[args=a,prec=nobrackets]{aseqdots}{#1\comp{,\ellipses}}{##1\comp,##2}
  \symdef[args=ai,prec=nobrackets]{aseqfromto}{#1\comp{,\ellipses,}#2}{##1\comp,##2}
  \symdef[args=aii,prec=nobrackets]{aseqfromtovia}{#1\comp{,\ellipses,}#2\comp{,\ellipses,}#3}{##1\comp,##2}

  % letin (``let'', local definitions, variable substitution)
  \symdecl[args=bii]{letin}
  \notation[let]{letin}{\comp{{\rm let}}\;#1\comp{=}#2\;\comp{{\rm in}}\;#3}
  \notation[subst]{letin}{#3 \comp[ #1 \comp/ #2 \comp]}
  \notation[frac]{letin}{#3 \comp[ \frac{#2}{#1} \comp]}

  % structures
  \symdecl*[args=1]{module-type}
  \notation{module-type}{\mathtt{MOD} #1}
  \symdecl[name=mathematical-structure,args=a]{mathstruct} % TODO
  \notation[angle,prec=nobrackets]{mathstruct}{\comp\langle #1 \comp\rangle}{##1 \comp, ##2}

}
  \ExplSyntaxOn
  \stex_add_to_current_module:n{
    \let\nappa\apply
    \def\nappli#1#2#3#4{\apply{#1}{\naseqli{#2}{#3}{#4}}}
    \def\nappui#1#2#3#4{\apply{#1}{\nasequi{#2}{#3}{#4}}}
    \def\livar{\csname sequence-index\endcsname[li]}
    \def\uivar{\csname sequence-index\endcsname[ui]}
    \def\naseqli#1#2#3{\aseqfromto{\livar{#1}{#2}}{\livar{#1}{#3}}}
    \def\nasequi#1#2#3{\aseqfromto{\uivar{#1}{#2}}{\uivar{#1}{#3}}}
    \def\nappe#1#2#3{\apply{#1}{\aseqfromto{#2}{#3}}}
  }
\_@@_end_module:
\endgroup
%    \end{macrocode}
%
%
%    \begin{macrocode}
%</package>
%    \end{macrocode}
%
% \end{implementation}
%
% \PrintIndex

% \endinput
% Local Variables:
% mode: doctex
% TeX-master: t
% End:

  \end{omgroup}
\end{omgroup}

\begin{omgroup}{\sTeX Statements (Definitions, Theorems, Examples, ...)}
  % \iffalse meta-comment
% An Infrastructure for Semantic Macros and Module Scoping
% Copyright (c) 2019 Michael Kohlhase, all rights reserved
%                this file is released under the
%                LaTeX Project Public License (LPPL)
% 
% The original of this file is in the public repository at 
% http://github.com/sLaTeX/sTeX/
%
% TODO update copyright  
%
%<*driver>
\providecommand\bibfolder{../../lib/bib}
\RequirePackage{paralist}
\documentclass[full,kernel]{l3doc}
\usepackage[dvipsnames]{xcolor}
\usepackage[utf8]{inputenc}
\usepackage[T1]{fontenc}
\RequirePackage{morewrites}
\RequirePackage{tikzinput}
\usetikzlibrary{fit}

\usepackage[debug=all,lang=en, mathhub=./tests]{stex}
\usepackage{url,array,float,textcomp}
\usepackage[show]{ed}
\usepackage[hyperref=auto,style=alphabetic]{biblatex}
\addbibresource{\bibfolder/kwarcpubs.bib}
\addbibresource{\bibfolder/extpubs.bib}
\addbibresource{\bibfolder/kwarccrossrefs.bib}
\addbibresource{\bibfolder/extcrossrefs.bib}
\usepackage{amssymb}
\usepackage{amsfonts}
\usepackage{xspace}
\usepackage{hyperref}

\makeindex
\floatstyle{boxed}
\newfloat{exfig}{thp}{lop}
\floatname{exfig}{Example}

\usepackage{stex-tests}

\MakeShortVerb{\|}

\def\scsys#1{{{\sc #1}}\index{#1@{\sc #1}}\xspace}
\def\mmt{\textsc{Mmt}\xspace}
\def\xml{\scsys{Xml}}
\def\mathml{\scsys{MathML}}
\def\omdoc{\scsys{OMDoc}}
\def\openmath{\scsys{OpenMath}}
\def\latexml{\scsys{LaTeXML}}
\def\perl{\scsys{Perl}}
\def\cmathml{Content-{\sc MathML}\index{Content {\sc MathML}}\index{MathML@{\sc MathML}!content}}
\def\activemath{\scsys{ActiveMath}}
\def\twin#1#2{\index{#1!#2}\index{#2!#1}}
\def\twintoo#1#2{{#1 #2}\twin{#1}{#2}}
\def\atwin#1#2#3{\index{#1!#2!#3}\index{#3!#2 (#1)}}
\def\atwintoo#1#2#3{{#1 #2 #3}\atwin{#1}{#2}{#3}}
\def\cT{\mathcal{T}}\def\cD{\mathcal{D}}

\def\fileversion{3.0}
\def\filedate{\today}

\RequirePackage{pdfcomment}

\ExplSyntaxOn\makeatletter
\cs_set_protected:Npn \@comp #1 #2 {
  \pdftooltip {
    \textcolor{blue}{#1}
  } { #2 }
}

\cs_set_protected:Npn \@defemph #1 #2 {
  \pdftooltip { 
    \textbf{\textcolor{magenta}{#1}}
  } { #2 }
}

\def\__omtext_lec#1{#1}
\cs_new_protected:Npn \lec #1 {
  \strut\hfil\strut\null\hfill\__omtext_lec{#1}
}
\makeatother\ExplSyntaxOff

\makeatletter
\let\@stex@oldcomment\comment
\let\@stex@oldendcomment\endcomment

%\RequirePackage{comment}
\RequirePackage{document-structure}
\RequirePackage[hints,solutions,notes]{problem}
\RequirePackage{hwexam}

\let\comment\@stex@oldcomment
\let\endcomment\@stex@oldendcomment

\newif\ifinfulldoc\infulldocfalse
\makeatother

\def\basedocurl{https://github.com/slatex/sTeX/blob/latex3/doc}
\newcounter{module}

\NewDocumentEnvironment {module}{}{
  \stepcounter{module}
  \textbf{Module \themodule: \smoduletitle}
}{

}
\stexpatchmodule{\begin{module}}{\end{module}}

\def\compemph#1{\textcolor{blue}{#1}}
\def\symrefemph#1{\textcolor{green}{#1}}

\RequirePackage{pdfcomment}
\makeatletter
\protected\def\compemph@uri#1#2{%
  \pdftooltip{%
    \srefsymuri{#2}{\compemph{#1}}%
  }{%
    URI: \detokenize{#2}%
  }%
}
\protected\def\symrefemph@uri#1#2{%
  \pdftooltip{%
    \srefsymuri{#2}{\symrefemph{#1}}%
  }{%
    URI: \detokenize{#2}%
  }%
}
\makeatother

\begin{document}
  \DocInput{\jobname.dtx}
\end{document}
%</driver>
% \fi
%
% \title{ \sTeX-Statements
% 	\thanks{Version {\fileversion} (last revised {\filedate})} 
% }
%
% \author{Michael Kohlhase, Dennis Müller\\
% 	FAU Erlangen-Nürnberg\\
% 	\url{http://kwarc.info/}
% }
%
% \maketitle
%
% \begin{documentation}\label{pkg:statements:doc}
%
% Code related to statements, e.g. definitions, theorems
%
% \section{Macros and Environments}\label{pkg:statements:doc:macros}
%
% \begin{environment}{symboldoc}
%    \begin{syntax} \cs{begin}\Arg{symboldoc}\Arg{symbols} \meta{text} \cs{end}\Arg{symboldoc} \end{syntax}
%  Declares \meta{text} to be a (natural language, encyclopaedic) description
% of \Arg{symbols} (a comma separated list of symbol identifiers).
% \end{environment}
%
% \end{documentation}
%
% \begin{implementation}\label{pkg:statements:impl}
%
% \section{\sTeX-Statements Implementation}
%
%    \begin{macrocode}
%<*package>

%%%%%%%%%%%%%   features.dtx   %%%%%%%%%%%%%

\protected\def\ignorespacesandpars{
  \begingroup\catcode13=10\relax
  \@ifnextchar\par{
    \endgroup\expandafter\ignorespacesandpars\@gobble
  }{
    \endgroup
  }
}

%<@@=stex_statements>
%    \end{macrocode}
%
% Warnings and error messages
%
%    \begin{macrocode}

%    \end{macrocode}
% \begin{macro}{\titleemph}
%    \begin{macrocode}
\def\titleemph#1{\textbf{#1}}
%    \end{macrocode}
% \end{macro}
%
% \subsection{Definitions}
%
% \begin{macro}{definiendum}
%    \begin{macrocode}
\keys_define:nn {stex / definiendum }{
  post    .tl_set:N     = \l_@@_definiendum_post_tl,
  root    .str_set_x:N  = \l_@@_definiendum_root_str,
  gfa     .str_set_x:N  = \l_@@_definiendum_gfa_str
}
\cs_new_protected:Nn \_@@_definiendum_args:n {
  \str_clear:N \l_@@_definiendum_root_str
  \tl_clear:N \l_@@_definiendum_post_tl
  \str_clear:N \l_@@_definiendum_gfa_str
  \keys_set:nn { stex / definiendum }{ #1 }
}
\NewDocumentCommand \definiendum { O{} m m} {
  \_@@_definiendum_args:n { #1 }
  \stex_get_symbol:n { #2 }
  \stex_ref_new_sym_target:n \l_stex_get_symbol_uri_str
  \str_if_empty:NTF \l_@@_definiendum_root_str {
    \tl_if_empty:NTF \l_@@_definiendum_post_tl {
      \tl_set:Nn \l_tmpa_tl { #3 }
    } {
      \str_set:Nx \l_@@_definiendum_root_str { #3 }
      \tl_set:Nn \l_tmpa_tl {
        \l_@@_definiendum_root_str\l_@@_definiendum_post_tl
       }
    }
  } {
    \tl_set:Nn \l_tmpa_tl { #3 }
  }

  % TODO root
  \rustex_if:TF {
    \stex_annotate:nnn { definiendum } { \l_stex_get_symbol_uri_str } { \l_tmpa_tl }
  } {
    \exp_args:Nnx \defemph@uri { \l_tmpa_tl } { \l_stex_get_symbol_uri_str }
  }
}
\stex_deactivate_macro:Nn \definiendum {definition~environments}
%    \end{macrocode}
% \end{macro}
%
% \begin{macro}{definame}
%    \begin{macrocode}
\NewDocumentCommand \definame { O{} m } {
  \_@@_definiendum_args:n { #1 }
  % TODO: root
  \stex_get_symbol:n { #2 }
  \stex_ref_new_sym_target:n \l_stex_get_symbol_uri_str
  \str_set:Nx \l_tmpa_str {
    \prop_item:cn { g_stex_symdecl_ \l_stex_get_symbol_uri_str _prop } { name }
  }
  \exp_args:NNno \str_replace_all:Nnn \l_tmpa_str {-} {~}
  \rustex_if:TF {
    \stex_annotate:nnn { definiendum } { \l_stex_get_symbol_uri_str } { 
      \l_tmpa_str\l_@@_definiendum_post_tl
      }
  } {
    \defemph@uri {
      \l_tmpa_str\l_@@_definiendum_post_tl
    } { \l_stex_get_symbol_uri_str }
  }
}
\stex_deactivate_macro:Nn \definame {definition~environments}
%    \end{macrocode}
% \end{macro}
%
% \begin{environment}{sdefinition}
%    \begin{macrocode}

\keys_define:nn {stex / sdefinition }{
  type    .str_set_x:N  = \sdefinitiontype,
  id      .str_set_x:N  = \sdefinitionid,
  title   .tl_set:N     = \sdefinitiontitle
}
\cs_new_protected:Nn \_@@_sdefinition_args:n {
  \str_clear:N \sdefinitiontype
  \str_clear:N \sdefinitionid
  \tl_clear:N \sdefinitiontitle
  \keys_set:nn { stex / sdefinition }{ #1 }
}

\NewDocumentEnvironment{sdefinition}{O{}}{
  \_@@_sdefinition_args:n{ #1 }
  \stex_reactivate_macro:N \definiendum
  \stex_reactivate_macro:N \definame
  \stex_ref_new_doc_target:n \sdefinitionid
  \stex_smsmode_set_codes:
  \clist_set:No \l_tmpa_clist \sdefinitiontype
  \tl_clear:N \l_tmpa_tl
  \clist_map_inline:Nn \l_tmpa_clist {
    \tl_if_exist:cT {_@@_sdefinition_##1_start:}{
      \tl_set:Nn \l_tmpa_tl {\use:c{_@@_sdefinition_##1_start:}}
    }
  }
  \tl_if_empty:NTF \l_tmpa_tl {
    \_@@_sdefinition_start:
  }{
    \l_tmpa_tl
  }
  \stex_if_smsmode:F {
    \exp_args:Nnnx
    \begin{stex_annotate_env}{definition}{}
    \str_if_empty:NF \sdefinitiontype {
      \stex_annotate_invisible:nnn{type}{\sdefinitiontype}{}
    }
  }
}{
  \stex_if_smsmode:F {
    \end{stex_annotate_env}
  }
  \clist_set:No \l_tmpa_clist \sdefinitiontype
  \tl_clear:N \l_tmpa_tl
  \clist_map_inline:Nn \l_tmpa_clist {
    \tl_if_exist:cT {_@@_sdefinition_##1_end:}{
      \tl_set:Nn \l_tmpa_tl {\use:c{_@@_sdefinition_##1_end:}}
    }
  }
  \tl_if_empty:NTF \l_tmpa_tl {
    \_@@_sdefinition_end:
  }{
    \l_tmpa_tl
  }
}
%    \end{macrocode}
% \end{environment}
%
% \begin{macro}{\stexpatchdefinition}
%    \begin{macrocode}
\cs_new_protected:Nn \_@@_sdefinition_start: {
  \titleemph{Definition\tl_if_empty:NF \sdefinitiontitle {
    ~(\sdefinitiontitle)
  }~}
}
\cs_new_protected:Nn \_@@_sdefinition_end: {}

\newcommand\stexpatchdefinition[3][] {
    \str_if_empty:nTF{#1}{
      \tl_set:Nn \_@@_sdefinition_start: { #2 }
      \tl_set:Nn \_@@_sdefinition_end: { #3 }
    }{
      \exp_after:wN \tl_set:Nn \csname _@@_sdefinition_#1_start:\endcsname{ #2 }
      \exp_after:wN \tl_set:Nn \csname _@@_sdefinition_#1_end:\endcsname{ #3 }
    }
}
%    \end{macrocode}
% \end{macro}
%
% \begin{macro}{\inlinedef}
% inline:
%    \begin{macrocode}
\NewDocumentCommand \inlinedef { m } {
  \begingroup
  \stex_reactivate_macro:N \definiendum
  \stex_reactivate_macro:N \definame
  \stex_ref_new_doc_target:n{}
  #1
  \endgroup
}
%    \end{macrocode}
% \end{macro}
%
% \subsection{Assertions}
%
% \begin{environment}{sassertion}
%    \begin{macrocode}

\keys_define:nn {stex / sassertion }{
  type    .str_set_x:N  = \sassertiontype,
  id      .str_set_x:N  = \sassertionid,
  title   .tl_set:N     = \sassertiontitle
}
\cs_new_protected:Nn \_@@_sassertion_args:n {
  \str_clear:N \sassertiontype
  \str_clear:N \sassertionid
  \tl_clear:N \sassertiontitle
  \keys_set:nn { stex / sassertion }{ #1 }
}

\NewDocumentEnvironment{sassertion}{O{}}{
  \_@@_sassertion_args:n{ #1 }
  \stex_ref_new_doc_target:n \sassertionid
  \stex_smsmode_set_codes:
  \clist_set:No \l_tmpa_clist \sassertiontype
  \tl_clear:N \l_tmpa_tl
  \clist_map_inline:Nn \l_tmpa_clist {
    \tl_if_exist:cT {_@@_sassertion_##1_start:}{
      \tl_set:Nn \l_tmpa_tl {\use:c{_@@_sassertion_##1_start:}}
    }
  }
  \tl_if_empty:NTF \l_tmpa_tl {
    \_@@_sassertion_start:
  }{
    \l_tmpa_tl
  }
  \stex_if_smsmode:F {
    \exp_args:Nnnx
    \begin{stex_annotate_env}{assertion}{}
    \str_if_empty:NF \sassertiontype {
      \stex_annotate_invisible:nnn{type}{\sassertiontype}{}
    }
  }
}{
  \stex_if_smsmode:F {
    \end{stex_annotate_env}
  }
  \clist_set:No \l_tmpa_clist \sassertiontype
  \tl_clear:N \l_tmpa_tl
  \clist_map_inline:Nn \l_tmpa_clist {
    \tl_if_exist:cT {_@@_sassertion_##1_end:}{
      \tl_set:Nn \l_tmpa_tl {\use:c{_@@_sassertion_##1_end:}}
    }
  }
  \tl_if_empty:NTF \l_tmpa_tl {
    \_@@_sassertion_end:
  }{
    \l_tmpa_tl
  }
}
%    \end{macrocode}
% \end{environment}
%
% \begin{macro}{\stexpatchassertion}
%    \begin{macrocode}

\cs_new_protected:Nn \_@@_sassertion_start: {
  \titleemph{Assertion~\tl_if_empty:NF \sassertiontitle {
    (\sassertiontitle)
  }~}
}
\cs_new_protected:Nn \_@@_sassertion_end: {}

\newcommand\stexpatchassertion[3][] {
    \str_if_empty:nTF{#1}{
      \tl_set:Nn \_@@_sassertion_start: { #2 }
      \tl_set:Nn \_@@_sassertion_end: { #3 }
    }{
      \exp_after:wN \tl_set:Nn \csname _@@_sassertion_#1_start:\endcsname{ #2 }
      \exp_after:wN \tl_set:Nn \csname _@@_sassertion_#1_end:\endcsname{ #3 }
    }
}
%    \end{macrocode}
% \end{macro}
%
% \begin{macro}{\inlineass}
% inline:
%    \begin{macrocode}
\NewDocumentCommand \inlineass { m } {
  \begingroup
  \stex_ref_new_doc_target:n{}
  #1
  \endgroup
}
%    \end{macrocode}
% \end{macro}
%
% \subsection{Examples}
%
% \begin{environment}{sexample}
%    \begin{macrocode}

\keys_define:nn {stex / sexample }{
  type    .str_set_x:N  = \exampletype,
  id      .str_set_x:N  = \sexampleid,
  title   .tl_set:N  = \sexampletitle,
  for     .clist_set:N   = \sexamplefor,
}
\cs_new_protected:Nn \_@@_sexample_args:n {
  \str_clear:N \sexampletype
  \str_clear:N \sexampleid
  \tl_clear:N \sexampletitle
  \clist_clear:N \sexamplefor
  \keys_set:nn { stex / sexample }{ #1 }
}

\NewDocumentEnvironment{sexample}{O{}}{
  \_@@_sexample_args:n{ #1 }
  \stex_ref_new_doc_target:n \sexampleid
  \stex_smsmode_set_codes:
  \clist_set:No \l_tmpa_clist \sexampletype
  \tl_clear:N \l_tmpa_tl
  \clist_map_inline:Nn \l_tmpa_clist {
    \tl_if_exist:cT {_@@_sexample_##1_start:}{
      \tl_set:Nn \l_tmpa_tl {\use:c{_@@_sexample_##1_start:}}
    }
  }
  \tl_if_empty:NTF \l_tmpa_tl {
    \_@@_sexample_start:
  }{
    \l_tmpa_tl
  }
  \stex_if_smsmode:F {
    \seq_clear:N \l_tmpa_seq
    \clist_map_inline:Nn \sexamplefor {
      \str_if_eq:nnF{ ##1 }{}{
        \stex_get_symbol:n { ##1 }
        \exp_args:NNo \seq_put_right:Nn \l_tmpa_seq {
          \l_stex_get_symbol_uri_str
        }
      }
    }
    \exp_args:Nnnx
    \begin{stex_annotate_env}{example}{\seq_use:Nn \l_tmpa_seq {,}}
    \str_if_empty:NF \sexampletype {
      \stex_annotate_invisible:nnn{type}{\sexampletype}{}
    }
  }
}{
  \stex_if_smsmode:F {
    \end{stex_annotate_env}
  }
  \clist_set:No \l_tmpa_clist \sexampletype
  \tl_clear:N \l_tmpa_tl
  \clist_map_inline:Nn \l_tmpa_clist {
    \tl_if_exist:cT {_@@_sexample_##1_end:}{
      \tl_set:Nn \l_tmpa_tl {\use:c{_@@_sexample_##1_end:}}
    }
  }
  \tl_if_empty:NTF \l_tmpa_tl {
    \_@@_sexample_end:
  }{
    \l_tmpa_tl
  }
}
%    \end{macrocode}
% \end{environment}
%
% \begin{macro}{\stexpatchexample}
%    \begin{macrocode}

\cs_new_protected:Nn \_@@_sexample_start: {
  \titleemph{Example~\tl_if_empty:NF \sexampletitle {
    (\sexampletitle)
  }~}
}
\cs_new_protected:Nn \_@@_sexample_end: {}

\newcommand\stexpatchexample[3][] {
    \str_if_empty:nTF{#1}{
      \tl_set:Nn \_@@_sexample_start: { #2 }
      \tl_set:Nn \_@@_sexample_end: { #3 }
    }{
      \exp_after:wN \tl_set:Nn \csname _@@_sexample_#1_start:\endcsname{ #2 }
      \exp_after:wN \tl_set:Nn \csname _@@_sexample_#1_end:\endcsname{ #3 }
    }
}
%    \end{macrocode}
% \end{macro}
%
% \begin{macro}{\inlineex}
% inline:
%    \begin{macrocode}
\NewDocumentCommand \inlineex { m } {
  \begingroup
  \stex_ref_new_doc_target:n{}
  #1
  \endgroup
}
%    \end{macrocode}
% \end{macro}
%
% \subsection{Logical Paragraphs}
%
% \begin{environment}{sparagraph}
%    \begin{macrocode}
\keys_define:nn { stex / sparagraph} {
  id      .str_set_x:N   = \sparagraphid , 
  title   .tl_set:N      = \l_stex_sparagraph_title_tl , 
  type    .str_set_x:N   = \sparagraphtype ,
  for     .str_set_x:N   = \sparagraphfor ,
  from    .tl_set_x:N    = \sparagraphfrom ,
  start   .tl_set:N      = \l_stex_sparagraph_start_tl ,
}

\cs_new_protected:Nn \stex_sparagraph_args:n {
  \tl_clear:N \l_stex_sparagraph_title_tl
  \tl_clear:N \sparagraphfrom
  \tl_clear:N \l_stex_sparagraph_start_tl
  \str_clear:N \sparagraphid
  \str_clear:N \sparagraphtype
  \str_clear:N \sparagraphfor
  \keys_set:nn { stex / sparagraph }{ #1 }
}
\newif\if@in@omtext\@in@omtextfalse

\NewDocumentEnvironment {sparagraph} { O{} } {
  \stex_sparagraph_args:n { #1 }
  \tl_if_empty:NTF \l_stex_sparagraph_start_tl {
    \tl_set_eq:NN \sparagraphtitle \l_stex_sparagraph_title_tl
  }{
    \tl_set_eq:NN \sparagraphtitle \l_stex_sparagraph_start_tl
  }
  \@in@omtexttrue
  \stex_ref_new_doc_target:n \sparagraphid
  \stex_smsmode_set_codes:
  \clist_set:No \l_tmpa_clist \sparagraphtype
  \tl_clear:N \l_tmpa_tl
  \clist_map_inline:Nn \l_tmpa_clist {
    \tl_if_exist:cT {_@@_sparagraph_##1_start:}{
      \tl_set:Nn \l_tmpa_tl {\use:c{_@@_sparagraph_##1_start:}}
    }
  }
  \tl_if_empty:NTF \l_tmpa_tl {
    \_@@_sparagraph_start:
  }{
    \l_tmpa_tl
  }
  \stex_if_smsmode:F {
    \exp_args:Nnnx
    \begin{stex_annotate_env}{paragraph}{}
    \str_if_empty:NF \sparagraphtype {
      \stex_annotate_invisible:nnn{type}{\sparagraphtype}{}
    }
  }
  \ignorespacesandpars
}{}
%    \end{macrocode}
% \end{environment}
%
% \begin{macro}{\stexpatchparagraph}
%    \begin{macrocode}

\cs_new_protected:Nn \_@@_sparagraph_start: {  
  \tl_if_empty:NTF \l_stex_sparagraph_start_tl {
    \tl_if_empty:NF \l_stex_sparagraph_title_tl {
      \titleemph{\l_stex_sparagraph_title_tl}:~
    }
  }{
    \titleemph{\l_stex_sparagraph_start_tl}~
  }
}
\cs_new_protected:Nn \_@@_sparagraph_end: {}

\newcommand\stexpatchparagraph[3][] {
    \str_if_empty:nTF{#1}{
      \tl_set:Nn \_@@_sparagraph_start: { #2 }
      \tl_set:Nn \_@@_sparagraph_end: { #3 }
    }{
      \exp_after:wN \tl_set:Nn \csname _@@_sparagraph_#1_start:\endcsname{ #2 }
      \exp_after:wN \tl_set:Nn \csname _@@_sparagraph_#1_end:\endcsname{ #3 }
    }
}
%    \end{macrocode}
% \end{macro}
%
%
%
%
%
%
%
%
%
%
%
%
%
%
%
%
%
%
%
% \begin{environment}{symboldoc}
%    \begin{macrocode}
\NewDocumentEnvironment{symboldoc}{ m }{
  \seq_set_split:Nnn \l_tmpa_seq , { #1 }
  \seq_clear:N \l_tmpb_seq
  \seq_map_inline:Nn \l_tmpa_seq {
    \str_if_eq:nnF{ ##1 }{}{
      \stex_get_symbol:n { ##1 }
      \exp_args:NNo \seq_put_right:Nn \l_tmpb_seq {
        \l_stex_get_symbol_uri_str
      }
    }
  }
  \par
  \exp_args:Nnnx
  \begin{stex_annotate_env}{symboldoc}{\seq_use:Nn \l_tmpb_seq {,}}
}{
  \end{stex_annotate_env}
}
%    \end{macrocode}
% \end{environment}
%
%
%    \begin{macrocode}
%</package>
%    \end{macrocode}
%
% \end{implementation}
%
% \PrintIndex

% \endinput
% Local Variables:
% mode: doctex
% TeX-master: t
% End:

  \textcolor{red}{TODO: sproofs documentation}
\end{omgroup}

\begin{omgroup}{Additional Packages}
  \NeedsTeXFormat{LaTeX2e}[1999/12/01]
\ProvidesPackage{tikzinput}[2012/09/26 v1.0 including tikz source or images]
\newif\iftikzinput@image\tikzinput@imagefalse
\DeclareOption{image}{\tikzinput@imagetrue}
\DeclareOption*{\PassOptionsToPackage{\CurrentOption}{tikz}}
\ProcessOptions
\iftikzinput@image
\RequirePackage{graphicx}
\newcommand\tikzinput[2][]{\message{image!!!}\includegraphics[#1]{#2}}
\providecommand\usetikzlibrary[1]{}
\else
\RequirePackage{etoolbox}
\RequirePackage{tikz}
\RequirePackage{standalone}
% \def\tizki{}
% \define@key{scale}{\edef\tikzi{\tikzi,scale=#1}}
% \define@key{xscale}{\edef\tikzi{\tikzi,xscale=#1}}
% \define@key{yscale}{\edef\tikzi{\tikzi,yscale=#1}}
\newcommand\tikzinput[2][]{\input{#2}}
\newcommand\ctikzinput[2][]{\begin{center}\input{#2}\end{center}}
\fi

  \begin{omgroup}{Modular Document Structuring}
    \textcolor{red}{TODO: document-structure documentation}
  \end{omgroup}
  \begin{omgroup}{Slides and Course Notes}
    \input{packages/slides}
  \end{omgroup}
  \begin{omgroup}{Homework, Problems and Exams}
    %%
%% This is file `problem.sty',
%% generated with the docstrip utility.
%%
%% The original source files were:
%%
%% problem.dtx  (with options: `package')
%% 
\ProvidesExplPackage{problem}{2019/03/20}{1.3}{Semantic Markup for Problems}
\RequirePackage{l3keys2e,expl-keystr-compat}

\keys_define:nn { problem / pkg }{
  notes     .default:n    = { true },
  notes     .bool_set:N   = \c__problems_notes_bool,
  gnotes    .default:n    = { true },
  gnotes    .bool_set:N   = \c__problems_gnotes_bool,
  hints     .default:n    = { true },
  hints     .bool_set:N   = \c__problems_hints_bool,
  solutions .default:n    = { true },
  solutions .bool_set:N   = \c__problems_solutions_bool,
  pts       .default:n    = { true },
  pts       .bool_set:N   = \c__problems_pts_bool,
  min       .default:n    = { true },
  min       .bool_set:N   = \c__problems_min_bool,
  boxed     .default:n    = { true },
  boxed     .bool_set:N   = \c__problems_boxed_bool
}
\def\solutionstrue{
  \bool_set_true:N \c__problems_solutions_bool
}
\def\solutionsfalse{
  \bool_set_false:N \c__problems_solutions_bool
}

\ProcessKeysOptions{ problem / pkg }
\RequirePackage{stex-compatibility}
\RequirePackage{comment}
\bool_if:NT \c__problems_boxed_bool { \RequirePackage{mdframed} }
\def\prob@problem@kw{Problem}
\def\prob@solution@kw{Solution}
\def\prob@hint@kw{Hint}
\def\prob@note@kw{Note}
\def\prob@gnote@kw{Grading}
\def\prob@pt@kw{pt}
\def\prob@min@kw{min}
\@ifpackageloaded{babel}{
    \clist_set:Nx \l_tmpa_clist {\bbl@loaded}
    \clist_if_in:NnT \l_tmpa_clist {ngerman}{
      \input{problem-ngerman.ldf}
    }
    \clist_if_in:NnT \l_tmpa_clist {finnish}{
      \input{problem-finnish.ldf}
    }
    \clist_if_in:NnT \l_tmpa_clist {french}{
      \input{problem-french.ldf}
    }
    \clist_if_in:NnT \l_tmpa_clist {russian}{
      \input{problem-russian.ldf}
    }
}{}
\keys_define:nn{ problem / problem }{
  id      .str_set_x:N  = \l__problems_prob_id_str,
  pts     .tl_set:N     = \l__problems_prob_pts_tl,
  min     .tl_set:N     = \l__problems_prob_min_tl,
  title   .tl_set:N     = \l__problems_prob_title_tl,
  refnum  .int_set:N    = \l__problems_prob_refnum_int
}
\cs_new_protected:Nn \__problems_prob_args:n {
  \str_clear:N \l__problems_prob_id_str
  \tl_clear:N \l__problems_prob_pts_tl
  \tl_clear:N \l__problems_prob_min_tl
  \tl_clear:N \l__problems_prob_title_tl
  \int_zero_new:N \l__problems_prob_refnum_int
  \keys_set:nn { problem / problem }{ #1 }
}
\newcounter{problem}
\newcommand\numberproblemsin[1]{\@addtoreset{problem}{#1}}
\newcommand\prob@label[1]{#1}
\newcommand\prob@number{
  \int_if_exist:NTF \l__problems_inclprob_refnum_int {
    \prob@label{\int_use:N \l__problems_inclprob_refnum_int }
  }{
    \int_if_exist:NTF \l__problems_prob_refnum_int {
      \prob@label{\int_use:N \l__problems_prob_refnum_int }
    }{
        \prob@label\theproblem
    }
  }
}
\newcommand\prob@title[3]{%
  \tl_if_exist:NTF \l__problems_inclprob_title_tl {
    #2 \l__problems_inclprob_title_tl #3
  }{
    \tl_if_exist:NTF \l__problems_prob_title_tl {
      #2 \l__problems_prob_title_tl #3
    }{
      #1
    }
  }
}
\def\prob@heading{
  \prob@problem@kw~\prob@number\prob@title{ }{ (}{)\strut}
  %\sref@label@id{\prob@problem@kw~\prob@number}{}
}
\newenvironment{problem}[1][]{
  \__problems_prob_args:n{#1}%\sref@target%
  \@in@omtexttrue% we are in a statement (for inline definitions)
  \stepcounter{problem}\record@problem
  \def\current@section@level{\prob@problem@kw}
  \par\noindent\textbf\prob@heading\show@pts\show@min\\\ignorespacesandpars
}%
{\smallskip}
\bool_if:NT \c__problems_boxed_bool {
  \surroundwithmdframed{problem}
}
\def\record@problem{
  \protected@write\@auxout{}
  {
    \string\@problem{\prob@number}
    {
      \tl_if_exist:NTF \l__problems_inclprob_pts_tl {
        \l__problems_inclprob_pts_tl
      }{
        \l__problems_prob_pts_tl
      }
    }%
    {
      \tl_if_exist:NTF \l__problems_inclprob_min_tl {
        \l__problems_inclprob_min_tl
      }{
        \l__problems_prob_min_tl
      }
    }
  }
}
\def\@problem#1#2#3{}
\keys_define:nn { problem / solution }{
  id            .str_set_x:N  = \l__problems_solution_id_str ,
  for           .tl_set:N     = \l__problems_solution_for_tl ,
  height        .dim_set:N    = \l__problems_solution_height_dim ,
  creators      .clist_set:N  = \l__problems_solution_creators_clist ,
  contributors  .clist_set:N  = \l__problems_solution_contributors_clist ,
  srccite       .tl_set:N     = \l__problems_solution_srccite_tl
}
\cs_new_protected:Nn \__problems_solution_args:n {
  \str_clear:N \l__problems_solution_id_str
  \tl_clear:N \l__problems_solution_for_tl
  \tl_clear:N \l__problems_solution_srccite_tl
  \clist_clear:N \l__problems_solution_creators_clist
  \clist_clear:N \l__problems_solution_contributors_clist
  \dim_zero:N \l__problems_solution_height_dim
  \keys_set:nn { problem / solution }{ #1 }
}
\newcommand\@startsolution[1][]{
  \__problems_solution_args:n { #1 }
  \@in@omtexttrue% we are in a statement.
  \bool_if:NF \c__problems_boxed_bool { \hrule }
  \smallskip\noindent
  {\textbf\prob@solution@kw :\enspace}
  \begin{small}
  \def\current@section@level{\prob@solution@kw}
  \ignorespacesandpars
}
\newcommand\startsolutions{
  \specialcomment{solution}{\@startsolution}{
    \bool_if:NF \c__problems_boxed_bool {
      \hrule\medskip
    }
    \end{small}%
  }
  \bool_if:NT \c__problems_boxed_bool {
    \surroundwithmdframed{solution}
  }
}
\newcommand\stopsolutions{\excludecomment{solution}}
\bool_if:NTF \c__problems_solutions_bool {
  \startsolutions
}{
  \stopsolutions
}
\bool_if:NTF \c__problems_notes_bool {
  \newenvironment{exnote}[1][]{
    \par\smallskip\hrule\smallskip
    \noindent\textbf{\prob@note@kw : }\small
  }{
    \smallskip\hrule
  }
}{
  \excludecomment{exnote}
}
\bool_if:NTF \c__problems_notes_bool {
  \newenvironment{hint}[1][]{
    \par\smallskip\hrule\smallskip
    \noindent\textbf{\prob@hint@kw : }\small
  }{
    \smallskip\hrule
  }
  \newenvironment{exhint}[1][]{
    \par\smallskip\hrule\smallskip
    \noindent\textbf{\prob@hint@kw : }\small
  }{
    \smallskip\hrule
  }
}{
  \excludecomment{hint}
  \excludecomment{exhint}
}
\bool_if:NTF \c__problems_notes_bool {
  \newenvironment{gnote}[1][]{
    \par\smallskip\hrule\smallskip
    \noindent\textbf{\prob@gnote@kw : }\small
  }{
    \smallskip\hrule
  }
}{
  \excludecomment{gnote}
}
\newenvironment{mcb}{
  \begin{enumerate}
}{
  \end{enumerate}
}
\cs_new_protected:Nn \__problems_do_yes_param:Nn {
  \exp_args:Nx \str_if_eq:nnTF { \str_lowercase:n{ #2 } }{ yes }{
    \bool_set_true:N #1
  }{
    \bool_set_false:N #1
  }
}
\keys_define:nn { problem / mcc }{
  id        .str_set_x:N  = \l__problems_mcc_id_str ,
  feedback  .tl_set:N     = \l__problems_mcc_feedback_tl ,
  T         .default:n    = { true } ,
  T         .bool_set:N   = \l__problems_mcc_t_bool ,
  F         .default:n    = { true } ,
  F         .bool_set:N   = \l__problems_mcc_f_bool ,
  Ttext     .code:n       = {
    \__problems_do_yes_param:Nn \l__problems_mcc_Ttext_bool { #1 }
  } ,
  Ftext     .code:n       = {
    \__problems_do_yes_param:Nn \l__problems_mcc_Ftext_bool { #1 }
  }
}
\cs_new_protected:Nn \l__problems_mcc_args:n {
  \str_clear:N \l__problems_mcc_id_str
  \tl_clear:N \l__problems_mcc_feedback_tl
  \bool_set_true:N \l__problems_mcc_t_bool
  \bool_set_true:N \l__problems_mcc_f_bool
  \bool_set_true:N \l__problems_mcc_Ttext_bool
  \bool_set_false:N \l__problems_mcc_Ftext_bool
  \keys_set:nn { problem / mcc }{ #1 }
}
\newcommand\mcc[2][]{
  \l__problems_mcc_args:n{ #1 }
  \item #2
  \bool_if:NT \c__problems_solutions_bool {
    \\
    \bool_if:NT \l__problems_mcc_t_bool {
      % TODO!
      % \ifcsstring{mcc@T}{T}{}{\mcc@Ttext}%
    }
    \bool_if:NT \l__problems_mcc_f_bool {
      % TODO!
      % \ifcsstring{mcc@F}{F}{}{\mcc@Ftext}%
    }
    \tl_if_empty:NTF \l__problems_mcc_feedback_tl {
      !
    }{
      \l__problems_mcc_feedback_tl
    }
  }
} %solutions

\keys_define:nn{ problem / inclproblem }{
 % id      .str_set_x:N  = \l__problems_inclprob_id_str,
  pts     .tl_set:N     = \l__problems_inclprob_pts_tl,
  min     .tl_set:N     = \l__problems_inclprob_min_tl,
  title   .tl_set:N     = \l__problems_inclprob_title_tl,
  refnum  .int_set:N    = \l__problems_inclprob_refnum_int
}
\cs_new_protected:Nn \__problems_inclprob_args:n {
  \tl_clear:N \l__problems_inclprob_pts_tl
  \tl_clear:N \l__problems_inclprob_min_tl
  \tl_clear:N \l__problems_inclprob_title_tl
  \int_zero_new:N \l__problems_inclprob_refnum_int
  \keys_set:nn { problem / inclproblem }{ #1 }
}

\cs_new_protected:Nn \__problems_inclprob_clear: {
  \let\l__problems_inclprob_pts_tl\undefined
  \let\l__problems_inclprob_min_tl\undefined
  \let\l__problems_inclprob_title_tl\undefined
  \let\l__problems_inclprob_refnum_int\undefined
}

\newcommand\includeproblem[2][]{
  \__problems_inclprob_args:n{ #1 }
  \edef\temp@path{#2}
  \if@iswindows@\path@to@windows\temp@path\fi %TODO ?
  \input{\temp@path}
  \__problems_inclprob:clear:
}
\AddToHook{enddocument}{
  \bool_if:NT \c__problems_pts_bool {
    \message{Total:~\arabic{pts}~points}
  }
  \bool_if:NT \c__problems_min_bool {
    \message{Total:~\arabic{min}~minutes}
  }
}
\def\pts#1{
  \bool_if:NT \c__problems_pts_bool {
    \marginpar{#1~\prob@pt@kw}
  }
}
\def\min#1{
  \bool_if:NT \c__problems_min_bool {
    \marginpar{#1~\prob@min@kw}
  }
}
\newcounter{pts}
\def\show@pts{
  \tl_if_exist:NTF \l__problems_inclprob_pts_tl {
    \bool_if:NT \c__problems_pts_bool {
      \marginpar{\l__problems_inclprob_pts_tl;\prob@pt@kw\smallskip}
      \addtocounter{pts}{\l__problems_inclprob_pts_tl}
    }
  }{
    \tl_if_exist:NT \l__problems_prob_pts_tl {
      \bool_if:NT \c__problems_pts_bool {
        \marginpar{\l__problems_prob_pts_tl;\prob@pt@kw\smallskip}
        \addtocounter{pts}{\l__problems_prob_pts_tl}
      }
    }
  }
}
\newcounter{min}
\def\show@min{
  \tl_if_exist:NTF \l__problems_inclprob_min_tl {
    \bool_if:NT \c__problems_min_bool {
      \marginpar{\l__problems_inclprob_pts_tl;min}
      \addtocounter{min}{\l__problems_inclprob_min_tl}
    }
  }{
    \tl_if_exist:NT \l__problems_prob_min_tl {
      \bool_if:NT \c__problems_min_bool {
        \marginpar{\l__problems_prob_min_tl;min}
        \addtocounter{min}{\l__problems_prob_min_tl}
      }
    }
  }
}
\endinput
%%
%% End of file `problem.sty'.

    % \iffalse meta-comment
% An Infrastructure for marking up Assignments
% Copyright (c) 2019 Michael Kohlhase, all rights reserved
%               this file is released under the
%               LaTeX Project Public License (LPPL)
% The original of this file is in the public repository at 
% http://github.com/sLaTeX/sTeX/
% \fi
% 
% \iffalse
%
%<*driver>
\def\stexdocpath{../doc}
\RequirePackage{paralist}
\ifcsname stexdocpath\endcsname\else\def\stexdocpath{.}\fi
\documentclass[full]{l3doc}
%\RequirePackage{document-structure}
\usepackage[hyperref=auto,style=alphabetic]{biblatex}
%\usepackage[mathhub=\stexdocpath/mh,usedeps]{stex}
\usepackage[lang={en,de}]{stex}

\usepackage{rustex}
\usepackage{stex-highlighting,stexthm}

\srefsetin[sTeX/Documentation]{documentation}{the \stex Documentation}

\makeatletter
\providecommand{\HTML}{\textsc{html}\xspace}%
\providecommand{\XML}{\textsc{xml}\xspace}%
\providecommand{\PDF}{\textsc{pdf}\xspace}%
\providecommand\openmath{\textsc{OpenMath}\xspace}
\providecommand\OMDoc{\textsc{OMDoc}\xspace}
\DeclareRobustCommand\LaTeXML{L\kern-.36em%
        {\sbox\z@ T%
         \vbox to\ht\z@{\hbox{\check@mathfonts
                              \fontsize\sf@size\z@
                              \math@fontsfalse\selectfont
                              A}%
                        \vss}%
        }%
        \kern-.15em%
%        T\kern-.1667em\lower.5ex\hbox{E}\kern-.125em\relax
%        {\tt XML}}
        T\kern-.1667em\lower.4ex\hbox{E}\kern-0.05em\relax
        {\scshape xml}\xspace}%
\def\mmt{\textsc{Mmt}\xspace}
\makeatother


\newif\ifhadtitle\hadtitlefalse

\def\stexversion{3.3.0}
\def\changedate{\today}
\def\stextoptitle#1#2{\title{#1\thanks{Version {\stexversion} (last revised {\changedate})} }\def\thispkg{#2}}

\author{Michael Kohlhase, Dennis Müller\\
 	FAU Erlangen-Nürnberg\\
 	\url{http://kwarc.info/}
}

\def\stexmaketitle{\ifhadtitle\else\hadtitletrue\maketitle\fi}

\ExplSyntaxOn

  \def\docmodule{
    \begin{document}
      \EnableManual
      \EnableDocumentation
      \EnableImplementation
      \stexmaketitle
      \tableofcontents
      \int_gincr:N \l_stex_docheader_sect
      \exp_args:Ne \__stex_mathhub_find_manifest:n {\stex_file_use:N \c_stex_mathhub_file / sTeX / Documentation}
      \str_if_empty:NF \l__stex_mathhub_manifest_str {
        \usemodule[sTeX/Documentation]{macros?AllMacros}
      }
      \DocInput{\jobname.dtx}
      \clearpage
      \PrintIndex
      \printbibliography
    \end{document}
  }

  \bool_new:N \g_stexdoc_typeset_manual_bool
  \NewDocumentCommand \EnableManual {}{
    \bool_gset_true:N \g_stexdoc_typeset_manual_bool
  }
  \NewDocumentCommand \DisableManual {}{
    \bool_gset_false:N \g_stexdoc_typeset_manual_bool
  }
  \NewDocumentEnvironment {stexmanual} {} {
    \bool_if:NTF \g_stexdoc_typeset_manual_bool
      {\bool_set_false:N \l__codedoc_in_implementation_bool}
      {\comment}
  }{
    \bool_if:NF \g_stexdoc_typeset_manual_bool {\endcomment}
  }
\ExplSyntaxOff

%\usepackage{makeidx}
%\makeindex

%\usepackage{document-structure}


\usepackage{lststex,mdframed}
\usepackage[most]{tcolorbox}

\lstset{literate=%
    {Ö}{{\"O}}1
    {Ä}{{\"A}}1
    {Ü}{{\"U}}1
    {ß}{{\ss}}1
    {ü}{{\"u}}1
    {ä}{{\"a}}1
    {ö}{{\"o}}1
    {~}{{\textasciitilde}}1
}

\newenvironment{framed}[1][]{
  \ifstexhtml\par\vbox\bgroup
    \csname exp_args:Nne\endcsname\begin{stex_annotate_env}{%
      style:border=solid 1px black,%
      style:width=var(--this-width),%
      style:min-width=var(--this-width),%
      style:--this-width=calc(var(--current-width) - 6px),%
      style:padding=3px,%
      style:margin-top=5px,%
      style:margin-bottom=5px%
    }
    \csname stex_annotate_invisible:n\endcsname{ }%
    \begin{stex_annotate_env}{%
      style:--current-width=var(--this-width);%
    }\csname stex_annotate_invisible:n\endcsname{ }
  \else\begin{mdframed}[#1]\fi
  %\begin{center}%
}{%
  %\end{center}%
  \ifstexhtml
    \end{stex_annotate_env}\end{stex_annotate_env}\egroup\par
  \else\end{mdframed}\fi
}
\newcommand{\scaled}[2][0.9\hsize]{\begin{center}\resizebox{#1}{!}{\begin{minipage}{\textwidth} #2 \end{minipage}}\end{center}}

\makeatletter
\ExplSyntaxOn

\def\doc_exbox:nnn#1#2#3{
  \begin{sexample}[#3]
    Input:
    \begin{framed}[linewidth=1pt,backgroundcolor=white]\small
      #1
    \end{framed}
    Output:
    \begin{framed}[linewidth=1pt,backgroundcolor=white]\small
      #2
    \end{framed}
  \end{sexample}
}


\NewDocumentCommand\stexexamplefile{O{} m O{} O{}}{
  \stex_resolve_path_pair:Nxx \l_@@_filepath_str {\tl_to_str:n{#1}} {\tl_to_str:n{#2}}
  \doc_exbox:nnn{
    \hfill File~\str_if_empty:nTF{#1}{
      \prop_if_exist:NT \l_stex_current_archive_prop {
        [\texttt{\prop_item:Nn \l_stex_current_archive_prop {id}}]
      }
    }{[#1]}\texttt{\tl_to_str:n{#2}}
    \_lststex_parse_args:n{#3}
    \exp_args:Nno \use:nn{\lstinputlisting[} \l_lststex_return_tl ]{\l_@@_filepath_str}
  }{
    \inputref[#1]{#2}
  }{#4}
}

\newwrite\testoutfile
\NewDocumentCommand\stexexample{O{} O{}}{
  \begingroup 
  \catcode`\\=12\relax
  \catcode`\#=12\relax
  \catcode`\&=12\relax
  \catcode`\$=12\relax
  \catcode`\^=12\relax
  \catcode`\_=12\relax
  \catcode`\ =12\relax
  \catcode`^^J=12\relax
  \endlinechar=`^^J
  \newlinechar=-1
%^^A    \everyeof{\noexpand}
  \example_a:nnn{#1}{#2}
}
\long\def\example_a:nnn #1 #2 #3 {
  \endgroup
  \immediate\openout\testoutfile=\jobname.exmpl
  \immediate\write\testoutfile{
    \c_backslash_str begin{stexcode}[#1]
    \detokenize{^^J}#3
    \c_backslash_str end{stexcode}
  }
  \immediate\closeout\testoutfile
  \doc_exbox:nnn{
    \catcode`\#=12\relax
    \csname @ @ input\endcsname{\jobname.exmpl}
  }{
    \immediate\openout\testoutfile=\jobname.exmpl
    \immediate\write\testoutfile{#3}
    \immediate\closeout\testoutfile
    \csname @ @ input\endcsname \jobname.exmpl\relax
  }{#2}
  \peek_charcode_remove:NT ^^J
}

\ExplSyntaxOff
\makeatother

\makeatletter
\newcount\example@counter\example@counter=0
\newtcolorbox{exampleborderbox}[1][]{
  empty,
  title={Example \the\example@counter #1},
  attach boxed title to top left,
     minipage boxed title,
  boxed title style={empty,size=minimal,toprule=0pt,top=1pt,left=3mm,overlay={}},
  coltitle=blue,fonttitle=\bfseries,
  parbox=false,boxsep=0pt,left=3mm,right=0mm,top=2pt,breakable,pad at break=0mm,
     before upper=\csname @totalleftmargin\endcsname0pt, 
  overlay unbroken={\draw[blue,line width=2pt] ([xshift=-0pt]title.north west) -- ([xshift=-0pt]frame.south west); },
  overlay first={\draw[blue,line width=2pt] ([xshift=-0pt]title.north west) -- ([xshift=-0pt]frame.south west); },
  overlay middle={\draw[blue,line width=2pt] ([xshift=-0pt]frame.north west) -- ([xshift=-0pt]frame.south west); },
  overlay last={\draw[blue,line width=2pt] ([xshift=-0pt]frame.north west) -- ([xshift=-0pt]frame.south west); },
  outer arc=4pt%
}

\ExplSyntaxOn
\stexstyleexample{
  \global\advance\example@counter by 1
  \tl_if_empty:NTF\thistitle{
    \begin{exampleborderbox}
  }{
    \begin{exampleborderbox}[ (\thistitle)]
  }
}{
    \end{exampleborderbox}
}

\ExplSyntaxOff\makeatother

\usetikzlibrary{calc}

\def\textwarning{\includegraphics[width=1.2em]{stex-dangerous-bend}\xspace}
\newtcolorbox{dangerbox}{
  breakable,
  enhanced,
  left=0pt,
  right=0pt,
  top=8pt,
  bottom=8pt,
  colback=white,
  colframe=red,
  width=\textwidth,
  enlarge left by=0mm,
  boxsep=5pt,
  fontupper=\small,
  arc=4pt,
  outer arc=4pt,
  leftupper=1.5cm,
  overlay={
    \node[anchor=west] at ([xshift=10pt]$(frame.north west)!0.5!(frame.south west)$)
       {\includegraphics[width=1cm,height=1cm]{stex-dangerous-bend}};}
}

\protected\def\TODO#1{\textcolor{red}{TODO}\footnote{\textcolor{red}{TODO: #1}}}

\definecolor{darkgreen}{rgb}{0.0, 0.5, 0.0}

\usepackage[solutions]{problem}
\usepackage{hwexam}
\newtcolorbox{problemborderbox}[1][]{
  empty,
  title={Exercise #1},
  attach boxed title to top left,
     minipage boxed title,
  boxed title style={empty,size=minimal,toprule=0pt,top=1pt,left=3mm,overlay={}},
  coltitle=darkgreen,fonttitle=\bfseries,
  parbox=false,boxsep=0pt,left=3mm,right=0mm,top=2pt,breakable,pad at break=0mm,
     before upper=\csname @totalleftmargin\endcsname0pt, 
  overlay unbroken={\draw[darkgreen,line width=2pt] ([xshift=-0pt]title.north west) -- ([xshift=-0pt]frame.south west); },
  overlay first={\draw[darkgreen,line width=2pt] ([xshift=-0pt]title.north west) -- ([xshift=-0pt]frame.south west); },
  overlay middle={\draw[darkgreen,line width=2pt] ([xshift=-0pt]frame.north west) -- ([xshift=-0pt]frame.south west); },
  overlay last={\draw[darkgreen,line width=2pt] ([xshift=-0pt]frame.north west) -- ([xshift=-0pt]frame.south west); },
  outer arc=4pt%
}

\ExplSyntaxOn
\stexstyleproblem{
  \tl_if_empty:NTF\thistitle{
    \begin{problemborderbox}
  }{
    \begin{problemborderbox}[ (\thistitle)]
  }
}{
    \end{problemborderbox}
}
\ExplSyntaxOff

\newtcolorbox{experimental}{
  breakable,
  enhanced,
  left=0pt,
  right=0pt,
  top=8pt,
  bottom=8pt,
  colback=white,
  colframe=gray,
  width=\textwidth,
  enlarge left by=0mm,
  boxsep=5pt,
  fontupper=\small,
  arc=4pt,
  outer arc=4pt,
  leftupper=1.5cm,
  overlay={
    \node[anchor=west] at ([xshift=10pt]$(frame.north west)!0.5!(frame.south west)$)
       {\includegraphics[height=1cm]{stex-experimental}};}
}


\usetikzlibrary{decorations.pathmorphing,shapes,arrows,calc}
% Taken from pgflibrarytikzmmt.code.tex
\newcommand{\mmtarrowtip}{angle 45}
\newcommand{\mmtarrowtipmonoright}{right hook}

\tikzstyle{include}=[\mmtarrowtipmonoright-\mmtarrowtip,thick]
\tikzstyle{morph}=[-\mmtarrowtip,thick]
\tikzstyle{preview}=[decorate, decoration={coil,aspect=0,amplitude=1pt,
                                                  segment length=6pt,
                                                  pre=lineto,pre length=3pt,
                                                  post=lineto,post length=5pt}, thick]
\tikzstyle{view}=[preview,-\mmtarrowtip]


% TIKZ RULES
\def\mmtlogo{
\begin{tikzpicture}

  % White Background (Margins are eyeballed)
  % This is necessary because we paste white over arrows later.
  % If somebody want's to do the full song and dance with
  % interrupted arrows to get transparent background, be my guest.

  \fill[white!] (-0.01,0.15) rectangle (1.11,-0.95);

  % Arrows
  \draw [blue, include] (0,0)     -- (1.1,0);
  \draw [green, morph] (0,-0.4)  -- (1.1,-0.4);
  \draw [red, view]   (-0,-0.8) -- (1.1,-0.8);

  % Cutout for letters
  \fill[white] (0.33,0.1) rectangle (0.66,-0.9);

  % Letters
  \node at (0.18,0)    (nodeM1) {\large M};
  \node at (0.18,-0.4) (nodeM2) {\large M};
  \node at (0.21,-0.8) (nodeT)  {\large T};

\end{tikzpicture}
}

\newtcolorbox{mmtbox}{
  breakable,
  enhanced,
  left=0pt,
  right=0pt,
  top=8pt,
  bottom=8pt,
  colback=white,
  colframe=green,
  width=\textwidth,
  enlarge left by=0mm,
  boxsep=5pt,
  fontupper=\small,
  arc=4pt,
  outer arc=4pt,
  leftupper=1.5cm,
  overlay={
    \node[anchor=west] at ([xshift=10pt]$(frame.north west)!0.5!(frame.south west)$)
       {\mmtlogo};}
}

\AtBeginDocument{\catcode`_=8}
\stextoptitle{The \texttt{hwexam} Package}{hwexam}
\docmodule
%</driver>
% \fi
%
% \begin{stexmanual}
%    \begin{sfragment}{HWExam Manual}
	%      
The \pkg{wexam} package and class supplies an infrastructure that allows to format
nice-looking assignment sheets by simply including problems from problem files marked up
with the \pkg{roblem} package.  It is designed to be compatible with |problems.sty|, and
inherits some of the functionality.

\begin{sfragment}[id=sec:user:options]{Package Options}

\begin{variable}{solutions,notes,hints,gnotes,pts,min}
The \pkg{wexam} package and class take the options |solutions|, |notes|, |hints|,
|gnotes|, |pts|, |min|, and |boxed| that are just passed on to the |problems| package
(cf. its documentation for a description of the intended behavior).
\end{variable}
\end{sfragment}

\begin{sfragment}{Assignments}
This package supplies the \DescribeEnv{assignment}|assignment| environment that groups
problems into assignment sheets. It takes an optional KeyVal argument with the keys
\DescribeMacro{number}|number| (for the assignment number; if none is given, 1 is
assumed as the default or --- in multi-assignment documents --- the ordinal of the
|assignment| environment), \DescribeMacro{title}|title| (for the assignment title; this
is referenced in the title of the assignment sheet), \DescribeMacro{type}|type| (for the
assignment type; e.g. ``quiz'', or ``homework''), \DescribeMacro{given}|given| (for the
date the assignment was given), and \DescribeMacro{due}|due| (for the date the
assignment is due).
\end{sfragment}

\begin{sfragment}{Typesetting Exams}

Furthermore, the |hwexam| package takes the option
\DescribeMacro{multiple}|multiple| that allows to combine multiple assignment sheets into
a compound document (the assignment sheets are treated as section, there is a table of
contents, etc.). 

Finally, there is the option \DescribeMacro{test}|test| that modifies the behavior to
facilitate formatting tests. Only in |test| mode, the macros |\testspace|,
|\testnewpage|, and |\testemptypage| have an effect: they generate space for the
students to solve the given problems. Thus they can be left in the {\LaTeX} source. 

\DescribeMacro{\testspace}|\testspace| takes an argument that expands to a dimension,
and leaves vertical space accordingly. \DescribeMacro{\testnewpage}|\testnewpage| makes
a new page in |test| mode, and \DescribeMacro{\testemptypage}|\testemptypage| generates
an empty page with the cautionary message that this page was intentionally left empty.

Finally, the \DescribeEnv{testheading}|\testheading| takes an optional keyword argument
where the keys \DescribeMacro{duration}|duration| specifies a string that specifies the
duration of the test, \DescribeMacro{min}|min| specifies the equivalent in number of
minutes, and \DescribeMacro{reqpts}|reqpts| the points that are required for a perfect
grade.

\begin{latexcode}
\title{320101 General Computer Science (Fall 2010)}
\begin{testheading}[duration=one hour,min=60,reqpts=27]
  Good luck to all students!
\end{testheading}
\end{latexcode}

Will result in
\begin{center}
  \begin{minipage}{.9\textwidth}
\makeatletter
\@problem{1.1}{4}{10}
\@problem{2.1}{4}{8}
\@problem{2.2}{6}{10}
\@problem{2.3}{6}{10}
\@problem{3.1}{4}{8}
\@problem{3.2}{4}{8}
\@problem{3.3}{2}{4}
\makeatother
\vspace*{-3ex}\hrule\vspace*{.5ex}  formats to\vspace*{1ex}
\hrule\par\noindent\vspace*{2ex}
\title{320101 General Computer Science (Fall 2010)}
\begin{testheading}[duration=one hour,min=60,reqpts=27]
  good luck
\end{testheading}
\end{minipage}
\end{center}
\end{sfragment}

\begin{sfragment}{Including Assignments}

The \DescribeMacro{\inputassignment}|\inputassignment| macro can be used to input
an assignment from another file. It takes an optional KeyVal argument and a second
argument which is a path to the file containing the problem (the macro assumes that
there is only one |assignment| environment in the included file).  The keys
\DescribeMacro{number}|number|, \DescribeMacro{title}|title|,
\DescribeMacro{type}|type|, \DescribeMacro{given}|given|, and \DescribeMacro{due}|due|
are just as for the |assignment| environment and (if given) overwrite the ones specified
in the |assignment| environment in the included file.
\end{sfragment}

%%% Local Variables:
%%% mode: latex
%%% TeX-master: "../stex-manual"
%%% End:

%    \end{sfragment}
% \end{stexmanual}
%
% \begin{documentation}
%    \begin{sfragment}{HWExam Documentation}
%       TODO
%    \end{sfragment}
% \end{documentation}
%
%\begin{implementation}
% 
% \section{Implementation: The hwexam Package} 
%
% \subsection{Package Options}
%
% The first step is to declare (a few) package options that handle whether certain
% information is printed or not. Some come with their own conditionals that are set by the
% options, the rest is just passed on to the |problems| package.
%
%    \begin{macrocode}
%<*package>
\ProvidesExplPackage{hwexam}{2024/10/26}{4.0.0}{homework assignments and exams}
\RequirePackage{l3keys2e}

\keys_define:nn {hwexam / pkg}{
	multiple  .default:n 		= { false },
	multiple	.bool_set:N 	= \c_hwexam_multiple_bool,
	qrcode		.default:n 	  = { false },
	qrcode		.bool_set:N 	= \c_hwexam_qrcode_bool,
	unknown   .code:n 			= {
		\PassOptionsToPackage{\CurrentOption}{problem}
	}
}
\ProcessKeysOptions{ hwexam /pkg }
\RequirePackage{problem}
%    \end{macrocode}
%
%
% \begin{macro}{\hwexam_kw_*}
%   For multilinguality, we define internal macros for keywords that can be specialized in
%   |*.ldf| files.
%    \begin{macrocode}
\AddToHook{begindocument}{
	\ExplSyntaxOn\makeatletter
	\input{hwexam-english.ldf}
	\ltx@ifpackageloaded{babel}{
			\clist_set:Nx \l_tmpa_clist {\exp_args:No \tl_to_str:n \bbl@loaded}
			\exp_args:NNx \clist_if_in:NnT \l_tmpa_clist {\detokenize{ngerman}}{
				\input{hwexam-ngerman.ldf}
			}
			\exp_args:NNx \clist_if_in:NnT \l_tmpa_clist {\detokenize{finnish}}{
				\input{hwexam-finnish.ldf}
			}
			\exp_args:NNx \clist_if_in:NnT \l_tmpa_clist {\detokenize{french}}{
				\input{hwexam-french.ldf}
			}
			\exp_args:NNx \clist_if_in:NnT \l_tmpa_clist {\detokenize{russian}}{
				\input{hwexam-russian.ldf}
			}
	}{}
	\makeatother\ExplSyntaxOff
}
%    \end{macrocode}
% \end{macro}
%
% \subsection{QR Codes}
%    \begin{macrocode}
\group_begin:
  \escapechar=-1
  \xdef\_@@_qr_backslash{\string\\}
\group_end:

\bool_if:NT \c_hwexam_qrcode_bool {
	\RequirePackage{qrcode}
	\RequirePackage{marginnote}
	\str_new:N \g_@@_qr_json_str
	\bool_new:N \g_@@_qr_in_problems_json_bool
	\bool_set_false:N \g_@@_qr_in_problems_json_bool
	\bool_new:N \g_@@_qr_in_subproblems_json_bool
	\bool_set_false:N \g_@@_qr_in_subproblems_json_bool
	\bool_new:N \g_@@_qr_in_anscls_json_bool
	\bool_set_false:N \g_@@_qr_in_anscls_json_bool
	\bool_if:NTF \c__problems_gnotes_bool {
  	\gdef \qrjson { \str_gput_right:Nx \g_@@_qr_json_str}
	}{
		\gdef \qrjson #1 {}
	}
	\qrjson{[}
	\AtEndDocument{\qrjson{]}}
	\def\_@@_qr_escape_char:n #1 {
		#1\exp_args:Nno\use:nn{\if_charcode:w#1}\_@@_qr_backslash#1\fi
	}

	\gdef\_@@_qr_escape:n #1{\exp_args:Ne\str_map_function:nN{\tl_to_str:n{#1}}\_@@_qr_escape_char:n}
	\gdef \_@@_qr_escape:o #1{\exp_args:No\_@@_qr_escape:n#1}

	\def\qrschema{TO:DO:\examnumber}

	\bool_if:NTF \c__problems_test_bool {
		\def\doproblemqr{
			\ifstexhtml\else{
				\reversemarginpar\marginnote{\qrcode[height=1.5cm]{\qrschema}}
				\normalmarginpar
			}\fi
		}
		\def\insertexamnumber{
			\ifstexhtml\else
				\tl_if_exist:NTF \examnumber {
					{\Large\bfseries ID:~\examnumber}
				}{
					{\color{red}\Large\bfseries!!!~ WARNING:~NO~{\string\examnumber}~SET~!!!}\gdef\examnumber{0}
				}
				\global\def\insertexamnumber{}
			\fi
		}
	}{
		\def\doproblemqr{}
		\def\insertexamnumber{}
	}

	\stexstyleproblem[noqr]{
		\par\noindent\problemheader \xdef\qrid{\thesproblem}
		\bool_if:NT \c__problems_pts_bool {
			\tl_if_eq:NnF \l__problems_pts_tl {0}{
				\marginpar{\l__problems_pts_tl{}~\problem@kw@points\smallskip}
			}
		}
		\bool_if:NT \c__problems_min_bool {
			\tl_if_eq:NnF \l__problems_min_tl {0} {
				\marginpar{\l__problems_min_tl{}~\problem@kw@minutes\smallskip}
			}
		}
		\par
		\stex_ignore_spaces_and_pars:
	}{
		\par\bigskip
	}

	\stexstyleproblem{
		\par\noindent\problemheader \xdef\qrid{\thesproblem}
		\doproblemqr
		\bool_if:NT \c__problems_pts_bool {
			\tl_if_eq:NnF \l__problems_pts_tl {0}{
				\marginpar{\l__problems_pts_tl{}~\problem@kw@points\smallskip}
			}
		}
		\bool_if:NT \c__problems_min_bool {
			\tl_if_eq:NnF \l__problems_min_tl {0} {
				\marginpar{\l__problems_min_tl{}~\problem@kw@minutes\smallskip}
			}
		}

		\bool_if:NTF \g_@@_qr_in_problems_json_bool {
			\qrjson{,}
		}{
			\bool_gset_true:N \g_@@_qr_in_problems_json_bool
		}

		\qrjson {
			\c_left_brace_str
				"id":"\_@@_qr_escape:o\l_stex_key_id_str","title":"\_@@_qr_escape:o\l_stex_key_title_tl",
				"number":"\thesproblem"
		}

		\tl_if_eq:NnF \l__problems_pts_tl {0}{
			\qrjson {
				,"pts":\l__problems_pts_tl
			}
		}
		\par
		\stex_ignore_spaces_and_pars:
	}{
		\bool_if:NT \g_@@_qr_in_subproblems_json_bool {
			\qrjson {]}
		}
		\bool_gset_false:N \g_@@_qr_in_subproblems_json_bool
		\qrjson {\c_right_brace_str}
		\par\bigskip
	}

	\stexstylesubproblem[noqr]{
		\begin{list}{}{
			\setlength\topsep{0pt}
			\setlength\parsep{0pt}
			\setlength\rightmargin{0pt}
		}\item[\int_use:N \g__problems_subproblem_int .]
		\xdef\qrid{\thesproblem.\int_use:N \g__problems_subproblem_int}
		\bool_if:NT \c__problems_pts_bool {
			\bool_if:NF \l__problems_has_pts_bool {
				\marginpar{\smallskip\l_stex_key_pts_tl{}~\problem@kw@points}
			}
		}
		\bool_if:NT \c__problems_min_bool {
			\bool_if:NF \l__problems_has_min_bool{
				\marginpar{\smallskip\l_stex_key_min_tl{}~\problem@kw@minutes}
			}
		}
	}{\end{list}}

	\stexstylesubproblem{
		\begin{list}{}{
			\setlength\topsep{0pt}
			\setlength\parsep{0pt}
			\setlength\rightmargin{0pt}
		}\item[\int_use:N \g__problems_subproblem_int .]
		\xdef\qrid{\thesproblem.\int_use:N \g__problems_subproblem_int}
		\doproblemqr
		\bool_if:NT \c__problems_pts_bool {
			\bool_if:NF \l__problems_has_pts_bool {
				\marginpar{\smallskip\l_stex_key_pts_tl{}~\problem@kw@points}
			}
		}
		\bool_if:NT \c__problems_min_bool {
			\bool_if:NF \l__problems_has_min_bool{
				\marginpar{\smallskip\l_stex_key_min_tl{}~\problem@kw@minutes}
			}
		}
		\bool_if:NTF \g_@@_qr_in_subproblems_json_bool {
			\qrjson {,}
		}{
			\qrjson {
				,"subproblems":[
			}
			\bool_gset_true:N \g_@@_qr_in_subproblems_json_bool
		}
		\qrjson {
			\c_left_brace_str
				"id":"\_@@_qr_escape:o\l_stex_key_id_str","title":"\_@@_qr_escape:o\l_stex_key_title_tl",
				"number":"\thesproblem.\int_use:N \g__problems_subproblem_int"
		}
		\bool_if:NF \l__problems_has_pts_bool {
			\qrjson {
				,"pts":\l_stex_key_pts_tl
			}
		}
	}{
		\qrjson {\c_right_brace_str}
		\end{list}
	}

	\stexstylegnote{
		\par\smallskip\rule[.3em]{\linewidth}{0.4pt}\newline\smallskip
		\noindent\emph{\problem@kw@grading\str_if_empty:NF \l_stex_key_title_tl{
			{~}\l_stex_key_title_tl
		} :~}
		\qrjson {
			,"gnote": \c_left_brace_str
			"id":"\_@@_qr_escape:o\l_stex_key_id_str","title":"\_@@_qr_escape:o\l_stex_key_title_tl",
			"anscls":[
		}
	}{
		\qrjson {
			]\c_right_brace_str
		}
		\par\rule[.3em]{\linewidth}{0.4pt}\newline
	}{}

	\renewcommand \anscls [2][] {
		\stex_keys_set:nn{ anscls }{#1}
		\str_if_empty:NT \l_stex_key_id_str {
			\int_incr:N \l__problems_anscls_int
			\str_set:Nx \l_stex_key_id_str {
				AC\int_use:N \l__problems_anscls_int
			}
		}
		\bool_if:NTF \g_@@_qr_in_anscls_json_bool {
			\qrjson{,}
		}{
			\bool_set_true:N \g_@@_qr_in_anscls_json_bool
		}
		\qrjson{
			\c_left_brace_str
				"id":"\_@@_qr_escape:o\l_stex_key_id_str","description":"\_@@_qr_escape:n{#2}",
				"feedback":"\_@@_qr_escape:o\l_stex_key_feedback_tl",
				"pts":"\l_stex_key_pts_str"
			\c_right_brace_str
		}
		\begin{list}{}{
			\setlength\topsep{0pt}
			\setlength\parsep{0pt}
			\setlength\rightmargin{0pt}
		}\item[\l_stex_key_id_str] 
			\stex_if_do_html:TF{
				\exp_args:Ne \stex_annotate:nn{
					shtml:answerclass={\l_stex_key_id_str}
					\str_if_empty:NF \l_stex_key_pts_str{
						,shtml:answerclass-pts={\l_stex_key_pts_str}
					}
				}{
					#2
					\tl_if_empty:NF \l_stex_key_feedback_tl{
						\stex_annotate_invisible:nn{
							shtml:answerclass-feedback={true}
						}{\l_stex_key_feedback_tl}
					}
				}
			}{#2}
			\str_if_empty:NF \l_stex_key_pts_str {\par
				~ \problem@kw@points :~\l_stex_key_pts_str
			}
			\str_if_empty:NF \l_stex_key_feedback_tl {\par
				~ \problem@kw@feedback :~\l_stex_key_feedback_tl
			}
		\end{list}
	}
	\stex_deactivate_macro:Nn \anscls {gnote~environments}

	\AtEndDocument{
		\message{^^J^^J\g_@@_qr_json_str^^J^^J}
		\bool_if:NT \c__problems_gnotes_bool {
			\iow_new:N \c_@@_qr_json_iow
			\iow_open:Nn \c_@@_qr_json_iow {\jobname-vollkorn.json}
			\iow_now:Nx \c_@@_qr_json_iow {\g_@@_qr_json_str}
			\iow_close:N \c_@@_qr_json_iow
		}
	}

	\renewcommand\testemptypage[1][]{%
		\bool_if:NT \c__problems_test_bool {\ 
		\xdef\qrid{P\thepage}
		\doproblemqr
		\vfill\begin{center}\hwexam@kw@testemptypage\end{center}\eject
		}
	}
}
%    \end{macrocode}
%
% \subsection{Assignments}
%
% Then we set up a counter for problems and make the problem counter inherited from
% |problem.sty| depend on it. Furthermore, we specialize the |\prob@label| macro to take
% the assignment counter into account.
%
% \begin{environment}{assignment}
%    \begin{macrocode}
\stex_keys_define:nnnn{ assignment }{
  \tl_clear:N \l_stex_key_number_tl 
	\tl_clear:N \l_stex_key_given_tl 
	\tl_clear:N \l_stex_key_due_tl
}{
  number  .tl_set:N     = \l_stex_key_number_tl,
  given   .tl_set:N     = \l_stex_key_given_tl,
  due     .tl_set:N     = \l_stex_key_due_tl,
	unknown .code:n = {}
}{id,title,style}

\newcounter{assignment}
\stex_new_stylable_env:nnnnnnn {assignment}{O{}}{
  \cs_if_exist:NTF \l_hwexam_includeassignment_keys_tl {
    \tl_put_left:Nn \l_hwexam_includeassignment_keys_tl {#1,}
    \exp_args:Nno \stex_keys_set:nn{assignment}{
      \l_hwexam_includeassignment_keys_tl
    }
  }{
    \stex_keys_set:nn{assignment}{#1}
  }
	\tl_if_empty:NF \l_stex_key_number_tl {
		\global\setcounter{assignment}{\int_eval:n{\l_stex_key_number_tl-1}}
	}
	\global\refstepcounter{assignment}
	\setcounter{sproblem}{0}
	\def\thesproblem{\theassignment.\arabic{sproblem}}
	\stex_style_apply:
	\_stex_do_id:
}{
	\stex_style_apply:
}{
	\par\begin{center}
	\textbf{\Large\assignmentautorefname~\theassignment
		\tl_if_empty:NF \l_stex_key_title_tl {
			{~}--~\l_stex_key_title_tl
		}
	}\par\smallskip
	\textbf{
		\tl_if_empty:NF \l_stex_key_given_tl {
			\hwexam@kw@given :~\l_stex_key_given_tl\quad
		}
		\tl_if_empty:NF \l_stex_key_due_tl {
			\hwexam@kw@due :~\l_stex_key_due_tl\quad
		}
	}
	\end{center}
	\par\bigskip
}{
	\par\pagebreak
}{}
%    \end{macrocode}
% \end{environment}
%
% \begin{macro}{\includeassignment}
%    \begin{macrocode}
\NewDocumentCommand\includeassignment{O{} m}{
	\group_begin:
	\tl_set:Nn \l_hwexam_includeassignment_keys_tl {#1}
	\stex_keys_set:nn{includeproblem}{#1}
	\exp_args:Nno \use:nn{\inputref[}\l_stex_key_mhrepos_str]{#2}
	\group_end:
}
%    \end{macrocode}
% \end{macro}
%
% Restoring information about problems:
%
%    \begin{macrocode}
\prop_new:N \c_@@_problems_prop
\tl_set:Nn \c_@@_total_mins_tl {0}
\tl_set:Nn \c_@@_total_pts_tl {0}
\int_new:N \c_@@_total_problems_int
\cs_set_protected:Npn \problem@restore #1 #2 #3 {
	\int_gincr:N \c_@@_total_problems_int
	\prop_gput:Nnn \c_@@_problems_prop {#1}{{#2}{#3}}
	\tl_gset:Nx \c_@@_total_pts_tl { \int_eval:n { \c_@@_total_pts_tl + #2 }}
	\tl_gset:Nx \c_@@_total_mins_tl { \int_eval:n { \c_@@_total_mins_tl + #2 }}
}
%    \end{macrocode}
%
% \begin{macro}{\correction@table}
%   This macro generates the correction table
%    \begin{macrocode}
\newcommand\correction@table{
	\int_compare:nNnT \c_@@_total_problems_int = 0 {
		\int_incr:N \c_@@_total_problems_int
		\prop_put:Nnn \c_@@_problems_prop {~}{{~}{~}}
	}
	\tl_clear:N \l_tmpa_tl
	\tl_clear:N \l_tmpb_tl
	\tl_clear:N \l_tmpc_tl
	\prop_map_inline:Nn \c_@@_problems_prop {
		\tl_put_right:Nn \l_tmpa_tl { ##1 & }
		\tl_put_right:Nx \l_tmpb_tl { \use_i:nn ##2 & }
		\tl_put_right:Nn \l_tmpc_tl { & }
	}
	\resizebox{\textwidth}{!}{%
\exp_args:Nne \begin{tabular}{|l|*{\int_use:N \c_@@_total_problems_int}{c|}c||l|}\hline
&\exp_args:Ne \multicolumn{\int_eval:n{ \c_@@_total_problems_int + 1}}{c||}
{\footnotesize\hwexam@kw@forgrading} &\\\hline
\hwexam@kw@probs & \l_tmpa_tl \hwexam@kw@sum & \hwexam@kw@grade\\\hline
\hwexam@kw@pts & \l_tmpb_tl \c_@@_total_pts_tl & \\\hline
\hwexam@kw@reached & \l_tmpc_tl & \\[.7cm]\hline
\end{tabular}}}
%    \end{macrocode}
% \end{macro}
%
% \begin{macro}{\testheading}
%    \begin{macrocode}
\def\hwexamheader{\input{hwexam-default.header}}

\def\hwexamminutes{
	\tl_if_empty:NTF \hwexam@duration {
		{\hwexam@min}~\hwexam@minutes@kw
	}{
		\hwexam@duration
	}
}

\stex_keys_define:nnnn{ hwexam / testheading }{
	\tl_clear:N \hwexam@min
	\tl_clear:N \hwexam@duration
	\tl_clear:N \hwexam@reqpts
	\tl_clear:N \hwexam@tools
}{
	min 		.tl_set:N 	= \hwexam@min,
	duration	.tl_set:N 	= \hwexam@duration,
	reqpts		.tl_set:N 	= \hwexam@reqpts,
	tools		.tl_set:N 	= \hwexam@tools 
}{}

\newenvironment{testheading}[1][]{
	\stex_keys_set:nn { hwexam / testheading}{#1}

	\tl_set_eq:NN \hwexam@totalpts \c_@@_total_pts_tl
	\tl_set_eq:NN \hwexam@totalmin \c_@@_total_mins_tl
	\tl_set:Nx \hwexam@checktime {\int_eval:n { \hwexam@min - \hwexam@totalmin }}

	\newif\if@bonuspoints
	\tl_if_empty:NTF \hwexam@reqpts {
		\@bonuspointsfalse
	}{
		\tl_set:Nx \hwexam@bonuspts {
			\int_eval:n{\hwexam@totalpts - \hwexam@reqpts}
		}
		\@bonuspointstrue
	}

	\makeatletter\hwexamheader\makeatother
}{
	\newpage
}
%    \end{macrocode}
% \end{macro}
%
%
%    \begin{macrocode}
%</package>
%    \end{macrocode}
% 
% \subsection{Leftovers}
%
% at some point, we may want to reactivate the logos font, then we use
% \begin{verbatim}
% here we define the logos that characterize the assignment
% \font\bierfont=../assignments/bierglas
% \font\denkerfont=../assignments/denker
% \font\uhrfont=../assignments/uhr
% \font\warnschildfont=../assignments/achtung
%
% \newcommand\bierglas{{\bierfont\char65}}
% \newcommand\denker{{\denkerfont\char65}}
% \newcommand\uhr{{\uhrfont\char65}}
% \newcommand\warnschild{{\warnschildfont\char 65}}
% \newcommand\hardA{\warnschild}
% \newcommand\longA{\uhr}
% \newcommand\thinkA{\denker}
% \newcommand\discussA{\bierglas}
% \end{verbatim}
% \end{implementation}
\endinput
% \iffalse

%%% Local Variables: 
%%% mode: doctex
%%% TeX-master: t
%%% End: 
% \fi
%  LocalWords:  texttt scsys sc latexml fileversion filedate maketitle setcounter newpage
%  LocalWords:  tocdepth tableofcontents pts showmeta showmeta showignores omdoc extrefs
%  LocalWords:  testspace testnewpage testemptypage testheading testheading reqpts reqpts
%  LocalWords:  exfig makeatletter makeatother vspace hrule vspace vspace noindent textsf
%  LocalWords:  includeassignment includeassignment HorIacJuc cscpnrr11 importmodule baz
%  LocalWords:  includemhassignment includemhassignment importmhmodule foobar ldots sref
%  LocalWords:  mhcurrentrepos mh-variants mh-variant compactenum printbibliography Cwd
%  LocalWords:  langle rangle langle rangle ltxml.cls ltxml.sty respetively metakeys qw
%  LocalWords:  cwd stex graphicx amssymb amstext amsmath newif iftest testfalse testtrue
%  LocalWords:  ifsolutions solutionsfalse ifmultiple multiplefalse multipletrue keyval
%  LocalWords:  ltxml assig srefaddidkey addmetakey ifx assignment@titleblock stepcounter
%  LocalWords:  document@hwexamtype importmodules metasetkeys inclassig@title inclassig
%  LocalWords:  inclassig@title inclassig@type inclassig@type inclassig@number xspace kv
%  LocalWords:  inclassig@number inclassig@due inclassig@due inclassig@given ignorespaces
%  LocalWords:  inclassig@given newenvironment currentsectionlevel OptionalKeyVals kvi
%  LocalWords:  omgroup vals hwexamtype ednote textbackslash newcommand inputassignment
%  LocalWords:  unlist quizheading tas hspace hfill textbf newcount vfill addtocounter
%  LocalWords:  theassignment@totalmin theassignment@totalpts assignment@probs xdef hline
%  LocalWords:  assignment@totalpts assignment@totalmin correction@probs correction@probs
%  LocalWords:  newcounter theassignment@probs footnotesize mh@currentrepos endinput
%  LocalWords:  inclassig@mhrepos inclassig@mhrepos doctex inputmhassignment
%  LocalWords:  GPL structuresharing STR iffalse cls NeedsTeXFormat hwexam hwexam.dtx sc 

  \end{omgroup}

\end{omgroup}

% \iffalse meta-comment
% An Infrastructure for Semantic Macros and Module Scoping
% Copyright (c) 2019 Michael Kohlhase, all rights reserved
%                this file is released under the
%                LaTeX Project Public License (LPPL)
% 
% The original of this file is in the public repository at 
% http://github.com/sLaTeX/sTeX/
%
% TODO update copyright  
%
%<*driver>
\providecommand\bibfolder{../../lib/bib}
\RequirePackage{paralist}
\documentclass[full,kernel]{l3doc}
\usepackage[dvipsnames]{xcolor}
\usepackage[utf8]{inputenc}
\usepackage[T1]{fontenc}
\RequirePackage{morewrites}
\RequirePackage{tikzinput}
\usetikzlibrary{fit}

\usepackage[debug=all,lang=en, mathhub=./tests]{stex}
\usepackage{url,array,float,textcomp}
\usepackage[show]{ed}
\usepackage[hyperref=auto,style=alphabetic]{biblatex}
\addbibresource{\bibfolder/kwarcpubs.bib}
\addbibresource{\bibfolder/extpubs.bib}
\addbibresource{\bibfolder/kwarccrossrefs.bib}
\addbibresource{\bibfolder/extcrossrefs.bib}
\usepackage{amssymb}
\usepackage{amsfonts}
\usepackage{xspace}
\usepackage{hyperref}

\makeindex
\floatstyle{boxed}
\newfloat{exfig}{thp}{lop}
\floatname{exfig}{Example}

\usepackage{stex-tests}

\MakeShortVerb{\|}

\def\scsys#1{{{\sc #1}}\index{#1@{\sc #1}}\xspace}
\def\mmt{\textsc{Mmt}\xspace}
\def\xml{\scsys{Xml}}
\def\mathml{\scsys{MathML}}
\def\omdoc{\scsys{OMDoc}}
\def\openmath{\scsys{OpenMath}}
\def\latexml{\scsys{LaTeXML}}
\def\perl{\scsys{Perl}}
\def\cmathml{Content-{\sc MathML}\index{Content {\sc MathML}}\index{MathML@{\sc MathML}!content}}
\def\activemath{\scsys{ActiveMath}}
\def\twin#1#2{\index{#1!#2}\index{#2!#1}}
\def\twintoo#1#2{{#1 #2}\twin{#1}{#2}}
\def\atwin#1#2#3{\index{#1!#2!#3}\index{#3!#2 (#1)}}
\def\atwintoo#1#2#3{{#1 #2 #3}\atwin{#1}{#2}{#3}}
\def\cT{\mathcal{T}}\def\cD{\mathcal{D}}

\def\fileversion{3.0}
\def\filedate{\today}

\RequirePackage{pdfcomment}

\ExplSyntaxOn\makeatletter
\cs_set_protected:Npn \@comp #1 #2 {
  \pdftooltip {
    \textcolor{blue}{#1}
  } { #2 }
}

\cs_set_protected:Npn \@defemph #1 #2 {
  \pdftooltip { 
    \textbf{\textcolor{magenta}{#1}}
  } { #2 }
}

\def\__omtext_lec#1{#1}
\cs_new_protected:Npn \lec #1 {
  \strut\hfil\strut\null\hfill\__omtext_lec{#1}
}
\makeatother\ExplSyntaxOff

\makeatletter
\let\@stex@oldcomment\comment
\let\@stex@oldendcomment\endcomment

%\RequirePackage{comment}
\RequirePackage{document-structure}
\RequirePackage[hints,solutions,notes]{problem}
\RequirePackage{hwexam}

\let\comment\@stex@oldcomment
\let\endcomment\@stex@oldendcomment

\newif\ifinfulldoc\infulldocfalse
\makeatother

\def\basedocurl{https://github.com/slatex/sTeX/blob/latex3/doc}
\newcounter{module}

\NewDocumentEnvironment {module}{}{
  \stepcounter{module}
  \textbf{Module \themodule: \smoduletitle}
}{

}
\stexpatchmodule{\begin{module}}{\end{module}}

\def\compemph#1{\textcolor{blue}{#1}}
\def\symrefemph#1{\textcolor{green}{#1}}

\RequirePackage{pdfcomment}
\makeatletter
\protected\def\compemph@uri#1#2{%
  \pdftooltip{%
    \srefsymuri{#2}{\compemph{#1}}%
  }{%
    URI: \detokenize{#2}%
  }%
}
\protected\def\symrefemph@uri#1#2{%
  \pdftooltip{%
    \srefsymuri{#2}{\symrefemph{#1}}%
  }{%
    URI: \detokenize{#2}%
  }%
}
\makeatother

\begin{document}
  \DocInput{\jobname.dtx}
\end{document}
%</driver>
% \fi
%
% \title{ \sTeX-Basics
% 	\thanks{Version {\fileversion} (last revised {\filedate})} 
% }
%
% \author{Michael Kohlhase, Dennis Müller\\
% 	FAU Erlangen-Nürnberg\\
% 	\url{http://kwarc.info/}
% }
%
% \maketitle
%
%\ifinfulldoc\else
% This is the documentation for the \pkg{stex-basics} package.
% For a more high-level introduction, 
%  see \href{\basedocurl/manual.pdf}{the \sTeX Manual} or the
% \href{\basedocurl/stex.pdf}{full \sTeX documentation}.
% \fi
%
% \begin{documentation}\label{pkg:basics:doc}
%
% This sub package provides general set up code, auxiliary methods
% and abstractions for |xhtml| annotations.
%
%\ifinfulldoc\else
% % \iffalse meta-comment
% An Infrastructure for Semantic Macros and Module Scoping
% Copyright (c) 2019 Michael Kohlhase, all rights reserved
%                this file is released under the
%                LaTeX Project Public License (LPPL)
% 
% The original of this file is in the public repository at 
% http://github.com/sLaTeX/sTeX/
%
% TODO update copyright  
%
%<*driver>
\providecommand\bibfolder{../../lib/bib}
\RequirePackage{paralist}
\documentclass[full,kernel]{l3doc}
\usepackage[dvipsnames]{xcolor}
\usepackage[utf8]{inputenc}
\usepackage[T1]{fontenc}
\RequirePackage{morewrites}
\RequirePackage{tikzinput}
\usetikzlibrary{fit}

\usepackage[debug=all,lang=en, mathhub=./tests]{stex}
\usepackage{url,array,float,textcomp}
\usepackage[show]{ed}
\usepackage[hyperref=auto,style=alphabetic]{biblatex}
\addbibresource{\bibfolder/kwarcpubs.bib}
\addbibresource{\bibfolder/extpubs.bib}
\addbibresource{\bibfolder/kwarccrossrefs.bib}
\addbibresource{\bibfolder/extcrossrefs.bib}
\usepackage{amssymb}
\usepackage{amsfonts}
\usepackage{xspace}
\usepackage{hyperref}

\makeindex
\floatstyle{boxed}
\newfloat{exfig}{thp}{lop}
\floatname{exfig}{Example}

\usepackage{stex-tests}

\MakeShortVerb{\|}

\def\scsys#1{{{\sc #1}}\index{#1@{\sc #1}}\xspace}
\def\mmt{\textsc{Mmt}\xspace}
\def\xml{\scsys{Xml}}
\def\mathml{\scsys{MathML}}
\def\omdoc{\scsys{OMDoc}}
\def\openmath{\scsys{OpenMath}}
\def\latexml{\scsys{LaTeXML}}
\def\perl{\scsys{Perl}}
\def\cmathml{Content-{\sc MathML}\index{Content {\sc MathML}}\index{MathML@{\sc MathML}!content}}
\def\activemath{\scsys{ActiveMath}}
\def\twin#1#2{\index{#1!#2}\index{#2!#1}}
\def\twintoo#1#2{{#1 #2}\twin{#1}{#2}}
\def\atwin#1#2#3{\index{#1!#2!#3}\index{#3!#2 (#1)}}
\def\atwintoo#1#2#3{{#1 #2 #3}\atwin{#1}{#2}{#3}}
\def\cT{\mathcal{T}}\def\cD{\mathcal{D}}

\def\fileversion{3.0}
\def\filedate{\today}

\RequirePackage{pdfcomment}

\ExplSyntaxOn\makeatletter
\cs_set_protected:Npn \@comp #1 #2 {
  \pdftooltip {
    \textcolor{blue}{#1}
  } { #2 }
}

\cs_set_protected:Npn \@defemph #1 #2 {
  \pdftooltip { 
    \textbf{\textcolor{magenta}{#1}}
  } { #2 }
}

\def\__omtext_lec#1{#1}
\cs_new_protected:Npn \lec #1 {
  \strut\hfil\strut\null\hfill\__omtext_lec{#1}
}
\makeatother\ExplSyntaxOff

\makeatletter
\let\@stex@oldcomment\comment
\let\@stex@oldendcomment\endcomment

%\RequirePackage{comment}
\RequirePackage{document-structure}
\RequirePackage[hints,solutions,notes]{problem}
\RequirePackage{hwexam}

\let\comment\@stex@oldcomment
\let\endcomment\@stex@oldendcomment

\newif\ifinfulldoc\infulldocfalse
\makeatother

\def\basedocurl{https://github.com/slatex/sTeX/blob/latex3/doc}
\newcounter{module}

\NewDocumentEnvironment {module}{}{
  \stepcounter{module}
  \textbf{Module \themodule: \smoduletitle}
}{

}
\stexpatchmodule{\begin{module}}{\end{module}}

\def\compemph#1{\textcolor{blue}{#1}}
\def\symrefemph#1{\textcolor{green}{#1}}

\RequirePackage{pdfcomment}
\makeatletter
\protected\def\compemph@uri#1#2{%
  \pdftooltip{%
    \srefsymuri{#2}{\compemph{#1}}%
  }{%
    URI: \detokenize{#2}%
  }%
}
\protected\def\symrefemph@uri#1#2{%
  \pdftooltip{%
    \srefsymuri{#2}{\symrefemph{#1}}%
  }{%
    URI: \detokenize{#2}%
  }%
}
\makeatother

\begin{document}
  \DocInput{\jobname.dtx}
\end{document}
%</driver>
% \fi
%
% \title{ \sTeX-Basics
% 	\thanks{Version {\fileversion} (last revised {\filedate})} 
% }
%
% \author{Michael Kohlhase, Dennis Müller\\
% 	FAU Erlangen-Nürnberg\\
% 	\url{http://kwarc.info/}
% }
%
% \maketitle
%
%\ifinfulldoc\else
% This is the documentation for the \pkg{stex-basics} package.
% For a more high-level introduction, 
%  see \href{\basedocurl/manual.pdf}{the \sTeX Manual} or the
% \href{\basedocurl/stex.pdf}{full \sTeX documentation}.
% \fi
%
% \begin{documentation}\label{pkg:basics:doc}
%
% This sub package provides general set up code, auxiliary methods
% and abstractions for |xhtml| annotations.
%
%\ifinfulldoc\else
% % \iffalse meta-comment
% An Infrastructure for Semantic Macros and Module Scoping
% Copyright (c) 2019 Michael Kohlhase, all rights reserved
%                this file is released under the
%                LaTeX Project Public License (LPPL)
% 
% The original of this file is in the public repository at 
% http://github.com/sLaTeX/sTeX/
%
% TODO update copyright  
%
%<*driver>
\providecommand\bibfolder{../../lib/bib}
\input{../../doc/docheader}

\begin{document}
  \DocInput{\jobname.dtx}
\end{document}
%</driver>
% \fi
%
% \title{ \sTeX-Basics
% 	\thanks{Version {\fileversion} (last revised {\filedate})} 
% }
%
% \author{Michael Kohlhase, Dennis Müller\\
% 	FAU Erlangen-Nürnberg\\
% 	\url{http://kwarc.info/}
% }
%
% \maketitle
%
%\ifinfulldoc\else
% This is the documentation for the \pkg{stex-basics} package.
% For a more high-level introduction, 
%  see \href{\basedocurl/manual.pdf}{the \sTeX Manual} or the
% \href{\basedocurl/stex.pdf}{full \sTeX documentation}.
% \fi
%
% \begin{documentation}\label{pkg:basics:doc}
%
% This sub package provides general set up code, auxiliary methods
% and abstractions for |xhtml| annotations.
%
%\ifinfulldoc\else
% \input{../../doc/packages/basics}
% \fi
%
%
% \section{Macros and Environments}\label{pkg:basics:doc:macros}
%
% \begin{function}{\sTeX , \stex}
%   Both print this \stex logo.
% \end{function}
%
% \begin{function}{\stex_debug:nn}
%   \begin{syntax}
%     \cs{stex_debug:nn} \Arg{log-prefix} \Arg{message} ^^A \meta{comma list}
%   \end{syntax}
% Logs \meta{message}, if the package option |debug| contains \meta{log-prefix}.
% \end{function}
%
% \subsection{HTML Annotations}
%
% \begin{function}{\if@latexml}
%   \LaTeX2e conditional for \latexml
% \end{function}
%
% \begin{function}[pTF]{\latexml_if:}
%   \LaTeX3 conditionals for \latexml.
% \end{function}
%
% \begin{function}[pTF]{\stex_if_do_html:}
%   Whether to currently produce any HTML annotations (can be false
%   in some advanced structuring environments, for example)
% \end{function}
%
% \begin{function}{\stex_suppress_html:n}
%   Temporarily disables HTML annotations in its argument code
% \end{function}
% 
%
% We have four macros for annotating generated HTML (via \latexml
% or \rustex) with attributes:
%
% \begin{function}{\stex_annotate:nnn, \stex_annotate_invisible:nnn,
%   \stex_annotate_invisible:n}
%   \begin{syntax} \cs{stex_annotate:nnn} \Arg{property} \Arg{resource} \Arg{content} \end{syntax}
% Annotates the HTML generated by \meta{content} with\\
% \begin{center}
%  |property="stex:|\meta{property}|", resource="|\meta{resource}|"|.
% \end{center}
%
% \cs{stex_annotate_invisible:n} adds the attributes\\
% \begin{center}
% |stex:visible="false", style="display:none"|.
% \end{center}
%
% \cs{stex_annotate_invisible:nnn} combines the functionality of both.
% \end{function}
%
% \begin{environment}{stex_annotate_env}
%   \begin{syntax} \cs{begin}|{stex_annotate_env}|\Arg{property}\Arg{resource}
%       \meta{content}
%     \cs{end}|{stex_annotate_env}|
%   \end{syntax}
% behaves like \cs{stex_annotate:nnn} \Arg{property} \Arg{resource}
%     \Arg{content}.
% \end{environment}
%
% \subsection{Babel Languages}
%
% \begin{variable}{\c_stex_languages_prop,\c_stex_language_abbrevs_prop}
%   Map language abbreviations to their full babel names and vice versa.
%   e.g. \cs{c_stex_languages_prop}|{en}| yields |english|, and
%   \cs{c_stex_language_abbrevs_prop}|{english}| yields |en|.
% \end{variable}
%
% \subsection{Auxiliary Methods}
%
% \begin{function}{\stex_deactivate_macro:Nn , \stex_reactivate_macro:N}
%   \begin{syntax}\cs{stex_deactivate_macro:Nn}\meta{cs}\Arg{environments}\end{syntax}
%   Makes the macro \meta{cs} throw an error, indicating that it
%   is only allowed in the context of \meta{environments}.
%
%   \cs{stex_reactivate_macro:N}\meta{cs} reactivates it again, i.e.
%   this happens ideally in the \meta{begin}-code of the associated
%   environments.
% \end{function}
%
% \begin{function}{\ignorespacesandpars}
%   ignores white space characters and |\par| control sequences.
%   Expands tokens in the process.
% \end{function}
%
% \end{documentation}
%
% \begin{implementation}
%
% \section{\sTeX-Basics Implementation}\label{pkg:basics:impl}
%
%   \subsection{The \sTeX Document Class}
%
% The \cls{stex} document class is pretty straight-forward: It largely extends the \cls{standalone} package
% and loads the \pkg{stex} package, passing all provided options on to the package.
%
%    \begin{macrocode}
%<*cls>

%%%%%%%%%%%%%   basics.dtx   %%%%%%%%%%%%%

\RequirePackage{expl3,l3keys2e}
\ProvidesExplClass{stex}{2022/02/24}{3.0.0}{sTeX document class}
\LoadClass[border=1px,varwidth]{standalone}
\setlength\textwidth{15cm}

\DeclareOption*{\PassOptionsToPackage{\CurrentOption}{stex}}
\ProcessOptions

\RequirePackage{stex}
%</cls>
%    \end{macrocode}
%
% \subsection{Preliminaries}
%
%    \begin{macrocode}
%<*package>

%%%%%%%%%%%%%   basics.dtx   %%%%%%%%%%%%%

\RequirePackage{expl3,l3keys2e,ltxcmds}
\ProvidesExplPackage{stex}{2022/02/24}{3.0.0}{sTeX package}

%\RequirePackage{morewrites}
%\RequirePackage{amsmath}

%    \end{macrocode}
%
% Package options:
%
%    \begin{macrocode}
\keys_define:nn { stex } {
  debug     .clist_set:N  = \c_stex_debug_clist ,
  lang      .clist_set:N  = \c_stex_languages_clist ,
  mathhub   .tl_set_x:N   = \mathhub ,
  sms       .bool_set:N   = \c_stex_persist_mode_bool ,
  image     .bool_set:N   = \c_tikzinput_image_bool,
  unknown   .code:n       = {}
}
\ProcessKeysOptions { stex }
%    \end{macrocode}
%
% \begin{macro}{\stex,\sTeX}
%   The \sTeX logo:
%
%    \begin{macrocode}
\protected\def\stex{%
  \@ifundefined{texorpdfstring}%
  {\let\texorpdfstring\@firstoftwo}%
  {}%
  \texorpdfstring{\raisebox{-.5ex}S\kern-.5ex\TeX}{sTeX}\xspace%
}
\def\sTeX{\stex}
%    \end{macrocode}
% \end{macro}
%
%
% \subsection{Messages and logging}
%
%    \begin{macrocode}
%<@@=stex_log>
%    \end{macrocode}
%
% Warnings and error messages
%
%    \begin{macrocode}
\msg_new:nnn{stex}{error/unknownlanguage}{
  Unknown~language:~#1
}
\msg_new:nnn{stex}{warning/nomathhub}{
  MATHHUB~system~variable~not~found~and~no~
  \detokenize{\mathhub}-value~set!
}
\msg_new:nnn{stex}{error/deactivated-macro}{
  The~\detokenize{#1}~command~is~only~allowed~in~#2!
}
%    \end{macrocode}
% 
% \begin{macro}{\stex_debug:nn}
%
%  A simple macro issuing package messages with subpath.
%
%    \begin{macrocode}
\cs_new_protected:Nn \stex_debug:nn {
  \clist_if_in:NnTF \c_stex_debug_clist { all } {
    \exp_args:Nnnx\msg_set:nnn{stex}{debug / #1}{
      \\Debug~#1:~#2\\
    }
    \msg_none:nn{stex}{debug / #1}
  }{
    \clist_if_in:NnT \c_stex_debug_clist { #1 } {
      \exp_args:Nnnx\msg_set:nnn{stex}{debug / #1}{
        \\Debug~#1:~#2\\
      }
      \msg_none:nn{stex}{debug / #1}
    }  
  }
}
%    \end{macrocode}
% \end{macro}
%
% Redirecting messages:
%
%    \begin{macrocode}
\clist_if_in:NnTF \c_stex_debug_clist {all} {
    \msg_redirect_module:nnn{ stex }{ none }{ term }
}{
  \clist_map_inline:Nn \c_stex_debug_clist {
    \msg_redirect_name:nnn{ stex }{ debug / ##1 }{ term }
  }
}

\stex_debug:nn{log}{debug~mode~on}
%    \end{macrocode}
%
%
% \subsection{HTML Annotations}
%    \begin{macrocode}
%<@@=stex_annotate>
\RequirePackage{rustex}
%    \end{macrocode}
%
% We add the namespace abbreviation |ns:stex="http://kwarc.info/ns/sTeX"| to \rustex:
%
%    \begin{macrocode}
\rustex_add_Namespace:nn{stex}{http://kwarc.info/ns/sTeX}
%    \end{macrocode}
%
% Conditionals for \latexml:
%
% \begin{macro}{\if@latexml}
%    \begin{macrocode}
\ifcsname if@latexml\endcsname\else
    \expandafter\newif\csname if@latexml\endcsname\@latexmlfalse
\fi
%    \end{macrocode}
% \end{macro}
%
% \begin{macro}[pTF]{\latexml_if:}
%    \begin{macrocode}
\prg_new_conditional:Nnn \latexml_if: {p, T, F, TF} {
  \if@latexml
    \prg_return_true:
  \else:
    \prg_return_false:
  \fi:
}
%    \end{macrocode}
% \end{macro}
%
% \begin{variable}{\l_@@_arg_tl, \c_@@_emptyarg_tl}
%
% Used by annotation macros to ensure that the HTML output to annotate
% is not empty.
%
%    \begin{macrocode}
\tl_new:N \l_@@_arg_tl
\tl_const:Nx \c_@@_emptyarg_tl {
  \rustex_if:TF {
    \rustex_direct_HTML:n { \c_ampersand_str lrm; }
  }{~}
}
%    \end{macrocode}
% \end{variable}
%
% \begin{macro}{\_@@_checkempty:n}
%    \begin{macrocode}
\cs_new_protected:Nn \_@@_checkempty:n {
  \tl_set:Nn \l_@@_arg_tl { #1 }
  \tl_if_empty:NT \l_@@_arg_tl {
    \tl_set_eq:NN \l_@@_arg_tl \c_@@_emptyarg_tl
  }
}
%    \end{macrocode}
% \end{macro}
%
% \begin{macro}[pTF]{\stex_if_do_html:}
%  Whether to (locally) produce HTML output
%    \begin{macrocode}
\bool_new:N \_stex_html_do_output_bool
\bool_set_true:N \_stex_html_do_output_bool

\prg_new_conditional:Nnn \stex_if_do_html: {p,T,F,TF} {
  \bool_if:nTF \_stex_html_do_output_bool
    \prg_return_true: \prg_return_false:
}
%    \end{macrocode}
% \end{macro}
%
% \begin{macro}{\stex_suppress_html:n}
%  Whether to (locally) produce HTML output
%    \begin{macrocode}
\cs_new_protected:Nn \stex_suppress_html:n {
  \exp_args:Nne \use:nn {
    \bool_set_false:N \_stex_html_do_output_bool
    #1
  }{
    \stex_if_do_html:T {
      \bool_set_true:N \_stex_html_do_output_bool
    }
  }
}
%    \end{macrocode}
% \end{macro}
%
%
% \begin{environment}{stex_annotate_env}
% \begin{macro}{\stex_annotate:nnn, \stex_annotate_invisible:n,
%    \stex_annotate_invisible:nnn}
%
% We define four macros for introducing attributes in the HTML
% output. The definitions depend on the ``backend'' used
% (\latexml, \rustex, \texttt{pdflatex}). 
%
% The \texttt{pdflatex}-macros largely do nothing; the
% \rustex-implementations are pretty clear in what they do,
%  the \latexml-implementations resort to perl bindings.
%
%    \begin{macrocode}
\rustex_if:TF{
  \cs_new_protected:Nn \stex_annotate:nnn {
    \_@@_checkempty:n { #3 }
    \rustex_annotate_HTML:nn {
      property="stex:#1" ~
      resource="#2"
    } {
      \mode_if_vertical:TF{
        \tl_use:N \l_@@_arg_tl\par
      }{
        \tl_use:N \l_@@_arg_tl
      }
    }
  }
  \cs_new_protected:Nn \stex_annotate_invisible:n {
    \_@@_checkempty:n { #1 }
    \rustex_annotate_HTML:nn {
      stex:visible="false" ~
      style:display="none"
    } {
      \mode_if_vertical:TF{
        \tl_use:N \l_@@_arg_tl\par
      }{
        \tl_use:N \l_@@_arg_tl
      }
    }
  }
  \cs_new_protected:Nn \stex_annotate_invisible:nnn {
    \_@@_checkempty:n { #3 }
    \rustex_annotate_HTML:nn {
      property="stex:#1" ~
      resource="#2" ~
      stex:visible="false" ~
      style:display="none"
    } {
      \mode_if_vertical:TF{
        \tl_use:N \l_@@_arg_tl\par
      }{
        \tl_use:N \l_@@_arg_tl
      }
    }
  }
  \NewDocumentEnvironment{stex_annotate_env} { m m } {
    \par
    \rustex_annotate_HTML_begin:n {
      property="stex:#1" ~
      resource="#2"
    }
  }{
    \par\rustex_annotate_HTML_end:
  }
}{
  \latexml_if:TF {
    \cs_new_protected:Nn \stex_annotate:nnn {
      \_@@_checkempty:n { #3 }
      \mode_if_math:TF {
        \cs:w latexml@annotate@math\cs_end:{#1}{#2}{
          \tl_use:N \l_@@_arg_tl
        }
      }{
        \cs:w latexml@annotate@text\cs_end:{#1}{#2}{
          \tl_use:N \l_@@_arg_tl
        }
      }
    }
    \cs_new_protected:Nn \stex_annotate_invisible:n {
      \_@@_checkempty:n { #1 }
      \mode_if_math:TF {
        \cs:w latexml@invisible@math\cs_end:{
          \tl_use:N \l_@@_arg_tl
        }
      } {
        \cs:w latexml@invisible@text\cs_end:{
          \tl_use:N \l_@@_arg_tl
        }
      }
    }
    \cs_new_protected:Nn \stex_annotate_invisible:nnn {
      \_@@_checkempty:n { #3 }
      \cs:w latexml@annotate@invisible\cs_end:{#1}{#2}{
        \tl_use:N \l_@@_arg_tl
      }
    }
    \NewDocumentEnvironment{stex_annotate_env} { m m } {
      \par\begin{latexml@annotateenv}{#1}{#2}
    }{
      \par\end{latexml@annotateenv}
    }
  }{
    \cs_new_protected:Nn \stex_annotate:nnn {#3}
    \cs_new_protected:Nn \stex_annotate_invisible:n {}
    \cs_new_protected:Nn \stex_annotate_invisible:nnn {}
    \NewDocumentEnvironment{stex_annotate_env} { m m } {}{}
  }
}
%    \end{macrocode}
% \end{macro}
% \end{environment}
%
% \subsection{Babel Languages}
%    \begin{macrocode}
%<@@=stex_language>
%    \end{macrocode}
%
% \begin{variable}{\c_stex_languages_prop,\c_stex_language_abbrevs_prop}
%
% We store language abbreviations in two (mutually inverse) 
% property lists:
%    \begin{macrocode}
\prop_const_from_keyval:Nn \c_stex_languages_prop {
  en = english ,
  de = ngerman ,
  ar = arabic ,
  bg = bulgarian ,
  ru = russian ,
  fi = finnish ,
  ro = romanian ,
  tr = turkish ,
  fr = french
}

\prop_const_from_keyval:Nn \c_stex_language_abbrevs_prop {
  english   = en ,
  ngerman   = de ,
  arabic    = ar ,
  bulgarian = bg ,
  russian   = ru ,
  finnish   = fi ,
  romanian  = ro ,
  turkish   = tr ,
  french    = fr
}
% todo: chinese simplified (zhs)
%       chinese traditional (zht)
%    \end{macrocode}
% \end{variable}
%
% we use the |lang|-package option to load the corresponding
% babel languages:
%
%    \begin{macrocode}
\clist_if_empty:NF \c_stex_languages_clist {
  \clist_clear:N \l_tmpa_clist
  \clist_map_inline:Nn \c_stex_languages_clist {
    \prop_get:NnNTF \c_stex_languages_prop { #1 } \l_tmpa_str {
      \clist_put_right:No \l_tmpa_clist \l_tmpa_str
    } {
      \msg_error:nnx{stex}{error/unknownlanguage}{\l_tmpa_str}
    }
  }
  \stex_debug:nn{lang} {Languages:~\clist_use:Nn \l_tmpa_clist {,~} }
  \RequirePackage[\clist_use:Nn \l_tmpa_clist,]{babel}
}
%    \end{macrocode}
%
% \subsection{Auxiliary Methods}
%
% \begin{macro}{\stex_deactivate_macro:Nn}
%    \begin{macrocode}
\cs_new_protected:Nn \stex_deactivate_macro:Nn {
  \exp_after:wN\let\csname \detokenize{#1} - orig\endcsname#1
  \def#1{
    \msg_error:nnnn{stex}{error/deactivated-macro}{#1}{#2}
  }
}
%    \end{macrocode}
% \end{macro}
%
% \begin{macro}{\stex_reactivate_macro:N}
%    \begin{macrocode}
\cs_new_protected:Nn \stex_reactivate_macro:N {
  \exp_after:wN\let\exp_after:wN#1\csname \detokenize{#1} - orig\endcsname
}
%    \end{macrocode}
% \end{macro}
%
% \begin{macro}{\ignorespacesandpars}
%    \begin{macrocode}
\protected\def\ignorespacesandpars{
  \begingroup\catcode13=10\relax
  \@ifnextchar\par{
    \endgroup\expandafter\ignorespacesandpars\@gobble
  }{
    \endgroup
  }
}
%</package>
%    \end{macrocode}
% \end{macro}
%
% \end{implementation}
%
% \PrintIndex

% \fi
%
%
% \section{Macros and Environments}\label{pkg:basics:doc:macros}
%
% \begin{function}{\sTeX , \stex}
%   Both print this \stex logo.
% \end{function}
%
% \begin{function}{\stex_debug:nn}
%   \begin{syntax}
%     \cs{stex_debug:nn} \Arg{log-prefix} \Arg{message} ^^A \meta{comma list}
%   \end{syntax}
% Logs \meta{message}, if the package option |debug| contains \meta{log-prefix}.
% \end{function}
%
% \subsection{HTML Annotations}
%
% \begin{function}{\if@latexml}
%   \LaTeX2e conditional for \latexml
% \end{function}
%
% \begin{function}[pTF]{\latexml_if:}
%   \LaTeX3 conditionals for \latexml.
% \end{function}
%
% \begin{function}[pTF]{\stex_if_do_html:}
%   Whether to currently produce any HTML annotations (can be false
%   in some advanced structuring environments, for example)
% \end{function}
%
% \begin{function}{\stex_suppress_html:n}
%   Temporarily disables HTML annotations in its argument code
% \end{function}
% 
%
% We have four macros for annotating generated HTML (via \latexml
% or \rustex) with attributes:
%
% \begin{function}{\stex_annotate:nnn, \stex_annotate_invisible:nnn,
%   \stex_annotate_invisible:n}
%   \begin{syntax} \cs{stex_annotate:nnn} \Arg{property} \Arg{resource} \Arg{content} \end{syntax}
% Annotates the HTML generated by \meta{content} with\\
% \begin{center}
%  |property="stex:|\meta{property}|", resource="|\meta{resource}|"|.
% \end{center}
%
% \cs{stex_annotate_invisible:n} adds the attributes\\
% \begin{center}
% |stex:visible="false", style="display:none"|.
% \end{center}
%
% \cs{stex_annotate_invisible:nnn} combines the functionality of both.
% \end{function}
%
% \begin{environment}{stex_annotate_env}
%   \begin{syntax} \cs{begin}|{stex_annotate_env}|\Arg{property}\Arg{resource}
%       \meta{content}
%     \cs{end}|{stex_annotate_env}|
%   \end{syntax}
% behaves like \cs{stex_annotate:nnn} \Arg{property} \Arg{resource}
%     \Arg{content}.
% \end{environment}
%
% \subsection{Babel Languages}
%
% \begin{variable}{\c_stex_languages_prop,\c_stex_language_abbrevs_prop}
%   Map language abbreviations to their full babel names and vice versa.
%   e.g. \cs{c_stex_languages_prop}|{en}| yields |english|, and
%   \cs{c_stex_language_abbrevs_prop}|{english}| yields |en|.
% \end{variable}
%
% \subsection{Auxiliary Methods}
%
% \begin{function}{\stex_deactivate_macro:Nn , \stex_reactivate_macro:N}
%   \begin{syntax}\cs{stex_deactivate_macro:Nn}\meta{cs}\Arg{environments}\end{syntax}
%   Makes the macro \meta{cs} throw an error, indicating that it
%   is only allowed in the context of \meta{environments}.
%
%   \cs{stex_reactivate_macro:N}\meta{cs} reactivates it again, i.e.
%   this happens ideally in the \meta{begin}-code of the associated
%   environments.
% \end{function}
%
% \begin{function}{\ignorespacesandpars}
%   ignores white space characters and |\par| control sequences.
%   Expands tokens in the process.
% \end{function}
%
% \end{documentation}
%
% \begin{implementation}
%
% \section{\sTeX-Basics Implementation}\label{pkg:basics:impl}
%
%   \subsection{The \sTeX Document Class}
%
% The \cls{stex} document class is pretty straight-forward: It largely extends the \cls{standalone} package
% and loads the \pkg{stex} package, passing all provided options on to the package.
%
%    \begin{macrocode}
%<*cls>

%%%%%%%%%%%%%   basics.dtx   %%%%%%%%%%%%%

\RequirePackage{expl3,l3keys2e}
\ProvidesExplClass{stex}{2022/02/24}{3.0.0}{sTeX document class}
\LoadClass[border=1px,varwidth]{standalone}
\setlength\textwidth{15cm}

\DeclareOption*{\PassOptionsToPackage{\CurrentOption}{stex}}
\ProcessOptions

\RequirePackage{stex}
%</cls>
%    \end{macrocode}
%
% \subsection{Preliminaries}
%
%    \begin{macrocode}
%<*package>

%%%%%%%%%%%%%   basics.dtx   %%%%%%%%%%%%%

\RequirePackage{expl3,l3keys2e,ltxcmds}
\ProvidesExplPackage{stex}{2022/02/24}{3.0.0}{sTeX package}

%\RequirePackage{morewrites}
%\RequirePackage{amsmath}

%    \end{macrocode}
%
% Package options:
%
%    \begin{macrocode}
\keys_define:nn { stex } {
  debug     .clist_set:N  = \c_stex_debug_clist ,
  lang      .clist_set:N  = \c_stex_languages_clist ,
  mathhub   .tl_set_x:N   = \mathhub ,
  sms       .bool_set:N   = \c_stex_persist_mode_bool ,
  image     .bool_set:N   = \c_tikzinput_image_bool,
  unknown   .code:n       = {}
}
\ProcessKeysOptions { stex }
%    \end{macrocode}
%
% \begin{macro}{\stex,\sTeX}
%   The \sTeX logo:
%
%    \begin{macrocode}
\protected\def\stex{%
  \@ifundefined{texorpdfstring}%
  {\let\texorpdfstring\@firstoftwo}%
  {}%
  \texorpdfstring{\raisebox{-.5ex}S\kern-.5ex\TeX}{sTeX}\xspace%
}
\def\sTeX{\stex}
%    \end{macrocode}
% \end{macro}
%
%
% \subsection{Messages and logging}
%
%    \begin{macrocode}
%<@@=stex_log>
%    \end{macrocode}
%
% Warnings and error messages
%
%    \begin{macrocode}
\msg_new:nnn{stex}{error/unknownlanguage}{
  Unknown~language:~#1
}
\msg_new:nnn{stex}{warning/nomathhub}{
  MATHHUB~system~variable~not~found~and~no~
  \detokenize{\mathhub}-value~set!
}
\msg_new:nnn{stex}{error/deactivated-macro}{
  The~\detokenize{#1}~command~is~only~allowed~in~#2!
}
%    \end{macrocode}
% 
% \begin{macro}{\stex_debug:nn}
%
%  A simple macro issuing package messages with subpath.
%
%    \begin{macrocode}
\cs_new_protected:Nn \stex_debug:nn {
  \clist_if_in:NnTF \c_stex_debug_clist { all } {
    \exp_args:Nnnx\msg_set:nnn{stex}{debug / #1}{
      \\Debug~#1:~#2\\
    }
    \msg_none:nn{stex}{debug / #1}
  }{
    \clist_if_in:NnT \c_stex_debug_clist { #1 } {
      \exp_args:Nnnx\msg_set:nnn{stex}{debug / #1}{
        \\Debug~#1:~#2\\
      }
      \msg_none:nn{stex}{debug / #1}
    }  
  }
}
%    \end{macrocode}
% \end{macro}
%
% Redirecting messages:
%
%    \begin{macrocode}
\clist_if_in:NnTF \c_stex_debug_clist {all} {
    \msg_redirect_module:nnn{ stex }{ none }{ term }
}{
  \clist_map_inline:Nn \c_stex_debug_clist {
    \msg_redirect_name:nnn{ stex }{ debug / ##1 }{ term }
  }
}

\stex_debug:nn{log}{debug~mode~on}
%    \end{macrocode}
%
%
% \subsection{HTML Annotations}
%    \begin{macrocode}
%<@@=stex_annotate>
\RequirePackage{rustex}
%    \end{macrocode}
%
% We add the namespace abbreviation |ns:stex="http://kwarc.info/ns/sTeX"| to \rustex:
%
%    \begin{macrocode}
\rustex_add_Namespace:nn{stex}{http://kwarc.info/ns/sTeX}
%    \end{macrocode}
%
% Conditionals for \latexml:
%
% \begin{macro}{\if@latexml}
%    \begin{macrocode}
\ifcsname if@latexml\endcsname\else
    \expandafter\newif\csname if@latexml\endcsname\@latexmlfalse
\fi
%    \end{macrocode}
% \end{macro}
%
% \begin{macro}[pTF]{\latexml_if:}
%    \begin{macrocode}
\prg_new_conditional:Nnn \latexml_if: {p, T, F, TF} {
  \if@latexml
    \prg_return_true:
  \else:
    \prg_return_false:
  \fi:
}
%    \end{macrocode}
% \end{macro}
%
% \begin{variable}{\l_@@_arg_tl, \c_@@_emptyarg_tl}
%
% Used by annotation macros to ensure that the HTML output to annotate
% is not empty.
%
%    \begin{macrocode}
\tl_new:N \l_@@_arg_tl
\tl_const:Nx \c_@@_emptyarg_tl {
  \rustex_if:TF {
    \rustex_direct_HTML:n { \c_ampersand_str lrm; }
  }{~}
}
%    \end{macrocode}
% \end{variable}
%
% \begin{macro}{\_@@_checkempty:n}
%    \begin{macrocode}
\cs_new_protected:Nn \_@@_checkempty:n {
  \tl_set:Nn \l_@@_arg_tl { #1 }
  \tl_if_empty:NT \l_@@_arg_tl {
    \tl_set_eq:NN \l_@@_arg_tl \c_@@_emptyarg_tl
  }
}
%    \end{macrocode}
% \end{macro}
%
% \begin{macro}[pTF]{\stex_if_do_html:}
%  Whether to (locally) produce HTML output
%    \begin{macrocode}
\bool_new:N \_stex_html_do_output_bool
\bool_set_true:N \_stex_html_do_output_bool

\prg_new_conditional:Nnn \stex_if_do_html: {p,T,F,TF} {
  \bool_if:nTF \_stex_html_do_output_bool
    \prg_return_true: \prg_return_false:
}
%    \end{macrocode}
% \end{macro}
%
% \begin{macro}{\stex_suppress_html:n}
%  Whether to (locally) produce HTML output
%    \begin{macrocode}
\cs_new_protected:Nn \stex_suppress_html:n {
  \exp_args:Nne \use:nn {
    \bool_set_false:N \_stex_html_do_output_bool
    #1
  }{
    \stex_if_do_html:T {
      \bool_set_true:N \_stex_html_do_output_bool
    }
  }
}
%    \end{macrocode}
% \end{macro}
%
%
% \begin{environment}{stex_annotate_env}
% \begin{macro}{\stex_annotate:nnn, \stex_annotate_invisible:n,
%    \stex_annotate_invisible:nnn}
%
% We define four macros for introducing attributes in the HTML
% output. The definitions depend on the ``backend'' used
% (\latexml, \rustex, \texttt{pdflatex}). 
%
% The \texttt{pdflatex}-macros largely do nothing; the
% \rustex-implementations are pretty clear in what they do,
%  the \latexml-implementations resort to perl bindings.
%
%    \begin{macrocode}
\rustex_if:TF{
  \cs_new_protected:Nn \stex_annotate:nnn {
    \_@@_checkempty:n { #3 }
    \rustex_annotate_HTML:nn {
      property="stex:#1" ~
      resource="#2"
    } {
      \mode_if_vertical:TF{
        \tl_use:N \l_@@_arg_tl\par
      }{
        \tl_use:N \l_@@_arg_tl
      }
    }
  }
  \cs_new_protected:Nn \stex_annotate_invisible:n {
    \_@@_checkempty:n { #1 }
    \rustex_annotate_HTML:nn {
      stex:visible="false" ~
      style:display="none"
    } {
      \mode_if_vertical:TF{
        \tl_use:N \l_@@_arg_tl\par
      }{
        \tl_use:N \l_@@_arg_tl
      }
    }
  }
  \cs_new_protected:Nn \stex_annotate_invisible:nnn {
    \_@@_checkempty:n { #3 }
    \rustex_annotate_HTML:nn {
      property="stex:#1" ~
      resource="#2" ~
      stex:visible="false" ~
      style:display="none"
    } {
      \mode_if_vertical:TF{
        \tl_use:N \l_@@_arg_tl\par
      }{
        \tl_use:N \l_@@_arg_tl
      }
    }
  }
  \NewDocumentEnvironment{stex_annotate_env} { m m } {
    \par
    \rustex_annotate_HTML_begin:n {
      property="stex:#1" ~
      resource="#2"
    }
  }{
    \par\rustex_annotate_HTML_end:
  }
}{
  \latexml_if:TF {
    \cs_new_protected:Nn \stex_annotate:nnn {
      \_@@_checkempty:n { #3 }
      \mode_if_math:TF {
        \cs:w latexml@annotate@math\cs_end:{#1}{#2}{
          \tl_use:N \l_@@_arg_tl
        }
      }{
        \cs:w latexml@annotate@text\cs_end:{#1}{#2}{
          \tl_use:N \l_@@_arg_tl
        }
      }
    }
    \cs_new_protected:Nn \stex_annotate_invisible:n {
      \_@@_checkempty:n { #1 }
      \mode_if_math:TF {
        \cs:w latexml@invisible@math\cs_end:{
          \tl_use:N \l_@@_arg_tl
        }
      } {
        \cs:w latexml@invisible@text\cs_end:{
          \tl_use:N \l_@@_arg_tl
        }
      }
    }
    \cs_new_protected:Nn \stex_annotate_invisible:nnn {
      \_@@_checkempty:n { #3 }
      \cs:w latexml@annotate@invisible\cs_end:{#1}{#2}{
        \tl_use:N \l_@@_arg_tl
      }
    }
    \NewDocumentEnvironment{stex_annotate_env} { m m } {
      \par\begin{latexml@annotateenv}{#1}{#2}
    }{
      \par\end{latexml@annotateenv}
    }
  }{
    \cs_new_protected:Nn \stex_annotate:nnn {#3}
    \cs_new_protected:Nn \stex_annotate_invisible:n {}
    \cs_new_protected:Nn \stex_annotate_invisible:nnn {}
    \NewDocumentEnvironment{stex_annotate_env} { m m } {}{}
  }
}
%    \end{macrocode}
% \end{macro}
% \end{environment}
%
% \subsection{Babel Languages}
%    \begin{macrocode}
%<@@=stex_language>
%    \end{macrocode}
%
% \begin{variable}{\c_stex_languages_prop,\c_stex_language_abbrevs_prop}
%
% We store language abbreviations in two (mutually inverse) 
% property lists:
%    \begin{macrocode}
\prop_const_from_keyval:Nn \c_stex_languages_prop {
  en = english ,
  de = ngerman ,
  ar = arabic ,
  bg = bulgarian ,
  ru = russian ,
  fi = finnish ,
  ro = romanian ,
  tr = turkish ,
  fr = french
}

\prop_const_from_keyval:Nn \c_stex_language_abbrevs_prop {
  english   = en ,
  ngerman   = de ,
  arabic    = ar ,
  bulgarian = bg ,
  russian   = ru ,
  finnish   = fi ,
  romanian  = ro ,
  turkish   = tr ,
  french    = fr
}
% todo: chinese simplified (zhs)
%       chinese traditional (zht)
%    \end{macrocode}
% \end{variable}
%
% we use the |lang|-package option to load the corresponding
% babel languages:
%
%    \begin{macrocode}
\clist_if_empty:NF \c_stex_languages_clist {
  \clist_clear:N \l_tmpa_clist
  \clist_map_inline:Nn \c_stex_languages_clist {
    \prop_get:NnNTF \c_stex_languages_prop { #1 } \l_tmpa_str {
      \clist_put_right:No \l_tmpa_clist \l_tmpa_str
    } {
      \msg_error:nnx{stex}{error/unknownlanguage}{\l_tmpa_str}
    }
  }
  \stex_debug:nn{lang} {Languages:~\clist_use:Nn \l_tmpa_clist {,~} }
  \RequirePackage[\clist_use:Nn \l_tmpa_clist,]{babel}
}
%    \end{macrocode}
%
% \subsection{Auxiliary Methods}
%
% \begin{macro}{\stex_deactivate_macro:Nn}
%    \begin{macrocode}
\cs_new_protected:Nn \stex_deactivate_macro:Nn {
  \exp_after:wN\let\csname \detokenize{#1} - orig\endcsname#1
  \def#1{
    \msg_error:nnnn{stex}{error/deactivated-macro}{#1}{#2}
  }
}
%    \end{macrocode}
% \end{macro}
%
% \begin{macro}{\stex_reactivate_macro:N}
%    \begin{macrocode}
\cs_new_protected:Nn \stex_reactivate_macro:N {
  \exp_after:wN\let\exp_after:wN#1\csname \detokenize{#1} - orig\endcsname
}
%    \end{macrocode}
% \end{macro}
%
% \begin{macro}{\ignorespacesandpars}
%    \begin{macrocode}
\protected\def\ignorespacesandpars{
  \begingroup\catcode13=10\relax
  \@ifnextchar\par{
    \endgroup\expandafter\ignorespacesandpars\@gobble
  }{
    \endgroup
  }
}
%</package>
%    \end{macrocode}
% \end{macro}
%
% \end{implementation}
%
% \PrintIndex

% \fi
%
%
% \section{Macros and Environments}\label{pkg:basics:doc:macros}
%
% \begin{function}{\sTeX , \stex}
%   Both print this \stex logo.
% \end{function}
%
% \begin{function}{\stex_debug:nn}
%   \begin{syntax}
%     \cs{stex_debug:nn} \Arg{log-prefix} \Arg{message} ^^A \meta{comma list}
%   \end{syntax}
% Logs \meta{message}, if the package option |debug| contains \meta{log-prefix}.
% \end{function}
%
% \subsection{HTML Annotations}
%
% \begin{function}{\if@latexml}
%   \LaTeX2e conditional for \latexml
% \end{function}
%
% \begin{function}[pTF]{\latexml_if:}
%   \LaTeX3 conditionals for \latexml.
% \end{function}
%
% \begin{function}[pTF]{\stex_if_do_html:}
%   Whether to currently produce any HTML annotations (can be false
%   in some advanced structuring environments, for example)
% \end{function}
%
% \begin{function}{\stex_suppress_html:n}
%   Temporarily disables HTML annotations in its argument code
% \end{function}
% 
%
% We have four macros for annotating generated HTML (via \latexml
% or \rustex) with attributes:
%
% \begin{function}{\stex_annotate:nnn, \stex_annotate_invisible:nnn,
%   \stex_annotate_invisible:n}
%   \begin{syntax} \cs{stex_annotate:nnn} \Arg{property} \Arg{resource} \Arg{content} \end{syntax}
% Annotates the HTML generated by \meta{content} with\\
% \begin{center}
%  |property="stex:|\meta{property}|", resource="|\meta{resource}|"|.
% \end{center}
%
% \cs{stex_annotate_invisible:n} adds the attributes\\
% \begin{center}
% |stex:visible="false", style="display:none"|.
% \end{center}
%
% \cs{stex_annotate_invisible:nnn} combines the functionality of both.
% \end{function}
%
% \begin{environment}{stex_annotate_env}
%   \begin{syntax} \cs{begin}|{stex_annotate_env}|\Arg{property}\Arg{resource}
%       \meta{content}
%     \cs{end}|{stex_annotate_env}|
%   \end{syntax}
% behaves like \cs{stex_annotate:nnn} \Arg{property} \Arg{resource}
%     \Arg{content}.
% \end{environment}
%
% \subsection{Babel Languages}
%
% \begin{variable}{\c_stex_languages_prop,\c_stex_language_abbrevs_prop}
%   Map language abbreviations to their full babel names and vice versa.
%   e.g. \cs{c_stex_languages_prop}|{en}| yields |english|, and
%   \cs{c_stex_language_abbrevs_prop}|{english}| yields |en|.
% \end{variable}
%
% \subsection{Auxiliary Methods}
%
% \begin{function}{\stex_deactivate_macro:Nn , \stex_reactivate_macro:N}
%   \begin{syntax}\cs{stex_deactivate_macro:Nn}\meta{cs}\Arg{environments}\end{syntax}
%   Makes the macro \meta{cs} throw an error, indicating that it
%   is only allowed in the context of \meta{environments}.
%
%   \cs{stex_reactivate_macro:N}\meta{cs} reactivates it again, i.e.
%   this happens ideally in the \meta{begin}-code of the associated
%   environments.
% \end{function}
%
% \begin{function}{\ignorespacesandpars}
%   ignores white space characters and |\par| control sequences.
%   Expands tokens in the process.
% \end{function}
%
% \end{documentation}
%
% \begin{implementation}
%
% \section{\sTeX-Basics Implementation}\label{pkg:basics:impl}
%
%   \subsection{The \sTeX Document Class}
%
% The \cls{stex} document class is pretty straight-forward: It largely extends the \cls{standalone} package
% and loads the \pkg{stex} package, passing all provided options on to the package.
%
%    \begin{macrocode}
%<*cls>

%%%%%%%%%%%%%   basics.dtx   %%%%%%%%%%%%%

\RequirePackage{expl3,l3keys2e}
\ProvidesExplClass{stex}{2022/02/24}{3.0.0}{sTeX document class}
\LoadClass[border=1px,varwidth]{standalone}
\setlength\textwidth{15cm}

\DeclareOption*{\PassOptionsToPackage{\CurrentOption}{stex}}
\ProcessOptions

\RequirePackage{stex}
%</cls>
%    \end{macrocode}
%
% \subsection{Preliminaries}
%
%    \begin{macrocode}
%<*package>

%%%%%%%%%%%%%   basics.dtx   %%%%%%%%%%%%%

\RequirePackage{expl3,l3keys2e,ltxcmds}
\ProvidesExplPackage{stex}{2022/02/24}{3.0.0}{sTeX package}

%\RequirePackage{morewrites}
%\RequirePackage{amsmath}

%    \end{macrocode}
%
% Package options:
%
%    \begin{macrocode}
\keys_define:nn { stex } {
  debug     .clist_set:N  = \c_stex_debug_clist ,
  lang      .clist_set:N  = \c_stex_languages_clist ,
  mathhub   .tl_set_x:N   = \mathhub ,
  sms       .bool_set:N   = \c_stex_persist_mode_bool ,
  image     .bool_set:N   = \c_tikzinput_image_bool,
  unknown   .code:n       = {}
}
\ProcessKeysOptions { stex }
%    \end{macrocode}
%
% \begin{macro}{\stex,\sTeX}
%   The \sTeX logo:
%
%    \begin{macrocode}
\protected\def\stex{%
  \@ifundefined{texorpdfstring}%
  {\let\texorpdfstring\@firstoftwo}%
  {}%
  \texorpdfstring{\raisebox{-.5ex}S\kern-.5ex\TeX}{sTeX}\xspace%
}
\def\sTeX{\stex}
%    \end{macrocode}
% \end{macro}
%
%
% \subsection{Messages and logging}
%
%    \begin{macrocode}
%<@@=stex_log>
%    \end{macrocode}
%
% Warnings and error messages
%
%    \begin{macrocode}
\msg_new:nnn{stex}{error/unknownlanguage}{
  Unknown~language:~#1
}
\msg_new:nnn{stex}{warning/nomathhub}{
  MATHHUB~system~variable~not~found~and~no~
  \detokenize{\mathhub}-value~set!
}
\msg_new:nnn{stex}{error/deactivated-macro}{
  The~\detokenize{#1}~command~is~only~allowed~in~#2!
}
%    \end{macrocode}
% 
% \begin{macro}{\stex_debug:nn}
%
%  A simple macro issuing package messages with subpath.
%
%    \begin{macrocode}
\cs_new_protected:Nn \stex_debug:nn {
  \clist_if_in:NnTF \c_stex_debug_clist { all } {
    \exp_args:Nnnx\msg_set:nnn{stex}{debug / #1}{
      \\Debug~#1:~#2\\
    }
    \msg_none:nn{stex}{debug / #1}
  }{
    \clist_if_in:NnT \c_stex_debug_clist { #1 } {
      \exp_args:Nnnx\msg_set:nnn{stex}{debug / #1}{
        \\Debug~#1:~#2\\
      }
      \msg_none:nn{stex}{debug / #1}
    }  
  }
}
%    \end{macrocode}
% \end{macro}
%
% Redirecting messages:
%
%    \begin{macrocode}
\clist_if_in:NnTF \c_stex_debug_clist {all} {
    \msg_redirect_module:nnn{ stex }{ none }{ term }
}{
  \clist_map_inline:Nn \c_stex_debug_clist {
    \msg_redirect_name:nnn{ stex }{ debug / ##1 }{ term }
  }
}

\stex_debug:nn{log}{debug~mode~on}
%    \end{macrocode}
%
%
% \subsection{HTML Annotations}
%    \begin{macrocode}
%<@@=stex_annotate>
\RequirePackage{rustex}
%    \end{macrocode}
%
% We add the namespace abbreviation |ns:stex="http://kwarc.info/ns/sTeX"| to \rustex:
%
%    \begin{macrocode}
\rustex_add_Namespace:nn{stex}{http://kwarc.info/ns/sTeX}
%    \end{macrocode}
%
% Conditionals for \latexml:
%
% \begin{macro}{\if@latexml}
%    \begin{macrocode}
\ifcsname if@latexml\endcsname\else
    \expandafter\newif\csname if@latexml\endcsname\@latexmlfalse
\fi
%    \end{macrocode}
% \end{macro}
%
% \begin{macro}[pTF]{\latexml_if:}
%    \begin{macrocode}
\prg_new_conditional:Nnn \latexml_if: {p, T, F, TF} {
  \if@latexml
    \prg_return_true:
  \else:
    \prg_return_false:
  \fi:
}
%    \end{macrocode}
% \end{macro}
%
% \begin{variable}{\l_@@_arg_tl, \c_@@_emptyarg_tl}
%
% Used by annotation macros to ensure that the HTML output to annotate
% is not empty.
%
%    \begin{macrocode}
\tl_new:N \l_@@_arg_tl
\tl_const:Nx \c_@@_emptyarg_tl {
  \rustex_if:TF {
    \rustex_direct_HTML:n { \c_ampersand_str lrm; }
  }{~}
}
%    \end{macrocode}
% \end{variable}
%
% \begin{macro}{\_@@_checkempty:n}
%    \begin{macrocode}
\cs_new_protected:Nn \_@@_checkempty:n {
  \tl_set:Nn \l_@@_arg_tl { #1 }
  \tl_if_empty:NT \l_@@_arg_tl {
    \tl_set_eq:NN \l_@@_arg_tl \c_@@_emptyarg_tl
  }
}
%    \end{macrocode}
% \end{macro}
%
% \begin{macro}[pTF]{\stex_if_do_html:}
%  Whether to (locally) produce HTML output
%    \begin{macrocode}
\bool_new:N \_stex_html_do_output_bool
\bool_set_true:N \_stex_html_do_output_bool

\prg_new_conditional:Nnn \stex_if_do_html: {p,T,F,TF} {
  \bool_if:nTF \_stex_html_do_output_bool
    \prg_return_true: \prg_return_false:
}
%    \end{macrocode}
% \end{macro}
%
% \begin{macro}{\stex_suppress_html:n}
%  Whether to (locally) produce HTML output
%    \begin{macrocode}
\cs_new_protected:Nn \stex_suppress_html:n {
  \exp_args:Nne \use:nn {
    \bool_set_false:N \_stex_html_do_output_bool
    #1
  }{
    \stex_if_do_html:T {
      \bool_set_true:N \_stex_html_do_output_bool
    }
  }
}
%    \end{macrocode}
% \end{macro}
%
%
% \begin{environment}{stex_annotate_env}
% \begin{macro}{\stex_annotate:nnn, \stex_annotate_invisible:n,
%    \stex_annotate_invisible:nnn}
%
% We define four macros for introducing attributes in the HTML
% output. The definitions depend on the ``backend'' used
% (\latexml, \rustex, \texttt{pdflatex}). 
%
% The \texttt{pdflatex}-macros largely do nothing; the
% \rustex-implementations are pretty clear in what they do,
%  the \latexml-implementations resort to perl bindings.
%
%    \begin{macrocode}
\rustex_if:TF{
  \cs_new_protected:Nn \stex_annotate:nnn {
    \_@@_checkempty:n { #3 }
    \rustex_annotate_HTML:nn {
      property="stex:#1" ~
      resource="#2"
    } {
      \mode_if_vertical:TF{
        \tl_use:N \l_@@_arg_tl\par
      }{
        \tl_use:N \l_@@_arg_tl
      }
    }
  }
  \cs_new_protected:Nn \stex_annotate_invisible:n {
    \_@@_checkempty:n { #1 }
    \rustex_annotate_HTML:nn {
      stex:visible="false" ~
      style:display="none"
    } {
      \mode_if_vertical:TF{
        \tl_use:N \l_@@_arg_tl\par
      }{
        \tl_use:N \l_@@_arg_tl
      }
    }
  }
  \cs_new_protected:Nn \stex_annotate_invisible:nnn {
    \_@@_checkempty:n { #3 }
    \rustex_annotate_HTML:nn {
      property="stex:#1" ~
      resource="#2" ~
      stex:visible="false" ~
      style:display="none"
    } {
      \mode_if_vertical:TF{
        \tl_use:N \l_@@_arg_tl\par
      }{
        \tl_use:N \l_@@_arg_tl
      }
    }
  }
  \NewDocumentEnvironment{stex_annotate_env} { m m } {
    \par
    \rustex_annotate_HTML_begin:n {
      property="stex:#1" ~
      resource="#2"
    }
  }{
    \par\rustex_annotate_HTML_end:
  }
}{
  \latexml_if:TF {
    \cs_new_protected:Nn \stex_annotate:nnn {
      \_@@_checkempty:n { #3 }
      \mode_if_math:TF {
        \cs:w latexml@annotate@math\cs_end:{#1}{#2}{
          \tl_use:N \l_@@_arg_tl
        }
      }{
        \cs:w latexml@annotate@text\cs_end:{#1}{#2}{
          \tl_use:N \l_@@_arg_tl
        }
      }
    }
    \cs_new_protected:Nn \stex_annotate_invisible:n {
      \_@@_checkempty:n { #1 }
      \mode_if_math:TF {
        \cs:w latexml@invisible@math\cs_end:{
          \tl_use:N \l_@@_arg_tl
        }
      } {
        \cs:w latexml@invisible@text\cs_end:{
          \tl_use:N \l_@@_arg_tl
        }
      }
    }
    \cs_new_protected:Nn \stex_annotate_invisible:nnn {
      \_@@_checkempty:n { #3 }
      \cs:w latexml@annotate@invisible\cs_end:{#1}{#2}{
        \tl_use:N \l_@@_arg_tl
      }
    }
    \NewDocumentEnvironment{stex_annotate_env} { m m } {
      \par\begin{latexml@annotateenv}{#1}{#2}
    }{
      \par\end{latexml@annotateenv}
    }
  }{
    \cs_new_protected:Nn \stex_annotate:nnn {#3}
    \cs_new_protected:Nn \stex_annotate_invisible:n {}
    \cs_new_protected:Nn \stex_annotate_invisible:nnn {}
    \NewDocumentEnvironment{stex_annotate_env} { m m } {}{}
  }
}
%    \end{macrocode}
% \end{macro}
% \end{environment}
%
% \subsection{Babel Languages}
%    \begin{macrocode}
%<@@=stex_language>
%    \end{macrocode}
%
% \begin{variable}{\c_stex_languages_prop,\c_stex_language_abbrevs_prop}
%
% We store language abbreviations in two (mutually inverse) 
% property lists:
%    \begin{macrocode}
\prop_const_from_keyval:Nn \c_stex_languages_prop {
  en = english ,
  de = ngerman ,
  ar = arabic ,
  bg = bulgarian ,
  ru = russian ,
  fi = finnish ,
  ro = romanian ,
  tr = turkish ,
  fr = french
}

\prop_const_from_keyval:Nn \c_stex_language_abbrevs_prop {
  english   = en ,
  ngerman   = de ,
  arabic    = ar ,
  bulgarian = bg ,
  russian   = ru ,
  finnish   = fi ,
  romanian  = ro ,
  turkish   = tr ,
  french    = fr
}
% todo: chinese simplified (zhs)
%       chinese traditional (zht)
%    \end{macrocode}
% \end{variable}
%
% we use the |lang|-package option to load the corresponding
% babel languages:
%
%    \begin{macrocode}
\clist_if_empty:NF \c_stex_languages_clist {
  \clist_clear:N \l_tmpa_clist
  \clist_map_inline:Nn \c_stex_languages_clist {
    \prop_get:NnNTF \c_stex_languages_prop { #1 } \l_tmpa_str {
      \clist_put_right:No \l_tmpa_clist \l_tmpa_str
    } {
      \msg_error:nnx{stex}{error/unknownlanguage}{\l_tmpa_str}
    }
  }
  \stex_debug:nn{lang} {Languages:~\clist_use:Nn \l_tmpa_clist {,~} }
  \RequirePackage[\clist_use:Nn \l_tmpa_clist,]{babel}
}
%    \end{macrocode}
%
% \subsection{Auxiliary Methods}
%
% \begin{macro}{\stex_deactivate_macro:Nn}
%    \begin{macrocode}
\cs_new_protected:Nn \stex_deactivate_macro:Nn {
  \exp_after:wN\let\csname \detokenize{#1} - orig\endcsname#1
  \def#1{
    \msg_error:nnnn{stex}{error/deactivated-macro}{#1}{#2}
  }
}
%    \end{macrocode}
% \end{macro}
%
% \begin{macro}{\stex_reactivate_macro:N}
%    \begin{macrocode}
\cs_new_protected:Nn \stex_reactivate_macro:N {
  \exp_after:wN\let\exp_after:wN#1\csname \detokenize{#1} - orig\endcsname
}
%    \end{macrocode}
% \end{macro}
%
% \begin{macro}{\ignorespacesandpars}
%    \begin{macrocode}
\protected\def\ignorespacesandpars{
  \begingroup\catcode13=10\relax
  \@ifnextchar\par{
    \endgroup\expandafter\ignorespacesandpars\@gobble
  }{
    \endgroup
  }
}
%</package>
%    \end{macrocode}
% \end{macro}
%
% \end{implementation}
%
% \PrintIndex


\chapter{Stuff}

\section{Modules}


\begin{function}{\sTeX , \stex}
  Both print this \stex logo.
\end{function}

 \subsection{Semantic Macros and Notations}

 Semantic macros invoke a formally declared symbol.

 To declare a symbol (in a module), we use \cs{symdecl},
 which takes as argument the name of the corresponding
 semantic macro, e.g. |\symdecl{foo}| introduces the macro
 \cs{foo}. Additionally, \cs{symdecl} takes several options,
 the most important one being its arity. |foo| as declared above
 yields a \emph{constant} symbol. To introduce an \emph{operator}
 which takes arguments, we have to specify which arguments it takes.

 \begin{@module}{SemanticMacrosExample}
   For example, to introduce binary multiplication,
   we can do |\symdecl[args=2]{mult}|. We can then supply
   the semantic macro with arbitrarily many notations, such as
   |\notation{mult}{#1 #2}|.
   
   \stexexample{
 \symdecl[args=2]{mult}
 \notation{mult}{#1 #2}
 $\mult{a}{b}$
}

 Since usually, a freshly introduced symbol also comes with a
 notation from the start, the \cs{symdef} command combines
 \cs{symdecl} and \cs{notation}. So instead of the above,
 we could have also written
 \begin{center} |\symdef[args=2]{mult}{#1 #2}| \end{center}

 \symdecl[args=2]{mult}
 \notation{mult}{#1 #2}

   \notation[cdot]{mult}{#1 \comp{\cdot} #2}
   \notation[times]{mult}{#1 \comp{\times} #2}
   Adding more notations like
   |\notation[cdot]{mult}{#1 \comp{\cdot} #2}| or 
   |\notation[times]{mult}{#1 \comp{\times} #2}|
   allows us to write |$\mult[cdot]{a}{b}$| and
   |$\mult[times]{a}{b}$|:
   \stexexample{
   \notation[cdot]{mult}{#1 \comp{\cdot} #2}
   \notation[times]{mult}{#1 \comp{\times} #2}
 $\mult[cdot]{a}{b}$ and $\mult[times]{a}{b}$
}
   \notation[cdot]{mult}{#1 \comp{\cdot} #2}
   \notation[times]{mult}{#1 \comp{\times} #2}

   Not using an explicit option with a semantic macro yields
   the first declared notation, unless changed\ednote{TODO}.

   Outside of math mode, or by using the starred variant
   |\foo*|, allows to provide a custom notation, where
   notational (or textual) components can be given
   explicitly in square brackets.
   \stexexample{
 $\mult*{a}[\comp{\ast}]{b}$ is the 
 \mult[\comp{product of} ]{$a$}[ \comp{and} ]{$b$}
}

   In custom mode, prefixing an argument with a star will not
   print that argument, but still export it to \omdoc:
   \stexexample{
 \mult[\comp{Multiplying}]*{$\mult{a}{b}$}[ again by ]{$b$} yields...
}
   The syntax |*[|\meta{int}|]| allows switching
   the order of arguments. For example, given a 2-ary semantic
   macro |\forevery| with exemplary notation
   |\forall #1. #2|, we can write
   \stexexample{
     \symdecl[args=2]{forevery}
     \forevery*[2]{The proposition $P$}[ \comp{holds for every} ]*[1]{$x\in A$}
}

 When using |*[|$n$|]|, after reading the provided ($n$th) argument,
  the ``argument counter'' automatically 
 continues where we left off, so the |*[1]| in the above example
 can be omitted.

   For a macro with arity $>0$, we can refer to the operator
   \emph{itself} semantically by suffixing the semantic macro
   with an exclamation point |!| in either text or math mode.
   For that reason \cs{notation} (and thus \cs{symdef}) take an
   additional optional argument |op=|, which allows to assign
   a notation for the operator itself. e.g.
   \stexexample{
     \symdef[args=2,op={+}]{add}{#1 \comp+ #2}
     The operator $\add!$ adds two elements, as in $\add ab$.
   }

  |*| is composable with |!| for custom notations, as in:

   \stexexample{
 \mult![\comp{Multiplication}] (denoted by $\mult*![\comp\cdot]$) is defined by...
}

 The macro \cs{comp} as used everywhere above is responsible
 for highlighting, linking, and tooltips, and should be wrapped
 around the notation (or text) components that should be treated
 accordingly. While it is attractive to just wrap a whole notation,
 this would also wrap around e.g. the arguments themselves, so
 instead, the user is tasked with marking the notation components
 themself.

 The precise behaviour of \cs{comp} is governed by
 the macro \cs{@comp}, which takes two arguments: The tex code
 of the text
 (unexpanded) to highlight, and the URI of the current symbol.
 \cs{@comp} can be safely redefined to customize the behaviour.


 The starred variant |\symdecl*{foo}| does not introduce a semantic
 macro, but still declares a corresponding symbol. |foo| (like
 any other symbol, for that matter) can
 then be accessed via \cs{STEXsymbol}|{foo}| or (if |foo| was declared
 in a module |Foo|) via \cs{STEXModule}|{Foo}?{foo}|.

 both \cs{STEXsymbol} and \cs{STEXModule} take any
 arbitrary ending segment of a full URI to determine
 which symbol or module is meant. e.g.
 \cs{STEXsymbol}|{Foo?foo}| is also valid, as are e.g.
 \cs{STEXModule}|{path?Foo}?{foo}| or
 \cs{STEXsymbol}|{path?Foo?foo}|

 There's also a convient shortcut \cs{symref}|{?foo}{some text}| for
 \cs{STEXsymbol}|{?foo}![some text]|.

 \end{@module}

 \subsubsection{Other Argument Types}

 So far, we have stated the arity of a semantic macro directly.
 This works if we only have ``normal'' (or more precisely: |i|-type) arguments.
  To make use of other argument types, instead of providing the arity
 numerically, we can provide it as a sequence of characters representing
 the argument types -- e.g. instead of writing |args=2|, we
 can equivalently write |args=ii|, indicating that the macro
 takes two |i|-type arguments.

 Besides |i|-type arguments, \sTeX has two other types, which we will
 discuss now.

 The first are \emph{binding} (|b|-type) arguments, representing
 variables that are \emph{bound} by the operator. This is the
 case for example in the above \cs{forevery}-macro:
 The first argument is not actually an argument that the
 |forevery| ``function'' is ``applied'' to; rather, the first argument
 is a new variable (e.g. $x$) that is \emph{bound} in the subsequent
 argument. More accurately, the macro should therefore have been
 implemented thusly:
   \begin{center}|\symdef[args=bi]{forevery}{\forall #1.\; #2}|\end{center}

 \begin{@module}{OtherArgs}
 |b|-type arguments are indistinguishable from |i|-type arguments
 within \sTeX, but are treated very differently in \omdoc and by \mmt.
 More interesting \emph{within} \sTeX are |a|-type arguments,
 which represent (associative) arguments of flexible arity, which are
 provided as comma-separated lists.
 This allows e.g. better representing the \cs{mult}-macro above:
 
   \stexexample{
 \symdef[args=a]{mult}{#1}{#1 \comp\cdot #2}
 $\mult{a,b,c,{d^e},f}$
}
 As the example above shows, notations get a little more complicated
 for associative arguments. For every |a|-type argument, the
 \cs{notation}-macro takes an additional argument that declares
 how individual entries in an |a|-type argument list are aggregated.
 The first notation argument then describes how the aggregated
 expression is combined into the full representation.

 For a more interesting example, consider a flexary operator
 for ordered sequences in ordered set, that taking 
 arguments |{a,b,c}| and |\mathbb{R}| prints
 $a \leq b \leq c\in \mathbb R$. This operator takes
 two arguments (an |a|-type argument and an |i|-type argument),
 aggregates the individuals of the associative argument using |\leq|,
 and combines the result with |\in| and the second argument thusly:

   \stexexample{
 \symdef[args=ai]{numseq}{#1 \comp\in #2}{#1 \comp\leq #2}
 $\numseq{a,b,c}{\mathbb R}$
}

 Finally, |B|-type arguments combine the functionalities of |a|
 and |b|, i.e. they represent flexary binding operator arguments.

\ednote{what about e.g. \detokenize{\int_x\int_y\int_z f dx dy dz}?}
\ednote{``decompose'' a-type arguments into fixed-arity operators?}

 \end{@module}

 \subsubsection{Precedences}

 Every notation has an (upwards) \emph{operator precedence} and
 for each argument a (downwards) \emph{argument precedence}
 used for automated bracketing. For example, a notation
 for a binary operator \cs{foo} could be declared like this:
 \begin{center} |\notation[prec=200;500x600]{foo}{#1 \comp{+} #2}| \end{center}
 assigning an operator precedence of 200, an argument precedence
 of 500 for the first argument, and an argument precedence of 600
 for the second argument.

 \sTeX insert brackets thusly: Upon encountering a semantic
 macro (such as \cs{foo}), its operator precedence (e.g. 200)
 is compared to the current downwards precedence (initially 
 \cs{neginfprec}). If the operator precedence is \emph{larger}
 than the current downwards precedence, parentheses are inserted
 around the semantic macro.

 Notations for symbols of arity 0 have a default precedence of \cs{infprec},
 i.e. by default, parentheses are never inserted around constants.
 Notations for symbols with arity $>0$ have a default operator
 precedence of $0$.
 If no argument precedences are explicitly provided, then by
 default they are equal to the operator precedence.

 Consequently, if some operator $A$ should bind stronger than
 some operator $B$, then $A$s operator precedence should be
 smaller than $B$s argument precedences.

 For example:
 \begin{@module}{NotationsEx}
 \symdecl[args=2]{plus}
 \symdecl[args=2]{times}
 \stexexample{
\notation[prec=100]{plus}{#1 \comp{+} #2}
\notation[prec=50]{times}{#1 \comp{\cdot} #2}
$\plus{a}{\times{b}{c}}$ and $\times{a}{\plus{b}{c}}$
}


 \end{@module}

 \subsection{Archives and Imports}

 \subsubsection{Namespaces}
   Ideally, \sTeX would use arbitrary URIs for modules, with no
   forced relationships between the \emph{logical} namespace
   of a module and the \emph{physical} location of the file
   declaring the module -- like \mmt does things.

   Unfortunately, \TeX\ only provides very restricted access to
   the file system, so we are forced to generate namespaces
   systematically in such a way that they reflect the physical
   location of the associated files, so that \sTeX can resolve
   them accordingly. Largely, users need not concern themselves
   with namespaces at all, but for completenesses sake, we describe
   how they are constructed:

   \begin{itemize}
     \item If \cs{begin}|{module}{Foo}| occurs in a file
       |/path/to/file/Foo[.|\meta{lang}|].tex| which does not belong
       to an archive, the namespace is |file://path/to/file|.
     \item If the same statement occurs in a file
       |/path/to/file/bar[.|\meta{lang}|].tex|, the namespace is 
       |file://path/to/file/bar|.
   \end{itemize}

   In other words: outside of archives, the namespace corresponds to
   the file URI with the filename dropped iff it is equal to the
   module name, and ignoring the (optional) language suffix^^A
   \footnote{which is internally attached to the module name instead,
   but a user need not worry about that.}.

   If the current file is in an archive, the procedure is the same
   except that the initial segment of the file path up to the archive's
   |source|-folder is replaced by the archive's namespace URI.

 \subsubsection{Paths in Import-Statements}

 Conversely, here is how namespaces/URIs and file paths are computed
 in import statements, examplary \cs{importmodule}:

 \begin{itemize}
   \item \cs{importmodule}|{Foo}| outside of an archive refers 
     to module |Foo| in the current namespace. Consequently, |Foo|
     must have been declared earlier in the same document or, if not,
     in a file |Foo[.|\meta{lang}|].tex| in the same directory.
   \item The same statement \emph{within} an archive refers to either
     the module |Foo| declared earlier in the same document, or
     otherwise to the module |Foo| in the archive's top-level namespace.
     In the latter case, is has to be declared in a file |Foo[.|\meta{lang}|].tex|
     directly in the archive's |source|-folder.
   \item Similarly, in \cs{importmodule}|{some/path?Foo}| the path
     |some/path| refers to either the sub-directory and relative 
     namespace path of the current directory and namespace outside of an archive,
     or relative to the current archive's top-level namespace and |source|-folder,
     respectively.

     The module |Foo| must either be declared in the file
     \meta{top-directory}|/some/path/Foo[.|\meta{lang}|].tex|, or in
     \meta{top-directory}|/some/path[.|\meta{lang}|].tex| (which are
     checked in that order).
   \item Similarly, \cs{importmodule}|[Some/Archive]{some/path?Foo}|
     is resolved like the previous cases, but relative to the archive
     |Some/Archive| in the mathhub-directory.
   \item Finally, \cs{importmodule}|{full://uri?Foo}| naturally refers to the
     module |Foo| in the namespace |full://uri|. Since the file this module
     is declared in can not be determined directly from the URI, the module
     must be in memory already, e.g. by being referenced earlier in the
     same document.

     Since this is less compatible with a modular development, using full
     URIs directly is discouraged.

 \end{itemize} 


	
	
\csname if@infulldoc\endcsname\else\end{document}\fi
