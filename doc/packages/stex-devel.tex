\usemodule[sTeX/Documentation]{macros?AllMacros}
%\begin{smodule}{Aux Macros}

  \begin{function}{\stex_debug:nn}
    \begin{syntax}
      \cs{stex_debug:nn} \marg{prefix} \marg{msg}
    \end{syntax}
    Logs the debug message \marg{msg} under the prefix
    \marg{prefix}. A message is shown if its prefix
    is in a list of prefixes given either via the
    package option |debug=|\meta{prefixes} or
    the environment variable |STEX_DEBUG=|\meta{prefixes},
    where the latter overrides the former.
  \end{function}

\begin{sfragment}{Language Handling}
  \begin{variable}{\c_stex_languages_prop,\c_stex_language_abbrevs_prop}
    Property lists converting babel languages to/from their abreviations;
    e.g.
    \begin{itemize}
      \item |\prop_item:Nn \c_stex_languages_prop {de}| yields |ngerman|, and 
      \item |\c_stex_language_abbrevs_prop {ngerman}| yields |de|.
    \end{itemize}  
  \end{variable}
  \begin{variable}{\l_stex_current_language_str}
    Always stores the current language.
  \end{variable}

  \begin{function}{\stex_set_language:n, \stex_set_language:x, \stex_set_language:o}
    \begin{syntax}\cs{stex_set_language:n}\marg{abbrev}\end{syntax}
    Sets \cs{l_stex_current_language_str}, and, if the \pkg{babel}
    package is loaded, calls \cs{selectlangage} on the
    language corresponding to \marg{abbrev}.

    If \marg{abbrev}|=tr|, additionally call \cs{selectlanguage}
    with the option |shorthands=:!|.

    Throws |error/unknownlanguage| if no language with abbreviation
    \marg{abbrev} is known.
  \end{function}

  \begin{function}{\stex_language_from_file:}
    Infers the current language from file ending (e.g. |.en.tex|)
    and sets it appropriately.

    Is called in a file hook, i.e. always switches language when inputting
    a file |.<lang>.<ext>|.
  \end{function}

\end{sfragment}

\begin{sfragment}{Inserting Annotations}
  \stex can be used to produce either \HTML or \PDF. In \HTML-mode,
  multiple macros exist to insert annotations. The same macros
  are also valid in \PDF mode but implemented as null operations.

  \begin{function}{\stex_suppress_html:n}
    \begin{syntax}\cs{stex_suppress_html:n}\marg{code}\end{syntax}
    Turns annotations off temporarily in \marg{code} (e.g. as to not 
    generate additional annotations for elaborated declarations, 
    or in sms-mode).
  \end{function}

  For that to work, code that inserts annotations should use
  \begin{function}[pTF]{\stex_if_do_html:}
    Whether to generate \HTML annotations.
  \end{function}

  \begin{function}{\stex@backend}
    Should be set by a backend engine, such that a file
    |stex-backend-|\cs{stex@backend}|.cfg| exists.
  \end{function}

  Such a backend config file should provide the following:

  \begin{function}[pTF]{\stex_if_html_backend:}
    Can be used to determine whether the backend produces \HTML (e.g.
    \rustex or \LaTeXML) or not (e.g. |pdflatex|).
    
    \cs{ifstexhtml} is set accordingly.
  \end{function}

  \begin{function}{\stex_annotate:nnn}
    \begin{syntax} \cs{stex_annotate:nnn}\marg{attr}\marg{value}\marg{code}
    \end{syntax}
    In \HTML mode, annotates the output of \marg{code} with the 
    \XML-attribute \marg{attr}|="|\marg{value}|"|. In \PDF mode, just
    calls \marg{code}.
  \end{function}

  \begin{function}{\stex_annotate_invisible:nnn,\stex_annotate_invisible:n}
    \begin{syntax} \cs{stex_annotate_invisible:n}\marg{code}
    \end{syntax}

    Should annotate \marg{code} with
    |shtml:visible="false" style="display:none;"|. In \PDF mode, does nothing.

    \cs{stex_annotate_invisible:nnn} combines \cs{stex_annotate_invisible:n}
    and \cs{stex_annotate:nnn}.
  \end{function}

  \DescribeEnv{stex_annotate_env}
    \cs{begin}|{stex_annotate_env}|\marg{key}\marg{value}
    \meta{code}
    \cs{end}|{stex_annotate_env}| should behave
    like \cs{stex_annotate:nnn}\marg{attr}\marg{value}\marg{code}

  \begin{function}{\stex_mathml_intent:nn,\stex_mathml_arg:nn}
    MathML Intent (TODO)
  \end{function}

\end{sfragment}


\begin{sfragment}{Utility Methods}

  \begin{function}{\stex_kpsewhich:Nn}
    \begin{syntax}
      \cs{stex_kpsewhich:Nn} \marg{\cs{macro}} \marg{args}
    \end{syntax}
    Calls ``|kpsewhich| \marg{args}'' and stores the result
    in \cs{macro},
%^^A     Foo
    \begin{texnote} 
      Does not require |shell-escape|
    \end{texnote}
%^^A
%^^A
%^^A     \begin{arguments}
%^^A       \item Narf?
%^^A     \end{arguments}
%^^A
  \end{function}

  \begin{function}{\stex_get_env:Nn}
    \begin{syntax}
      \cs{stex_get_env:Nn} \marg{\cs{macro}} \marg{envvar}
    \end{syntax}
    Stores the value of the environment variable \marg{envvar}
    in \cs{macro}.
  \end{function}

  \begin{function}{\_stex_fatal_error:n,\_stex_fatal_error:nnn,\_stex_fatal_error:nxx}
    Mimic the \cs{msg_error:}-macros, but make sure that \TeX\ stops
    processing.
    \begin{texnote} 
      Calls |\input{non-existent file}|.
    \end{texnote}
  \end{function}

  \begin{sfragment}{Group-like Behaviours}

    \begin{function}{\stex_pseudogroup_with:nn}
      \begin{syntax}\cs{stex_pseudogroup_with:nn}\marg{macros}\marg{code}
      \end{syntax}
      Calls \marg{code} and subsequently restores the values of the
      \marg{macros} given.
      \begin{texnote}
        Does \emph{not} work recursively!
      \end{texnote} 
    \end{function}

    \begin{function}{\stex_pseudogroup:nn}
      \begin{syntax}\cs{stex_pseudogroup:nn}\marg{code1}\marg{code2}
      \end{syntax}
      Expands \marg{code2}, and inserts the result after \marg{code1}. 
      Works recursively and
      allows for restoring the values of macros in combination with
      \cs{stex_pseudogroup_restore:N}, but \emph{only for macros
      that take no arguments}:
    \end{function}

    \begin{function}[EXP]{\stex_pseudogroup_restore:N}
      \begin{syntax}\cs{stex_pseudogroup_restore:N}\marg{macro}
      \end{syntax}
    \end{function}

    \begin{sexample}
      \begin{stexcode}[gobble=8]
        \stex_pseudogroup:nn{
          something changing \foo
          something changing \num
        }{
          \stex_pseudogroup_restore:N \foo
          \int_set:Nn \num {\int_use:N \num}
        }
      \end{stexcode}
      restores the values of macro \cs{foo} and register \cs{num}
      after calling the first block.
    \end{sexample}

    Commands like \cs{symdecl} and \cs{importmodule} that generate
    (semantic) macros should be local \emph{to the current module},
    e.g. \env{smodule}. For that purpose, we open a new ``metagroup''
    with some identifier (e.g. \cs{l_stex_current_module_str})
    and then execute the relevant code \emph{in the metagroup with that
    identifier}:
    
    \begin{function}{\stex_metagroup_new:n, \stex_metagroup_new:o}
      \begin{syntax}\cs{stex_metagroup_new:n} \marg{id}\end{syntax}
      Opens a new metagroup at the current \TeX\ group level with
      identifier \marg{id}.
    \end{function}

    \begin{function}{\stex_metagroup_do_in:nn, \stex_metagroup_do_in:nx}
      \begin{syntax}\cs{stex_metagroup_do_in:nn} \marg{id}\marg{code}\end{syntax}
      Executes \marg{code} and adds its content to \cs{aftergroup} up
      until the \TeX\ group level of the metagroup with identifier \marg{id}.
    \end{function}






  \end{sfragment}

\end{sfragment}
%\end{smodule}

%^^A^^A   \begin{TemplateInterfaceDescription}{foo}
%^^A^^A     \TemplateArgument{1}{Something Here}
%^^A^^A     \TemplateSemantics{Some Narf Here}
%^^A^^A   \end{TemplateInterfaceDescription}
%^^A^^A   \begin{TemplateDescription}{foo}{bar}
%^^A^^A     \TemplateKey{narf}{Something Here}
%^^A^^A     \TemplateSemantics{Some Narf Here}
%^^A^^A   \end{TemplateDescription}
%^^A^^A   \begin{InstanceDescription}{foo}{newinst}{bar}
%^^A^^A     \InstanceKey{narf}{Something Here}
%^^A^^A     \InstanceSemantics{Some Narf Here}
%^^A^^A   \end{InstanceDescription}
%^^A^^A   \cs{stex_kpsewhich:Nn}
%^^A   \begin{function}{\stex_kpsewhich:Nn}
%^^A     Foo
%^^A     \begin{texnote} Foo! \end{texnote}
%^^A
%^^A     \begin{syntax} \cs{stex_kpsewhich:Nn} \meta{something} \Arg{argh}
%^^A     \end{syntax}
%^^A
%^^A     \begin{arguments}
%^^A       \item Narf?
%^^A     \end{arguments}
%^^A
%^^A   \end{function}
%^^A   \begin{function}{\stex_get_env:Nn, \stex_get_env:Nnn}
%^^A     Foo
%^^A   \end{function}
%^^A   \begin{function}{\stex_debug:nn}
%^^A     Foo
%^^A   \end{function}
%^^A   \begin{variable}{\c_stex_pwd_file,\c_stex_main_file}
%^^A     Foo
%^^A   \end{variable}
%^^A   \begin{variable}{\c_stex_languages_prop, \c_stex_language_abbrevs_prop}
%^^A     Foo
%^^A   \end{variable}
%^^A   \begin{variable}{\l_stex_current_language_str}
%^^A     Foo
%^^A   \end{variable}
%^^A   \begin{function}{\stex_set_language:n, \stex_set_language:x, \stex_set_language:o}
%^^A     Foo
%^^A   \end{function}
%^^A   \begin{function}{\stex_pseudogroup:nn,\stex_pseudogroup_restore:N}
%^^A     Foo
%^^A   \end{function}
%^^A   \begin{function}{\stex_pseudogroup_with:nn}
%^^A     Foo
%^^A   \end{function}
%^^A   \begin{function}{\stex_metagroup_new:n, \stex_metagroup_new:o}
%^^A     Foo
%^^A   \end{function}
%^^A   \begin{function}{\stex_metagroup_do_in:nn, \stex_metagroup_do_in:nx}
%^^A     Foo
%^^A   \end{function}
%^^A   \begin{function}[pTF]{\stex_if_do_html:}
%^^A     Whether to currently produce any HTML annotations (can be false
%^^A     in some advanced structuring environments, for example)
%^^A   \end{function}
%^^A   \begin{function}{\stex_suppress_html:n}
%^^A     Foo
%^^A   \end{function}
%^^A   \begin{function}[pTF]{\stex_html_backend:}
%^^A     Foo
%^^A   \end{function}
%^^A   \begin{function}[EXP]{\ifstexhtml}
%^^A     Foo
%^^A   \end{function}
%^^A   \begin{function}{\stex_keys_define:nnnn, \stex_keys_set:nn}
%^^A     Foo
%^^A   \end{function}
%^^A   \begin{function}{\stex_new_stylable_env:nnnnnnn, \stex_new_stylable_cmd:nnnnn, \stex_style_apply:}
%^^A     Foo
%^^A   \end{function}
%^^A   \begin{function}{\stex_deactivate_macro:Nn, \stex_reactivate_macro:N}
%^^A     Foo
%^^A   \end{function}
%^^A   \begin{function}{\ignorespacesandpars}
%^^A     Foo
%^^A   \end{function}
%^^A   \begin{function}[pTF]{\stex_str_if_ends_with:nn}
%^^A     Foo
%^^A   \end{function}
%^^A   \begin{function}[pTF]{\stex_str_if_starts_with:nn}
%^^A     Foo
%^^A   \end{function}
%^^A   \begin{function}[EXP]{\stex_macro_body:N}
%^^A     Foo
%^^A   \end{function}
%^^A   \begin{function}[EXP]{\stex_macro_definition:N}
%^^A     Foo
%^^A   \end{function}
%^^A   \begin{function}{\stex_persist:n,\stex_persist:o,\stex_persist:x}
%^^A     Foo
%^^A   \end{function}
%^^A   \begin{function}{\stex_file_set:Nn, \stex_file_set:No, \stex_file_set:Nx}
%^^A     Foo
%^^A   \end{function}
%^^A   \begin{function}{\stex_file_resolve:Nn, \stex_file_resolve:No, \stex_file_resolve:Nx}
%^^A     Foo
%^^A   \end{function}
%^^A   \begin{function}{\stex_map_uri:Nnnnn}
%^^A     Foo
%^^A   \end{function}
%^^A   \begin{function}{\stex_uri_set:Nn, \stex_uri_set:No, \stex_uri_set:Nx}
%^^A     Foo
%^^A   \end{function}
%^^A   \begin{function}{\stex_uri_resolve:Nn, \stex_uri_resolve:No, \stex_uri_resolve:Nx}
%^^A     Foo
%^^A   \end{function}
%^^A   \begin{function}[EXP]{\stex_file_use:N}
%^^A     Foo
%^^A   \end{function}
%^^A   \begin{function}[EXP]{\stex_uri_use:N}
%^^A     Foo
%^^A   \end{function}
%^^A   \begin{function}{\stex_uri_from_repo_file:NNNn, \stex_uri_from_repo_file_nolang:NNNn}
%^^A     Foo
%^^A   \end{function}
%^^A   \begin{function}{\stex_uri_from_current_file:Nn, \stex_uri_from_current_file_nolang:Nn}
%^^A     Foo
%^^A   \end{function}
%^^A   \begin{function}{\stex_uri_add_module:NNn, \stex_uri_add_module:NNo}
%^^A     Foo
%^^A   \end{function}
%^^A   \begin{function}[pTF]{\stex_file_absolute:N}
%^^A     Foo
%^^A   \end{function}
%^^A   \begin{function}{\stex_filestack_push:n, \stex_filestack_pop:}
%^^A     Foo
%^^A   \end{function}
%^^A   \begin{function}[TF]{\stex_file_starts_with:N}
%^^A     Foo
%^^A   \end{function}
%^^A   \begin{function}{\stex_file_split_off_ext:NN,\stex_file_split_off_lang:NN}
%^^A     Foo
%^^A   \end{function}
%^^A   \begin{variable}{\mathhub,\c_stex_mathhub_file}
%^^A     Foo
%^^A   \end{variable}
%^^A   \begin{function}{\stex_require_repository:n, \stex_require_repository:o}
%^^A     Foo
%^^A   \end{function}
%^^A   \begin{function}{\stex_set_current_repository:n}
%^^A     Foo
%^^A   \end{function}
%^^A   \begin{variable}{\c_stex_mathhub_main_manifest_prop, \l_stex_current_repository_prop}
%^^A     Foo
%^^A   \end{variable}
%^^A   \begin{variable}{\l_stex_current_doc_uri}
%^^A     Foo
%^^A   \end{variable}
%^^A   \begin{function}{\stex_get_document_uri:}
%^^A     Foo
%^^A   \end{function}
%^^A   \begin{function}{\STEXtitle,\stex_document_title:n}
%^^A     Foo
%^^A   \end{function}
%^^A   \begin{function}{\stex_ref_new_doc_target:n,\sreflabel}
%^^A     Foo
%^^A   \end{function}
%^^A   \begin{function}{\stex_ref_new_sym_target:n}
%^^A     Foo
%^^A   \end{function}
%^^A   \begin{function}{\sref,\extref}
%^^A     Foo
%^^A   \end{function}
%^^A   \begin{function}{\stex_sms_allow:N,\stex_sms_allow_escape:N,\stex_sms_allow_env:n}
%^^A     Foo
%^^A   \end{function}
%^^A   \begin{function}{\stex_sms_allow_import:Nn, \stex_sms_allow_import_env:nn}
%^^A     Foo
%^^A   \end{function}
%^^A   \begin{variable}{\g_stex_sms_import_code}
%^^A     Foo
%^^A   \end{variable}
%^^A   \begin{function}[pTF]{\stex_if_smsmode:}
%^^A     Foo
%^^A   \end{function}
%^^A   \begin{function}{\stex_file_in_smsmode:nn,\stex_file_in_smsmode:on}
%^^A     Foo
%^^A   \end{function}
%^^A   \begin{function}{\stex_smsmode_do:}
%^^A     Foo
%^^A   \end{function}
%^^A   \begin{variable}{\l_stex_current_module_str}
%^^A     Foo
%^^A   \end{variable}
%^^A   \begin{function}[EXP]{\stex_current_module_prop:}
%^^A     Foo
%^^A   \end{function}
%^^A   \begin{variable}{\l_stex_all_modules_seq}
%^^A     Foo
%^^A   \end{variable}
%^^A   \begin{function}[pTF]{\stex_if_in_module:}
%^^A     Foo
%^^A   \end{function}
%^^A   \begin{function}[pTF]{\stex_if_module_exists:n}
%^^A     Foo
%^^A   \end{function}
%^^A   \begin{function}{\stex_do_up_to_module:n, \stex_do_up_to_module:x}
%^^A     Foo
%^^A   \end{function}
%^^A   \begin{function}{\stex_add_to_current_module:n,\stex_add_to_current_module:x}
%^^A     Foo
%^^A   \end{function}
%^^A   \begin{function}{\stex_add_module_dependency:nnnn,\stex_add_module_dependency:oonn}
%^^A     Foo
%^^A   \end{function}
%^^A   \begin{function}{\stex_iterate_modules:n}
%^^A     Foo
%^^A   \end{function}
%^^A   \begin{function}{\stex_iterate_modules:nn}
%^^A     Foo
%^^A   \end{function}
%^^A   \begin{function}{\stex_add_module_decl:nnnnnnnN}
%^^A     Foo
%^^A   \end{function}
%^^A   \begin{function}{\stex_iterate_decls:n}
%^^A     Foo
%^^A   \end{function}
%^^A   \begin{function}{\stex_iterate_decls:nn}
%^^A     Foo
%^^A   \end{function}
%^^A   \begin{function}{\stex_add_module_notation:nnnnn,\stex_add_module_notation:eoexo}
%^^A     Foo
%^^A   \end{function}
%^^A   \begin{function}{\stex_execute_in_module:n,\stex_execute_in_module:x,\STEXexport}
%^^A     Foo
%^^A   \end{function}
%^^A   \begin{function}{\stex_every_module:n}
%^^A     Foo
%^^A   \end{function}
%^^A   \begin{variable}{\l_stex_metatheory_uri}
%^^A     Foo
%^^A   \end{variable}
%^^A   \DescribeEnv{smodule}
%^^A     Foo
%^^A
%^^A   \begin{function}{\setmetatheory}
%^^A     Foo
%^^A   \end{function}
%^^A   \begin{variable}{\l_stex_current_ns_uri}
%^^A     Foo
%^^A   \end{variable}
%^^A   \begin{function}{\stex_get_current_namespace:}
%^^A     Foo
%^^A   \end{function}
%^^A   \begin{function}{\stex_module_setup:n}
%^^A     Foo
%^^A   \end{function}
%^^A   \begin{function}{\stex_close_module:}
%^^A     Foo
%^^A   \end{function}
%^^A   \begin{function}{\stex_activate_module:n,\stex_activate_module:o,\stex_activate_module:x}
%^^A     Foo
%^^A   \end{function}
%^^A   \DescribeEnv{mmtinterface}
%^^A     Foo
%^^A
%^^A   \DescribeEnv{structural_feature_module}
%^^A     Foo
%^^A
%^^A   \DescribeEnv{structural_feature_morphism}
%^^A     Foo
%^^A
%^^A   \begin{function}{\stex_import_module_uri:nn}
%^^A     Destructs a relative \verb|[some/archive]{some/path?Name}|-pair into
%^^A     absolute values \cs{l_stex_import_archive_str},
%^^A     \cs{l_stex_import_path_str}, \cs{l_stex_import_name_str}:
%^^A   \end{function}
%^^A   \begin{function}{\stex_import_require_module:nnn}
%^^A     Foo
%^^A   \end{function}
%^^A   \begin{function}{\usemodule}
%^^A     Foo
%^^A   \end{function}
%^^A   \begin{function}{\importmodule}
%^^A     Foo
%^^A   \end{function}
%^^A   \begin{function}{\MMTinclude}
%^^A     Foo
%^^A   \end{function}
%^^A   \begin{function}{\stex_decl_parse_arity:}
%^^A     Foo
%^^A   \end{function}
%^^A   \begin{function}{\stex_symdecl_top:n}
%^^A     Foo
%^^A   \end{function}
%^^A   \begin{function}{\stex_symdecl_do:}
%^^A     Foo
%^^A   \end{function}
%^^A   \begin{function}{\textsymdecl}
%^^A     Foo
%^^A   \end{function}
%^^A   \begin{function}{\stex_get_symbol:n}
%^^A     Foo
%^^A   \end{function}
%^^A   \begin{function}{\stex_notation_top:nnn}
%^^A     Foo
%^^A   \end{function}
%^^A   \begin{function}{\stex_notation_do:nn}
%^^A     Foo
%^^A   \end{function}
%^^A   \begin{function}{\notation}
%^^A     Foo
%^^A   \end{function}
%^^A   \begin{function}{\symdef}
%^^A     Foo
%^^A   \end{function}
%^^A   \begin{function}{\vardef}
%^^A     Foo
%^^A   \end{function}
%^^A   \begin{function}{\_stex_invoke_symbol:nnnnnnnN}
%^^A     Foo
%^^A   \end{function}
%^^A   \begin{function}{\_stex_invoke_variable:nnnnnn}
%^^A     Foo
%^^A   \end{function}
%^^A   \begin{function}{\stex_invoke_symbol:}
%^^A     Foo
%^^A   \end{function}
%^^A   \begin{function}{\infprec,\neginfprec}
%^^A     Foo
%^^A   \end{function}
%^^A   \begin{function}{\STEXInternalTermMathArgiii,\STEXInternalTermMathOMSiii,\STEXInternalTermMathOMAiii,\STEXInternalTermMathOMBiii}
%^^A     Foo
%^^A   \end{function}
%^^A   \begin{function}{\dobrackets,\withbrackets,\dowithbrackets}
%^^A     Foo
%^^A   \end{function}
%^^A   \begin{function}{\STEXinvisible}
%^^A     Foo
%^^A   \end{function}
%^^A   \begin{function}{\symname,\sn,\sns,\Symname,\Sn,\Sns,\symref,\sr}
%^^A     Foo
%^^A   \end{function}
%^^A   \begin{function}{\comp}
%^^A     Foo
%^^A   \end{function}
%^^A   \begin{function}{\compemph,\compemph@uri}
%^^A     Foo
%^^A   \end{function}
%^^A   \begin{function}{\defemph,\defemph@uri}
%^^A     Foo
%^^A   \end{function}
%^^A   \begin{function}{\symrefemph,\symrefemph@uri}
%^^A     Foo
%^^A   \end{function}
%^^A   \begin{function}{\varemph,\varemph@uri}
%^^A     Foo
%^^A   \end{function}
%^^A
%^^A   \DescribeEnv{mathstructure}
%^^A     Foo
