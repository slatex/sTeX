%%
%% This is file `tikzmarklibraryhighlight.code.tex',
%% generated with the docstrip utility.
%%
%% The original source files were:
%%
%% tikzmark.dtx  (with options: `highlight')
%% ----------------------------------------------------------------
%% tikzmark --- remembering absolute positioning with TikZ.
%% E-mail: stacey@math.ntnu.no
%% Released under the LaTeX Project Public License v1.3c or later
%% See http://www.latex-project.org/lppl.txt
%% ----------------------------------------------------------------
%% 
\newcounter{highlight}
\newcommand{\hlstart}{\tikz[remember picture with id=hlstart\the\value{highlight},baseline=-0.7ex];\hl@start}
\newcommand{\hlend}{\tikz[remember picture with id=hlend\the\value{highlight},baseline=-0.7ex];\hl@end\stepcounter{highlight}}
\newcommand{\fdstart}{\tikz[remember picture with id=hlstart\the\value{highlight},baseline=-0.7ex];\fd@start}
\newcommand{\fdend}{\tikz[remember picture with id=hlend\the\value{highlight},baseline=-0.7ex];\fd@end\stepcounter{highlight}}
\newcommand{\vlstart}{\tikz[remember picture with id=hlstart\the\value{highlight},baseline=-1em];\vl@start}
\newcommand{\vlend}{\tikz[remember picture with id=hlend\the\value{highlight},baseline=0.3ex];\vl@end\stepcounter{highlight}}
\newcommand{\hl@start}[1][]{%
  \hl@draw{highlighter}{#1}{\the\value{highlight}}}

\newcommand{\hl@end}{}

\newcommand{\fd@start}[1][]{%
  \def\fd@args{#1}}

\newcommand{\fd@end}{\def\@tempa{\hl@draw{fader}}\expandafter\@tempa\expandafter{\fd@args}{\the\value{highlight}}\def\fd@args{}}

\newcommand{\vl@start}[1][]{%
  \vl@draw{highlighter}{#1}{\the\value{highlight}}}

\newcommand{\vl@end}{}

\def\hl@sets{%
  \edef\hl@sx{\the\pgf@x}%
  \edef\hl@sy{\the\pgf@y}%
}
\def\hl@sete{%
  \edef\hl@ex{\the\pgf@x}%
  \edef\hl@ey{\the\pgf@y}%
}

\@ifclassloaded{beamer}{

\def\page@node{
  \path (current page.south east)
      ++(-\beamer@rightmargin,\footheight)
  node[
    minimum width=\textwidth,
    minimum height=\textheight,
    anchor=south east
  ] (page) {};
}

}{

  \def\page@node{
    \path (current page.north west)
    ++(\hoffset + 1in + \oddsidemargin + \leftskip,\voffset + 1in + \topmargin + \headheight + \headsep)
    node[
      minimum width=\textwidth - \leftskip - \rightskip,
      minimum height=\textheight,
      anchor=north west
    ] (page) {};
  }

}

\newcommand{\hl@draw}[3]{%
  \begin{tikzpicture}[remember picture,overlay]%
  \page@node
  \tikzset{#2,highlight=#1,every path/.append style={highlight=#1}}%
  \pgfmathsetlengthmacro{\hl@width}{\the\pgflinewidth - 1pt}%
  \coordinate (hlstart) at (pic cs:hlstart#3);
  \coordinate (hlend) at (pic cs:hlend#3);
  \tikz@scan@one@point\hl@sets(pic cs:hlstart#3)
  \tikz@scan@one@point\hl@sete(pic cs:hlend#3)
  \ifdim\hl@sy=\hl@ey\relax
  \draw (hlstart) -- (hlend);
  \else
  \draw (hlstart) -- (hlstart -| page.east);
  \pgfmathsetlengthmacro{\hl@sy}{\hl@sy -\hl@width}%
  \pgfmathsetlengthmacro{\hl@ey}{\hl@ey +\hl@width}%
  \loop\ifdim\hl@sy>\hl@ey\relax
  \draw (0,\hl@sy -| page.west) -- (0,\hl@sy -| page.east);
  \pgfmathsetlengthmacro{\hl@sy}{\hl@sy -\hl@width}%
  \repeat
  \draw (hlend -| page.west) -- (hlend);
  \fi
  \end{tikzpicture}%
}

\newcommand{\vl@draw}[3]{%
  \begin{tikzpicture}[remember picture,overlay]%
  \page@node
  \tikzset{#2,highlight=#1,every path/.append style={highlight=#1}}%
  \pgfmathsetlengthmacro{\hl@width}{\the\pgflinewidth - 1pt}%
  \coordinate (hlstart) at (pic cs:hlstart#3);
  \coordinate (hlend) at (pic cs:hlend#3);
  \tikz@scan@one@point\hl@sets(pic cs:hlstart#3)
  \tikz@scan@one@point\hl@sete(pic cs:hlend#3)
  \ifdim\hl@sx=\hl@ex\relax
  \draw (hlstart) -- (hlend);
  \else
  \draw (hlstart) -- (hlstart |- page.south);
  \pgfmathsetlengthmacro{\hl@sx}{\hl@sx -\hl@width}%
  \pgfmathsetlengthmacro{\hl@ex}{\hl@ex +\hl@width}%
  \loop\ifdim\hl@sx>\hl@ex\relax
  \draw (\hl@sx,0 |- page.north) -- (\hl@sx,0 |- page.south);
  \pgfmathsetlengthmacro{\hl@sx}{\hl@sx -\hl@width}%
  \repeat
  \draw (hlend |- page.north) -- (hlend);
  \fi
  \end{tikzpicture}%
}

\tikzset{%
  highlight/.default=highlighter,
  highlight/.style={
    color=\pgfkeysvalueof{/tikz/#1 colour},
    line width=\pgfkeysvalueof{/tikz/#1 width},
    line cap=\pgfkeysvalueof{/tikz/#1 cap},
    opacity=\pgfkeysvalueof{/tikz/#1 opacity},
  },
  highlighter colour/.initial=yellow,
  highlighter width/.initial=12pt,
  highlighter cap/.initial=butt,
  highlighter opacity/.initial=1,
  fader colour/.initial=gray,
  fader width/.initial=12pt,
  fader cap/.initial=butt,
  fader opacity/.initial=.5,
}

\@ifclassloaded{beamer}{

%% Beamer variants

\setbeamercolor{highlighted text}{bg=yellow}
\setbeamercolor{faded text}{fg=gray}

\newcommand<>{\highlight}[2][]{%
  \only#3{\hlstart[#1]}#2\only#3{\hlend}}

\newcommand<>{\fade}[2][]{%
  \only#3{\fdstart[#1]}#2\only#3{\fdend}}

\newcommand<>{\vhighlight}[2][]{%
  \only#3{\vlstart[#1]}#2\only#3{\vlend}}

}{

\newcommand{\highlight}[2][]{%
\hlstart[#1]#2\hlend}

\newcommand{\fade}[2][]{%
\fdstart[#1]#2\fdend}

\newcommand{\vhighlight}[2][]{%
\vlstart[#1]#2\vlend}

}

%% 
%% Copyright (C) 2011 by Andrew Stacey <stacey@math.ntnu.no>
%% 
%% This work may be distributed and/or modified under the
%% conditions of the LaTeX Project Public License (LPPL), either
%% version 1.3c of this license or (at your option) any later
%% version.  The latest version of this license is in the file:
%% 
%% http://www.latex-project.org/lppl.txt
%% 
%% This work is "maintained" (as per LPPL maintenance status) by
%% Andrew Stacey.
%% 
%% This work consists of the file  tikzmark.dtx
%% and the derived files           tikzmark.ins,
%%                                 tikzmark.pdf,
%%                                 tikzlibrarytikzmark.code.tex, and
%%                                 tikzmarklibrarylistings.code.tex
%%                                 tikzmarklibraryhighlight.code.tex
%% 
%%
%% End of file `tikzmarklibraryhighlight.code.tex'.
