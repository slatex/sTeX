% \iffalse meta-comment 
% An Infrastructure for Structural Markup for Proofs
% Copyright (c) 2019 Michael Kohlhase, all rights reserved
%                this file is released under the
%                LaTeX Project Public License (LPPL)
% 
% The original of this file is in the public repository at 
% http://github.com/sLaTeX/sTeX/
%
% TODO update copyright  
%
%<*driver>
\def\bibfolder#1{../../lib/bib/#1}
\RequirePackage{paralist}
\ifcsname stexdocpath\endcsname\else\def\stexdocpath{.}\fi
\documentclass[full]{l3doc}
%\RequirePackage{document-structure}
\usepackage[hyperref=auto,style=alphabetic]{biblatex}
%\usepackage[mathhub=\stexdocpath/mh,usedeps]{stex}
\usepackage[lang={en,de}]{stex}

\usepackage{rustex}
\usepackage{stex-highlighting,stexthm}

\srefsetin[sTeX/Documentation]{documentation}{the \stex Documentation}

\makeatletter
\providecommand{\HTML}{\textsc{html}\xspace}%
\providecommand{\XML}{\textsc{xml}\xspace}%
\providecommand{\PDF}{\textsc{pdf}\xspace}%
\providecommand\openmath{\textsc{OpenMath}\xspace}
\providecommand\OMDoc{\textsc{OMDoc}\xspace}
\DeclareRobustCommand\LaTeXML{L\kern-.36em%
        {\sbox\z@ T%
         \vbox to\ht\z@{\hbox{\check@mathfonts
                              \fontsize\sf@size\z@
                              \math@fontsfalse\selectfont
                              A}%
                        \vss}%
        }%
        \kern-.15em%
%        T\kern-.1667em\lower.5ex\hbox{E}\kern-.125em\relax
%        {\tt XML}}
        T\kern-.1667em\lower.4ex\hbox{E}\kern-0.05em\relax
        {\scshape xml}\xspace}%
\def\mmt{\textsc{Mmt}\xspace}
\makeatother


\newif\ifhadtitle\hadtitlefalse

\def\stexversion{3.3.0}
\def\changedate{\today}
\def\stextoptitle#1#2{\title{#1\thanks{Version {\stexversion} (last revised {\changedate})} }\def\thispkg{#2}}

\author{Michael Kohlhase, Dennis Müller\\
 	FAU Erlangen-Nürnberg\\
 	\url{http://kwarc.info/}
}

\def\stexmaketitle{\ifhadtitle\else\hadtitletrue\maketitle\fi}

\ExplSyntaxOn

  \def\docmodule{
    \begin{document}
      \EnableManual
      \EnableDocumentation
      \EnableImplementation
      \stexmaketitle
      \tableofcontents
      \int_gincr:N \l_stex_docheader_sect
      \exp_args:Ne \__stex_mathhub_find_manifest:n {\stex_file_use:N \c_stex_mathhub_file / sTeX / Documentation}
      \str_if_empty:NF \l__stex_mathhub_manifest_str {
        \usemodule[sTeX/Documentation]{macros?AllMacros}
      }
      \DocInput{\jobname.dtx}
      \clearpage
      \PrintIndex
      \printbibliography
    \end{document}
  }

  \bool_new:N \g_stexdoc_typeset_manual_bool
  \NewDocumentCommand \EnableManual {}{
    \bool_gset_true:N \g_stexdoc_typeset_manual_bool
  }
  \NewDocumentCommand \DisableManual {}{
    \bool_gset_false:N \g_stexdoc_typeset_manual_bool
  }
  \NewDocumentEnvironment {stexmanual} {} {
    \bool_if:NTF \g_stexdoc_typeset_manual_bool
      {\bool_set_false:N \l__codedoc_in_implementation_bool}
      {\comment}
  }{
    \bool_if:NF \g_stexdoc_typeset_manual_bool {\endcomment}
  }
\ExplSyntaxOff

%\usepackage{makeidx}
%\makeindex

%\usepackage{document-structure}


\usepackage{lststex,mdframed}
\usepackage[most]{tcolorbox}

\lstset{literate=%
    {Ö}{{\"O}}1
    {Ä}{{\"A}}1
    {Ü}{{\"U}}1
    {ß}{{\ss}}1
    {ü}{{\"u}}1
    {ä}{{\"a}}1
    {ö}{{\"o}}1
    {~}{{\textasciitilde}}1
}

\newenvironment{framed}[1][]{
  \ifstexhtml\par\vbox\bgroup
    \csname exp_args:Nne\endcsname\begin{stex_annotate_env}{%
      style:border=solid 1px black,%
      style:width=var(--this-width),%
      style:min-width=var(--this-width),%
      style:--this-width=calc(var(--current-width) - 6px),%
      style:padding=3px,%
      style:margin-top=5px,%
      style:margin-bottom=5px%
    }
    \csname stex_annotate_invisible:n\endcsname{ }%
    \begin{stex_annotate_env}{%
      style:--current-width=var(--this-width);%
    }\csname stex_annotate_invisible:n\endcsname{ }
  \else\begin{mdframed}[#1]\fi
  %\begin{center}%
}{%
  %\end{center}%
  \ifstexhtml
    \end{stex_annotate_env}\end{stex_annotate_env}\egroup\par
  \else\end{mdframed}\fi
}
\newcommand{\scaled}[2][0.9\hsize]{\begin{center}\resizebox{#1}{!}{\begin{minipage}{\textwidth} #2 \end{minipage}}\end{center}}

\makeatletter
\ExplSyntaxOn

\def\doc_exbox:nnn#1#2#3{
  \begin{sexample}[#3]
    Input:
    \begin{framed}[linewidth=1pt,backgroundcolor=white]\small
      #1
    \end{framed}
    Output:
    \begin{framed}[linewidth=1pt,backgroundcolor=white]\small
      #2
    \end{framed}
  \end{sexample}
}


\NewDocumentCommand\stexexamplefile{O{} m O{} O{}}{
  \stex_resolve_path_pair:Nxx \l_@@_filepath_str {\tl_to_str:n{#1}} {\tl_to_str:n{#2}}
  \doc_exbox:nnn{
    \hfill File~\str_if_empty:nTF{#1}{
      \prop_if_exist:NT \l_stex_current_archive_prop {
        [\texttt{\prop_item:Nn \l_stex_current_archive_prop {id}}]
      }
    }{[#1]}\texttt{\tl_to_str:n{#2}}
    \_lststex_parse_args:n{#3}
    \exp_args:Nno \use:nn{\lstinputlisting[} \l_lststex_return_tl ]{\l_@@_filepath_str}
  }{
    \inputref[#1]{#2}
  }{#4}
}

\newwrite\testoutfile
\NewDocumentCommand\stexexample{O{} O{}}{
  \begingroup 
  \catcode`\\=12\relax
  \catcode`\#=12\relax
  \catcode`\&=12\relax
  \catcode`\$=12\relax
  \catcode`\^=12\relax
  \catcode`\_=12\relax
  \catcode`\ =12\relax
  \catcode`^^J=12\relax
  \endlinechar=`^^J
  \newlinechar=-1
%^^A    \everyeof{\noexpand}
  \example_a:nnn{#1}{#2}
}
\long\def\example_a:nnn #1 #2 #3 {
  \endgroup
  \immediate\openout\testoutfile=\jobname.exmpl
  \immediate\write\testoutfile{
    \c_backslash_str begin{stexcode}[#1]
    \detokenize{^^J}#3
    \c_backslash_str end{stexcode}
  }
  \immediate\closeout\testoutfile
  \doc_exbox:nnn{
    \catcode`\#=12\relax
    \csname @ @ input\endcsname{\jobname.exmpl}
  }{
    \immediate\openout\testoutfile=\jobname.exmpl
    \immediate\write\testoutfile{#3}
    \immediate\closeout\testoutfile
    \csname @ @ input\endcsname \jobname.exmpl\relax
  }{#2}
  \peek_charcode_remove:NT ^^J
}

\ExplSyntaxOff
\makeatother

\makeatletter
\newcount\example@counter\example@counter=0
\newtcolorbox{exampleborderbox}[1][]{
  empty,
  title={Example \the\example@counter #1},
  attach boxed title to top left,
     minipage boxed title,
  boxed title style={empty,size=minimal,toprule=0pt,top=1pt,left=3mm,overlay={}},
  coltitle=blue,fonttitle=\bfseries,
  parbox=false,boxsep=0pt,left=3mm,right=0mm,top=2pt,breakable,pad at break=0mm,
     before upper=\csname @totalleftmargin\endcsname0pt, 
  overlay unbroken={\draw[blue,line width=2pt] ([xshift=-0pt]title.north west) -- ([xshift=-0pt]frame.south west); },
  overlay first={\draw[blue,line width=2pt] ([xshift=-0pt]title.north west) -- ([xshift=-0pt]frame.south west); },
  overlay middle={\draw[blue,line width=2pt] ([xshift=-0pt]frame.north west) -- ([xshift=-0pt]frame.south west); },
  overlay last={\draw[blue,line width=2pt] ([xshift=-0pt]frame.north west) -- ([xshift=-0pt]frame.south west); },
  outer arc=4pt%
}

\ExplSyntaxOn
\stexstyleexample{
  \global\advance\example@counter by 1
  \tl_if_empty:NTF\thistitle{
    \begin{exampleborderbox}
  }{
    \begin{exampleborderbox}[ (\thistitle)]
  }
}{
    \end{exampleborderbox}
}

\ExplSyntaxOff\makeatother

\usetikzlibrary{calc}

\def\textwarning{\includegraphics[width=1.2em]{stex-dangerous-bend}\xspace}
\newtcolorbox{dangerbox}{
  breakable,
  enhanced,
  left=0pt,
  right=0pt,
  top=8pt,
  bottom=8pt,
  colback=white,
  colframe=red,
  width=\textwidth,
  enlarge left by=0mm,
  boxsep=5pt,
  fontupper=\small,
  arc=4pt,
  outer arc=4pt,
  leftupper=1.5cm,
  overlay={
    \node[anchor=west] at ([xshift=10pt]$(frame.north west)!0.5!(frame.south west)$)
       {\includegraphics[width=1cm,height=1cm]{stex-dangerous-bend}};}
}

\protected\def\TODO#1{\textcolor{red}{TODO}\footnote{\textcolor{red}{TODO: #1}}}

\definecolor{darkgreen}{rgb}{0.0, 0.5, 0.0}

\usepackage[solutions]{problem}
\usepackage{hwexam}
\newtcolorbox{problemborderbox}[1][]{
  empty,
  title={Exercise #1},
  attach boxed title to top left,
     minipage boxed title,
  boxed title style={empty,size=minimal,toprule=0pt,top=1pt,left=3mm,overlay={}},
  coltitle=darkgreen,fonttitle=\bfseries,
  parbox=false,boxsep=0pt,left=3mm,right=0mm,top=2pt,breakable,pad at break=0mm,
     before upper=\csname @totalleftmargin\endcsname0pt, 
  overlay unbroken={\draw[darkgreen,line width=2pt] ([xshift=-0pt]title.north west) -- ([xshift=-0pt]frame.south west); },
  overlay first={\draw[darkgreen,line width=2pt] ([xshift=-0pt]title.north west) -- ([xshift=-0pt]frame.south west); },
  overlay middle={\draw[darkgreen,line width=2pt] ([xshift=-0pt]frame.north west) -- ([xshift=-0pt]frame.south west); },
  overlay last={\draw[darkgreen,line width=2pt] ([xshift=-0pt]frame.north west) -- ([xshift=-0pt]frame.south west); },
  outer arc=4pt%
}

\ExplSyntaxOn
\stexstyleproblem{
  \tl_if_empty:NTF\thistitle{
    \begin{problemborderbox}
  }{
    \begin{problemborderbox}[ (\thistitle)]
  }
}{
    \end{problemborderbox}
}
\ExplSyntaxOff

\newtcolorbox{experimental}{
  breakable,
  enhanced,
  left=0pt,
  right=0pt,
  top=8pt,
  bottom=8pt,
  colback=white,
  colframe=gray,
  width=\textwidth,
  enlarge left by=0mm,
  boxsep=5pt,
  fontupper=\small,
  arc=4pt,
  outer arc=4pt,
  leftupper=1.5cm,
  overlay={
    \node[anchor=west] at ([xshift=10pt]$(frame.north west)!0.5!(frame.south west)$)
       {\includegraphics[height=1cm]{stex-experimental}};}
}


\usetikzlibrary{decorations.pathmorphing,shapes,arrows,calc}
% Taken from pgflibrarytikzmmt.code.tex
\newcommand{\mmtarrowtip}{angle 45}
\newcommand{\mmtarrowtipmonoright}{right hook}

\tikzstyle{include}=[\mmtarrowtipmonoright-\mmtarrowtip,thick]
\tikzstyle{morph}=[-\mmtarrowtip,thick]
\tikzstyle{preview}=[decorate, decoration={coil,aspect=0,amplitude=1pt,
                                                  segment length=6pt,
                                                  pre=lineto,pre length=3pt,
                                                  post=lineto,post length=5pt}, thick]
\tikzstyle{view}=[preview,-\mmtarrowtip]


% TIKZ RULES
\def\mmtlogo{
\begin{tikzpicture}

  % White Background (Margins are eyeballed)
  % This is necessary because we paste white over arrows later.
  % If somebody want's to do the full song and dance with
  % interrupted arrows to get transparent background, be my guest.

  \fill[white!] (-0.01,0.15) rectangle (1.11,-0.95);

  % Arrows
  \draw [blue, include] (0,0)     -- (1.1,0);
  \draw [green, morph] (0,-0.4)  -- (1.1,-0.4);
  \draw [red, view]   (-0,-0.8) -- (1.1,-0.8);

  % Cutout for letters
  \fill[white] (0.33,0.1) rectangle (0.66,-0.9);

  % Letters
  \node at (0.18,0)    (nodeM1) {\large M};
  \node at (0.18,-0.4) (nodeM2) {\large M};
  \node at (0.21,-0.8) (nodeT)  {\large T};

\end{tikzpicture}
}

\newtcolorbox{mmtbox}{
  breakable,
  enhanced,
  left=0pt,
  right=0pt,
  top=8pt,
  bottom=8pt,
  colback=white,
  colframe=green,
  width=\textwidth,
  enlarge left by=0mm,
  boxsep=5pt,
  fontupper=\small,
  arc=4pt,
  outer arc=4pt,
  leftupper=1.5cm,
  overlay={
    \node[anchor=west] at ([xshift=10pt]$(frame.north west)!0.5!(frame.south west)$)
       {\mmtlogo};}
}

\AtBeginDocument{\catcode`_=8}

\begin{document}
  \DocInput{\jobname.dtx}
\end{document}
%</driver>
% \fi
%
% \GetFileInfo{sproof.sty}
% 
% \title{\sTeX-Proofs: Structural Markup for Proofs\thanks{Version {\fileversion} (last revised {\filedate})}
% }
%
% \author{Michael Kohlhase, Dennis Müller\\
% 	FAU Erlangen-Nürnberg\\
% 	\url{http://kwarc.info/}
%c }
%
% \maketitle
%
%\ifinfulldoc\else
% \begin{abstract}
%   This is the documentation for the \pkg{stex-proofs} package.
%
% The |sproof| package is part of the {\stex} collection, a version of {\TeX/\LaTeX} that
% allows to markup {\TeX/\LaTeX} documents semantically without leaving the document
% format, essentially turning {\TeX/\LaTeX} into a document format for mathematical
% knowledge management (MKM).
%
% The \kpg{stex-proofs} package supplies macros and environment that allow to annotate the
% structure of mathematical proofs in {\stex} files. This structure can be used by MKM
% systems for added-value services, either directly from the \sTeX sources, or after
% translation.
%
% For a more high-level introduction, see \href{\basedocurl/manual.pdf}{the \sTeX Manual}
% or the \href{\basedocurl/stex.pdf}{full \sTeX documentation}.
% \end{abstract}
% 
% \tableofcontents
%
% The \pkg{stex-proof} package supplies macros and environment that allow to annotate the
structure of mathematical proofs in \sTeX document. This structure can be used by MKM
systems for added-value services, either directly from the \sTeX sources, or after
translation.

We will go over the general intuition by way of a running example: 

\begin{latexcode}
\begin{sproof}[id=simple-proof]
   {We prove that $\sum_{i=1}^n{2i-1}=n^{2}$ by induction over $n$}
  \begin{spfcases}{For the induction we have to consider three cases:}
   \begin{spfcase}{$n=1$}
    \begin{spfstep}[type=inline] then we compute $1=1^2$\end{spfstep}
   \end{spfcase}
   \begin{spfcase}{$n=2$}
      \begin{spfcomment}[type=inline]
        This case is not really necessary, but we do it for the
        fun of it (and to get more intuition).
      \end{spfcomment}
      \begin{spfstep}[type=inline] We compute $1+3=2^{2}=4$.\end{spfstep}
   \end{spfcase}
   \begin{spfcase}{$n>1$}
      \begin{spfstep}[type=assumption,id=ind-hyp]
        Now, we assume that the assertion is true for a certain $k\geq 1$,
        i.e. $\sum_{i=1}^k{(2i-1)}=k^{2}$.
      \end{spfstep}
      \begin{spfcomment}
        We have to show that we can derive the assertion for $n=k+1$ from
        this assumption, i.e. $\sum_{i=1}^{k+1}{(2i-1)}=(k+1)^{2}$.
      \end{spfcomment}
      \begin{spfstep}
        We obtain $\sum_{i=1}^{k+1}{2i-1}=\sum_{i=1}^k{2i-1}+2(k+1)-1$
        \spfjust[method=arith:split-sum]{by splitting the sum}.
      \end{spfstep}
      \begin{spfstep}
        Thus we have $\sum_{i=1}^{k+1}{(2i-1)}=k^2+2k+1$
        \spfjust[method=fertilize]{by inductive hypothesis}.
      \end{spfstep}
      \begin{spfstep}[type=conclusion]
        We can \spfjust[method=simplify]{simplify} the right-hand side to
        ${k+1}^2$, which proves the assertion.
      \end{spfstep}
   \end{spfcase}
    \begin{spfstep}[type=conclusion]
      We have considered all the cases, so we have proven the assertion.
    \end{spfstep}
  \end{spfcases}
\end{sproof}
\end{latexcode}

This yields the following result: 

\begin{mdframed}
  \begin{sproof}[id=simple-proof]
  {We prove that $\sum_{i=1}^n{2i-1}=n^{2}$ by induction over $n$}
  \begin{spfcases}{For the induction we have to consider the following cases:}
    \begin{spfcase}{$n=1$}
      \begin{spfstep}[type=inline] then we compute $1=1^2$\end{spfstep}
    \end{spfcase}
    \begin{spfcase}{$n=2$}
      \begin{spfcomment}[type=inline]
         This case is not really necessary, but we do it for the fun
         of it (and to get more intuition).
      \end{spfcomment}
      \begin{spfstep}[type=inline]
         We compute $1+3=2^{2}=4$
      \end{spfstep}
    \end{spfcase}
    \begin{spfcase}{$n>1$}
      \begin{spfstep}[type=hypothesis,id=ind-hyp]
        Now, we assume that the assertion is true for a certain $k\geq 1$, i.e.
        $\sum_{i=1}^k{(2i-1)}=k^{2}$.
      \end{spfstep}
      \begin{spfcomment}
        We have to show that we can derive the assertion for $n=k+1$ from this
        assumption, i.e.  $\sum_{i=1}^{k+1}{(2i-1)}=(k+1)^{2}$.
      \end{spfcomment}
      \begin{spfstep}[id=splitit]
        We obtain $\sum_{i=1}^{k+1}{(2i-1)}=\sum_{i=1}^k{(2i-1)}+2(k+1)-1$
       \spfjust[method=arith:split-sum]{by splitting the sum}.
     \end{spfstep}
     \begin{spfstep}[id=byindhyp]
       Thus we have $\sum_{i=1}^{k+1}{(2i-1)}=k^2+2k+1$
       \spfjust[method=fertilize]{by \premise[ind-hyp]{inductive hypothesis}}.
     \end{spfstep}
     \begin{spfstep}[type=conclusion]
       We can \spfjust[method=simplify-eq]{simplify the \justarg[rhs]{right-hand side}} to
       $(k+1)^2$, which proves the assertion.
     \end{spfstep}
   \end{spfcase}
   \begin{spfstep}[type=conclusion]
     We have considered all the cases, so we have proven the assertion.
   \end{spfstep}
  \end{spfcases}
\end{sproof}
\end{mdframed}

\begin{environment}{sproof}
  The |sproof| environment is the main container for proofs. It takes an optional |KeyVal|
  argument that allows to specify the |id| (identifier) and |for| (for which assertion is
  this a proof) keys. The regular argument of the |proof| environment contains an
  introductory comment, that may be used to announce the proof style. The |proof|
  environment contains a sequence of |spfstep|, |spfcomment|, and |spfcases| environments
  that are used to markup the proof steps.
\end{environment}
  
\begin{function}{\spfidea}
  The |\spfidea| macro allows to give a one-paragraph description of the proof idea.
\end{function}

\begin{function}{\spfsketch}
  For one-line proof sketches, we use the |\spfsketch| macro, which takes the same
  optional argument as |sproof| and another one: a natural language text that sketches
  the proof.
\end{function}

\begin{environment}{spfstep}
  Regular proof steps are marked up with the |step| environment, which takes an optional
  |KeyVal| argument for annotations. A proof step usually contains a local assertion
  (the text of the step) together with some kind of evidence that this can be derived
  from already established assertions.
\end{environment}

\begin{function}{\spfjust}
  This evidence is marked up with the |\spfjust| macro in the \pkg{stex-proofs}
  package. This environment totally invisible to the formatted result; it wraps the text
  in the proof step that corresponds to the evidence. The environment takes an optional
  |KeyVal| argument, which can have the |method| key, whose value is the name of a proof
  method (this will only need to mean something to the application that consumes the
  semantic annotations). Furthermore, the justification can contain ``premises''
  (specifications to assertions that were used justify the step) and ``arguments''
  (other information taken into account by the proof method).
\end{function}

\begin{function}{\premise}
  The |\premise| macro allows to mark up part of the text as reference to an assertion
  that is used in the argumentation. In the running example we have used the |\premise|
  macro to identify the inductive hypothesis.
\end{function}

\begin{function}{\justarg}
  The |\justarg| macro is very similar to |\premise| with the difference that it is used
  to mark up arguments to the proof method. Therefore the content of the first argument
  is interpreted as a mathematical object rather than as an identifier as in the case of
  |\premise|. In our example, we specified that the simplification should take place on
  the right hand side of the equation. Other examples include proof methods that
  instantiate. Here we would indicate the substituted object in a |\justarg| macro.
\end{function}

Note that both |\premise| and |\justarg| can be used with an empty second argument to
mark up premises and arguments that are not explicitly mentioned in the text.

\begin{environment}{subproof}
  The |spfcases| environment is used to mark up a subproof. This environment takes an
  optional |KeyVal| argument for semantic annotations and a second argument that allows
  to specify an introductory comment (just like in the |proof| environment). The
  |method| key can be used to give the name of the proof method
  executed to make this subproof.
\end{environment}

\begin{environment}{spfcases}
  The |spfcases| environment is used to mark up a proof by cases. Technically it is a
  variant of the |subproof| where the |method| is |by-cases|. Its contents are |spfcase|
  environments that mark up the cases one by one.
\end{environment}

\begin{environment}{spfcase}
  The content of a |spfcases| environment are a sequence of case proofs marked up in the
  |spfcase| environment, which takes an optional |KeyVal| argument for semantic
  annotations. The second argument is used to specify the the description of the case
  under consideration. The content of a |spfcase| environment is the same as that of a
  |sproof|, i.e. |spfstep|s, |spfcomment|s, and |spfcases| environments.
\end{environment}

\begin{function}{\spfcasesketch}
  |\spfcasesketch| is a variant of the |spfcase| environment that takes the same
  arguments, but instead of the |spfstep|s in the body uses a third argument for a proof
  sketch.
\end{function}

\begin{environment}{spfcomment}
  The |spfcomment| environment is much like a |step|, only that it does not have an
  object-level assertion of its own. Rather than asserting some fact that is relevant
  for the proof, it is used to explain where the proof is going, what we are attempting
  to to, or what we have achieved so far. As such, it cannot be the target of a
  |\premise|.
\end{environment}

\begin{function}{\sproofend}
  Traditionally, the end of a mathematical proof is marked with a little box at the end of
  the last line of the proof (if there is space and on the end of the next line if there
  isn't), like so:\sproofend

  The \pkg{stex-proofs} package provides the |\sproofend| macro for this.
\end{function}
  
\begin{variable}{\sProofEndSymbol}
  If a different symbol for the proof end is to be used (e.g. {\sl{q.e.d}}), then this can
  be obtained by specifying it using the |\sProofEndSymbol| configuration macro (e.g. by
  specifying |\sProofEndSymbol{q.e.d}|).
\end{variable}
  
Some of the proof structuring macros above will insert proof end symbols for sub-proofs,
in most cases, this is desirable to make the proof structure explicit, but sometimes this
wastes space (especially, if a proof ends in a case analysis which will supply its own
proof end marker). To suppress it locally, just set |proofend={}| in them or use use
|\sProofEndSymbol{}|.

%%% Local Variables:
%%% mode: latex
%%% TeX-master: "../stex-manual"
%%% End:

%  LocalWords:  hypothesis,id geq splitit arith:split-sum byindhyp rhs proofend

% \fi
%
% \begin{documentation}
% \changes{v0.9}{2022/02/14}{Moved over from the \pkg{sproofs} package}
% 
%
%
% \end{documentation}
%
% \begin{implementation}
%
% \section{The Implementation} 
% 
% \subsection{Package Options}\label{sec:impl:options}
%
% We declare some switches which will modify the behavior according to the package
% options. Generally, an option |xxx| will just set the appropriate switches to true
% (otherwise they stay false).\ednote{need an implementation for {\latexml}}
%
%    \begin{macrocode}
%<*package>
%<@@=stex_sproof>

%%%%%%%%%%%%%   sproof.dtx   %%%%%%%%%%%%%

%    \end{macrocode}
%
%
% \subsection{Proofs}\label{sec:impl:proofs}
% 
% We first define some keys for the |proof| environment.
%    \begin{macrocode}
\keys_define:nn { stex / spf } {
  id          .str_set_x:N  = \spfid,
  for         .clist_set:N  = \l_@@_spf_for_clist ,
  from        .tl_set:N     = \l_@@_spf_from_tl ,
  proofend    .tl_set:N     = \l_@@_spf_proofend_tl,
  type        .str_set_x:N  = \spftype,
  title       .tl_set:N     = \spftitle,
  continues   .tl_set:N     = \l_@@_spf_continues_tl,
  functions   .tl_set:N     = \l_@@_spf_functions_tl,
  method      .tl_set:N     = \l_@@_spf_method_tl
}
\cs_new_protected:Nn \_@@_spf_args:n {
	\str_clear:N \spfid
	\tl_clear:N \l_@@_spf_for_tl
	\tl_clear:N \l_@@_spf_from_tl
	\tl_set:Nn \l_@@_spf_proofend_tl {\sproof@box}
	\str_clear:N \spftype
	\tl_clear:N \spftitle
	\tl_clear:N \l_@@_spf_continues_tl
	\tl_clear:N \l_@@_spf_functions_tl
	\tl_clear:N \l_@@_spf_method_tl
  \bool_set_false:N \l_@@_inc_counter_bool
	\keys_set:nn { stex / spf }{ #1 }
}
%    \end{macrocode}
%
% \begin{macro}{\c_@@_flow_str}
% We define this macro, so that we can test whether the |display| key has the value |flow|
%    \begin{macrocode}
\str_set:Nn\c_@@_flow_str{inline}
%    \end{macrocode}
% \end{macro}
%
% For proofs, we will have to have deeply nested structures of enumerated list-like
% environments. However, {\LaTeX} only allows |enumerate| environments up to nesting depth
% 4 and general list environments up to listing depth 6. This is not enough for us.
% Therefore we have decided to go along the route proposed by Leslie Lamport to use a
% single top-level list with dotted sequences of numbers to identify the position in the
% proof tree. Unfortunately, we could not use his |pf.sty| package directly, since it does
% not do automatic numbering, and we have to add keyword arguments all over the place, to
% accomodate semantic information.
%
% \begin{environment}{pst@with@label}
%   This environment manages\footnote{This gets the labeling right but only works 8 levels
%   deep} the path labeling of the proof steps in the description environment of the
%   outermost |proof| environment. The argument is the label prefix up to now; which we
%   cache in |\pst@label| (we need evaluate it first, since are in the right place
%   now!). Then we increment the proof depth which is stored in |\count10| (lower counters
%   are used by {\TeX} for page numbering) and initialize the next level counter
%   |\count\count10| with 1. In the end call for this environment, we just decrease the
%   proof depth counter by 1 again.
%    \begin{macrocode}
\intarray_new:Nn\l_@@_counter_intarray{50}
\cs_new_protected:Npn \sproofnumber {
  \int_set:Nn \l_tmpa_int {1}
  \bool_while_do:nn {
    \int_compare_p:nNn {
      \intarray_item:Nn \l_@@_counter_intarray \l_tmpa_int
    } > 0
  }{
    \intarray_item:Nn \l_@@_counter_intarray \l_tmpa_int .
    \int_incr:N \l_tmpa_int
  }
}
\cs_new_protected:Npn \_@@_inc_counter: {
  \int_set:Nn \l_tmpa_int {1}
  \bool_while_do:nn {
    \int_compare_p:nNn {
      \intarray_item:Nn \l_@@_counter_intarray \l_tmpa_int
    } > 0
  }{
    \int_incr:N \l_tmpa_int
  }
  \int_compare:nNnF \l_tmpa_int = 1 {
    \int_decr:N \l_tmpa_int
  }
  \intarray_gset:Nnn \l_@@_counter_intarray \l_tmpa_int {
    \intarray_item:Nn \l_@@_counter_intarray \l_tmpa_int + 1
  }
}

\cs_new_protected:Npn \_@@_add_counter: {
  \int_set:Nn \l_tmpa_int {1}
  \bool_while_do:nn {
    \int_compare_p:nNn {
      \intarray_item:Nn \l_@@_counter_intarray \l_tmpa_int
    } > 0
  }{
    \int_incr:N \l_tmpa_int
  }
  \intarray_gset:Nnn \l_@@_counter_intarray \l_tmpa_int { 1 }
}

\cs_new_protected:Npn \_@@_remove_counter: {
  \int_set:Nn \l_tmpa_int {1}
  \bool_while_do:nn {
    \int_compare_p:nNn {
      \intarray_item:Nn \l_@@_counter_intarray \l_tmpa_int
    } > 0
  }{
    \int_incr:N \l_tmpa_int
  }
  \int_decr:N \l_tmpa_int
  \intarray_gset:Nnn \l_@@_counter_intarray \l_tmpa_int { 0 }
}
%    \end{macrocode}
% \end{environment}
%
%
%\begin{macro}{\sproofend}
%    This macro places a little box at the end of the line if there is space, or at the
%    end of the next line if there isn't
%    \begin{macrocode}
\def\sproof@box{
  \hbox{\vrule\vbox{\hrule width 6 pt\vskip 6pt\hrule}\vrule}
}
\def\sproofend{
  \tl_if_empty:NF \l_@@_spf_proofend_tl {
    \hfil\null\nobreak\hfill\l_@@_spf_proofend_tl\par\smallskip
  }
}
%    \end{macrocode}
% \end{macro}
%
% \begin{macro}{spf@*@kw}
%    \begin{macrocode}
\def\spf@proofsketch@kw{Proof~Sketch}
\def\spf@proof@kw{Proof}
\def\spf@step@kw{Step}
%    \end{macrocode}
% \end{macro}
%
% For the other languages, we set up triggers
%    \begin{macrocode}
\AddToHook{begindocument}{
  \ltx@ifpackageloaded{babel}{
    \makeatletter
    \clist_set:Nx \l_tmpa_clist {\bbl@loaded}
    \clist_if_in:NnT \l_tmpa_clist {ngerman}{
      \input{sproof-ngerman.ldf}
    }
    \clist_if_in:NnT \l_tmpa_clist {finnish}{
      \input{sproof-finnish.ldf}
    }
    \clist_if_in:NnT \l_tmpa_clist {french}{
      \input{sproof-french.ldf}
    }
    \clist_if_in:NnT \l_tmpa_clist {russian}{
      \input{sproof-russian.ldf}
    }
    \makeatother
  }{}
}
%    \end{macrocode}
%
% \begin{macro}{spfsketch}
%    \begin{macrocode}
\newcommand\spfsketch[2][]{
  \begingroup
  \let \premise \stex_proof_premise:
  \_@@_spf_args:n{#1}
  \stex_if_smsmode:TF {
    \str_if_empty:NF \spfid {
      \stex_ref_new_doc_target:n \spfid
    }
  }{
    \seq_clear:N \l_tmpa_seq
    \clist_map_inline:Nn \l_@@_spf_for_clist {
      \tl_if_empty:nF{ ##1 }{
        \stex_get_symbol:n { ##1 }
        \exp_args:NNo \seq_put_right:Nn \l_tmpa_seq {
          \l_stex_get_symbol_uri_str
        }
      }
    }
    \exp_args:Nnx
    \stex_annotate:nnn{proofsketch}{\seq_use:Nn \l_tmpa_seq {,}}{
      \str_if_empty:NF \spftype {
        \stex_annotate_invisible:nnn{type}{\spftype}{}
      }
      \clist_set:No \l_tmpa_clist \spftype
      \tl_set:Nn \l_tmpa_tl {
        \titleemph{
          \tl_if_empty:NTF \spftitle {
            \spf@proofsketch@kw
          }{
            \spftitle
          }
        }:~
      }
      \clist_map_inline:Nn \l_tmpa_clist {
        \exp_args:No \str_if_eq:nnT \c_@@_flow_str {##1} {
          \tl_clear:N \l_tmpa_tl
        }
      }
      \str_if_empty:NF \spfid {
        \stex_ref_new_doc_target:n \spfid
      }
      \l_tmpa_tl #2 \sproofend
    }
  }
  \endgroup
  \stex_smsmode_do:
}

%    \end{macrocode}
% \end{macro}
%
% \begin{macro}{spfeq}
%   This is very similar to |\spfsketch|, but uses a computation array\ednote{This should
%   really be more like a tabular with an ensuremath in it. or invoke text on the last
%   column}\ednote{document above}
%    \begin{macrocode}
\newenvironment{spfeq}[2][]{
  \_@@_spf_args:n{#1}
  \let \premise \stex_proof_premise:
  \stex_if_smsmode:TF {
    \str_if_empty:NF \spfid {
      \stex_ref_new_doc_target:n \spfid
    }
  }{
    \seq_clear:N \l_tmpa_seq
    \clist_map_inline:Nn \l_@@_spf_for_clist {
      \tl_if_empty:nF{ ##1 }{
        \stex_get_symbol:n { ##1 }
        \exp_args:NNo \seq_put_right:Nn \l_tmpa_seq {
          \l_stex_get_symbol_uri_str
        }
      }
    }
    \exp_args:Nnnx
    \begin{stex_annotate_env}{spfeq}{\seq_use:Nn \l_tmpa_seq {,}}
    \str_if_empty:NF \spftype {
      \stex_annotate_invisible:nnn{type}{\spftype}{}
    }

    \clist_set:No \l_tmpa_clist \spftype
    \tl_clear:N \l_tmpa_tl
    \clist_map_inline:Nn \l_tmpa_clist {
      \tl_if_exist:cT {_@@_spfeq_##1_start:}{
        \tl_set:Nn \l_tmpa_tl {\use:c{_@@_spfeq_##1_start:}}
      }
      \exp_args:No \str_if_eq:nnT \c_@@_flow_str {##1} {
        \tl_set:Nn \l_tmpa_tl {\use:n{}}
      }
    }
    \tl_if_empty:NTF \l_tmpa_tl {
      \_@@_spfeq_start:
    }{
      \l_tmpa_tl
    }{~#2}
    \str_if_empty:NF \spfid {
      \stex_ref_new_doc_target:n \spfid
    }
    \begin{displaymath}\begin{array}{rcll}
  }
  \stex_smsmode_do:
}{
  \stex_if_smsmode:F {
    \end{array}\end{displaymath}
    \clist_set:No \l_tmpa_clist \spftype
    \tl_clear:N \l_tmpa_tl
    \clist_map_inline:Nn \l_tmpa_clist {
      \tl_if_exist:cT {_@@_spfeq_##1_end:}{
        \tl_set:Nn \l_tmpa_tl {\use:c{_@@_spfeq_##1_end:}}
      }
    }
    \tl_if_empty:NTF \l_tmpa_tl {
      \_@@_spfeq_end:
    }{
      \l_tmpa_tl
    }
    \end{stex_annotate_env}
  }
}

\cs_new_protected:Nn \_@@_spfeq_start: {
  \titleemph{
    \tl_if_empty:NTF \spftitle {
      \spf@proof@kw
    }{
      \spftitle
    }
  }:
}
\cs_new_protected:Nn \_@@_spfeq_end: {\sproofend}

\newcommand\stexpatchspfeq[3][] {
    \str_set:Nx \l_tmpa_str{ #1 }
    \str_if_empty:NTF \l_tmpa_str {
      \tl_set:Nn \_@@_spfeq_start: { #2 }
      \tl_set:Nn \_@@_spfeq_end: { #3 }
    }{
      \exp_after:wN \tl_set:Nn \csname _@@_spfeq_#1_start:\endcsname{ #2 }
      \exp_after:wN \tl_set:Nn \csname _@@_spfeq_#1_end:\endcsname{ #3 }
    }
}

%    \end{macrocode}
% \end{macro}
%
% \begin{environment}{sproof}
%    In this environment, we initialize the proof depth counter |\count10| to 10, and set
%    up the description environment that will take the proof steps. At the end of the
%    proof, we position the proof end into the last line.
%    \begin{macrocode}
\newenvironment{sproof}[2][]{
  \let \premise \stex_proof_premise:
  \intarray_gzero:N \l_@@_counter_intarray
  \intarray_gset:Nnn \l_@@_counter_intarray 1 1
  \_@@_spf_args:n{#1}
  \stex_if_smsmode:TF {
    \str_if_empty:NF \spfid {
      \stex_ref_new_doc_target:n \spfid
    }
  }{
    \seq_clear:N \l_tmpa_seq
    \clist_map_inline:Nn \l_@@_spf_for_clist {
      \tl_if_empty:nF{ ##1 }{
        \stex_get_symbol:n { ##1 }
        \exp_args:NNo \seq_put_right:Nn \l_tmpa_seq {
          \l_stex_get_symbol_uri_str
        }
      }
    }
    \exp_args:Nnnx
    \begin{stex_annotate_env}{sproof}{\seq_use:Nn \l_tmpa_seq {,}}
    \str_if_empty:NF \spftype {
      \stex_annotate_invisible:nnn{type}{\spftype}{}
    }

    \clist_set:No \l_tmpa_clist \spftype
    \tl_clear:N \l_tmpa_tl
    \clist_map_inline:Nn \l_tmpa_clist {
      \tl_if_exist:cT {_@@_sproof_##1_start:}{
        \tl_set:Nn \l_tmpa_tl {\use:c{_@@_sproof_##1_start:}}
      }
      \exp_args:No \str_if_eq:nnT \c_@@_flow_str {##1} {
        \tl_set:Nn \l_tmpa_tl {\use:n{}}
      }
    }
    \tl_if_empty:NTF \l_tmpa_tl {
      \_@@_sproof_start:
    }{
      \l_tmpa_tl
    }{~#2}
    \str_if_empty:NF \spfid {
      \stex_ref_new_doc_target:n \spfid
    }
    \begin{description}
  }
  \stex_smsmode_do:
}{
  \stex_if_smsmode:F{
    \end{description}
    \clist_set:No \l_tmpa_clist \spftype
    \tl_clear:N \l_tmpa_tl
    \clist_map_inline:Nn \l_tmpa_clist {
      \tl_if_exist:cT {_@@_sproof_##1_end:}{
        \tl_set:Nn \l_tmpa_tl {\use:c{_@@_sproof_##1_end:}}
      }
    }
    \tl_if_empty:NTF \l_tmpa_tl {
      \_@@_sproof_end:
    }{
      \l_tmpa_tl
    }
    \end{stex_annotate_env}
  }
}

\cs_new_protected:Nn \_@@_sproof_start: {
  \par\noindent\titleemph{
    \tl_if_empty:NTF \spftype {
      \spf@proof@kw
    }{
      \spftype
    }
  }:
}
\cs_new_protected:Nn \_@@_sproof_end: {\sproofend}

\newcommand\stexpatchproof[3][] {
  \str_set:Nx \l_tmpa_str{ #1 }
  \str_if_empty:NTF \l_tmpa_str {
    \tl_set:Nn \_@@_sproof_start: { #2 }
    \tl_set:Nn \_@@_sproof_end: { #3 }
  }{
    \exp_after:wN \tl_set:Nn \csname _@@_sproof_#1_start:\endcsname{ #2 }
    \exp_after:wN \tl_set:Nn \csname _@@_sproof_#1_end:\endcsname{ #3 }
  }
}
%    \end{macrocode}
% \end{environment}
% 
% \begin{macro}{\spfidea}
%    \begin{macrocode}
\newcommand\spfidea[2][]{
  \_@@_spf_args:n{#1}
  \titleemph{
    \tl_if_empty:NTF \spftype {Proof~Idea}{
      \spftype
    }:
  }~#2
  \sproofend
}
%    \end{macrocode}
% \end{macro}
%
% The next two environments (proof steps) and comments, are mostly semantical, they take
% |KeyVal| arguments that specify their semantic role. In draft mode, they read these
% values and show them. If the surrounding proof had |display=flow|, then no new |\item| is
% generated, otherwise it is. In any case, the proof step number (at the current level) is
% incremented.
% \begin{environment}{spfstep}
%    \begin{macrocode}
\newenvironment{spfstep}[1][]{
  \_@@_spf_args:n{#1}
  \stex_if_smsmode:TF {
    \str_if_empty:NF \spfid {
      \stex_ref_new_doc_target:n \spfid
    }
  }{
    \@in@omtexttrue
    \seq_clear:N \l_tmpa_seq
    \clist_map_inline:Nn \l_@@_spf_for_clist {
      \tl_if_empty:nF{ ##1 }{
        \stex_get_symbol:n { ##1 }
        \exp_args:NNo \seq_put_right:Nn \l_tmpa_seq {
          \l_stex_get_symbol_uri_str
        }
      }
    }
    \exp_args:Nnnx
    \begin{stex_annotate_env}{spfstep}{\seq_use:Nn \l_tmpa_seq {,}}
    \str_if_empty:NF \spftype {
      \stex_annotate_invisible:nnn{type}{\spftype}{}
    }
    \clist_set:No \l_tmpa_clist \spftype
    \tl_set:Nn \l_tmpa_tl {
      \item[\sproofnumber]
      \bool_set_true:N \l_@@_inc_counter_bool
    }
    \clist_map_inline:Nn \l_tmpa_clist {
      \exp_args:No \str_if_eq:nnT \c_@@_flow_str {##1} {
        \tl_clear:N \l_tmpa_tl
      }
    }
    \l_tmpa_tl
    \tl_if_empty:NF \spftitle {
      {(\titleemph{\spftitle})\enspace}
    }
    \str_if_empty:NF \spfid {
      \stex_ref_new_doc_target:n \spfid
    }
  }
  \stex_smsmode_do:
  \ignorespacesandpars
}{
  \bool_if:NT \l_@@_inc_counter_bool {
    \_@@_inc_counter:
  }
  \stex_if_smsmode:F {
    \end{stex_annotate_env}
  }
}
%    \end{macrocode}
% \end{environment}
%
% \begin{environment}{sproofcomment}
%    \begin{macrocode}
\newenvironment{sproofcomment}[1][]{
  \_@@_spf_args:n{#1}
  \clist_set:No \l_tmpa_clist \spftype
  \tl_set:Nn \l_tmpa_tl {
    \item[\sproofnumber]
    \bool_set_true:N \l_@@_inc_counter_bool
  }
  \clist_map_inline:Nn \l_tmpa_clist {
    \exp_args:No \str_if_eq:nnT \c_@@_flow_str {##1} {
      \tl_clear:N \l_tmpa_tl
    }
  }
  \l_tmpa_tl
}{
  \bool_if:NT \l_@@_inc_counter_bool {
    \_@@_inc_counter:
  }
}
%    \end{macrocode}
% \end{environment}
%
% The next two environments also take a |KeyVal| argument, but also a regular one, which
% contains a start text. Both environments start a new numbered proof level.
%
% \begin{environment}{subproof}
%   In the |subproof| environment, a new (lower-level) proproofof environment is started.
%    \begin{macrocode}
\newenvironment{subproof}[2][]{
  \_@@_spf_args:n{#1}
  \stex_if_smsmode:TF{
    \str_if_empty:NF \spfid {
      \stex_ref_new_doc_target:n \spfid
    }
  }{
    \seq_clear:N \l_tmpa_seq
    \clist_map_inline:Nn \l_@@_spf_for_clist {
      \tl_if_empty:nF{ ##1 }{
        \stex_get_symbol:n { ##1 }
        \exp_args:NNo \seq_put_right:Nn \l_tmpa_seq {
          \l_stex_get_symbol_uri_str
        }
      }
    }
    \exp_args:Nnnx
    \begin{stex_annotate_env}{subproof}{\seq_use:Nn \l_tmpa_seq {,}}
    \str_if_empty:NF \spftype {
      \stex_annotate_invisible:nnn{type}{\spftype}{}
    }

    \clist_set:No \l_tmpa_clist \spftype
    \tl_set:Nn \l_tmpa_tl {
      \item[\sproofnumber]
      \bool_set_true:N \l_@@_inc_counter_bool
    }
    \clist_map_inline:Nn \l_tmpa_clist {
      \exp_args:No \str_if_eq:nnT \c_@@_flow_str {##1} {
        \tl_clear:N \l_tmpa_tl
      }
    }
    \l_tmpa_tl
    \tl_if_empty:NF \spftitle {
      {(\titleemph{\spftitle})\enspace}
    }
    {~#2}
    \str_if_empty:NF \spfid {
      \stex_ref_new_doc_target:n \spfid
    }
  }
  \_@@_add_counter:
  \stex_smsmode_do:
}{
  \_@@_remove_counter:
  \bool_if:NT \l_@@_inc_counter_bool {
    \_@@_inc_counter:
  }
  \stex_if_smsmode:F{
    \end{stex_annotate_env}
  }
}
%    \end{macrocode}
% \end{environment}
%
% \begin{environment}{spfcases}
%   In the |pfcases| environment, the start text is displayed as the first comment of the
%   proof.
%    \begin{macrocode}
\newenvironment{spfcases}[2][]{
  \tl_if_empty:nTF{#1}{
    \begin{subproof}[method=by-cases]{#2}
  }{
    \begin{subproof}[#1,method=by-cases]{#2}
  }
}{
  \end{subproof}
}
%    \end{macrocode}
% \end{environment}
%
% \begin{environment}{spfcase}
%    In the |pfcase| environment, the start text is displayed specification of the case
%    after the |\item|
%    \begin{macrocode}
\newenvironment{spfcase}[2][]{
  \_@@_spf_args:n{#1}
  \stex_if_smsmode:TF {
    \str_if_empty:NF \spfid {
      \stex_ref_new_doc_target:n \spfid
    }
  }{
    \seq_clear:N \l_tmpa_seq
    \clist_map_inline:Nn \l_@@_spf_for_clist {
      \tl_if_empty:nF{ ##1 }{
        \stex_get_symbol:n { ##1 }
        \exp_args:NNo \seq_put_right:Nn \l_tmpa_seq {
          \l_stex_get_symbol_uri_str
        }
      }
    }
    \exp_args:Nnnx
    \begin{stex_annotate_env}{spfcase}{\seq_use:Nn \l_tmpa_seq {,}}
    \str_if_empty:NF \spftype {
      \stex_annotate_invisible:nnn{type}{\spftype}{}
    }
    \clist_set:No \l_tmpa_clist \spftype
    \tl_set:Nn \l_tmpa_tl {
      \item[\sproofnumber]
      \bool_set_true:N \l_@@_inc_counter_bool
    }
    \clist_map_inline:Nn \l_tmpa_clist {
      \exp_args:No \str_if_eq:nnT \c_@@_flow_str {##1} {
        \tl_clear:N \l_tmpa_tl
      }
    }
    \l_tmpa_tl
    \tl_if_empty:nF{#2}{
      \titleemph{#2}:~
    }
  }
  \_@@_add_counter:
  \stex_smsmode_do:
}{
  \_@@_remove_counter:
  \bool_if:NT \l_@@_inc_counter_bool {
    \_@@_inc_counter:
  }
  \stex_if_smsmode:F{
    \clist_set:No \l_tmpa_clist \spftype
    \tl_set:Nn \l_tmpa_tl{\sproofend}
    \clist_map_inline:Nn \l_tmpa_clist {
      \exp_args:No \str_if_eq:nnT \c_@@_flow_str {##1} {
        \tl_clear:N \l_tmpa_tl
      }
    }
    \l_tmpa_tl
    \end{stex_annotate_env}
  }
}
%    \end{macrocode}
% \end{environment}
%
% \begin{environment}{spfcase}
%    similar to |spfcase|, takes a third argument. 
%    \begin{macrocode}
\newcommand\spfcasesketch[3][]{
  \begin{spfcase}[#1]{#2}#3\end{spfcase}
}
%    \end{macrocode}
% \end{environment}
%
% \subsection{Justifications}
%
% We define the actions that are undertaken, when the keys for justifications are
% encountered. Here this is very simple, we just define an internal macro with the value,
% so that we can use it later.
%    \begin{macrocode}
\keys_define:nn { stex / just }{
  id        .str_set_x:N  = \l_@@_just_id_str,
  method    .tl_set:N     = \l_@@_just_method_tl,
  premises  .tl_set:N     = \l_@@_just_premises_tl,
  args      .tl_set:N     = \l_@@_just_args_tl
}
%    \end{macrocode}
%
% The next three environments and macros are purely semantic, so we ignore the keyval
% arguments for now and only display the content.\ednote{need to do something about the
% premise in draft mode.}
%
% \begin{environment}{justification}
%    \begin{macrocode}
\newenvironment{justification}[1][]{}{}
%    \end{macrocode}
% \end{environment}
%
% \begin{macro}{\premise}
%    \begin{macrocode}
\newcommand\stex_proof_premise:[2][]{#2}
%    \end{macrocode}
% \end{macro}
%
% \begin{macro}{\justarg}
% the |\justarg| macro is purely semantic, so we ignore the keyval arguments for now and
% only display the content.
%    \begin{macrocode}
\newcommand\justarg[2][]{#2}
%</package>
%    \end{macrocode}
% \end{macro}
% \end{implementation}
% \Finale
\endinput
%%% Local Variables: 
%%% mode: doctex
%%% TeX-master: t
%%% End: 
% LocalWords:  GPL structuresharing STR sproof dtx CPERL keyval methodfalse env
% LocalWords:  methodtrue envtrue medhodtrue DefKeyVal Semiverbatim omdoc args
% LocalWords:  DefEnvironment OptionalKeyVals KeyVal omtext DefConstructor str
% LocalWords:  proofidea KeyVal pfstep DefCMPEnvironment KeyVal proofcomment eq
% LocalWords:  KeyVal pfcases KeyVal pfcase KeyVal extractBodyText unlist elsif
% LocalWords:  foreach getBody toString str str str LookupValue LastSeenCMP Thu
% LocalWords:  appendText getValue undef openElement closeElement DefMacro omd
% LocalWords:  afterClose nodeType childNodes firstCMP localname hasChildNodes
% LocalWords:  firstChild textContent removeChild iffalse kohlhase sref scsys
% LocalWords:  sproofs.sty sc sc mathml openmath latexml cmathml activemath geq
% LocalWords:  twintoo atwin atwintoo texttt fileversion maketitle stex newpage
% LocalWords:  tableofcontents newpage exfig scriptsize vspace ednote spfidea
% LocalWords:  spfidea spfsketch spfsketch spfstep justarg spfcases spfcase rhs
% LocalWords:  sproofcomment ind-hyp splitit arith byindhyp sproofend proofend
% LocalWords:  printbibliography textsf langle textsf langle ltxml ctancite spf
% LocalWords:  srefaddidkey pf.sty newenvironment hbox vrule vbox ifx showmeta
% LocalWords:  hrule vskip hrule vrule hfil nobreak hfill smallskip newcommand
% LocalWords:  stDMemph newcount endsproof xref doctex showmeta hline lec ldots
% LocalWords:  textbackslash makeatletter sketchproof compactenum tracissue
% LocalWords:  metakeys addmetakey metasetkeys stylable pstlabelstyle pstlabel
% LocalWords:  pstlabelstyle pstlabelstyle ldots ldots ensuremath inparaenum
% LocalWords:  nameuse prooflistenv spfcasesketch spfcasesketch spfeq rcll
% LocalWords:  displaymath noindent ignorespaces
