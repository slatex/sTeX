% \iffalse meta-comment
% A LaTeX Class for Semantic Lectures Slides (originally developed for Michael Kohlhase)
% Copyright (c) 2019 Michael Kohlhase, all rights reserved
%               this file is released under the
%               Gnu Library Public Licences (LGPL)
%
% The original of this file is in the public repository at 
% http://github.com/sLaTeX/sTeX/
% \fi
% 
% \iffalse
%
%<*driver>
\providecommand\bibfolder{../../lib/bib}
\RequirePackage{paralist}
\ifcsname stexdocpath\endcsname\else\def\stexdocpath{.}\fi
\documentclass[full]{l3doc}
%\RequirePackage{document-structure}
\usepackage[hyperref=auto,style=alphabetic]{biblatex}
%\usepackage[mathhub=\stexdocpath/mh,usedeps]{stex}
\usepackage[lang={en,de}]{stex}

\usepackage{rustex}
\usepackage{stex-highlighting,stexthm}

\srefsetin[sTeX/Documentation]{documentation}{the \stex Documentation}

\makeatletter
\providecommand{\HTML}{\textsc{html}\xspace}%
\providecommand{\XML}{\textsc{xml}\xspace}%
\providecommand{\PDF}{\textsc{pdf}\xspace}%
\providecommand\openmath{\textsc{OpenMath}\xspace}
\providecommand\OMDoc{\textsc{OMDoc}\xspace}
\DeclareRobustCommand\LaTeXML{L\kern-.36em%
        {\sbox\z@ T%
         \vbox to\ht\z@{\hbox{\check@mathfonts
                              \fontsize\sf@size\z@
                              \math@fontsfalse\selectfont
                              A}%
                        \vss}%
        }%
        \kern-.15em%
%        T\kern-.1667em\lower.5ex\hbox{E}\kern-.125em\relax
%        {\tt XML}}
        T\kern-.1667em\lower.4ex\hbox{E}\kern-0.05em\relax
        {\scshape xml}\xspace}%
\def\mmt{\textsc{Mmt}\xspace}
\makeatother


\newif\ifhadtitle\hadtitlefalse

\def\stexversion{3.3.0}
\def\changedate{\today}
\def\stextoptitle#1#2{\title{#1\thanks{Version {\stexversion} (last revised {\changedate})} }\def\thispkg{#2}}

\author{Michael Kohlhase, Dennis Müller\\
 	FAU Erlangen-Nürnberg\\
 	\url{http://kwarc.info/}
}

\def\stexmaketitle{\ifhadtitle\else\hadtitletrue\maketitle\fi}

\ExplSyntaxOn

  \def\docmodule{
    \begin{document}
      \EnableManual
      \EnableDocumentation
      \EnableImplementation
      \stexmaketitle
      \tableofcontents
      \int_gincr:N \l_stex_docheader_sect
      \exp_args:Ne \__stex_mathhub_find_manifest:n {\stex_file_use:N \c_stex_mathhub_file / sTeX / Documentation}
      \str_if_empty:NF \l__stex_mathhub_manifest_str {
        \usemodule[sTeX/Documentation]{macros?AllMacros}
      }
      \DocInput{\jobname.dtx}
      \clearpage
      \PrintIndex
      \printbibliography
    \end{document}
  }

  \bool_new:N \g_stexdoc_typeset_manual_bool
  \NewDocumentCommand \EnableManual {}{
    \bool_gset_true:N \g_stexdoc_typeset_manual_bool
  }
  \NewDocumentCommand \DisableManual {}{
    \bool_gset_false:N \g_stexdoc_typeset_manual_bool
  }
  \NewDocumentEnvironment {stexmanual} {} {
    \bool_if:NTF \g_stexdoc_typeset_manual_bool
      {\bool_set_false:N \l__codedoc_in_implementation_bool}
      {\comment}
  }{
    \bool_if:NF \g_stexdoc_typeset_manual_bool {\endcomment}
  }
\ExplSyntaxOff

%\usepackage{makeidx}
%\makeindex

%\usepackage{document-structure}


\usepackage{lststex,mdframed}
\usepackage[most]{tcolorbox}

\lstset{literate=%
    {Ö}{{\"O}}1
    {Ä}{{\"A}}1
    {Ü}{{\"U}}1
    {ß}{{\ss}}1
    {ü}{{\"u}}1
    {ä}{{\"a}}1
    {ö}{{\"o}}1
    {~}{{\textasciitilde}}1
}

\newenvironment{framed}[1][]{
  \ifstexhtml\par\vbox\bgroup
    \csname exp_args:Nne\endcsname\begin{stex_annotate_env}{%
      style:border=solid 1px black,%
      style:width=var(--this-width),%
      style:min-width=var(--this-width),%
      style:--this-width=calc(var(--current-width) - 6px),%
      style:padding=3px,%
      style:margin-top=5px,%
      style:margin-bottom=5px%
    }
    \csname stex_annotate_invisible:n\endcsname{ }%
    \begin{stex_annotate_env}{%
      style:--current-width=var(--this-width);%
    }\csname stex_annotate_invisible:n\endcsname{ }
  \else\begin{mdframed}[#1]\fi
  %\begin{center}%
}{%
  %\end{center}%
  \ifstexhtml
    \end{stex_annotate_env}\end{stex_annotate_env}\egroup\par
  \else\end{mdframed}\fi
}
\newcommand{\scaled}[2][0.9\hsize]{\begin{center}\resizebox{#1}{!}{\begin{minipage}{\textwidth} #2 \end{minipage}}\end{center}}

\makeatletter
\ExplSyntaxOn

\def\doc_exbox:nnn#1#2#3{
  \begin{sexample}[#3]
    Input:
    \begin{framed}[linewidth=1pt,backgroundcolor=white]\small
      #1
    \end{framed}
    Output:
    \begin{framed}[linewidth=1pt,backgroundcolor=white]\small
      #2
    \end{framed}
  \end{sexample}
}


\NewDocumentCommand\stexexamplefile{O{} m O{} O{}}{
  \stex_resolve_path_pair:Nxx \l_@@_filepath_str {\tl_to_str:n{#1}} {\tl_to_str:n{#2}}
  \doc_exbox:nnn{
    \hfill File~\str_if_empty:nTF{#1}{
      \prop_if_exist:NT \l_stex_current_archive_prop {
        [\texttt{\prop_item:Nn \l_stex_current_archive_prop {id}}]
      }
    }{[#1]}\texttt{\tl_to_str:n{#2}}
    \_lststex_parse_args:n{#3}
    \exp_args:Nno \use:nn{\lstinputlisting[} \l_lststex_return_tl ]{\l_@@_filepath_str}
  }{
    \inputref[#1]{#2}
  }{#4}
}

\newwrite\testoutfile
\NewDocumentCommand\stexexample{O{} O{}}{
  \begingroup 
  \catcode`\\=12\relax
  \catcode`\#=12\relax
  \catcode`\&=12\relax
  \catcode`\$=12\relax
  \catcode`\^=12\relax
  \catcode`\_=12\relax
  \catcode`\ =12\relax
  \catcode`^^J=12\relax
  \endlinechar=`^^J
  \newlinechar=-1
%^^A    \everyeof{\noexpand}
  \example_a:nnn{#1}{#2}
}
\long\def\example_a:nnn #1 #2 #3 {
  \endgroup
  \immediate\openout\testoutfile=\jobname.exmpl
  \immediate\write\testoutfile{
    \c_backslash_str begin{stexcode}[#1]
    \detokenize{^^J}#3
    \c_backslash_str end{stexcode}
  }
  \immediate\closeout\testoutfile
  \doc_exbox:nnn{
    \catcode`\#=12\relax
    \csname @ @ input\endcsname{\jobname.exmpl}
  }{
    \immediate\openout\testoutfile=\jobname.exmpl
    \immediate\write\testoutfile{#3}
    \immediate\closeout\testoutfile
    \csname @ @ input\endcsname \jobname.exmpl\relax
  }{#2}
  \peek_charcode_remove:NT ^^J
}

\ExplSyntaxOff
\makeatother

\makeatletter
\newcount\example@counter\example@counter=0
\newtcolorbox{exampleborderbox}[1][]{
  empty,
  title={Example \the\example@counter #1},
  attach boxed title to top left,
     minipage boxed title,
  boxed title style={empty,size=minimal,toprule=0pt,top=1pt,left=3mm,overlay={}},
  coltitle=blue,fonttitle=\bfseries,
  parbox=false,boxsep=0pt,left=3mm,right=0mm,top=2pt,breakable,pad at break=0mm,
     before upper=\csname @totalleftmargin\endcsname0pt, 
  overlay unbroken={\draw[blue,line width=2pt] ([xshift=-0pt]title.north west) -- ([xshift=-0pt]frame.south west); },
  overlay first={\draw[blue,line width=2pt] ([xshift=-0pt]title.north west) -- ([xshift=-0pt]frame.south west); },
  overlay middle={\draw[blue,line width=2pt] ([xshift=-0pt]frame.north west) -- ([xshift=-0pt]frame.south west); },
  overlay last={\draw[blue,line width=2pt] ([xshift=-0pt]frame.north west) -- ([xshift=-0pt]frame.south west); },
  outer arc=4pt%
}

\ExplSyntaxOn
\stexstyleexample{
  \global\advance\example@counter by 1
  \tl_if_empty:NTF\thistitle{
    \begin{exampleborderbox}
  }{
    \begin{exampleborderbox}[ (\thistitle)]
  }
}{
    \end{exampleborderbox}
}

\ExplSyntaxOff\makeatother

\usetikzlibrary{calc}

\def\textwarning{\includegraphics[width=1.2em]{stex-dangerous-bend}\xspace}
\newtcolorbox{dangerbox}{
  breakable,
  enhanced,
  left=0pt,
  right=0pt,
  top=8pt,
  bottom=8pt,
  colback=white,
  colframe=red,
  width=\textwidth,
  enlarge left by=0mm,
  boxsep=5pt,
  fontupper=\small,
  arc=4pt,
  outer arc=4pt,
  leftupper=1.5cm,
  overlay={
    \node[anchor=west] at ([xshift=10pt]$(frame.north west)!0.5!(frame.south west)$)
       {\includegraphics[width=1cm,height=1cm]{stex-dangerous-bend}};}
}

\protected\def\TODO#1{\textcolor{red}{TODO}\footnote{\textcolor{red}{TODO: #1}}}

\definecolor{darkgreen}{rgb}{0.0, 0.5, 0.0}

\usepackage[solutions]{problem}
\usepackage{hwexam}
\newtcolorbox{problemborderbox}[1][]{
  empty,
  title={Exercise #1},
  attach boxed title to top left,
     minipage boxed title,
  boxed title style={empty,size=minimal,toprule=0pt,top=1pt,left=3mm,overlay={}},
  coltitle=darkgreen,fonttitle=\bfseries,
  parbox=false,boxsep=0pt,left=3mm,right=0mm,top=2pt,breakable,pad at break=0mm,
     before upper=\csname @totalleftmargin\endcsname0pt, 
  overlay unbroken={\draw[darkgreen,line width=2pt] ([xshift=-0pt]title.north west) -- ([xshift=-0pt]frame.south west); },
  overlay first={\draw[darkgreen,line width=2pt] ([xshift=-0pt]title.north west) -- ([xshift=-0pt]frame.south west); },
  overlay middle={\draw[darkgreen,line width=2pt] ([xshift=-0pt]frame.north west) -- ([xshift=-0pt]frame.south west); },
  overlay last={\draw[darkgreen,line width=2pt] ([xshift=-0pt]frame.north west) -- ([xshift=-0pt]frame.south west); },
  outer arc=4pt%
}

\ExplSyntaxOn
\stexstyleproblem{
  \tl_if_empty:NTF\thistitle{
    \begin{problemborderbox}
  }{
    \begin{problemborderbox}[ (\thistitle)]
  }
}{
    \end{problemborderbox}
}
\ExplSyntaxOff

\newtcolorbox{experimental}{
  breakable,
  enhanced,
  left=0pt,
  right=0pt,
  top=8pt,
  bottom=8pt,
  colback=white,
  colframe=gray,
  width=\textwidth,
  enlarge left by=0mm,
  boxsep=5pt,
  fontupper=\small,
  arc=4pt,
  outer arc=4pt,
  leftupper=1.5cm,
  overlay={
    \node[anchor=west] at ([xshift=10pt]$(frame.north west)!0.5!(frame.south west)$)
       {\includegraphics[height=1cm]{stex-experimental}};}
}


\usetikzlibrary{decorations.pathmorphing,shapes,arrows,calc}
% Taken from pgflibrarytikzmmt.code.tex
\newcommand{\mmtarrowtip}{angle 45}
\newcommand{\mmtarrowtipmonoright}{right hook}

\tikzstyle{include}=[\mmtarrowtipmonoright-\mmtarrowtip,thick]
\tikzstyle{morph}=[-\mmtarrowtip,thick]
\tikzstyle{preview}=[decorate, decoration={coil,aspect=0,amplitude=1pt,
                                                  segment length=6pt,
                                                  pre=lineto,pre length=3pt,
                                                  post=lineto,post length=5pt}, thick]
\tikzstyle{view}=[preview,-\mmtarrowtip]


% TIKZ RULES
\def\mmtlogo{
\begin{tikzpicture}

  % White Background (Margins are eyeballed)
  % This is necessary because we paste white over arrows later.
  % If somebody want's to do the full song and dance with
  % interrupted arrows to get transparent background, be my guest.

  \fill[white!] (-0.01,0.15) rectangle (1.11,-0.95);

  % Arrows
  \draw [blue, include] (0,0)     -- (1.1,0);
  \draw [green, morph] (0,-0.4)  -- (1.1,-0.4);
  \draw [red, view]   (-0,-0.8) -- (1.1,-0.8);

  % Cutout for letters
  \fill[white] (0.33,0.1) rectangle (0.66,-0.9);

  % Letters
  \node at (0.18,0)    (nodeM1) {\large M};
  \node at (0.18,-0.4) (nodeM2) {\large M};
  \node at (0.21,-0.8) (nodeT)  {\large T};

\end{tikzpicture}
}

\newtcolorbox{mmtbox}{
  breakable,
  enhanced,
  left=0pt,
  right=0pt,
  top=8pt,
  bottom=8pt,
  colback=white,
  colframe=green,
  width=\textwidth,
  enlarge left by=0mm,
  boxsep=5pt,
  fontupper=\small,
  arc=4pt,
  outer arc=4pt,
  leftupper=1.5cm,
  overlay={
    \node[anchor=west] at ([xshift=10pt]$(frame.north west)!0.5!(frame.south west)$)
       {\mmtlogo};}
}

\AtBeginDocument{\catcode`_=8}

\begin{document}
  \DocInput{\jobname.dtx}
\end{document}
%</driver>
% \fi
% 
% \GetFileInfo{notesslides.cls}
% \MakeShortVerb{\|}
%
% \def\twin#1#2{\index{#1!#2}\index{#2!#1}}
% \def\twintoo#1#2{{#1 #2}\twin{#1}{#2}}
% \def\atwin#1#2#3{\index{#1!#2!#3}\index{#3!#2 (#1)}}
% \def\atwintoo#1#2#3{{#1 #2 #3}\atwin{#1}{#2}{#3}}
%
% \def\scsys#1{{{\sc #1}}\index{#1@{\sc #1}}}
% \def\stex{\hbox{\raisebox{-.5ex}S\kern-.5ex\TeX}}
% \def\sTeX{\stex}
% \def\cnxml{\scshape{CNXml}}
% \def\connexions{\scshape{Connexions}}
% \def\element#1{{\ttfamily{#1}}}
% \def\snippet#1{{\ttfamily{#1}}}
% \def\cnxlatex{CNX\LaTeX\xspace}
% \def\mathml{{\scshape{MathML}}\xspace}
% \def\omdoc{OMDoc\xspace}
% \def\activemath{{\scshape{ActiveMath}}\xspace}
% \def\textwarning{\includegraphics[width=1.2em]{stex-dangerous-bend}\xspace}
% 
% \title{NotesSlides -- Slides and Course Notes\thanks{Version {\fileversion}
% (last revised {\filedate})}}
%    \author{Michael Kohlhase\\
%            FAU Erlangen-N\"urnberg\\
%            \url{http://kwarc.info/kohlhase}}
% \maketitle
%
%\ifinfulldoc\else
% This is the documentation for the \pkg{notesslides} package.
% For a more high-level introduction, 
%  see \href{\basedocurl/manual.pdf}{the \sTeX Manual} or the
% \href{\basedocurl/stex.pdf}{full \sTeX documentation}.
%
% \textcolor{red}{TODO: notesslides documentation}
% \fi
%
% \begin{documentation}\label{pkg:notesslides:doc}
% \changes{3.1.0}{2022/02/28}{Added \detokenize{\libusetheme}}
%
%   We present a document class from which we can generate both course slides and course
%   notes in a transparent way.
%
%\section{Introduction}
%
% The |notesslides| document class is derived from |beamer.cls|~\cite{beamerclass:on}, it
% adds a ``notes version'' for course notes derived from the |omdoc|
% class~\cite{Kohlhase:smomdl} that is more suited to printing than the one supplied by
% |beamer.cls|.
%
% 
%\section{The User Interface}\label{sec:user}
%
% The |notesslides| class takes the notion of a slide frame from Till Tantau's excellent
% |beamer| class and adapts its notion of frames for use in the \sTeX and \omdoc. To
% support semantic course notes, it extends the notion of mixing frames and explanatory
% text, but rather than treating the frames as images (or integrating their contents into
% the flowing text), the |notesslides| package displays the slides as such in the course
% notes to give students a visual anchor into the slide presentation in the course (and to
% distinguish the different writing styles in slides and course notes).
% 
% In practice we want to generate two documents from the same source: the slides for
% presentation in the lecture and the course notes as a narrative document for home
% study. To achieve this, the |notesslides| class has two modes: \emph{slides mode} and
% \emph{notes mode} which are determined by the package option. 
%
% \subsection{Package Options}\label{sec:user:options}
% 
% The |notesslides| class takes a variety of class options:\ednote{leaving out noproblems
% for the moment until we decide what to do with it.}
% \begin{itemize}
% \item The options \DescribeMacro{slides}|slides| and \DescribeMacro{notes}|notes| switch
%   between slides mode and notes mode (see Section~\ref{sec:user:notesslides}).
% \item If the option \DescribeMacro{sectocframes}|sectocframes| is given, then for the
%   |omgroup|s, special frames with the |omgroup| title (and number) are generated.
% \item \DescribeMacro{showmeta}|showmeta|. If this is set, then the metadata keys are
%   shown (see~\cite{Kohlhase:metakeys:git} for details and customization options).
% \item If the option \DescribeMacro{frameimages}|frameimages| is set, then slide mode
%   also shows the |\frameimage|-generated frames (see
%   section~\ref{sec:user:frameimage}). If also the \DescribeMacro{fiboxed}|fiboxed|
%   option is given, the slides are surrounded by a box.
% \item \DescribeMacro{topsect}|topsect=|\meta{sect} can be used to specify the
%   top-level sectioning level; the default for \meta{sect} is |section|.
% \end{itemize}
% 
% \subsection{Notes and Slides}\label{sec:user:notesslides}
% 
% Slides are represented with the \DescribeEnv{frame}|frame| just like in the |beamer|
% class, see~\cite{Tantau:ugbc} for details. The |notesslides| class adds the
% \DescribeEnv{note}|note| environment for encapsulating the course note
% fragments.\footnote{MK: it would be very nice, if we did not need this environment, and
% this should be possible in principle, but not without intensive LaTeX trickery. Hints to
% the author are welcome.} 
%
% \textwarning Note that it is essential to start and end the |notes| environment at the
% start of the line -- in particular, there may not be leading blanks -- else {\LaTeX}
% becomes confused and throws error messages that are difficult to decipher.
%
% \begin{exfig}
% \begin{verbatim}
% \ifnotes\maketitle\else
% \frame[noframenumbering]\maketitle\fi
% 
% \begin{note}
%   We start this course with ...
% \end{note}
%
% \begin{frame}
%   \frametitle{The first slide}
%   ...
% \end{frame}
% \begin{note}
%   ... and more explanatory text
% \end{note}
%
% \begin{frame}
%   \frametitle{The second slide}
%   ...
% \end{frame}
% ...
% \end{verbatim}
% \caption{A typical Course Notes File}\label{fig:notesfile}
% \end{exfig}
% 
% By interleaving the |frame| and |note| environments, we can build course notes as shown
% in Figure~\ref{fig:notesfile}.
%
% Note the use of the \DescribeMacro{\ifnotes}|\ifnotes| conditional, which allows
% different treatment between |notes| and |slides| mode -- manually setting |\notestrue|
% or |\notesfalse| is strongly discouraged however.
% 
% \textwarning: We need to give the title frame the |noframenumbering| option so that the
% frame numbering is kept in sync between the slides and the course notes.
%
% \textwarning: The |beamer| class recommends not to use the |allowframebreaks| option on
% frames (even though it is very convenient). This holds even more in the |notesslides|
% case: At least in conjunction with |\newpage|, frame numbering behaves funnily (we have
% tried to fix this, but who knows). 
% 
% If we want to transclude a the contents of a file as a note, we can use a new variant
% \DescribeMacro{\inputref*}|\inputref*| of the |\inputref| macro from
% \cite{KohAmb:smmssl:git}: |\inputref*{foo}| is equivalent to
% |\begin{note}\inputref{foo}\end{note}|.
% 
% There are some environments that tend to occur at the top-level of |note|
% environments. We make convenience versions of these: e.g. the
% \DescribeEnv{nparagraph}|nparagraph| environment is just an |sparagraph| inside a |note|
% environment (but looks nicer in the source, since it avoids one level of source
% indenting). Similarly, we have the \DescribeEnv{nfragment}|nomgroup|,
% \DescribeEnv{ndefinition}|ndefinition|, \DescribeEnv{nexample}|nexample|,
% \DescribeEnv{nsproof}|nsproof|, and \DescribeEnv{nassertion}|nassertion| environments.
%  
% \subsection{Header and Footer Lines of the Slides}\label{sec:user:headfootlines}
%
% The default logo provided by the |notesslides| package is the {\sTeX} logo it can be
% customized using \DescribeMacro{\setslidelogo}|\setslidelogo{|\meta{logo name}|}|.
%
% The default footer line of the |notesslides| package mentions copyright and licensing. In
% the |beamer| class, |\source| stores the author's name as the copyright holder . By
% default it is {\it Michael Kohlhase} in the |notesslides| package since he is the main
% user and designer of this package. \DescribeMacro{\setsource}|\setsource{|\meta{name}|}|
% can change the writer's name. For licensing, we use the Creative Commons
% Attribuition-ShareAlike license by default to strengthen the public domain. If package
% |hyperref| is loaded, then we can attach a hyperlink to the license
% logo. \DescribeMacro{\setlicensing}|\setlicensing[|\meta{url}|]{|\meta{logo name}|}| is
% used for customization, where \meta{url} is optional.
%
% \subsection{Frame Images}\label{sec:user:frameimage}
% 
% Sometimes, we want to integrate slides as images after all -- e.g. because we already
% have a PowerPoint presentation, to which we want to add \sTeX notes. In this case we can
% use \DescribeMacro{\frameimage}|\frameimage[|\meta{opt}|]{|\meta{path}|}|, where
% \meta{opt} are the options of |\includegraphics| from the |graphicx|
% package~\cite{CarRah:tpp99} and \meta{path} is the file path (extension can be left off
% like in |\includegraphics|). We have added the |label| key that allows to give a frame
% label that can be referenced like a regular |beamer| frame.\ednote{MK: the hyperref link
% does not seem to work yet. I wonder why but do not have the time to fix it.}
% 
% The \DescribeMacro{\mhframeimage}|\mhframeimage| macro is a variant of |\frameimage|
% with repository support. Instead of writing
% \begin{verbatim}
% \frameimage{\MathHub{fooMH/bar/source/baz/foobar}}
% \end{verbatim}
% we can simply write (assuming that |\MathHub| is defined as above)
% \begin{verbatim}
% \mhframeimage[fooMH/bar]{baz/foobar}
% \end{verbatim}
% Note that the |\mhframeimage| form is more semantic, which allows more advanced document
% management features in \textsf{MathHub}.
% 
% If |baz/foobar| is the ``current module'', i.e. if we are on the \textsf{MathHub} path
% \ldots|MathHub/fooMH/bar|\ldots, then stating the repository in the first optional
% argument is redundant, so we can just use
% \begin{verbatim}
% \mhframeimage{baz/foobar}
% \end{verbatim}
%
% \subsection{Colors and Highlighting}\label{sec:user:highlighting}
% The \DescribeMacro{\textwarning}|\textwarning| macro generates a warning
% sign: \textwarning
%
% \subsection{Front Matter, Titles, etc.}\label{sec:user:matter}
%
% \subsection{Excursions}\label{sec:user:excursions}
%
% In course notes, we sometimes want to point to an ``excursion'' -- material that is
% either presupposed or tangential to the course at the moment -- e.g. in an appendix. The
% typical setup is the following:
% 
% \begin{verbatim}
% \excursion{founif}{../ex/founif}{We will cover first-order unification in}
% ...
% \begin{appendix}\printexcursions\end{appendix}
% \end{verbatim}
%
% The \DescribeMacro{\excursion}|\excursion{|\meta{ref}|}{|\meta{path}|}{|\meta{text}|}| is
% syntactic sugar for \DescribeMacro{\activateexcursion}
%
% \begin{verbatim}
% \begin{nparagraph}[title=Excursion]
%   \activateexcursion{founif}{../ex/founif}
%   We will cover first-order unification in \sref{founif}.
% \end{nparagraph}
% \end{verbatim}
%
% where \DescribeMacro{\activateexcursion}|\activateexcursion{|\meta{path}|}| augments the
% |\printexcursions| macro by a call |\inputref{|\meta{path}|}|. In this way, the3
% \DescribeMacro{\printexcursions}|\printexcursions| macro (usually in the appendix) will
% collect up all excursions that are specified in the main text.
%
% Sometimes, we want to reference -- in an excursion -- part of another. We can use
% \DescribeMacro{\excursionref}|\excursionref{|\meta{label}|}| for that.
%
% Finally, we usually want to put the excursions into an |omgroup| environment and add an
% introduction, therefore we provide the a variant of the |\printexcursions| macro:
% \DescribeMacro{\excursiongroup}|\excursiongroup[id=|\meta{id}|,intro=|\meta{path}|]| is
% equivalent to
% \begin{verbatim}
% \begin{note}
% \begin{sfragment}[id=<id>]{Excursions}
%   \inputref{<path>}
%   \printexcursions
% \end{sfragment}
% \end{note}
% \end{verbatim}
% 
% \subsection{Miscellaneous}\label{sec:user:misc}
%
% \section{Limitations}\label{sec:limitations}
% 
% In this section we document known limitations. If you want to help alleviate them,
% please feel free to contact the package author. Some of them are currently discussed in
% the \sTeX GitHub repository~\cite{sTeX:github:on}. 
% \begin{enumerate}
% \item when option |book| which uses |\pagestyle{headings}| is given and semantic macros
%   are given in the |omgroup| titles, then they sometimes are not defined by the time the
%   heading is formatted. Need to look into how the headings are made. This is a problem
%   of the underlying |omdoc| package.
% \end{enumerate}
% 
% \end{documentation}
%
% \begin{implementation}\label{pkg:notesslides:impl}
% 
%\section{NotesSlides -- Implementation}\label{sec:impl}
%
%\subsection{Class and Package Options}\label{sec:impl:init}
%
% We define some Package Options and switches for the |notesslides| class and activate them
% by passing them on to |beamer.cls| and |omdoc.cls| and the |notesslides| package. We pass
% the |nontheorem| option to the |statements| package when we are not in notes mode, since
% the |beamer| package has its own (overlay-aware) theorem environments. 
%
%    \begin{macrocode}
%<*cls>
%<@@=notesslides>
\ProvidesExplClass{notesslides}{2022/02/28}{3.1.0}{notesslides Class}
\RequirePackage{l3keys2e}

\keys_define:nn{notesslides / cls}{
  class   .str_set_x:N = \c_@@_class_str,
  notes   .bool_set:N  = \c_@@_notes_bool ,
  slides  .code:n      = { \bool_set_false:N \c_@@_notes_bool },
  docopt  .str_set_x:N = \c_@@_docopt_str,
  unknown .code:n      = {
    \PassOptionsToPackage{\CurrentOption}{document-structure}
    \PassOptionsToClass{\CurrentOption}{beamer}
    \PassOptionsToPackage{\CurrentOption}{notesslides}
    \PassOptionsToPackage{\CurrentOption}{stex}
  }
}
\ProcessKeysOptions{ notesslides / cls }


\exp_args:No \str_if_eq:nnT\c_@@_class_str{book}{
  \PassOptionsToPackage{defaulttopsec=part}{notesslides}
}
\exp_args:No \str_if_eq:nnT\c_@@_class_str{report}{
  \PassOptionsToPackage{defaulttopsec=part}{notesslides}
}

\RequirePackage{stex}
\stex_html_backend:T {
  \bool_set_true:N\c_@@_notes_bool
}

\bool_if:NTF \c_@@_notes_bool {
  \PassOptionsToPackage{notes=true}{notesslides}
}{
  \PassOptionsToPackage{notes=false}{notesslides}
}
%</cls>
%    \end{macrocode}
% now we do the same for the |notesslides| package. 
%    \begin{macrocode}
%<*package>
\ProvidesExplPackage{notesslides}{2022/02/28}{3.1.0}{notesslides Package}
\RequirePackage{l3keys2e}

\keys_define:nn{notesslides / pkg}{
  topsect         .str_set_x:N  = \c_@@_topsect_str,
  defaulttopsect  .str_set_x:N  = \c_@@_defaulttopsec_str,
  notes           .bool_set:N   = \c_@@_notes_bool ,
  slides          .code:n       = { \bool_set_false:N \c_@@_notes_bool },
  sectocframes    .bool_set:N   = \c_@@_sectocframes_bool ,
  frameimages     .bool_set:N   = \c_@@_frameimages_bool ,
  fiboxed         .bool_set:N   = \c_@@_fiboxed_bool ,
  noproblems      .bool_set:N   = \c_@@_noproblems_bool,
  unknown         .code:n       = {
    \PassOptionsToClass{\CurrentOption}{stex}
    \PassOptionsToClass{\CurrentOption}{tikzinput}
  }
}
\ProcessKeysOptions{ notesslides / pkg }

\RequirePackage{stex}
\stex_html_backend:T {
  \bool_set_true:N\c_@@_notes_bool
}

\newif\ifnotes
\bool_if:NTF \c_@@_notes_bool {
  \notestrue
}{
  \notesfalse
}

%    \end{macrocode}
% we give ourselves a macro |\@@topsect| that needs only be evaluated once, so that the
% |\ifdefstring| conditionals work below.
%    \begin{macrocode}
\str_if_empty:NTF \c_@@_topsect_str {
  \str_set_eq:NN \@@topsect \c_@@_defaulttopsec_str
}{
  \str_set_eq:NN \@@topsect \c_@@_topsect_str
}
%</package>
%    \end{macrocode}
%
% Depending on the options, we either load the |article|-based |document-structure| or the |beamer|
% class (and set some counters).
%    \begin{macrocode}
%<*cls>
\bool_if:NTF \c_@@_notes_bool {
  \str_if_empty:NT \c_@@_class_str {
    \str_set:Nn \c_@@_class_str {article}
  }
  \exp_after:wN\LoadClass\exp_after:wN[\c_@@_docopt_str]
    {\c_@@_class_str}
}{
  \LoadClass[10pt,notheorems,xcolor={dvipsnames,svgnames}]{beamer}
  \newcounter{Item}
  \newcounter{paragraph}
  \newcounter{subparagraph}
  \newcounter{Hfootnote}
}
\RequirePackage{document-structure}
%    \end{macrocode}
% now it only remains to load the |notesslides| package that does all the rest. 
%    \begin{macrocode}
\RequirePackage{notesslides}
%</cls>
%    \end{macrocode}
% 
% In |notes| mode, we also have to make the |beamer|-specific things available to
% |article| via the |beamerarticle| package. We use options to avoid loading theorem-like
% environments, since we want to use our own from the $\sTeX$ packages.  The first batch
% of packages we want are loaded on |notesslides.sty|. These are the general ones, we will
% load the \sTeX-specific ones after we have done some work (e.g. defined the counters
% |m*|). Only the |stex-logo| package is already needed now for the default theme.
%
%    \begin{macrocode}
%<*package>
\bool_if:NT \c_@@_notes_bool {
  \RequirePackage{a4wide}
  \RequirePackage{marginnote}
  \PassOptionsToPackage{usenames,dvipsnames,svgnames}{xcolor}
  \RequirePackage{mdframed}
  \RequirePackage[noxcolor,noamsthm]{beamerarticle}
  \RequirePackage[bookmarks,bookmarksopen,bookmarksnumbered,breaklinks,hidelinks]{hyperref}
}
\RequirePackage{stex-tikzinput}
\RequirePackage{etoolbox}
\RequirePackage{amssymb}
\RequirePackage{amsmath}
\RequirePackage{comment}
\RequirePackage{textcomp}
\RequirePackage{url}
\RequirePackage{graphicx}
\RequirePackage{pgf}
%    \end{macrocode}
% 
% \subsection{Notes and Slides}\label{sec:impl:noteslides}
%
% For the lecture notes cases, we also provide the |\usetheme| macro that would otherwise
% come from the the |beamer| class. While the latter loads
% |beamertheme|\meta{theme}{.sty}, the notes version loads
% |beamernotestheme|\meta{theme}|.sty|.\ednote{MK: This is not ideal, but I am not sure
% that I want to be able to provide the full theme functionality there.}
%    \begin{macrocode}
\bool_if:NT \c_@@_notes_bool {
  \renewcommand\usetheme[2][]{\usepackage[#1]{beamernotestheme#2}}
}


\NewDocumentCommand \libusetheme {O{} m} {
  \bool_if:NTF \c_@@_notes_bool {
    \libusepackage[#1]{beamernotestheme#2}
  }{
  \libusepackage[#1]{beamertheme#2}
  }
}
%    \end{macrocode}
% We define the sizes of slides in the notes. Somehow, we cannot get by with the same
% here. 
%
%    \begin{macrocode}
\newcounter{slide}
\newlength{\slidewidth}\setlength{\slidewidth}{13.5cm}
\newlength{\slideheight}\setlength{\slideheight}{9cm}
%    \end{macrocode}
% 
% \begin{environment}{note}
% The |note| environment is used to leave out text in the |slides| mode. It does not have
% a counterpart in OMDoc. So for course notes, we define the |note| environment to be a
% no-operation otherwise we declare the |note| environment as a comment via the |comment|
% package.
%    \begin{macrocode}
\bool_if:NTF \c_@@_notes_bool {
  \renewenvironment{note}{\ignorespaces}{}
}{
  \excludecomment{note}
}
%    \end{macrocode}
% \end{environment}
% 
% We first set up the slide boxes in |article| mode. We set up sizes and provide a
% box register for the frames and a counter for the slides.
% 
%    \begin{macrocode}
\bool_if:NT \c_@@_notes_bool {
  \newlength{\slideframewidth}
  \setlength{\slideframewidth}{1.5pt}
%    \end{macrocode}
% 
% \begin{environment}{frame}
%   We first define the keys. 
%    \begin{macrocode}
  \cs_new_protected:Nn \_@@_do_yes_param:Nn {
    \exp_args:Nx \str_if_eq:nnTF { \str_uppercase:n{ #2 } }{ yes }{
      \bool_set_true:N #1
    }{
      \bool_set_false:N #1
    }
  }
  \keys_define:nn{notesslides / frame}{
    label               .str_set_x:N  = \l_@@_frame_label_str,
    allowframebreaks    .code:n       = {
      \_@@_do_yes_param:Nn \l_@@_frame_allowframebreaks_bool { #1 }
    },
    allowdisplaybreaks  .code:n       = {
      \_@@_do_yes_param:Nn \l_@@_frame_allowdisplaybreaks_bool { #1 }
    },
    fragile             .code:n       = {
      \_@@_do_yes_param:Nn \l_@@_frame_fragile_bool { #1 }
    },
    shrink              .code:n       = {
      \_@@_do_yes_param:Nn \l_@@_frame_shrink_bool { #1 }
    },
    squeeze             .code:n       = {
      \_@@_do_yes_param:Nn \l_@@_frame_squeeze_bool { #1 }
    },
    t                   .code:n       = {
      \_@@_do_yes_param:Nn \l_@@_frame_t_bool { #1 }
    },
  }
  \cs_new_protected:Nn \_@@_frame_args:n {
    \str_clear:N \l_@@_frame_label_str
    \bool_set_true:N \l_@@_frame_allowframebreaks_bool
    \bool_set_true:N \l_@@_frame_allowdisplaybreaks_bool
    \bool_set_true:N \l_@@_frame_fragile_bool
    \bool_set_true:N \l_@@_frame_shrink_bool
    \bool_set_true:N \l_@@_frame_squeeze_bool
    \bool_set_true:N \l_@@_frame_t_bool
    \keys_set:nn { notesslides / frame }{ #1 }
  }
%    \end{macrocode}
% We define the environment, read them, and construct the slide number and label.
%    \begin{macrocode}
  \renewenvironment{frame}[1][]{
    \_@@_frame_args:n{#1}
    \sffamily
    \stepcounter{slide}
    \def\@currentlabel{\theslide}
    \str_if_empty:NF \l_@@_frame_label_str {
      \label{\l_@@_frame_label_str}
    }
%    \end{macrocode}
%   We redefine the |itemize| environment so that it looks more like the one in |beamer|. 
%    \begin{macrocode}
    \def\itemize@level{outer}
    \def\itemize@outer{outer}
    \def\itemize@inner{inner}
    \renewcommand\newpage{\addtocounter{framenumber}{1}}
    \newcommand\metakeys@show@keys[2]{\marginnote{{\scriptsize ##2}}}
    \renewenvironment{itemize}{
      \ifx\itemize@level\itemize@outer
        \def\itemize@label{$\rhd$}
      \fi
      \ifx\itemize@level\itemize@inner
        \def\itemize@label{$\scriptstyle\rhd$}
      \fi
      \begin{list}
      {\itemize@label}
      {\setlength{\labelsep}{.3em}
       \setlength{\labelwidth}{.5em}
       \setlength{\leftmargin}{1.5em}
      }
      \edef\itemize@level{\itemize@inner}
    }{
      \end{list}
    }
%    \end{macrocode}
% We create the box with the |mdframed| environment from the equinymous package.
%    \begin{macrocode}
    \stex_html_backend:TF {
      \begin{stex_annotate_env}{frame}{}\vbox\bgroup
    }{
      \begin{mdframed}[linewidth=\slideframewidth,skipabove=1ex,skipbelow=1ex,userdefinedwidth=\slidewidth,align=center]\sf
    }
  }{
    \stex_html_backend:TF {
      \miko@slidelabel\egroup\end{stex_annotate_env}
    }{\medskip\miko@slidelabel\end{mdframed}}
  }
%    \end{macrocode}
% \end{environment}
% 
% Now, we need to redefine the frametitle (we are still in course notes mode). 
% \begin{macro}{\frametitle}
%    \begin{macrocode}
  \renewcommand{\frametitle}[1]{{\Large\bf\sf\color{blue}{#1}}\medskip}
}
%    \end{macrocode}
% \end{macro}
%
% \begin{macro}{\pause}
%   \ednote{MK: fake it in notes mode for now}
%    \begin{macrocode}
\bool_if:NT \c_@@_notes_bool {
  \newcommand\pause{}
}
%    \end{macrocode}
% \end{macro}
% 
% \begin{environment}{nparagraph}
%    \begin{macrocode}
\bool_if:NTF \c_@@_notes_bool {
  \newenvironment{nparagraph}[1][]{\begin{sparagraph}[#1]}{\end{sparagraph}}
}{
  \excludecomment{nparagraph}
}
%    \end{macrocode}
% \end{environment}
%
% \begin{environment}{nfragment}
%    \begin{macrocode}
\bool_if:NTF \c_@@_notes_bool {
  \newenvironment{nfragment}[2][]{\begin{sfragment}[#1]{#2}}{\end{sfragment}}
}{
  \excludecomment{nfragment}
}
%    \end{macrocode}
% \end{environment}
%
% \begin{environment}{ndefinition}
%    \begin{macrocode}
\bool_if:NTF \c_@@_notes_bool {
  \newenvironment{ndefinition}[1][]{\begin{sdefinition}[#1]}{\end{sdefinition}}
}{
  \excludecomment{ndefinition}
}
%    \end{macrocode}
% \end{environment}
%
% \begin{environment}{nassertion}
%    \begin{macrocode}
\bool_if:NTF \c_@@_notes_bool {
  \newenvironment{nassertion}[1][]{\begin{sassertion}[#1]}{\end{sassertion}}
}{
  \excludecomment{nassertion}
}
%    \end{macrocode}
% \end{environment}
%
% \begin{environment}{nsproof}
%    \begin{macrocode}
\bool_if:NTF \c_@@_notes_bool {
  \newenvironment{nproof}[2][]{\begin{sproof}[#1]{#2}}{\end{sproof}}
}{
  \excludecomment{nproof}
}
%    \end{macrocode}
% \end{environment}
%
% \begin{environment}{nexample}
%    \begin{macrocode}
\bool_if:NTF \c_@@_notes_bool {
  \newenvironment{nexample}[1][]{\begin{sexample}[#1]}{\end{sexample}}
}{
  \excludecomment{nexample}
}
%    \end{macrocode}
% \end{environment}
%
% \begin{macro}{\inputref@*skip}
% We customize the hooks for in |\inputref|. 
%    \begin{macrocode}
\def\inputref@preskip{\smallskip}
\def\inputref@postskip{\medskip}
%    \end{macrocode}
% \end{macro}
%
% \begin{macro}{\inputref*}
%    \begin{macrocode}
\let\orig@inputref\inputref
\def\inputref{\@ifstar\ninputref\orig@inputref}
\newcommand\ninputref[2][]{
  \bool_if:NT \c_@@_notes_bool {
    \orig@inputref[#1]{#2}
  }
}
%    \end{macrocode}
% \end{macro}
% 
% \subsection{Header and Footer Lines}\label{sec:impl:headfootlines}
%
% Now, we set up the infrastructure for the footer line of the slides, we use boxes for
% the logos, so that they are only loaded once, that considerably speeds up processing.
% 
% \begin{macro}{\setslidelogo}
% The default logo is the {\sTeX} logo. Customization can be done by |\setslidelogo{|\meta{logo name}|}|.
%    \begin{macrocode}
\newlength{\slidelogoheight}

\bool_if:NTF \c_@@_notes_bool {
  \setlength{\slidelogoheight}{.4cm}
}{
  \setlength{\slidelogoheight}{1cm}
}
\newsavebox{\slidelogo}
\sbox{\slidelogo}{\sTeX}
\newrobustcmd{\setslidelogo}[1]{
  \sbox{\slidelogo}{\includegraphics[height=\slidelogoheight]{#1}}
}
%    \end{macrocode}
% \end{macro}
%
% \begin{macro}{\setsource}
% |\source| stores the writer's name. By default it is {\it Michael Kohlhase} since he is the main user and designer of this package. |\setsource{|\meta{name}|}| can change the writer's name.
%    \begin{macrocode}
\def\source{Michael Kohlhase}% customize locally
\newrobustcmd{\setsource}[1]{\def\source{#1}}
%    \end{macrocode}
% \end{macro}
%
% \begin{macro}{\setlicensing}
%   Now, we set up the copyright and licensing. By default we use the Creative Commons
%   Attribuition-ShareAlike license to strengthen the public domain. If package |hyperref|
%   is loaded, then we can attach a hyperlink to the license
%   logo. |\setlicensing[|\meta{url}|]{|\meta{logo name}|}| is used for customization,
%   where ||\meta{url}|| is optional.
%    \begin{macrocode}
\def\copyrightnotice{\footnotesize\copyright :\hspace{.3ex}{\source}}
\newsavebox{\cclogo}
\sbox{\cclogo}{\includegraphics[height=\slidelogoheight]{stex-cc_somerights}}
\newif\ifcchref\cchreffalse
\AtBeginDocument{
  \@ifpackageloaded{hyperref}{\cchreftrue}{\cchreffalse}
}
\def\licensing{
  \ifcchref
    \href{http://creativecommons.org/licenses/by-sa/2.5/}{\usebox{\cclogo}}
  \else
    {\usebox{\cclogo}}
  \fi
}
\newrobustcmd{\setlicensing}[2][]{
  \def\@url{#1}
  \sbox{\cclogo}{\includegraphics[height=\slidelogoheight]{#2}}
  \ifx\@url\@empty
    \def\licensing{{\usebox{\cclogo}}}
  \else
    \def\licensing{
      \ifcchref
      \href{#1}{\usebox{\cclogo}}
      \else
      {\usebox{\cclogo}}
      \fi
    }
  \fi
}
%    \end{macrocode}
% \end{macro} 
%
% \begin{macro}{\slidelabel}
% Now, we set up the slide label for the |article| mode.\ednote{see that we can use the themes for the slides some day. This is all fake.}
%    \begin{macrocode}
\newrobustcmd\miko@slidelabel{
  \vbox to \slidelogoheight{
    \vss\hbox to \slidewidth
    {\licensing\hfill\copyrightnotice\hfill\arabic{slide}\hfill\usebox{\slidelogo}}
  }
}
%    \end{macrocode}
% \end{macro}
% \subsection{Frame Images}\label{sec:impl:frameimage}
%
% \begin{macro}{\frameimage}
%   We have to make sure that the width is overwritten, for that we check the
%   |\Gin@ewidth| macro from the |graphicx| package. We also add the |label| key. 
%    \begin{macrocode}
\def\Gin@mhrepos{}
\define@key{Gin}{mhrepos}{\def\Gin@mhrepos{#1}}
\define@key{Gin}{label}{\def\@currentlabel{\arabic{slide}}\label{#1}}
\newrobustcmd\frameimage[2][]{
  \stepcounter{slide}
  \bool_if:NT \c_@@_frameimages_bool {
    \def\Gin@ewidth{}\setkeys{Gin}{#1}
    \bool_if:NF \c_@@_notes_bool { \vfill }
    \begin{center}
      \bool_if:NTF \c_@@_fiboxed_bool {
        \fbox{
          \ifx\Gin@ewidth\@empty
            \ifx\Gin@mhrepos\@empty
              \mhgraphics[width=\slidewidth,#1]{#2}
            \else
              \mhgraphics[width=\slidewidth,#1,mhrepos=\Gin@mhrepos]{#2}
            \fi
          \else% Gin@ewidth empty
            \ifx\Gin@mhrepos\@empty
              \mhgraphics[#1]{#2}
            \else
              \mhgraphics[#1,mhrepos=\Gin@mhrepos]{#2}
            \fi
          \fi% Gin@ewidth empty
        }
      }{
        \ifx\Gin@ewidth\@empty
          \ifx\Gin@mhrepos\@empty
            \mhgraphics[width=\slidewidth,#1]{#2}
          \else
            \mhgraphics[width=\slidewidth,#1,mhrepos=\Gin@mhrepos]{#2}
          \fi
          \ifx\Gin@mhrepos\@empty
            \mhgraphics[#1]{#2}
          \else
            \mhgraphics[#1,mhrepos=\Gin@mhrepos]{#2}
          \fi
        \fi% Gin@ewidth empty
      }
     \end{center}
    \par\strut\hfill{\footnotesize Slide \arabic{slide}}%
    \bool_if:NF \c_@@_notes_bool { \vfill }
  }
} % ifmks@sty@frameimages
%    \end{macrocode}
% \end{macro}
% 
% \subsection{Colors and Highlighting}\label{sec:impl:highlighting}
%
% We first specify sans serif fonts as the default. 
%
%    \begin{macrocode}
\sffamily
%    \end{macrocode}
%
% Now, we set up an infrastructure for highlighting phrases in slides. Note that we use
% content-oriented macros for highlighting rather than directly using color markup. 
% The first thing to to is to adapt the green so that it is dark enough for most beamers
%    \begin{macrocode}
\AddToHook{begindocument}{
  \definecolor{green}{rgb}{0,.5,0}
  \definecolor{purple}{cmyk}{.3,1,0,.17}
}
%    \end{macrocode}
%
% We customize the |\defemph|, |\symrefemph|, |\compemph|, and |\titleemph| macros with
% colors. Furthermore we customize the |\__omtextlec| macro for the appearance of line end
% comments in |\lec|.
%
%    \begin{macrocode}
% \def\STpresent#1{\textcolor{blue}{#1}}
\def\defemph#1{{\textcolor{magenta}{#1}}}
\def\symrefemph#1{{\textcolor{cyan}{#1}}}
\def\compemph#1{{\textcolor{blue}{#1}}}
\def\titleemph#1{{\textcolor{blue}{#1}}}
\def\__omtext_lec#1{(\textcolor{green}{#1})}
%    \end{macrocode}
%
% I like to use the dangerous bend symbol for warnings, so we provide it here.
% \begin{macro}{\textwarning}
%   as the macro can be used quite often we put it into a box register, so that it is only
%   loaded once. 
%    \begin{macrocode}
\pgfdeclareimage[width=.8em]{miko@small@dbend}{stex-dangerous-bend}
\def\smalltextwarning{
  \pgfuseimage{miko@small@dbend}
  \xspace
}
\pgfdeclareimage[width=1.2em]{miko@dbend}{stex-dangerous-bend}
\newrobustcmd\textwarning{
  \raisebox{-.05cm}{\pgfuseimage{miko@dbend}}
  \xspace
}
\pgfdeclareimage[width=2.5em]{miko@big@dbend}{stex-dangerous-bend}
\newrobustcmd\bigtextwarning{
  \raisebox{-.05cm}{\pgfuseimage{miko@big@dbend}}
  \xspace
}
%    \end{macrocode}
% \end{macro}
% 
%    \begin{macrocode}
\newrobustcmd\putgraphicsat[3]{
  \begin{picture}(0,0)\put(#1){\includegraphics[#2]{#3}}\end{picture}
}
\newrobustcmd\putat[2]{
  \begin{picture}(0,0)\put(#1){#2}\end{picture}
}
%    \end{macrocode}
%
% \subsection{Sectioning}
%
% If the |sectocframes| option is set, then we make section frames. We first define
% counters for |part| and |chapter|, which |beamer.cls| does not have and we make the
% |section| counter which it does dependent on |chapter|. 
%    \begin{macrocode}
\bool_if:NT \c_@@_sectocframes_bool {
  \str_if_eq:VnTF \@@topsect{part}{
    \newcounter{chapter}\counterwithin*{section}{chapter}
  }{
    \str_if_eq:VnT\@@topsect{chapter}{
      \newcounter{chapter}\counterwithin*{section}{chapter}
    }
  }
}
%    \end{macrocode}
%
% We set the \DescribeMacro{\section@level}|\section@level| counter that governs
% sectioning according to the class options. We also introduce the sectioning counters
% accordingly. 
%
% \begin{macro}{\section@level}
%    \begin{macrocode}
\def\part@prefix{}
\@ifpackageloaded{document-structure}{}{
  \str_case:VnF \@@topsect {
    {part}{
      \int_set:Nn \l_document_structure_section_level_int {0}
      \def\thesection{\arabic{chapter}.\arabic{section}}
      \def\part@prefix{\arabic{chapter}.}
    }
    {chapter}{
      \int_set:Nn \l_document_structure_section_level_int {1}
      \def\thesection{\arabic{chapter}.\arabic{section}}
      \def\part@prefix{\arabic{chapter}.}
    }
  }{
    \int_set:Nn \l_document_structure_section_level_int {2}
    \def\part@prefix{}
  }
}

\bool_if:NF \c_@@_notes_bool { % only in slides
%    \end{macrocode}
% \end{macro}
%
% The new counters are used in the |omgroup| environment that choses the {\LaTeX}
% sectioning macros according to |\section@level|. 
% 
% \begin{environment}{sfragment}
%    \begin{macrocode}
  \renewenvironment{sfragment}[2][]{
    \__document_structure_omgroup_args:n { #1 }
    \int_incr:N \l_document_structure_section_level_int
    \bool_if:NT \c_@@_sectocframes_bool {
      \stepcounter{slide}
      \begin{frame}[noframenumbering]
      \vfill\Large\centering
      \red{
        \ifcase\l_document_structure_section_level_int\or
          \stepcounter{part}
          \def\@@label{\omdoc@part@kw~\Roman{part}}
          \def\currentsectionlevel{\omdoc@part@kw}
        \or
          \stepcounter{chapter}
          \def\@@label{\omdoc@chapter@kw~\arabic{chapter}}
          \def\currentsectionlevel{\omdoc@chapter@kw}
        \or
          \stepcounter{section}
          \def\@@label{\part@prefix\arabic{section}}
          \def\currentsectionlevel{\omdoc@section@kw}
        \or
          \stepcounter{subsection}
          \def\@@label{\part@prefix\arabic{section}.\arabic{subsection}}
          \def\currentsectionlevel{\omdoc@subsection@kw}
        \or
          \stepcounter{subsubsection}
          \def\@@label{\part@prefix\arabic{section}.\arabic{subsection}.\arabic{subsubsection}}
          \def\currentsectionlevel{\omdoc@subsubsection@kw}
        \or
          \stepcounter{paragraph}
          \def\@@label{\part@prefix\arabic{section}.\arabic{subsection}.\arabic{subsubsection}.\arabic{paragraph}}
          \def\currentsectionlevel{\omdoc@paragraph@kw}
        \else
          \def\@@label{}
          \def\currentsectionlevel{\omdoc@paragraph@kw}
        \fi% end ifcase
        \@@label%\sref@label@id\@@label
        \quad #2%
      }%
      \vfill%
      \end{frame}%
    }
    \str_if_empty:NF \l__document_structure_omgroup_id_str {
      \stex_ref_new_doc_target:n\l__document_structure_omgroup_id_str
    }
  }{}
}
%    \end{macrocode}
% \end{environment}
%
% We set up a |beamer| template for theorems like ams style, but without a block
% environment.  
%    \begin{macrocode}
\def\inserttheorembodyfont{\normalfont}
%\bool_if:NF \c_@@_notes_bool {
%  \defbeamertemplate{theorem begin}{miko}
%  {\inserttheoremheadfont\inserttheoremname\inserttheoremnumber
%    \ifx\inserttheoremaddition\@empty\else\ (\inserttheoremaddition)\fi%
%    \inserttheorempunctuation\inserttheorembodyfont\xspace}
%  \defbeamertemplate{theorem end}{miko}{}
%    \end{macrocode}
% and we set it as the default one. 
%    \begin{macrocode}
%  \setbeamertemplate{theorems}[miko]
%    \end{macrocode}
% The following fixes an error I do not understand, this has something to do with
% beamer compatibility, which has similar definitions but only up to 1. 
%    \begin{macrocode}
%  \expandafter\def\csname Parent2\endcsname{}
%}

\AddToHook{begindocument}{ % this does not work for some reasone
  \setbeamertemplate{theorems}[ams style]
}
\bool_if:NT \c_@@_notes_bool {
  \renewenvironment{columns}[1][]{%
    \par\noindent%
    \begin{minipage}%
    \slidewidth\centering\leavevmode%
  }{%
    \end{minipage}\par\noindent%
  }%
  \newsavebox\columnbox%
  \renewenvironment<>{column}[2][]{%
    \begin{lrbox}{\columnbox}\begin{minipage}{#2}%
  }{%
    \end{minipage}\end{lrbox}\usebox\columnbox%
  }%
}
%    \end{macrocode}
%
%    \begin{macrocode}
\bool_if:NTF \c_@@_noproblems_bool {
  \newenvironment{problems}{}{}
}{
  \excludecomment{problems}
}
%    \end{macrocode}
%
% \subsection{Excursions}\label{sec:impl:excursions}
%
% \begin{macro}{\excursion}
%  The excursion macros are very simple, we define a new internal macro |\excursionref| and
%  use it in |\excursion|, which is just an |\inputref| that checks if the new macro is
%  defined before formatting the file in the argument. 
%    \begin{macrocode}
\gdef\printexcursions{}
\newcommand\excursionref[2]{% label, text
  \bool_if:NT \c_@@_notes_bool {
    \begin{sparagraph}[title=Excursion]
      #2 \sref[fallback=the appendix]{#1}.
    \end{sparagraph}
  }
}
\newcommand\activate@excursion[2][]{
  \gappto\printexcursions{\inputref[#1]{#2}}
}
\newcommand\excursion[4][]{% repos, label, path, text
  \bool_if:NT \c_@@_notes_bool {
    \activate@excursion[#1]{#3}\excursionref{#2}{#4}
  }
}
%    \end{macrocode}
% \end{macro}
%
% \begin{macro}{\excursiongroup}
%    \begin{macrocode}
\keys_define:nn{notesslides / excursiongroup }{
  id        .str_set_x:N  = \l_@@_excursion_id_str,
  intro     .tl_set:N     = \l_@@_excursion_intro_tl,
  mhrepos   .str_set_x:N  = \l_@@_excursion_mhrepos_str
}
\cs_new_protected:Nn \_@@_excursion_args:n {
  \tl_clear:N \l_@@_excursion_intro_tl
  \str_clear:N \l_@@_excursion_id_str
  \str_clear:N \l_@@_excursion_mhrepos_str
  \keys_set:nn {notesslides / excursiongroup }{ #1 }
}
\newcommand\excursiongroup[1][]{
  \_@@_excursion_args:n{ #1 }
  \ifdefempty\printexcursions{}% only if there are excursions
  {\begin{note}
    \begin{sfragment}[#1]{Excursions}%
      \ifdefempty\l_@@_excursion_intro_tl{}{
        \inputref[\l_@@_excursion_mhrepos_str]{
          \l_@@_excursion_intro_tl
        }
      }
      \printexcursions%
    \end{sfragment}
  \end{note}}
}
\ifcsname beameritemnestingprefix\endcsname\else\def\beameritemnestingprefix{}\fi
%</package>
%    \end{macrocode}
% \end{macro}
%
% \end{implementation}
\endinput
% \endinput
% Local Variables:
% mode: doctex
% TeX-master: t
% End:

%  LocalWords:  bibfolder jobname.dtx omdoc.dtx mytwocolumns beamer.cls sectocframes rhd
%  LocalWords:  frameimage twintoo atwin atwintoo scsys scshape ttfamily cnxlatex xspace
%  LocalWords:  mathml omdoc beamerclass:on Kohlhase:smomdl noproblems frameimages exfig
%  LocalWords:  frameimages fiboxed fiboxed topsect topsect Tantau:ugbc ifnotes ifnotes
%  LocalWords:  noframenumbering allowframebreaks newpage inputref inputref inputref sref
%  LocalWords:  setsource setsource setlicensing setlicensing CarRah:tpp99 mhframeimage
%  LocalWords:  mhframeimage printexcursions excursionref excursionref excursiongroup
%  LocalWords:  excursiongroup sec:impl nontheorem mks@cls,prefix mks@cls mks@sty,prefix
%  LocalWords:  mks@sty 10pt,notheorems,xcolor dvipsnames,svgnames Hfootnote mdframed
%  LocalWords:  beamerarticle noxcolor,noamsthm metakeys addmetakey usetheme beamertheme
%  LocalWords:  bookmarks,bookmarksopen,bookmarksnumbered,breaklinks,hidelinks sproof
%  LocalWords:  beamernotestheme slideframewidth allowdisplaybreaks framenumber sproof
%  LocalWords:  miko@slidelabel inputref@preskip smallskip ninputref slidelogoheight
%  LocalWords:  copyrightnotice cclogo cc_somerights ifcchref cchreffalse cchreftrue
%  LocalWords:  Gin@ewidth mh@currentrepos mhrepos if@iswindows defemph notemph STpresent
%  LocalWords:  smalltextwarning miko@dbend bigtextwarning counterwithin ifsectocframes
%  LocalWords:  subsubsection mparagraph csname stex_ref_new_doc_target:n columnbox
%  LocalWords:  inserttheorembodyfont defbeamertemplate inserttheoremheadfont mhexcursion
%  LocalWords:  inserttheoremname inserttheoremnumber inserttheoremaddition srefaddidkey
%  LocalWords:  inserttheoremaddition inserttheorempunctuation setbeamertemplate
%  LocalWords:  mhexcursiongroup
