% \iffalse meta-comment
% A LaTeX Class and Package for OMDoc Document Structures
% Copyright (c) 2019 Michael Kohlhase, all rights reserved
%               this file is released under the
%               LaTeX Project Public License (LPPL)
%
% The original of this file is in the public repository at 
% http://github.com/sLaTeX/sTeX/
%
%
%<*driver>
\def\bibfolder#1{../../lib/bib/#1}
\RequirePackage{paralist}
\ifcsname stexdocpath\endcsname\else\def\stexdocpath{.}\fi
\documentclass[full]{l3doc}
%\RequirePackage{document-structure}
\usepackage[hyperref=auto,style=alphabetic]{biblatex}
%\usepackage[mathhub=\stexdocpath/mh,usedeps]{stex}
\usepackage[lang={en,de}]{stex}

\usepackage{rustex}
\usepackage{stex-highlighting,stexthm}

\srefsetin[sTeX/Documentation]{documentation}{the \stex Documentation}

\makeatletter
\providecommand{\HTML}{\textsc{html}\xspace}%
\providecommand{\XML}{\textsc{xml}\xspace}%
\providecommand{\PDF}{\textsc{pdf}\xspace}%
\providecommand\openmath{\textsc{OpenMath}\xspace}
\providecommand\OMDoc{\textsc{OMDoc}\xspace}
\DeclareRobustCommand\LaTeXML{L\kern-.36em%
        {\sbox\z@ T%
         \vbox to\ht\z@{\hbox{\check@mathfonts
                              \fontsize\sf@size\z@
                              \math@fontsfalse\selectfont
                              A}%
                        \vss}%
        }%
        \kern-.15em%
%        T\kern-.1667em\lower.5ex\hbox{E}\kern-.125em\relax
%        {\tt XML}}
        T\kern-.1667em\lower.4ex\hbox{E}\kern-0.05em\relax
        {\scshape xml}\xspace}%
\def\mmt{\textsc{Mmt}\xspace}
\makeatother


\newif\ifhadtitle\hadtitlefalse

\def\stexversion{3.3.0}
\def\changedate{\today}
\def\stextoptitle#1#2{\title{#1\thanks{Version {\stexversion} (last revised {\changedate})} }\def\thispkg{#2}}

\author{Michael Kohlhase, Dennis Müller\\
 	FAU Erlangen-Nürnberg\\
 	\url{http://kwarc.info/}
}

\def\stexmaketitle{\ifhadtitle\else\hadtitletrue\maketitle\fi}

\ExplSyntaxOn

  \def\docmodule{
    \begin{document}
      \EnableManual
      \EnableDocumentation
      \EnableImplementation
      \stexmaketitle
      \tableofcontents
      \int_gincr:N \l_stex_docheader_sect
      \exp_args:Ne \__stex_mathhub_find_manifest:n {\stex_file_use:N \c_stex_mathhub_file / sTeX / Documentation}
      \str_if_empty:NF \l__stex_mathhub_manifest_str {
        \usemodule[sTeX/Documentation]{macros?AllMacros}
      }
      \DocInput{\jobname.dtx}
      \clearpage
      \PrintIndex
      \printbibliography
    \end{document}
  }

  \bool_new:N \g_stexdoc_typeset_manual_bool
  \NewDocumentCommand \EnableManual {}{
    \bool_gset_true:N \g_stexdoc_typeset_manual_bool
  }
  \NewDocumentCommand \DisableManual {}{
    \bool_gset_false:N \g_stexdoc_typeset_manual_bool
  }
  \NewDocumentEnvironment {stexmanual} {} {
    \bool_if:NTF \g_stexdoc_typeset_manual_bool
      {\bool_set_false:N \l__codedoc_in_implementation_bool}
      {\comment}
  }{
    \bool_if:NF \g_stexdoc_typeset_manual_bool {\endcomment}
  }
\ExplSyntaxOff

%\usepackage{makeidx}
%\makeindex

%\usepackage{document-structure}


\usepackage{lststex,mdframed}
\usepackage[most]{tcolorbox}

\lstset{literate=%
    {Ö}{{\"O}}1
    {Ä}{{\"A}}1
    {Ü}{{\"U}}1
    {ß}{{\ss}}1
    {ü}{{\"u}}1
    {ä}{{\"a}}1
    {ö}{{\"o}}1
    {~}{{\textasciitilde}}1
}

\newenvironment{framed}[1][]{
  \ifstexhtml\par\vbox\bgroup
    \csname exp_args:Nne\endcsname\begin{stex_annotate_env}{%
      style:border=solid 1px black,%
      style:width=var(--this-width),%
      style:min-width=var(--this-width),%
      style:--this-width=calc(var(--current-width) - 6px),%
      style:padding=3px,%
      style:margin-top=5px,%
      style:margin-bottom=5px%
    }
    \csname stex_annotate_invisible:n\endcsname{ }%
    \begin{stex_annotate_env}{%
      style:--current-width=var(--this-width);%
    }\csname stex_annotate_invisible:n\endcsname{ }
  \else\begin{mdframed}[#1]\fi
  %\begin{center}%
}{%
  %\end{center}%
  \ifstexhtml
    \end{stex_annotate_env}\end{stex_annotate_env}\egroup\par
  \else\end{mdframed}\fi
}
\newcommand{\scaled}[2][0.9\hsize]{\begin{center}\resizebox{#1}{!}{\begin{minipage}{\textwidth} #2 \end{minipage}}\end{center}}

\makeatletter
\ExplSyntaxOn

\def\doc_exbox:nnn#1#2#3{
  \begin{sexample}[#3]
    Input:
    \begin{framed}[linewidth=1pt,backgroundcolor=white]\small
      #1
    \end{framed}
    Output:
    \begin{framed}[linewidth=1pt,backgroundcolor=white]\small
      #2
    \end{framed}
  \end{sexample}
}


\NewDocumentCommand\stexexamplefile{O{} m O{} O{}}{
  \stex_resolve_path_pair:Nxx \l_@@_filepath_str {\tl_to_str:n{#1}} {\tl_to_str:n{#2}}
  \doc_exbox:nnn{
    \hfill File~\str_if_empty:nTF{#1}{
      \prop_if_exist:NT \l_stex_current_archive_prop {
        [\texttt{\prop_item:Nn \l_stex_current_archive_prop {id}}]
      }
    }{[#1]}\texttt{\tl_to_str:n{#2}}
    \_lststex_parse_args:n{#3}
    \exp_args:Nno \use:nn{\lstinputlisting[} \l_lststex_return_tl ]{\l_@@_filepath_str}
  }{
    \inputref[#1]{#2}
  }{#4}
}

\newwrite\testoutfile
\NewDocumentCommand\stexexample{O{} O{}}{
  \begingroup 
  \catcode`\\=12\relax
  \catcode`\#=12\relax
  \catcode`\&=12\relax
  \catcode`\$=12\relax
  \catcode`\^=12\relax
  \catcode`\_=12\relax
  \catcode`\ =12\relax
  \catcode`^^J=12\relax
  \endlinechar=`^^J
  \newlinechar=-1
%^^A    \everyeof{\noexpand}
  \example_a:nnn{#1}{#2}
}
\long\def\example_a:nnn #1 #2 #3 {
  \endgroup
  \immediate\openout\testoutfile=\jobname.exmpl
  \immediate\write\testoutfile{
    \c_backslash_str begin{stexcode}[#1]
    \detokenize{^^J}#3
    \c_backslash_str end{stexcode}
  }
  \immediate\closeout\testoutfile
  \doc_exbox:nnn{
    \catcode`\#=12\relax
    \csname @ @ input\endcsname{\jobname.exmpl}
  }{
    \immediate\openout\testoutfile=\jobname.exmpl
    \immediate\write\testoutfile{#3}
    \immediate\closeout\testoutfile
    \csname @ @ input\endcsname \jobname.exmpl\relax
  }{#2}
  \peek_charcode_remove:NT ^^J
}

\ExplSyntaxOff
\makeatother

\makeatletter
\newcount\example@counter\example@counter=0
\newtcolorbox{exampleborderbox}[1][]{
  empty,
  title={Example \the\example@counter #1},
  attach boxed title to top left,
     minipage boxed title,
  boxed title style={empty,size=minimal,toprule=0pt,top=1pt,left=3mm,overlay={}},
  coltitle=blue,fonttitle=\bfseries,
  parbox=false,boxsep=0pt,left=3mm,right=0mm,top=2pt,breakable,pad at break=0mm,
     before upper=\csname @totalleftmargin\endcsname0pt, 
  overlay unbroken={\draw[blue,line width=2pt] ([xshift=-0pt]title.north west) -- ([xshift=-0pt]frame.south west); },
  overlay first={\draw[blue,line width=2pt] ([xshift=-0pt]title.north west) -- ([xshift=-0pt]frame.south west); },
  overlay middle={\draw[blue,line width=2pt] ([xshift=-0pt]frame.north west) -- ([xshift=-0pt]frame.south west); },
  overlay last={\draw[blue,line width=2pt] ([xshift=-0pt]frame.north west) -- ([xshift=-0pt]frame.south west); },
  outer arc=4pt%
}

\ExplSyntaxOn
\stexstyleexample{
  \global\advance\example@counter by 1
  \tl_if_empty:NTF\thistitle{
    \begin{exampleborderbox}
  }{
    \begin{exampleborderbox}[ (\thistitle)]
  }
}{
    \end{exampleborderbox}
}

\ExplSyntaxOff\makeatother

\usetikzlibrary{calc}

\def\textwarning{\includegraphics[width=1.2em]{stex-dangerous-bend}\xspace}
\newtcolorbox{dangerbox}{
  breakable,
  enhanced,
  left=0pt,
  right=0pt,
  top=8pt,
  bottom=8pt,
  colback=white,
  colframe=red,
  width=\textwidth,
  enlarge left by=0mm,
  boxsep=5pt,
  fontupper=\small,
  arc=4pt,
  outer arc=4pt,
  leftupper=1.5cm,
  overlay={
    \node[anchor=west] at ([xshift=10pt]$(frame.north west)!0.5!(frame.south west)$)
       {\includegraphics[width=1cm,height=1cm]{stex-dangerous-bend}};}
}

\protected\def\TODO#1{\textcolor{red}{TODO}\footnote{\textcolor{red}{TODO: #1}}}

\definecolor{darkgreen}{rgb}{0.0, 0.5, 0.0}

\usepackage[solutions]{problem}
\usepackage{hwexam}
\newtcolorbox{problemborderbox}[1][]{
  empty,
  title={Exercise #1},
  attach boxed title to top left,
     minipage boxed title,
  boxed title style={empty,size=minimal,toprule=0pt,top=1pt,left=3mm,overlay={}},
  coltitle=darkgreen,fonttitle=\bfseries,
  parbox=false,boxsep=0pt,left=3mm,right=0mm,top=2pt,breakable,pad at break=0mm,
     before upper=\csname @totalleftmargin\endcsname0pt, 
  overlay unbroken={\draw[darkgreen,line width=2pt] ([xshift=-0pt]title.north west) -- ([xshift=-0pt]frame.south west); },
  overlay first={\draw[darkgreen,line width=2pt] ([xshift=-0pt]title.north west) -- ([xshift=-0pt]frame.south west); },
  overlay middle={\draw[darkgreen,line width=2pt] ([xshift=-0pt]frame.north west) -- ([xshift=-0pt]frame.south west); },
  overlay last={\draw[darkgreen,line width=2pt] ([xshift=-0pt]frame.north west) -- ([xshift=-0pt]frame.south west); },
  outer arc=4pt%
}

\ExplSyntaxOn
\stexstyleproblem{
  \tl_if_empty:NTF\thistitle{
    \begin{problemborderbox}
  }{
    \begin{problemborderbox}[ (\thistitle)]
  }
}{
    \end{problemborderbox}
}
\ExplSyntaxOff

\newtcolorbox{experimental}{
  breakable,
  enhanced,
  left=0pt,
  right=0pt,
  top=8pt,
  bottom=8pt,
  colback=white,
  colframe=gray,
  width=\textwidth,
  enlarge left by=0mm,
  boxsep=5pt,
  fontupper=\small,
  arc=4pt,
  outer arc=4pt,
  leftupper=1.5cm,
  overlay={
    \node[anchor=west] at ([xshift=10pt]$(frame.north west)!0.5!(frame.south west)$)
       {\includegraphics[height=1cm]{stex-experimental}};}
}


\usetikzlibrary{decorations.pathmorphing,shapes,arrows,calc}
% Taken from pgflibrarytikzmmt.code.tex
\newcommand{\mmtarrowtip}{angle 45}
\newcommand{\mmtarrowtipmonoright}{right hook}

\tikzstyle{include}=[\mmtarrowtipmonoright-\mmtarrowtip,thick]
\tikzstyle{morph}=[-\mmtarrowtip,thick]
\tikzstyle{preview}=[decorate, decoration={coil,aspect=0,amplitude=1pt,
                                                  segment length=6pt,
                                                  pre=lineto,pre length=3pt,
                                                  post=lineto,post length=5pt}, thick]
\tikzstyle{view}=[preview,-\mmtarrowtip]


% TIKZ RULES
\def\mmtlogo{
\begin{tikzpicture}

  % White Background (Margins are eyeballed)
  % This is necessary because we paste white over arrows later.
  % If somebody want's to do the full song and dance with
  % interrupted arrows to get transparent background, be my guest.

  \fill[white!] (-0.01,0.15) rectangle (1.11,-0.95);

  % Arrows
  \draw [blue, include] (0,0)     -- (1.1,0);
  \draw [green, morph] (0,-0.4)  -- (1.1,-0.4);
  \draw [red, view]   (-0,-0.8) -- (1.1,-0.8);

  % Cutout for letters
  \fill[white] (0.33,0.1) rectangle (0.66,-0.9);

  % Letters
  \node at (0.18,0)    (nodeM1) {\large M};
  \node at (0.18,-0.4) (nodeM2) {\large M};
  \node at (0.21,-0.8) (nodeT)  {\large T};

\end{tikzpicture}
}

\newtcolorbox{mmtbox}{
  breakable,
  enhanced,
  left=0pt,
  right=0pt,
  top=8pt,
  bottom=8pt,
  colback=white,
  colframe=green,
  width=\textwidth,
  enlarge left by=0mm,
  boxsep=5pt,
  fontupper=\small,
  arc=4pt,
  outer arc=4pt,
  leftupper=1.5cm,
  overlay={
    \node[anchor=west] at ([xshift=10pt]$(frame.north west)!0.5!(frame.south west)$)
       {\mmtlogo};}
}

\AtBeginDocument{\catcode`_=8}

\begin{document}
  \DocInput{\jobname.dtx}
\end{document}
%</driver>
% \fi
%
% \title{document-structure: Semantic Markup for Open Mathematical Documents in {\LaTeX}
% 	\thanks{Version {\fileversion} (last revised {\filedate})} 
% }
%
% \author{Michael Kohlhase, Dennis Müller\\
% 	FAU Erlangen-Nürnberg\\
% 	\url{http://kwarc.info/}
% }
%
% \maketitle
%
% \ifinfulldoc\else
% \begin{abstract}
%   This is the documentation for the \pkg{document-structure} package from \sTeX
%   collection, a version of {\TeX/\LaTeX} that allows to markup {\TeX/\LaTeX} documents
%   semantically without leaving the document format, essentially turning {\TeX/\LaTeX}
%   into a document format for mathematical knowledge management (MKM).
%
%   For a more high-level introduction, see \href{\basedocurl/manual.pdf}{the \sTeX
%   Manual} or the \href{\basedocurl/stex.pdf}{full \sTeX documentation}.
% \end{abstract}
%
% \tableofcontents
% 
% This package supplies an infrastructure for writing {\omdoc} documents in {\LaTeX}.
This includes a simple structure sharing mechanism for \sTeX that allows to to move from
a copy-and-paste document development model to a copy-and-reference model, which
conserves space and simplifies document management. The augmented structure can be used
by MKM systems for added-value services, either directly from the \sTeX sources, or
after translation.

\begin{sfragment}[id=sec:STR]{Introduction}

 The |document-structure| package supplies macros and environments that allow to label document
 fragments and to reference them later in the same document or in other documents. In
 essence, this enhances the document-as-trees model to
 documents-as-directed-acyclic-graphs (DAG) model. This structure can be used by MKM
 systems for added-value services, either directly from the \sTeX sources, or after
 translation. Currently, trans-document referencing provided by this package can only be
 used in the \sTeX collection.

 DAG models of documents allow to replace the ``Copy and Paste'' in the source document
 with a label-and-reference model where document are shared in the document source and the
 formatter does the copying during document formatting/presentation.
\end{sfragment}

\begin{sfragment}[id=sec:user]{The User Interface}
  The \pkg{document-structure} package accepts the following options:
  \begin{center}
    \begin{tabular}{|l|p{10cm}|}\hline
      \texttt{class=\meta{name}} & load \meta{name}|.cls| instead of |article.cls|\\\hline 
      \texttt{topsect=\meta{sect}} & The top-level sectioning level; the default for
      \meta{sect} is \texttt{section}\\\hline 
    \end{tabular}
  \end{center}

\begin{sfragment}[id=sec:user:struct]{Document Structure}

  \begin{environment}{sfragment}
    The structure of the document is given by the |sfragment| environment just like in
    {\omdoc}. In the {\LaTeX} route, the |sfragment| environment is flexibly mapped to
    sectioning commands, inducing the proper sectioning level from the nesting of
    |sfragment| environments. Correspondingly, the |sfragment| environment takes an optional
    key/value argument for metadata followed by a regular argument for the (section) title
    of the sfragment. The optional metadata argument has the keys |id| for an identifier,
    |creators| and |contributors| for the Dublin Core metadata~\cite{DCMI:dmt03}. The
    option |short| allows to give a short title for the generated section. If the title
    contains semantic macros, they need to be protected by |\protect|\ednote{MK: still?},
    and we need to give the |loadmodules| key it needs no value. For instance we would
    have
\begin{latexcode}
\begin{smodule}{foo}
  \symdef{bar}{B^a_r}
   ...
   \begin{sfragment}[id=sec.barderiv,loadmodules]
     {Introducing $\protect\bar$ Derivations}
\end{latexcode}

\sTeX automatically computes the sectioning level, from the nesting of |sfragment|
environments.
\end{environment}

But sometimes, we want to skip levels (e.g. to use a subsection* as an introduction for a
chapter).

\begin{environment}{blindfragment}
  Therefore the |document-structure| package provides a variant |blindfragment| that does
  not produce markup, but increments the sectioning level and logically groups document
  parts that belong together, but where traditional document markup relies on convention
  rather than explicit markup. The |blindfragment| environment is useful e.g. for creating
  frontmatter at the correct level. The example below shows a typical setup for the outer
  document structure of a book with parts and chapters.
  
\begin{latexcode}
\begin{document}
\begin{blindfragment}
\begin{blindfragment}
\begin{frontmatter}
\maketitle\newpage
\begin{sfragment}{Preface}
... <<preface>> ...
\end{sfragment}
\clearpage\setcounter{tocdepth}{4}\tableofcontents\clearpage
\end{frontmatter}
\end{blindfragment}
... <<introductory remarks>> ...
\end{blindfragment}
\begin{sfragment}{Introduction}
... <<intro>> ...
\end{sfragment}
... <<more chapters>> ... 
\bibliographystyle{alpha}\bibliography{kwarc}
\end{document}
\end{latexcode}

Here we use two levels of |blindfragment|:
\begin{itemize}
\item The outer one groups the introductory parts of the book (which we assume to have a
  sectioning hierarchy topping at the part level). This |blindfragment| makes sure that
  the introductory remarks become a ``chapter'' instead of a ``part''.
\item Th inner one groups the frontmatter\footnote{We shied away from redefining the
    |frontmatter| to induce a blindfragment, but this may be the ``right'' way to go in
    the future.} and makes the preface of the book a section-level construct.\ednote{MK:
    We need a substitute for the ``Note that here the |display=flow| on the |sfragment|
    environment prevents numbering as is traditional for prefaces.''}
\end{itemize}
\end{environment}

\begin{function}{\skipfragment}
  The |\skipfragment| ``skips an |sfragment|'', i.e. it just steps the respective sectioning
  counter. This macro is useful, when we want to keep two documents in sync structurally,
  so that section numbers match up: Any section that is left out in one becomes a
  |\skipfragment|.
\end{function}

\begin{function}{\currentsectionlevel,\CurrentSectionLevel}
  The |\currentsectionlevel| macro supplies the name of the current sectioning level,
  e.g. ``chapter'', or ``subsection''. |\CurrentSectionLevel| is the capitalized
  variant. They are useful to write something like ``In this |\currentsectionlevel|, we
  will\ldots'' in an |sfragment| environment, where we do not know which sectioning level we
  will end up.
\end{function}
\end{sfragment}

\begin{sfragment}[id=sec:user:ignore]{Ignoring Inputs}

\begin{function}{\prematurestop,\afterprematurestop}
  For prematurely stopping the formatting of a document, \sTeX provides the
  |\prematurestop| macro. It can be used everywhere in a document and ignores all input
  after that -- backing out of the sfragment environment as needed. After that -- and
  before the implicit |\end{document}| it calls the internal |\afterprematurestop|, which
  can be customized to do additional cleanup or e.g. print the bibliography.

  |\prematurestop| is useful when one has a driver file, e.g. for a course taught multiple
  years and wants to generate course notes up to the current point in the lecture. Instead
  of commenting out the remaining parts, one can just move the |\prematurestop| macro.
  This is especially useful, if we need the rest of the file for processing, e.g. to
  generate a theory graph of the whole course with the already-covered parts marked up as
  an overview over the progress; see |import_graph.py| from the |lmhtools|
  utilities~\cite{lmhtools:github:on}.
\end{function}
\end{sfragment}

\begin{sfragment}[id=sec:user:gvars]{Global Variables}

  Text fragments and modules can be made more re-usable by the use of global
  variables. For instance, the admin section of a course can be made course-independent
  (and therefore re-usable) by using variables (actually token registers)
  |courseAcronym| and |courseTitle| instead of the text itself. The variables can then
  be set in the \sTeX preamble of the course notes file.
  
  \begin{function}{\setSGvar,\useSGvar}
    |\setSGvar{|\meta{vname}|}{|\meta{text}|}| to set the global variable \meta{vname} to
    \meta{text} and |\useSGvar{|\meta{vname}|}| to reference it.
  \end{function}
  
  \begin{function}{\ifSGvar}
    With|\ifSGvar| we can test for the contents of a global variable: the macro call
    |\ifSGvar{|\meta{vname}|}{|\meta{val}|}{|\meta{ctext}|}| tests the content of the
    global variable \meta{vname}, only if (after expansion) it is equal to \meta{val}, the
    conditional text \meta{ctext} is formatted.
  \end{function}
\end{sfragment}
\end{sfragment}

%%% Local Variables:
%%% mode: latex
%%% TeX-master: "../stex-manual"
%%% End:

%  LocalWords:  article.cls topsect DCMI:dmt03 loadmodules lmhtools
%  LocalWords:  prematurestop afterprematurestop import_graph.py STRlabel STRcopy vname
%  LocalWords:  STRsemantics setSGvar ifSGvar ctext

% \fi
%
%
% \begin{documentation}\label{pkg:documentstructure:doc}
%
% \end{documentation}
%
% \begin{implementation}\label{pkg:documentstructure:impl}
%
% \begin{sfragment}{document-structure.sty Implementation}
%
%    \begin{macrocode}
%<*package>
%<@@=document_structure>
\ProvidesExplPackage{document-structure}{2022/02/26}{3.0.1}{Modular Document Structure}
\RequirePackage{l3keys2e}
%    \end{macrocode}
%
% \begin{sfragment}[id=sec:impl:options]{Package Options}
%
%   We declare some switches which will modify the behavior according to the package
%   options. Generally, an option |xxx| will just set the appropriate switches to true
%   (otherwise they stay false).
%
%
%    \begin{macrocode}

\keys_define:nn{ document-structure }{
  class       .str_set_x:N  = \c_document_structure_class_str,
  topsect     .str_set_x:N  = \c_document_structure_topsect_str,,
  unknown     .code:n       = {
    \PassOptionsToClass{\CurrentOption}{stex}
    \PassOptionsToClass{\CurrentOption}{tikzinput}
  }
%  showignores .bool_set:N   = \c_document_structure_showignores_bool,
}
\ProcessKeysOptions{ document-structure }
\str_if_empty:NT \c_document_structure_class_str {
  \str_set:Nn \c_document_structure_class_str {article}
}
\str_if_empty:NT \c_document_structure_topsect_str {
  \str_set:Nn \c_document_structure_topsect_str {section}
}
%    \end{macrocode}
%
% Then we need to set up the packages by requiring the |sref| package to be loaded,
% and set up triggers for other languages
%
%    \begin{macrocode}
\RequirePackage{xspace}
\RequirePackage{comment}
\RequirePackage{stex}
\AddToHook{begindocument}{
	\ltx@ifpackageloaded{babel}{
    \clist_set:Nx \l_tmpa_clist {\bbl@loaded}
    \clist_if_in:NnT \l_tmpa_clist {ngerman}{
      \makeatletter\input{document-structure-ngerman.ldf}\makeatother
    }
  }{}
}
%    \end{macrocode}
%
% Finally, we set the \DescribeMacro{\section@level}|\section@level| macro that governs
% sectioning. The default is two (corresponding to the |article| class), then we set the
% defaults for the standard classes |book| and |report| and then we take care of the
% levels passed in via the |topsect| option.
%
%    \begin{macrocode}
\int_new:N \l_document_structure_section_level_int
\str_case:VnF \c_document_structure_topsect_str {
  {part}{
    \int_set:Nn \l_document_structure_section_level_int {0}
  }
  {chapter}{
    \int_set:Nn \l_document_structure_section_level_int {1}
  }
}{
  \str_case:VnF \c_document_structure_class_str {
    {book}{
      \int_set:Nn \l_document_structure_section_level_int {0}
    }
    {report}{
      \int_set:Nn \l_document_structure_section_level_int {0}
    }
  }{
    \int_set:Nn \l_document_structure_section_level_int {2}
  }
}
%    \end{macrocode}
%
% \end{sfragment}
% 
% \begin{sfragment}[id=sec:impl:struct]{Document Structure}
% 
%   The structure of the document is given by the |sfragment| environment. The hierarchy
%   is adjusted automatically according to the {\LaTeX} class in effect.
% \begin{macro}{\currentsectionlevel}
%   For the |\currentsectionlevel| and |\Currentsectionlevel| macros we use an internal
%   macro |\current@section@level| that only contains the keyword (no markup). We
%   initialize it with ``document'' as a default. In the generated OMDoc, we only generate
%   a text element of class |omdoc_currentsectionlevel|, wich will be instantiated by CSS
%   later.\ednote{MK: we may have to experiment with the more powerful uppercasing macro
%   from \texttt{mfirstuc.sty} once we internationalize.}
%    \begin{macrocode}
\def\current@section@level{document}%
\newcommand\currentsectionlevel{\lowercase\expandafter{\current@section@level}\xspace}%
\newcommand\Currentsectionlevel{\expandafter\MakeUppercase\current@section@level\xspace}%
%    \end{macrocode}
% \end{macro}
%
% \begin{macro}{\skipfragment}
%    \begin{macrocode}
\cs_new_protected:Npn \skipfragment {
  \ifcase\l_document_structure_section_level_int
  \or\stepcounter{part}
  \or\stepcounter{chapter}
  \or\stepcounter{section}
  \or\stepcounter{subsection}
  \or\stepcounter{subsubsection}
  \or\stepcounter{paragraph}
  \or\stepcounter{subparagraph}
  \fi
}
%    \end{macrocode}
% \end{macro}
%
% \begin{environment}{blindfragment}
%    \begin{macrocode}
\newcommand\at@begin@blindsfragment[1]{}
\newenvironment{blindfragment}
{
  \int_incr:N\l_document_structure_section_level_int
  \at@begin@blindsfragment\l_document_structure_section_level_int
}{}
%    \end{macrocode}
% \end{environment}
%
% \begin{macro}{\sfragment@nonum}
%   convenience macro: |\sfragment@nonum{|\meta{level}|}{|\meta{title}|}| makes an unnumbered
%   sectioning with title \meta{title} at level \meta{level}.
%    \begin{macrocode}
\newcommand\sfragment@nonum[2]{
  \ifx\hyper@anchor\@undefined\else\phantomsection\fi
  \addcontentsline{toc}{#1}{#2}\@nameuse{#1}*{#2}
}
%    \end{macrocode}
% \end{macro}
%
% \begin{macro}{\sfragment@num}
%   convenience macro: |\sfragment@nonum{|\meta{level}|}{|\meta{title}|}| makes numbered
%   sectioning with title \meta{title} at level \meta{level}. We have to check the |short|
%   key was given in the |sfragment| environment and -- if it is use it. But how to do that
%   depends on whether the |rdfmeta| package has been loaded. In the end we call
%   |\sref@label@id| to enable crossreferencing.
%    \begin{macrocode}
\newcommand\sfragment@num[2]{
  \tl_if_empty:NTF \l_@@_sfragment_short_tl {
    \@nameuse{#1}{#2}
  }{
    \cs_if_exist:NTF\rdfmeta@sectioning{
      \@nameuse{rdfmeta@#1@old}[\l_@@_sfragment_short_tl]{#2}
    }{
      \@nameuse{#1}[\l_@@_sfragment_short_tl]{#2}
    }
  }
%\sref@label@id@arg{\omdoc@sect@name~\@nameuse{the#1}}\sfragment@id
}
%    \end{macrocode}
% \end{macro}
%
% \begin{environment}{sfragment}
%    \begin{macrocode}
\keys_define:nn { document-structure / sfragment }{
  id            .str_set_x:N = \l_@@_sfragment_id_str,
  date          .str_set_x:N = \l_@@_sfragment_date_str,
  creators      .clist_set:N = \l_@@_sfragment_creators_clist,
  contributors  .clist_set:N = \l_@@_sfragment_contributors_clist,
  srccite       .tl_set:N    = \l_@@_sfragment_srccite_tl,
  type          .tl_set:N    = \l_@@_sfragment_type_tl,
  short         .tl_set:N    = \l_@@_sfragment_short_tl,
  display       .tl_set:N    = \l_@@_sfragment_display_tl,
  intro         .tl_set:N    = \l_@@_sfragment_intro_tl,
  imports       .tl_set:N    = \l_@@_sfragment_imports_tl,
  loadmodules   .bool_set:N  = \l_@@_sfragment_loadmodules_bool
}
\cs_new_protected:Nn \_@@_sfragment_args:n {
  \str_clear:N \l_@@_sfragment_id_str
  \str_clear:N \l_@@_sfragment_date_str
  \clist_clear:N \l_@@_sfragment_creators_clist
  \clist_clear:N \l_@@_sfragment_contributors_clist
  \tl_clear:N \l_@@_sfragment_srccite_tl
  \tl_clear:N \l_@@_sfragment_type_tl
  \tl_clear:N \l_@@_sfragment_short_tl
  \tl_clear:N \l_@@_sfragment_display_tl
  \tl_clear:N \l_@@_sfragment_imports_tl
  \tl_clear:N \l_@@_sfragment_intro_tl
  \bool_set_false:N \l_@@_sfragment_loadmodules_bool
  \keys_set:nn { document-structure / sfragment } { #1 }
}
%    \end{macrocode}
% we define a switch for numbering lines and a hook for the beginning of groups: The
% \DescribeMacro{\at@begin@sfragment}|\at@begin@sfragment| macro allows customization. It is
% run at the beginning of the |sfragment|, i.e. after the section heading.
%    \begin{macrocode}
\newif\if@mainmatter\@mainmattertrue
\newcommand\at@begin@sfragment[3][]{}
%    \end{macrocode}
%
% Then we define a helper macro that takes care of the sectioning magic. It comes with its
% own key/value interface for customization.
%
%    \begin{macrocode}
\keys_define:nn { document-structure / sectioning }{
  name    .str_set_x:N  = \l_@@_sect_name_str   ,
  ref     .str_set_x:N  = \l_@@_sect_ref_str    ,
  clear   .bool_set:N   = \l_@@_sect_clear_bool ,
  clear   .default:n    = {true}                ,
  num     .bool_set:N   = \l_@@_sect_num_bool   ,
  num     .default:n    = {true}
}
\cs_new_protected:Nn \_@@_sect_args:n {
  \str_clear:N \l_@@_sect_name_str
  \str_clear:N \l_@@_sect_ref_str
  \bool_set_false:N \l_@@_sect_clear_bool
  \bool_set_false:N \l_@@_sect_num_bool
  \keys_set:nn { document-structure / sectioning } { #1 }
}
\newcommand\omdoc@sectioning[3][]{
  \_@@_sect_args:n {#1 }
  \let\omdoc@sect@name\l_@@_sect_name_str
  \bool_if:NT \l_@@_sect_clear_bool { \cleardoublepage }
  \if@mainmatter% numbering not overridden by frontmatter, etc.
    \bool_if:NTF \l_@@_sect_num_bool {
      \sfragment@num{#2}{#3}
    }{
      \sfragment@nonum{#2}{#3}
    }
    \def\current@section@level{\omdoc@sect@name}
  \else
    \sfragment@nonum{#2}{#3}
  \fi
}% if@mainmatter
%    \end{macrocode}
% and another one, if redefines the |\addtocontentsline| macro of {\LaTeX} to import the
% respective macros. It takes as an argument a list of module names.
%    \begin{macrocode}
\newcommand\sfragment@redefine@addtocontents[1]{%
%\edef\@@import{#1}%
%\@for\@I:=\@@import\do{%
%\edef\@path{\csname module@\@I  @path\endcsname}%
%\@ifundefined{tf@toc}\relax%
%     {\protected@write\tf@toc{}{\string\@requiremodules{\@path}}}}
%\ifx\hyper@anchor\@undefined% hyperref.sty loaded?
%\def\addcontentsline##1##2##3{%
%\addtocontents{##1}{\protect\contentsline{##2}{\string\withusedmodules{#1}{##3}}{\thepage}}}
%\else% hyperref.sty not loaded
%\def\addcontentsline##1##2##3{%
%\addtocontents{##1}{\protect\contentsline{##2}{\string\withusedmodules{#1}{##3}}{\thepage}{\@currentHref}}}%
%\fi
}% hypreref.sty loaded?
%    \end{macrocode}
% now the |sfragment| environment itself. This takes care of the table of contents via the
% helper macro above and then selects the appropriate sectioning command from
% |article.cls|. It also registeres the current level of sfragments in the |\sfragment@level|
% counter. 
%    \begin{macrocode}
\newenvironment{sfragment}[2][]% keys, title
{
  \_@@_sfragment_args:n { #1 }%\sref@target%
%    \end{macrocode}
% If the |loadmodules| key is set on |\begin{sfragment}|, we redefine the |\addcontetsline|
%   macro that determines how the sectioning commands below construct the entries for the
%   table of contents.
%    \begin{macrocode}
  \stex_csl_to_imports:No \usemodule \l_@@_sfragment_imports_tl

  \bool_if:NT \l_@@_sfragment_loadmodules_bool {
    \sfragment@redefine@addtocontents{
      %\@ifundefined{module@id}\used@modules%
      %{\@ifundefined{module@\module@id @path}{\used@modules}\module@id}
    }
  }
%    \end{macrocode}
% now we only need to construct the right sectioning depending on the value of
% |\section@level|.
%    \begin{macrocode}

  \stex_document_title:n { #2 }
  
  \int_incr:N\l_document_structure_section_level_int
  \ifcase\l_document_structure_section_level_int
    \or\omdoc@sectioning[name=\omdoc@part@kw,clear,num]{part}{#2}
    \or\omdoc@sectioning[name=\omdoc@chapter@kw,clear,num]{chapter}{#2}
    \or\omdoc@sectioning[name=\omdoc@section@kw,num]{section}{#2}
    \or\omdoc@sectioning[name=\omdoc@subsection@kw,num]{subsection}{#2}
    \or\omdoc@sectioning[name=\omdoc@subsubsection@kw,num]{subsubsection}{#2}
    \or\omdoc@sectioning[name=\omdoc@paragraph@kw,ref=this \omdoc@paragraph@kw]{paragraph}{#2}
    \or\omdoc@sectioning[name=\omdoc@subparagraph@kw,ref=this \omdoc@subparagraph@kw]{paragraph}{#2}
  \fi
  \at@begin@sfragment[#1]\l_document_structure_section_level_int{#2}
  \str_if_empty:NF \l_@@_sfragment_id_str {
    \stex_ref_new_doc_target:n\l_@@_sfragment_id_str
  }
}% for customization
{}
%    \end{macrocode}
% \end{environment}
%
% and finally, we localize the sections
%    \begin{macrocode}
\newcommand\omdoc@part@kw{Part}
\newcommand\omdoc@chapter@kw{Chapter}
\newcommand\omdoc@section@kw{Section}
\newcommand\omdoc@subsection@kw{Subsection}
\newcommand\omdoc@subsubsection@kw{Subsubsection}
\newcommand\omdoc@paragraph@kw{paragraph}
\newcommand\omdoc@subparagraph@kw{subparagraph}
%    \end{macrocode}
%
% \end{sfragment}
%
% \begin{sfragment}[id=sec:user:docmatter]{Front and Backmatter}
% 
%   Index markup is provided by the |omtext| package~\cite{Kohlhase:smmtf:git}, so in the
%   |document-structure| package we only need to supply the corresponding |\printindex| command, if it
%   is not already defined
% \begin{macro}{\printindex}
%    \begin{macrocode}
\providecommand\printindex{\IfFileExists{\jobname.ind}{\input{\jobname.ind}}{}}
%    \end{macrocode}
% \end{macro}
% 
% some classes (e.g. |book.cls|) already have |\frontmatter|, |\mainmatter|, and
% |\backmatter| macros. As we want to define |frontmatter| and |backmatter| environments,
% we save their behavior (possibly defining it) in |orig@*matter| macros and make them
% undefined (so that we can define the environments).
%    \begin{macrocode}
\cs_if_exist:NTF\frontmatter{
  \let\_@@_orig_frontmatter\frontmatter
  \let\frontmatter\relax
}{
  \tl_set:Nn\_@@_orig_frontmatter{
    \clearpage
    \@mainmatterfalse
    \pagenumbering{roman}
  }
}
\cs_if_exist:NTF\backmatter{
  \let\_@@_orig_backmatter\backmatter
  \let\backmatter\relax
}{
  \tl_set:Nn\_@@_orig_backmatter{
    \clearpage
    \@mainmatterfalse
    \pagenumbering{roman}
  }
}
%    \end{macrocode}
%
% Using these, we can now define the |frontmatter| and |backmatter| environments
% 
% \begin{environment}{frontmatter}
%  we use the |\orig@frontmatter| macro defined above and |\mainmatter| if it exists,
%  otherwise we define it.  
%    \begin{macrocode}
\newenvironment{frontmatter}{
  \_@@_orig_frontmatter
}{
  \cs_if_exist:NTF\mainmatter{
    \mainmatter
  }{
    \clearpage
    \@mainmattertrue
    \pagenumbering{arabic}
  }
}
%    \end{macrocode}
% \end{environment}
%
% \begin{environment}{backmatter}
%   As backmatter is at the end of the document, we do nothing for |\endbackmatter|. 
%    \begin{macrocode}
\newenvironment{backmatter}{
  \_@@_orig_backmatter
}{
  \cs_if_exist:NTF\mainmatter{
    \mainmatter
  }{
    \clearpage
    \@mainmattertrue
    \pagenumbering{arabic}
  }
}
%    \end{macrocode}
%
% finally, we make sure that page numbering is arabic and we have main matter as the default
%
%    \begin{macrocode}
\@mainmattertrue\pagenumbering{arabic}
%    \end{macrocode}
% \end{environment}
% \end{sfragment}
%
% \begin{macro}{\prematurestop}
%   We initialize |\afterprematurestop|, and provide 
%   |\prematurestop@endsfragment| which looks up |\sfragment@level| and recursively ends
%   enough |{sfragment}|s. 
%    \begin{macrocode}
\def \c_@@_document_str{document} 
\newcommand\afterprematurestop{}
\def\prematurestop@endsfragment{
  \unless\ifx\@currenvir\c_@@_document_str
    \expandafter\expandafter\expandafter\end\expandafter\expandafter\expandafter{\expandafter\@currenvir\expandafter}
    \expandafter\prematurestop@endsfragment
  \fi
}
\providecommand\prematurestop{
  \message{Stopping~sTeX~processing~prematurely}
  \prematurestop@endsfragment
  \afterprematurestop
  \end{document}
}
%    \end{macrocode}
% \end{macro}
%
% \begin{sfragment}[id=sec:impl:gvars]{Global Variables}
%
% \begin{macro}{\setSGvar}
%   set a global variable
%    \begin{macrocode}
\RequirePackage{etoolbox}
\newcommand\setSGvar[1]{\@namedef{sTeX@Gvar@#1}}
%    \end{macrocode}
% \end{macro}
%
% \begin{macro}{\useSGvar}
%   use a global variable
%    \begin{macrocode}
\newrobustcmd\useSGvar[1]{%
  \@ifundefined{sTeX@Gvar@#1}
  {\PackageError{document-structure}
    {The sTeX Global variable #1 is undefined}
    {set it with \protect\setSGvar}}
\@nameuse{sTeX@Gvar@#1}}
%    \end{macrocode}
% \end{macro}
%
% \begin{macro}{\ifSGvar}
%   execute something conditionally based on the state of the global variable. 
%    \begin{macrocode}
\newrobustcmd\ifSGvar[3]{\def\@test{#2}%
  \@ifundefined{sTeX@Gvar@#1}
  {\PackageError{document-structure}
    {The sTeX Global variable #1 is undefined}
    {set it with \protect\setSGvar}}
  {\expandafter\ifx\csname sTeX@Gvar@#1\endcsname\@test #3\fi}}
%    \end{macrocode}
% \end{macro}
%
% \end{sfragment}
% \end{sfragment}
%
% \end{implementation}
% \ifinfulldoc\else\printbibliography\fi
\endinput

%%% Local Variables: 
%%% mode: doctex
%%% TeX-master: t
%%% End: 
